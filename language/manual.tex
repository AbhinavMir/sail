\documentclass[11pt]{article}

\usepackage{amsmath,amssymb,supertabular,geometry,fullpage}
\geometry{a4paper,twoside,landscape,left=10.5mm,right=10.5mm,top=20mm,bottom=30mm}
\usepackage{color}

\begin{document}

\input{doc_in}

\title{Sail Manual}
\author{Kathryn E Gray, Gabriel Kerneis, Peter Sewell}

\maketitle

\tableofcontents

\newpage

\section{Introduction}

This is a manual describing the Sail specification language, its
common library, compiler, interpreter and type system. However it is
currently in early stages of being written, so questions to the
developers are highly encouraged.

\section{Sail syntax}

\ottgrammartabular{
\ottl\ottinterrule
\ottannot\ottinterrule
\ottid\ottinterrule
\ottkid\ottinterrule
\ottbaseXXkind\ottinterrule
\ottkind\ottinterrule
\ottnexp\ottinterrule
\ottorder\ottinterrule
\ottbaseXXeffect\ottinterrule
\otteffect\ottinterrule
\otttyp\ottinterrule
\otttypXXarg\ottinterrule
\ottnXXconstraint\ottinterrule
\ottkindedXXid\ottinterrule
\ottquantXXitem\ottinterrule
\otttypquant\ottinterrule
\otttypschm\ottinterrule
\ottnameXXscmXXopt\ottinterrule
\otttypeXXdef\ottinterrule
\otttypeXXunion\ottinterrule
\ottindexXXrange\ottinterrule
\ottlit\ottinterrule
\ottsemiXXopt\ottinterrule
\ottpat\ottinterrule
\ottfpat\ottinterrule
\ottexp\ottinterrule
\ottlexp\ottinterrule
\ottfexp\ottinterrule
\ottfexps\ottinterrule
\ottoptXXdefault\ottinterrule
\ottpexp\ottinterrule
\otttannotXXopt\ottinterrule
\ottrecXXopt\ottinterrule
\otteffectXXopt\ottinterrule
\ottfuncl\ottinterrule
\ottfundef\ottinterrule
\ottletbind\ottinterrule
\ottvalXXspec\ottinterrule
\ottdefaultXXspec\ottinterrule
\ottscatteredXXdef\ottinterrule
\ottregXXid\ottinterrule
\ottaliasXXspec\ottinterrule
\ottdecXXspec\ottinterrule
\ottdef\ottinterrule
\ottdefs\ottinterrule}

\newpage
\section{Sail primitive types and functions}

\ottgrammartabular{
\ottbuiltXXinXXtypes\ottinterrule}

\ottgrammartabular{
\ottbuiltXXinXXtypeXXabbreviations\ottinterrule
\ottfunctions\ottinterrule
\ottfunctionsXXwithXXcoercions\ottinterrule}
\newpage

\section{Tips for Writing Sail specifications}

This section attempts to offer advice for writing Sail specifications
that will work well with the Sail executable interpreter and
compilers.

Some tips might also be advice for good ways to specify instructions;
this will come from a combination of users and Sail developers.

\begin{itemize}
\item Be precise in numeric types. 

While Sail includes very wide types like int and nat, consider that
for bounds checking, numeric operations, and and clear understanding,
these really are unbounded quantities. If you know that a number in
the specification will range only between 0 and 32, 0 and 4, -32 to
32, it is better to use a specific range type such as [|32|]. 

Similarly, if you don't know the range precisely, it may also be best
to remain polymorphic and let Sail's type resolution work out bounds
in a particular use rather than removing all bounds; to do this, use
[:'n:] to say that it will polymorphically take some number.

\item Use bit vectors for registers.

Sail the language will readily allow a register to store a value of
any type. However, the Sail executable interpreter expects that it is
simulating a uni-processor machine where all registers are bit
vectors.

A vector of length one, such as \emph{a} can read the element \emph{a}
either with {\tt a} or {\tt a[0]}.

\item Have functions named decode and execute to evaluate
  instructions.

The sail interpreter is hard-wired to look for functions with these names.

\end{itemize}

\section{Sail type system}

\subsection{Internal type syntax}

\ottgrammartabular{
\ottk\ottinterrule
\ottt\ottinterrule
\ottoptx\ottinterrule
\otttag\ottinterrule
\ottne\ottinterrule
\otttXXarg\ottinterrule
\otttXXargs\ottinterrule
\ottnec\ottinterrule
\ottSXXN\ottinterrule
\ottEXXd\ottinterrule
\ottkinf\ottinterrule
\otttid\ottinterrule
\ottEXXk\ottinterrule
\otttinf\ottinterrule
\ottEXXa\ottinterrule
\ottfieldXXtyps\ottinterrule
\ottEXXr\ottinterrule
\ottenumerateXXmap\ottinterrule
\ottEXXe\ottinterrule
\ottEXXt\ottinterrule
\ottts\ottinterrule
\ottE\ottinterrule
\ottI\ottinterrule
\ottformula\ottinterrule}


\subsection{ Type relations }
\ottdefnss

\section{Sail operational semantics \{TODO\}}

\end{document}