\section{Using Sail}
\label{sec:usage}

In its most basic use-case Sail is a command-line tool, analogous to
a compiler: one gives it a list of input Sail files; it type-checks
them and provides translated output.

To simply typecheck Sail files, one can pass them on the command line
with no other options, so for our \riscv\ spec:
\begin{verbatim}
sail prelude.sail riscv_types.sail riscv_mem.sail riscv_sys.sail riscv_vmem.sail riscv.sail
\end{verbatim}
The sail files passed on the command line are simply treated as if
they are one large file concatenated together, although the parser
will keep track of locations on a per-file basis for
error-reporting. As can be seen, this specification is split into
several logical components. \verb+prelude.sail+ defines the initial
type environment and builtins, \verb+riscv_types.sail+ gives type
definitions used in the rest of the specification, \verb+riscv_mem.sail+
and \verb+riscv_vmem.sail+ describe the physical and virtual memory
interaction, and then \verb+riscv_sys.sail+ and \verb+riscv.sail+
implement most of the specification.

For more complex projects, one can use \ll{$include} statements in
Sail source, for example:
\begin{lstlisting}
$include <library.sail>
$include "file.sail"
\end{lstlisting}

Here, Sail will look for \verb+library.sail+ in the
\verb+$SAIL_DIR/lib+, where \verb+$SAIL_DIR+ is usually the root of
the sail repository. It will search for \verb+file.sail+ relative to
the location of the file containing the \ll{$include}. The space after
the \ll{$include} is mandatory. Sail also supports \ll{$define},
\ll{$ifdef}, and \ll{$ifndef}. These are things that are understood by
Sail itself, not a separate preprocessor, and are handled after the
AST is parsed~\footnote{This can affect precedence declarations for custom user defined operators---the precedence must be redeclared in the file you are including the operator into.}.

\subsection{OCaml compilation}

To compile a Sail specification into OCaml, one calls Sail as
\begin{verbatim}
sail -ocaml FILES
\end{verbatim}
This will produce a version of the specification translated into
OCaml, which is placed into a directory called \verb+_sbuild+, similar
to ocamlbuild's \verb+_build+ directory. The generated OCaml is
intended to be fairly close to the original Sail source, and currently
we do not attempt to do much optimisation on this output.

The contents of the \verb+_sbuild+ directory are set up as an
ocamlbuild project, so one can simply switch into that directory and run
\begin{verbatim}
ocamlbuild -use-ocamlfind out.cmx
\end{verbatim}
to compile the generated model. Currently the OCaml compilation
requires that lem, linksem, and zarith are available as ocamlfind
findable libraries, and also that the environment variable
\verb+$SAIL_DIR+ is set to the root of the Sail repository.

If the Sail specification contains a \ll{main} function with type
\ll{unit -> unit} that implements a fetch/decode/execute loop then the
OCaml backend can produce a working executable, by running
\begin{verbatim}
sail -o out -ocaml FILES
\end{verbatim}
Then one can run
\begin{verbatim}
./out ELF_FILE
\end{verbatim}
to simulate an ELF file on the specification. One can do \ll{$include
  <elf.sail>} to gain access to some useful functions for accessing
information about the loaded ELF file from within the Sail
specification. In particular \verb+elf.sail+ defines a function
\ll{elf_entry : unit -> int} which can be used to set the PC to the
correct location. ELF loading is done by the linksem
library\footnote{\url{https://github.com/rems-project/linksem}}.

There is also an \verb+-ocaml_trace+ option which is the same as
\verb+-ocaml+ except it instruments the generated OCaml code with
tracing information.

\subsection{C compilation}

To compile Sail into C, the \verb+-c+ option is used, like so:
\begin{verbatim}
sail -c FILES 1> out.c
\end{verbatim}
The transated C is currently printed to stdout, so this should be
redirected to a file as above. To produce an executable this needs to
be compiled and linked with the C files in the \verb+sail/lib+
directory:
\begin{verbatim}
gcc out.c $SAIL_DIR/lib/*.c -lgmp -lz -I $SAIL_DIR/lib/ -o out
\end{verbatim}
The C output requires the GMP library for arbitrary precision
arithmetic, as well as zlib for working with compressed ELF binaries.

There are several Sail options that affect the C output:
\begin{itemize}
  \item \verb+-O+ turns on optimisations. The generated C code will be
    quite slow unless this flag is set.
  \item \verb+-Oconstant_fold+ apply constant folding optimisations.
  \item \verb+-c_include+ Supply additional header files to be
    included in the generated C.
  \item \verb+-c_no_main+ Do not generate a \verb+main()+ function.
  \item \verb+-static+ Mark generated C functions as static where
    possible. This is useful for measuring code coverage.
\end{itemize}

The generated executable for the Sail specification (provided a main
function is generated) supports several options for loading ELF files
and binary data into the specification memory.
\begin{itemize}
\item \verb+-e/--elf+ Loads an ELF file into memory. Currently only
  AArch64 and RISC-V ELF files are supported.

\item \verb+-b/--binary+ Loads raw binary data into the
  specification's memory. It is used like so:
\begin{verbatim}
./out --binary=0xADDRESS,FILE
./out -b 0xADDRESS,FILE
\end{verbatim}
The contents of the supplied file will be placed in memory starting at
the given address, which must be given as a hexadecimal number.

\item \verb+-i/--image+ For ELF files that are not loadable via the
  \verb+--elf+ flag, they can be pre-processed by Sail using linksem
  into a special image file that can be loaded via this flag. This is
  done like so:
\begin{verbatim}
sail -elf ELF_FILE -o image.bin
./out --image=image.bin
\end{verbatim}
The advantage of this flag is that it uses Linksem to process the ELF
file, so it can handle any ELF file that linksem is able to
understand. This also guarantees that the contents of the ELF binary
loaded into memory is exactly the same as for the OCaml backend and
the interpreter as they both also use Linksem internally to load ELF
files.
\item \verb+-n/--entry+ sets a custom entry point returned by the
  \ll{elf_entry} function. Must be a hexadecimal address prefixed by
  \verb+0x+.
\item \verb+-l/--cyclelimit+ run the simulation until a set number of
  cycles have been reached. The main loop of the specification must
  call the \ll{cycle_count} function for this to work.
\end{itemize}

\subsection{Lem, Isabelle \& HOL4}

We have a separate document detailing how to generate Isabelle
theories from Sail models, and how to work with those models in
Isabelle, see:
\begin{center}
\anonymise{\url{https://github.com/rems-project/sail/raw/sail2/snapshots/isabelle/Manual.pdf}}
\end{center}
Currently there are generated Isabelle snapshots for some of our
models in \verb+snapshots/isabelle+ in the Sail repository. These
snapshots are provided for convenience, and are not guaranteed to be
up-to-date.

In order to open a theory of one of the specifications in Isabelle,
use the \verb+-l Sail+ command-line flag to load the session containing the
Sail library. Snapshots of the Sail and Lem libraries are in the
\verb+lib/sail+ and \verb+lib/lem+ directories, respectively. You can
tell Isabelle where to find them using the \verb+-d+ flag, as in
\begin{verbatim}
isabelle jedit -l Sail -d lib/lem -d lib/sail riscv/Riscv.thy
\end{verbatim}
When run from the \verb+snapshots/isabelle+ directory this will open
the RISC-V specification.

\subsection{Interactive mode}

Compiling Sail with
\begin{verbatim}
make isail
\end{verbatim}
builds it with a GHCi-style interactive interpreter. This can be used
by starting Sail with \verb+sail -i+. If Sail is not compiled with
interactive support the \verb+-i+ flag does nothing. Sail will still
handle any other command line arguments as per usual, including
compiling to OCaml or Lem. One can also pass a list of commands to the
interpreter by using the \verb+-is+ flag, as
\begin{verbatim}
sail -is FILE
\end{verbatim}
where \verb+FILE+ contains a list of commands. Once inside the interactive
mode, a list of commands can be accessed by typing \verb+:commands+,
while \verb+:help+ can be used to provide some documentation for each
command.

\subsection{\LaTeX\ Generation}

Sail can be used to generate latex for inclusion in documentation as:
\begin{verbatim}
sail -o DIRECTORY -latex FILES
\end{verbatim}
The list of \verb+FILES+ is a list of Sail files to produce latex for,
and \verb+DIRECTORY+ is the directory where the generated latex will
be placed. The list of files must be a valid type-checkable series of
Sail files. The intention behind this latex generation is for it to be
included within existing ISA manuals written in Latex, as such the
latex output generates a list of commands for each top-level Sail
declaration in \verb+DIRECTORY/commands.tex+. The rest of this section
discusses the stable features of the latex generation process---there
are additional features for including markdown doc-comments in Sail
code and formatting them into latex for inclusion id documentation,
among other things, but these features are not completely stable
yet. This manual itself makes use of the Sail latex generation, so
\verb+doc/manual.tex+, and \verb+doc/Makefile+ can be used to see how
the process is set up.


\paragraph{Requirements} The generated latex uses the \emph{listings} package for formatting
source code, uses the macros in the \emph{etoolbox} package for the
generated commands, and relies on the \emph{hyperref} package for
cross-referencing. These packages are available in most TeX
distributions, and are available as part of thetexlive packages for
Ubuntu.

\paragraph{Usage} Due to the oddities of latex verbatim environments each Sail
declaration must be placed in it's own file then the command in
\verb+commands.tex+ includes in with \verb+\lstinputlisting+. To
include the generated Sail in a document one would do something like:
\begin{lstlisting}[language=TeX]
  \newcommand{\sailsailregderefv}{\label{zregzyderef} \lstinputlisting[language=sail]{sail_latex/sailsailregderefv.tex}}

\newcommand{\sailregderef}{\label{zzyregzyderef} \lstinputlisting[language=sail]{sail_latex/sailregderef.tex}}

\newcommand{\saileqbittwo}{\label{zeqzybittwo} \lstinputlisting[language=sail]{sail_latex/saileqbittwo.tex}}

\newcommand{\sailsailsailsailzeightoperatorzzerozJzJzninevvv}{\label{zzeightoperatorzzerozJzJznine} \lstinputlisting[language=sail]{sail_latex/sailsailsailsailzeightoperatorzzerozJzJzninevvv.tex}}

\newcommand{\saildiv}{\label{zdiv} \lstinputlisting[language=sail]{sail_latex/saildiv.tex}}

\newcommand{\sailsailsailzeightoperatorzzerozFzninevv}{\label{zzeightoperatorzzerozFznine} \lstinputlisting[language=sail]{sail_latex/sailsailsailzeightoperatorzzerozFzninevv.tex}}

\newcommand{\sailmod}{\label{zmod} \lstinputlisting[language=sail]{sail_latex/sailmod.tex}}

\newcommand{\sailsailsailzeightoperatorzzerozfivezninevv}{\label{zzeightoperatorzzerozfiveznine} \lstinputlisting[language=sail]{sail_latex/sailsailsailzeightoperatorzzerozfivezninevv.tex}}

\newcommand{\sailabsatom}{\label{zabszyatom} \lstinputlisting[language=sail]{sail_latex/sailabsatom.tex}}

\newcommand{\sailnotbool}{\label{znotzybool} \lstinputlisting[language=sail]{sail_latex/sailnotbool.tex}}

\newcommand{\sailandbool}{\label{zandzybool} \lstinputlisting[language=sail]{sail_latex/sailandbool.tex}}

\newcommand{\sailorbool}{\label{zorzybool} \lstinputlisting[language=sail]{sail_latex/sailorbool.tex}}

\newcommand{\saileqatom}{\label{zeqzyatom} \lstinputlisting[language=sail]{sail_latex/saileqatom.tex}}

\newcommand{\sailneqatom}{\label{zneqzyatom} \lstinputlisting[language=sail]{sail_latex/sailneqatom.tex}}

\newcommand{\sailfnneqatom}{\label{zneqzyatom} \lstinputlisting[language=sail]{sail_latex/sailfnneqatom.tex}}

\newcommand{\saillteqatom}{\label{zlteqzyatom} \lstinputlisting[language=sail]{sail_latex/saillteqatom.tex}}

\newcommand{\sailgteqatom}{\label{zgteqzyatom} \lstinputlisting[language=sail]{sail_latex/sailgteqatom.tex}}

\newcommand{\sailltatom}{\label{zltzyatom} \lstinputlisting[language=sail]{sail_latex/sailltatom.tex}}

\newcommand{\sailgtatom}{\label{zgtzyatom} \lstinputlisting[language=sail]{sail_latex/sailgtatom.tex}}

\newcommand{\sailltrangeatom}{\label{zltzyrangezyatom} \lstinputlisting[language=sail]{sail_latex/sailltrangeatom.tex}}

\newcommand{\saillteqrangeatom}{\label{zlteqzyrangezyatom} \lstinputlisting[language=sail]{sail_latex/saillteqrangeatom.tex}}

\newcommand{\sailgtrangeatom}{\label{zgtzyrangezyatom} \lstinputlisting[language=sail]{sail_latex/sailgtrangeatom.tex}}

\newcommand{\sailgteqrangeatom}{\label{zgteqzyrangezyatom} \lstinputlisting[language=sail]{sail_latex/sailgteqrangeatom.tex}}

\newcommand{\sailltatomrange}{\label{zltzyatomzyrange} \lstinputlisting[language=sail]{sail_latex/sailltatomrange.tex}}

\newcommand{\saillteqatomrange}{\label{zlteqzyatomzyrange} \lstinputlisting[language=sail]{sail_latex/saillteqatomrange.tex}}

\newcommand{\sailgtatomrange}{\label{zgtzyatomzyrange} \lstinputlisting[language=sail]{sail_latex/sailgtatomrange.tex}}

\newcommand{\sailgteqatomrange}{\label{zgteqzyatomzyrange} \lstinputlisting[language=sail]{sail_latex/sailgteqatomrange.tex}}

\newcommand{\saileqrange}{\label{zeqzyrange} \lstinputlisting[language=sail]{sail_latex/saileqrange.tex}}

\newcommand{\saileqint}{\label{zeqzyint} \lstinputlisting[language=sail]{sail_latex/saileqint.tex}}

\newcommand{\saileqbool}{\label{zeqzybool} \lstinputlisting[language=sail]{sail_latex/saileqbool.tex}}

\newcommand{\sailneqrange}{\label{zneqzyrange} \lstinputlisting[language=sail]{sail_latex/sailneqrange.tex}}

\newcommand{\sailfnneqrange}{\label{zneqzyrange} \lstinputlisting[language=sail]{sail_latex/sailfnneqrange.tex}}

\newcommand{\sailneqint}{\label{zneqzyint} \lstinputlisting[language=sail]{sail_latex/sailneqint.tex}}

\newcommand{\sailfnneqint}{\label{zneqzyint} \lstinputlisting[language=sail]{sail_latex/sailfnneqint.tex}}

\newcommand{\sailneqbool}{\label{zneqzybool} \lstinputlisting[language=sail]{sail_latex/sailneqbool.tex}}

\newcommand{\sailfnneqbool}{\label{zneqzybool} \lstinputlisting[language=sail]{sail_latex/sailfnneqbool.tex}}

\newcommand{\saillteqint}{\label{zlteqzyint} \lstinputlisting[language=sail]{sail_latex/saillteqint.tex}}

\newcommand{\sailgteqint}{\label{zgteqzyint} \lstinputlisting[language=sail]{sail_latex/sailgteqint.tex}}

\newcommand{\sailltint}{\label{zltzyint} \lstinputlisting[language=sail]{sail_latex/sailltint.tex}}

\newcommand{\sailgtint}{\label{zgtzyint} \lstinputlisting[language=sail]{sail_latex/sailgtint.tex}}

\newcommand{\sailsailsailzeightoperatorzzerozJzJzninevv}{\label{zzeightoperatorzzerozJzJznine} \lstinputlisting[language=sail]{sail_latex/sailsailsailzeightoperatorzzerozJzJzninevv.tex}}

\newcommand{\sailsailzeightoperatorzzerozonezJzninev}{\label{zzeightoperatorzzerozonezJznine} \lstinputlisting[language=sail]{sail_latex/sailsailzeightoperatorzzerozonezJzninev.tex}}

\newcommand{\sailsailzeightoperatorzzerozUzninev}{\label{zzeightoperatorzzerozUznine} \lstinputlisting[language=sail]{sail_latex/sailsailzeightoperatorzzerozUzninev.tex}}

\newcommand{\sailsailzeightoperatorzzerozsixzninev}{\label{zzeightoperatorzzerozsixznine} \lstinputlisting[language=sail]{sail_latex/sailsailzeightoperatorzzerozsixzninev.tex}}

\newcommand{\sailzeightoperatorzzerozIzJznine}{\label{zzeightoperatorzzerozIzJznine} \lstinputlisting[language=sail]{sail_latex/sailzeightoperatorzzerozIzJznine.tex}}

\newcommand{\sailzeightoperatorzzerozIznine}{\label{zzeightoperatorzzerozIznine} \lstinputlisting[language=sail]{sail_latex/sailzeightoperatorzzerozIznine.tex}}

\newcommand{\sailzeightoperatorzzerozKzJznine}{\label{zzeightoperatorzzerozKzJznine} \lstinputlisting[language=sail]{sail_latex/sailzeightoperatorzzerozKzJznine.tex}}

\newcommand{\sailzeightoperatorzzerozKznine}{\label{zzeightoperatorzzerozKznine} \lstinputlisting[language=sail]{sail_latex/sailzeightoperatorzzerozKznine.tex}}

\newcommand{\sailaddatom}{\label{zaddzyatom} \lstinputlisting[language=sail]{sail_latex/sailaddatom.tex}}

\newcommand{\sailaddint}{\label{zaddzyint} \lstinputlisting[language=sail]{sail_latex/sailaddint.tex}}

\newcommand{\sailsailsailzeightoperatorzzerozBzninevv}{\label{zzeightoperatorzzerozBznine} \lstinputlisting[language=sail]{sail_latex/sailsailsailzeightoperatorzzerozBzninevv.tex}}

\newcommand{\sailsubatom}{\label{zsubzyatom} \lstinputlisting[language=sail]{sail_latex/sailsubatom.tex}}

\newcommand{\sailsubint}{\label{zsubzyint} \lstinputlisting[language=sail]{sail_latex/sailsubint.tex}}

\newcommand{\sailsailzeightoperatorzzerozDzninev}{\label{zzeightoperatorzzerozDznine} \lstinputlisting[language=sail]{sail_latex/sailsailzeightoperatorzzerozDzninev.tex}}

\newcommand{\sailnegateatom}{\label{znegatezyatom} \lstinputlisting[language=sail]{sail_latex/sailnegateatom.tex}}

\newcommand{\sailnegateint}{\label{znegatezyint} \lstinputlisting[language=sail]{sail_latex/sailnegateint.tex}}

\newcommand{\sailsailnegatev}{\label{znegate} \lstinputlisting[language=sail]{sail_latex/sailsailnegatev.tex}}

\newcommand{\sailmultatom}{\label{zmultzyatom} \lstinputlisting[language=sail]{sail_latex/sailmultatom.tex}}

\newcommand{\sailmultint}{\label{zmultzyint} \lstinputlisting[language=sail]{sail_latex/sailmultint.tex}}

\newcommand{\sailsailzeightoperatorzzerozAzninev}{\label{zzeightoperatorzzerozAznine} \lstinputlisting[language=sail]{sail_latex/sailsailzeightoperatorzzerozAzninev.tex}}

\newcommand{\sailprintint}{\label{zprintzyint} \lstinputlisting[language=sail]{sail_latex/sailprintint.tex}}

\newcommand{\sailprerrint}{\label{zprerrzyint} \lstinputlisting[language=sail]{sail_latex/sailprerrint.tex}}

\newcommand{\sailshlint}{\label{zshlzyint} \lstinputlisting[language=sail]{sail_latex/sailshlint.tex}}

\newcommand{\sailshrint}{\label{zshrzyint} \lstinputlisting[language=sail]{sail_latex/sailshrint.tex}}

\newcommand{\saildivint}{\label{zdivzyint} \lstinputlisting[language=sail]{sail_latex/saildivint.tex}}

\newcommand{\sailsailzeightoperatorzzerozFzninev}{\label{zzeightoperatorzzerozFznine} \lstinputlisting[language=sail]{sail_latex/sailsailzeightoperatorzzerozFzninev.tex}}

\newcommand{\sailmodint}{\label{zmodzyint} \lstinputlisting[language=sail]{sail_latex/sailmodint.tex}}

\newcommand{\sailsailzeightoperatorzzerozfivezninev}{\label{zzeightoperatorzzerozfiveznine} \lstinputlisting[language=sail]{sail_latex/sailsailzeightoperatorzzerozfivezninev.tex}}

\newcommand{\sailabsint}{\label{zabszyint} \lstinputlisting[language=sail]{sail_latex/sailabsint.tex}}

\newcommand{\sailisnone}{\label{ziszynone} \lstinputlisting[language=sail]{sail_latex/sailisnone.tex}}

\newcommand{\sailfnisnone}{\label{ziszynone} \lstinputlisting[language=sail]{sail_latex/sailfnisnone.tex}}

\newcommand{\sailissome}{\label{ziszysome} \lstinputlisting[language=sail]{sail_latex/sailissome.tex}}

\newcommand{\sailfnissome}{\label{ziszysome} \lstinputlisting[language=sail]{sail_latex/sailfnissome.tex}}

\newcommand{\sailbits}{\label{zbits} \lstinputlisting[language=sail]{sail_latex/sailbits.tex}}

\newcommand{\saileqbit}{\label{zeqzybit} \lstinputlisting[language=sail]{sail_latex/saileqbit.tex}}

\newcommand{\saileqbits}{\label{zeqzybits} \lstinputlisting[language=sail]{sail_latex/saileqbits.tex}}

\newcommand{\sailsailzeightoperatorzzerozJzJzninev}{\label{zzeightoperatorzzerozJzJznine} \lstinputlisting[language=sail]{sail_latex/sailsailzeightoperatorzzerozJzJzninev.tex}}

\newcommand{\sailbitvectorlength}{\label{zbitvectorzylength} \lstinputlisting[language=sail]{sail_latex/sailbitvectorlength.tex}}

\newcommand{\sailvectorlength}{\label{zvectorzylength} \lstinputlisting[language=sail]{sail_latex/sailvectorlength.tex}}

\newcommand{\saillength}{\label{zlength} \lstinputlisting[language=sail]{sail_latex/saillength.tex}}

\newcommand{\sailsailzzeros}{\label{zsailzyzzeros} \lstinputlisting[language=sail]{sail_latex/sailsailzzeros.tex}}

\newcommand{\sailprintbits}{\label{zprintzybits} \lstinputlisting[language=sail]{sail_latex/sailprintbits.tex}}

\newcommand{\sailprerrbits}{\label{zprerrzybits} \lstinputlisting[language=sail]{sail_latex/sailprerrbits.tex}}

\newcommand{\sailsailsignextend}{\label{zsailzysignzyextend} \lstinputlisting[language=sail]{sail_latex/sailsailsignextend.tex}}

\newcommand{\sailsailzzeroextend}{\label{zsailzyzzerozyextend} \lstinputlisting[language=sail]{sail_latex/sailsailzzeroextend.tex}}

\newcommand{\sailtruncate}{\label{ztruncate} \lstinputlisting[language=sail]{sail_latex/sailtruncate.tex}}

\newcommand{\sailsailmask}{\label{zsailzymask} \lstinputlisting[language=sail]{sail_latex/sailsailmask.tex}}

\newcommand{\sailfnsailmask}{\label{zsailzymask} \lstinputlisting[language=sail]{sail_latex/sailfnsailmask.tex}}

\newcommand{\sailsailzeightoperatorzzerozQzninev}{\label{zzeightoperatorzzerozQznine} \lstinputlisting[language=sail]{sail_latex/sailsailzeightoperatorzzerozQzninev.tex}}

\newcommand{\sailbitvectorconcat}{\label{zbitvectorzyconcat} \lstinputlisting[language=sail]{sail_latex/sailbitvectorconcat.tex}}

\newcommand{\sailappend}{\label{zappend} \lstinputlisting[language=sail]{sail_latex/sailappend.tex}}

\newcommand{\sailappendsixfour}{\label{zappendzysixfour} \lstinputlisting[language=sail]{sail_latex/sailappendsixfour.tex}}

\newcommand{\sailbitvectoraccess}{\label{zbitvectorzyaccess} \lstinputlisting[language=sail]{sail_latex/sailbitvectoraccess.tex}}

\newcommand{\sailplainvectoraccess}{\label{zplainzyvectorzyaccess} \lstinputlisting[language=sail]{sail_latex/sailplainvectoraccess.tex}}

\newcommand{\sailvectoraccess}{\label{zvectorzyaccess} \lstinputlisting[language=sail]{sail_latex/sailvectoraccess.tex}}

\newcommand{\sailbitvectorupdate}{\label{zbitvectorzyupdate} \lstinputlisting[language=sail]{sail_latex/sailbitvectorupdate.tex}}

\newcommand{\sailplainvectorupdate}{\label{zplainzyvectorzyupdate} \lstinputlisting[language=sail]{sail_latex/sailplainvectorupdate.tex}}

\newcommand{\sailvectorupdate}{\label{zvectorzyupdate} \lstinputlisting[language=sail]{sail_latex/sailvectorupdate.tex}}

\newcommand{\sailaddbits}{\label{zaddzybits} \lstinputlisting[language=sail]{sail_latex/sailaddbits.tex}}

\newcommand{\sailaddbitsint}{\label{zaddzybitszyint} \lstinputlisting[language=sail]{sail_latex/sailaddbitsint.tex}}

\newcommand{\sailsailzeightoperatorzzerozBzninev}{\label{zzeightoperatorzzerozBznine} \lstinputlisting[language=sail]{sail_latex/sailsailzeightoperatorzzerozBzninev.tex}}

\newcommand{\sailvectorsubrange}{\label{zvectorzysubrange} \lstinputlisting[language=sail]{sail_latex/sailvectorsubrange.tex}}

\newcommand{\sailvectorupdatesubrange}{\label{zvectorzyupdatezysubrange} \lstinputlisting[language=sail]{sail_latex/sailvectorupdatesubrange.tex}}

\newcommand{\sailgetsliceint}{\label{zgetzyslicezyint} \lstinputlisting[language=sail]{sail_latex/sailgetsliceint.tex}}

\newcommand{\sailsetsliceint}{\label{zsetzyslicezyint} \lstinputlisting[language=sail]{sail_latex/sailsetsliceint.tex}}

\newcommand{\sailsetslicebits}{\label{zsetzyslicezybits} \lstinputlisting[language=sail]{sail_latex/sailsetslicebits.tex}}

\newcommand{\sailslice}{\label{zslice} \lstinputlisting[language=sail]{sail_latex/sailslice.tex}}

\newcommand{\sailreplicatebits}{\label{zreplicatezybits} \lstinputlisting[language=sail]{sail_latex/sailreplicatebits.tex}}

\newcommand{\sailunsigned}{\label{zunsigned} \lstinputlisting[language=sail]{sail_latex/sailunsigned.tex}}

\newcommand{\sailsigned}{\label{zsigned} \lstinputlisting[language=sail]{sail_latex/sailsigned.tex}}

\newcommand{\saileqanything}{\label{zeqzyanything} \lstinputlisting[language=sail]{sail_latex/saileqanything.tex}}

\newcommand{\sailzeightoperatorzzerozJzJznine}{\label{zzeightoperatorzzerozJzJznine} \lstinputlisting[language=sail]{sail_latex/sailzeightoperatorzzerozJzJznine.tex}}

\newcommand{\sailnotvec}{\label{znotzyvec} \lstinputlisting[language=sail]{sail_latex/sailnotvec.tex}}

\newcommand{\sailzW}{\label{zzW} \lstinputlisting[language=sail]{sail_latex/sailzW.tex}}

\newcommand{\sailnot}{\label{znot} \lstinputlisting[language=sail]{sail_latex/sailnot.tex}}

\newcommand{\sailneqvec}{\label{zneqzyvec} \lstinputlisting[language=sail]{sail_latex/sailneqvec.tex}}

\newcommand{\sailfnneqvec}{\label{zneqzyvec} \lstinputlisting[language=sail]{sail_latex/sailfnneqvec.tex}}

\newcommand{\sailneqanything}{\label{zneqzyanything} \lstinputlisting[language=sail]{sail_latex/sailneqanything.tex}}

\newcommand{\sailfnneqanything}{\label{zneqzyanything} \lstinputlisting[language=sail]{sail_latex/sailfnneqanything.tex}}

\newcommand{\sailzeightoperatorzzerozonezJznine}{\label{zzeightoperatorzzerozonezJznine} \lstinputlisting[language=sail]{sail_latex/sailzeightoperatorzzerozonezJznine.tex}}

\newcommand{\sailandbits}{\label{zandzybits} \lstinputlisting[language=sail]{sail_latex/sailandbits.tex}}

\newcommand{\sailzeightoperatorzzerozsixznine}{\label{zzeightoperatorzzerozsixznine} \lstinputlisting[language=sail]{sail_latex/sailzeightoperatorzzerozsixznine.tex}}

\newcommand{\sailorbits}{\label{zorzybits} \lstinputlisting[language=sail]{sail_latex/sailorbits.tex}}

\newcommand{\sailzeightoperatorzzerozUznine}{\label{zzeightoperatorzzerozUznine} \lstinputlisting[language=sail]{sail_latex/sailzeightoperatorzzerozUznine.tex}}

\newcommand{\sailcastunitvec}{\label{zcastzyunitzyvec} \lstinputlisting[language=sail]{sail_latex/sailcastunitvec.tex}}

\newcommand{\sailfncastunitvec}{\label{zcastzyunitzyvec} \lstinputlisting[language=sail]{sail_latex/sailfncastunitvec.tex}}

\newcommand{\sailprint}{\label{zprint} \lstinputlisting[language=sail]{sail_latex/sailprint.tex}}

\newcommand{\sailprerrendline}{\label{zprerrzyendline} \lstinputlisting[language=sail]{sail_latex/sailprerrendline.tex}}

\newcommand{\sailprerrstring}{\label{zprerrzystring} \lstinputlisting[language=sail]{sail_latex/sailprerrstring.tex}}

\newcommand{\sailputchar}{\label{zputchar} \lstinputlisting[language=sail]{sail_latex/sailputchar.tex}}

\newcommand{\sailconcatstr}{\label{zconcatzystr} \lstinputlisting[language=sail]{sail_latex/sailconcatstr.tex}}

\newcommand{\sailstringofint}{\label{zstringzyofzyint} \lstinputlisting[language=sail]{sail_latex/sailstringofint.tex}}

\newcommand{\sailBitStr}{\label{zBitStr} \lstinputlisting[language=sail]{sail_latex/sailBitStr.tex}}

\newcommand{\sailxorvec}{\label{zxorzyvec} \lstinputlisting[language=sail]{sail_latex/sailxorvec.tex}}

\newcommand{\sailintpower}{\label{zintzypower} \lstinputlisting[language=sail]{sail_latex/sailintpower.tex}}

\newcommand{\sailzeightoperatorzzerozQznine}{\label{zzeightoperatorzzerozQznine} \lstinputlisting[language=sail]{sail_latex/sailzeightoperatorzzerozQznine.tex}}

\newcommand{\sailaddrange}{\label{zaddzyrange} \lstinputlisting[language=sail]{sail_latex/sailaddrange.tex}}

\newcommand{\sailaddvec}{\label{zaddzyvec} \lstinputlisting[language=sail]{sail_latex/sailaddvec.tex}}

\newcommand{\sailaddvecint}{\label{zaddzyveczyint} \lstinputlisting[language=sail]{sail_latex/sailaddvecint.tex}}

\newcommand{\sailzeightoperatorzzerozBznine}{\label{zzeightoperatorzzerozBznine} \lstinputlisting[language=sail]{sail_latex/sailzeightoperatorzzerozBznine.tex}}

\newcommand{\sailsubrange}{\label{zsubzyrange} \lstinputlisting[language=sail]{sail_latex/sailsubrange.tex}}

\newcommand{\sailsubvec}{\label{zsubzyvec} \lstinputlisting[language=sail]{sail_latex/sailsubvec.tex}}

\newcommand{\sailsubvecint}{\label{zsubzyveczyint} \lstinputlisting[language=sail]{sail_latex/sailsubvecint.tex}}

\newcommand{\sailnegaterange}{\label{znegatezyrange} \lstinputlisting[language=sail]{sail_latex/sailnegaterange.tex}}

\newcommand{\sailzeightoperatorzzerozDznine}{\label{zzeightoperatorzzerozDznine} \lstinputlisting[language=sail]{sail_latex/sailzeightoperatorzzerozDznine.tex}}

\newcommand{\sailnegate}{\label{znegate} \lstinputlisting[language=sail]{sail_latex/sailnegate.tex}}

\newcommand{\sailzeightoperatorzzerozAznine}{\label{zzeightoperatorzzerozAznine} \lstinputlisting[language=sail]{sail_latex/sailzeightoperatorzzerozAznine.tex}}

\newcommand{\sailquotientnat}{\label{zquotientzynat} \lstinputlisting[language=sail]{sail_latex/sailquotientnat.tex}}

\newcommand{\sailquotient}{\label{zquotient} \lstinputlisting[language=sail]{sail_latex/sailquotient.tex}}

\newcommand{\sailzeightoperatorzzerozFznine}{\label{zzeightoperatorzzerozFznine} \lstinputlisting[language=sail]{sail_latex/sailzeightoperatorzzerozFznine.tex}}

\newcommand{\sailquotroundzzero}{\label{zquotzyroundzyzzero} \lstinputlisting[language=sail]{sail_latex/sailquotroundzzero.tex}}

\newcommand{\sailremroundzzero}{\label{zremzyroundzyzzero} \lstinputlisting[language=sail]{sail_latex/sailremroundzzero.tex}}

\newcommand{\sailmodulus}{\label{zmodulus} \lstinputlisting[language=sail]{sail_latex/sailmodulus.tex}}

\newcommand{\sailzeightoperatorzzerozfiveznine}{\label{zzeightoperatorzzerozfiveznine} \lstinputlisting[language=sail]{sail_latex/sailzeightoperatorzzerozfiveznine.tex}}

\newcommand{\sailminnat}{\label{zminzynat} \lstinputlisting[language=sail]{sail_latex/sailminnat.tex}}

\newcommand{\sailminint}{\label{zminzyint} \lstinputlisting[language=sail]{sail_latex/sailminint.tex}}

\newcommand{\sailmaxnat}{\label{zmaxzynat} \lstinputlisting[language=sail]{sail_latex/sailmaxnat.tex}}

\newcommand{\sailmaxint}{\label{zmaxzyint} \lstinputlisting[language=sail]{sail_latex/sailmaxint.tex}}

\newcommand{\sailminatom}{\label{zminzyatom} \lstinputlisting[language=sail]{sail_latex/sailminatom.tex}}

\newcommand{\sailmaxatom}{\label{zmaxzyatom} \lstinputlisting[language=sail]{sail_latex/sailmaxatom.tex}}

\newcommand{\sailmin}{\label{zmin} \lstinputlisting[language=sail]{sail_latex/sailmin.tex}}

\newcommand{\sailsailmaxv}{\label{zmax} \lstinputlisting[language=sail]{sail_latex/sailsailmaxv.tex}}

\newcommand{\sailWriteRAM}{\label{zzyzyWriteRAM} \lstinputlisting[language=sail]{sail_latex/sailWriteRAM.tex}}

\newcommand{\sailMIPSwrite}{\label{zzyzyMIPSzywrite} \lstinputlisting[language=sail]{sail_latex/sailMIPSwrite.tex}}

\newcommand{\sailfnMIPSwrite}{\label{zzyzyMIPSzywrite} \lstinputlisting[language=sail]{sail_latex/sailfnMIPSwrite.tex}}

\newcommand{\sailReadRAM}{\label{zzyzyReadRAM} \lstinputlisting[language=sail]{sail_latex/sailReadRAM.tex}}

\newcommand{\sailMIPSread}{\label{zzyzyMIPSzyread} \lstinputlisting[language=sail]{sail_latex/sailMIPSread.tex}}

\newcommand{\sailfnMIPSread}{\label{zzyzyMIPSzyread} \lstinputlisting[language=sail]{sail_latex/sailfnMIPSread.tex}}

\newcommand{\sailzeightoperatorzzerozQzQznine}{\label{zzeightoperatorzzerozQzQznine} \lstinputlisting[language=sail]{sail_latex/sailzeightoperatorzzerozQzQznine.tex}}

\newcommand{\sailfnzeightoperatorzzerozQzQznine}{\label{zzeightoperatorzzerozQzQznine} \lstinputlisting[language=sail]{sail_latex/sailfnzeightoperatorzzerozQzQznine.tex}}

\newcommand{\sailpowtwo}{\label{zpowtwo} \lstinputlisting[language=sail]{sail_latex/sailpowtwo.tex}}

\newcommand{\sailmipssignextend}{\label{zmipszysignzyextend} \lstinputlisting[language=sail]{sail_latex/sailmipssignextend.tex}}

\newcommand{\sailmipszzeroextend}{\label{zmipszyzzerozyextend} \lstinputlisting[language=sail]{sail_latex/sailmipszzeroextend.tex}}

\newcommand{\sailfnmipssignextend}{\label{zmipszysignzyextend} \lstinputlisting[language=sail]{sail_latex/sailfnmipssignextend.tex}}

\newcommand{\sailfnmipszzeroextend}{\label{zmipszyzzerozyextend} \lstinputlisting[language=sail]{sail_latex/sailfnmipszzeroextend.tex}}

\newcommand{\sailsignextend}{\label{zsignzyextend} \lstinputlisting[language=sail]{sail_latex/sailsignextend.tex}}

\newcommand{\sailzzeroextend}{\label{zzzerozyextend} \lstinputlisting[language=sail]{sail_latex/sailzzeroextend.tex}}

\newcommand{\sailzzeros}{\label{zzzeros} \lstinputlisting[language=sail]{sail_latex/sailzzeros.tex}}

\newcommand{\sailfnzzeros}{\label{zzzeros} \lstinputlisting[language=sail]{sail_latex/sailfnzzeros.tex}}

\newcommand{\sailones}{\label{zones} \lstinputlisting[language=sail]{sail_latex/sailones.tex}}

\newcommand{\sailfnones}{\label{zones} \lstinputlisting[language=sail]{sail_latex/sailfnones.tex}}

\newcommand{\sailzeightoperatorzzerozIsznine}{\label{zzeightoperatorzzerozIzysznine} \lstinputlisting[language=sail]{sail_latex/sailzeightoperatorzzerozIsznine.tex}}

\newcommand{\sailzeightoperatorzzerozKzJsznine}{\label{zzeightoperatorzzerozKzJzysznine} \lstinputlisting[language=sail]{sail_latex/sailzeightoperatorzzerozKzJsznine.tex}}

\newcommand{\sailzeightoperatorzzerozIuznine}{\label{zzeightoperatorzzerozIzyuznine} \lstinputlisting[language=sail]{sail_latex/sailzeightoperatorzzerozIuznine.tex}}

\newcommand{\sailzeightoperatorzzerozKzJuznine}{\label{zzeightoperatorzzerozKzJzyuznine} \lstinputlisting[language=sail]{sail_latex/sailzeightoperatorzzerozKzJuznine.tex}}

\newcommand{\sailfnzeightoperatorzzerozIsznine}{\label{zzeightoperatorzzerozIzysznine} \lstinputlisting[language=sail]{sail_latex/sailfnzeightoperatorzzerozIsznine.tex}}

\newcommand{\sailfnzeightoperatorzzerozKzJsznine}{\label{zzeightoperatorzzerozKzJzysznine} \lstinputlisting[language=sail]{sail_latex/sailfnzeightoperatorzzerozKzJsznine.tex}}

\newcommand{\sailfnzeightoperatorzzerozIuznine}{\label{zzeightoperatorzzerozIzyuznine} \lstinputlisting[language=sail]{sail_latex/sailfnzeightoperatorzzerozIuznine.tex}}

\newcommand{\sailfnzeightoperatorzzerozKzJuznine}{\label{zzeightoperatorzzerozKzJzyuznine} \lstinputlisting[language=sail]{sail_latex/sailfnzeightoperatorzzerozKzJuznine.tex}}

\newcommand{\sailbooltobits}{\label{zboolzytozybits} \lstinputlisting[language=sail]{sail_latex/sailbooltobits.tex}}

\newcommand{\sailfnbooltobits}{\label{zboolzytozybits} \lstinputlisting[language=sail]{sail_latex/sailfnbooltobits.tex}}

\newcommand{\sailbittobool}{\label{zbitzytozybool} \lstinputlisting[language=sail]{sail_latex/sailbittobool.tex}}

\newcommand{\sailfnbittobool}{\label{zbitzytozybool} \lstinputlisting[language=sail]{sail_latex/sailfnbittobool.tex}}

\newcommand{\sailbitstobool}{\label{zbitszytozybool} \lstinputlisting[language=sail]{sail_latex/sailbitstobool.tex}}

\newcommand{\sailfnbitstobool}{\label{zbitszytozybool} \lstinputlisting[language=sail]{sail_latex/sailfnbitstobool.tex}}

\newcommand{\sailshiftbitsright}{\label{zshiftzybitszyright} \lstinputlisting[language=sail]{sail_latex/sailshiftbitsright.tex}}

\newcommand{\sailshiftbitsleft}{\label{zshiftzybitszyleft} \lstinputlisting[language=sail]{sail_latex/sailshiftbitsleft.tex}}

\newcommand{\sailshiftl}{\label{zshiftl} \lstinputlisting[language=sail]{sail_latex/sailshiftl.tex}}

\newcommand{\sailshiftr}{\label{zshiftr} \lstinputlisting[language=sail]{sail_latex/sailshiftr.tex}}

\newcommand{\sailzeightoperatorzzerozKzKznine}{\label{zzeightoperatorzzerozKzKznine} \lstinputlisting[language=sail]{sail_latex/sailzeightoperatorzzerozKzKznine.tex}}

\newcommand{\sailzeightoperatorzzerozIzIznine}{\label{zzeightoperatorzzerozIzIznine} \lstinputlisting[language=sail]{sail_latex/sailzeightoperatorzzerozIzIznine.tex}}

\newcommand{\sailzeightoperatorzzerozKzKsznine}{\label{zzeightoperatorzzerozKzKzysznine} \lstinputlisting[language=sail]{sail_latex/sailzeightoperatorzzerozKzKsznine.tex}}

\newcommand{\sailzeightoperatorzzerozAsznine}{\label{zzeightoperatorzzerozAzysznine} \lstinputlisting[language=sail]{sail_latex/sailzeightoperatorzzerozAsznine.tex}}

\newcommand{\sailzeightoperatorzzerozAuznine}{\label{zzeightoperatorzzerozAzyuznine} \lstinputlisting[language=sail]{sail_latex/sailzeightoperatorzzerozAuznine.tex}}

\newcommand{\sailtobits}{\label{ztozybits} 
\function{to\_bits} converts an integer to a bit vector of given length. If the integer is negative a twos-complement representation is used. If the integer is too large (or too negative) to fit in the requested length then it is truncated to the least significant bits.
\lstinputlisting[language=sail]{sail_latex/sailtobits.tex}}

\newcommand{\sailfntobits}{\label{ztozybits} \lstinputlisting[language=sail]{sail_latex/sailfntobits.tex}}

\newcommand{\sailmask}{\label{zmask} \lstinputlisting[language=sail]{sail_latex/sailmask.tex}}

\newcommand{\sailfnmask}{\label{zmask} \lstinputlisting[language=sail]{sail_latex/sailfnmask.tex}}

\newcommand{\sailgettimens}{\label{zgetzytimezyns} \lstinputlisting[language=sail]{sail_latex/sailgettimens.tex}}

\newcommand{\sailCauseReg}{\label{zCauseReg} \lstinputlisting[language=sail]{sail_latex/sailCauseReg.tex}}

\newcommand{\sailMkCauseReg}{\label{zMkzyCauseReg} \lstinputlisting[language=sail]{sail_latex/sailMkCauseReg.tex}}

\newcommand{\sailfnMkCauseReg}{\label{zMkzyCauseReg} \lstinputlisting[language=sail]{sail_latex/sailfnMkCauseReg.tex}}

\newcommand{\sailgetCauseRegbits}{\label{zzygetzyCauseRegzybits} \lstinputlisting[language=sail]{sail_latex/sailgetCauseRegbits.tex}}

\newcommand{\sailfngetCauseRegbits}{\label{zzygetzyCauseRegzybits} \lstinputlisting[language=sail]{sail_latex/sailfngetCauseRegbits.tex}}

\newcommand{\sailsetCauseRegbits}{\label{zzysetzyCauseRegzybits} \lstinputlisting[language=sail]{sail_latex/sailsetCauseRegbits.tex}}

\newcommand{\sailfnsetCauseRegbits}{\label{zzysetzyCauseRegzybits} \lstinputlisting[language=sail]{sail_latex/sailfnsetCauseRegbits.tex}}

\newcommand{\sailupdateCauseRegbits}{\label{zzyupdatezyCauseRegzybits} \lstinputlisting[language=sail]{sail_latex/sailupdateCauseRegbits.tex}}

\newcommand{\sailfnupdateCauseRegbits}{\label{zzyupdatezyCauseRegzybits} \lstinputlisting[language=sail]{sail_latex/sailfnupdateCauseRegbits.tex}}

\newcommand{\sailsailsailsailsailsailsailsailupdatebitsvvvvvvv}{\label{zupdatezybits} \lstinputlisting[language=sail]{sail_latex/sailsailsailsailsailsailsailsailupdatebitsvvvvvvv.tex}}

\newcommand{\sailsailsailsailsailsailsailsailmodbitsvvvvvvv}{\label{zzymodzybits} \lstinputlisting[language=sail]{sail_latex/sailsailsailsailsailsailsailsailmodbitsvvvvvvv.tex}}

\newcommand{\sailgetCauseRegBD}{\label{zzygetzyCauseRegzyBD} \lstinputlisting[language=sail]{sail_latex/sailgetCauseRegBD.tex}}

\newcommand{\sailfngetCauseRegBD}{\label{zzygetzyCauseRegzyBD} \lstinputlisting[language=sail]{sail_latex/sailfngetCauseRegBD.tex}}

\newcommand{\sailsetCauseRegBD}{\label{zzysetzyCauseRegzyBD} \lstinputlisting[language=sail]{sail_latex/sailsetCauseRegBD.tex}}

\newcommand{\sailfnsetCauseRegBD}{\label{zzysetzyCauseRegzyBD} \lstinputlisting[language=sail]{sail_latex/sailfnsetCauseRegBD.tex}}

\newcommand{\sailupdateCauseRegBD}{\label{zzyupdatezyCauseRegzyBD} \lstinputlisting[language=sail]{sail_latex/sailupdateCauseRegBD.tex}}

\newcommand{\sailfnupdateCauseRegBD}{\label{zzyupdatezyCauseRegzyBD} \lstinputlisting[language=sail]{sail_latex/sailfnupdateCauseRegBD.tex}}

\newcommand{\sailupdateBD}{\label{zupdatezyBD} \lstinputlisting[language=sail]{sail_latex/sailupdateBD.tex}}

\newcommand{\sailmodBD}{\label{zzymodzyBD} \lstinputlisting[language=sail]{sail_latex/sailmodBD.tex}}

\newcommand{\sailgetCauseRegCE}{\label{zzygetzyCauseRegzyCE} \lstinputlisting[language=sail]{sail_latex/sailgetCauseRegCE.tex}}

\newcommand{\sailfngetCauseRegCE}{\label{zzygetzyCauseRegzyCE} \lstinputlisting[language=sail]{sail_latex/sailfngetCauseRegCE.tex}}

\newcommand{\sailsetCauseRegCE}{\label{zzysetzyCauseRegzyCE} \lstinputlisting[language=sail]{sail_latex/sailsetCauseRegCE.tex}}

\newcommand{\sailfnsetCauseRegCE}{\label{zzysetzyCauseRegzyCE} \lstinputlisting[language=sail]{sail_latex/sailfnsetCauseRegCE.tex}}

\newcommand{\sailupdateCauseRegCE}{\label{zzyupdatezyCauseRegzyCE} \lstinputlisting[language=sail]{sail_latex/sailupdateCauseRegCE.tex}}

\newcommand{\sailfnupdateCauseRegCE}{\label{zzyupdatezyCauseRegzyCE} \lstinputlisting[language=sail]{sail_latex/sailfnupdateCauseRegCE.tex}}

\newcommand{\sailupdateCE}{\label{zupdatezyCE} \lstinputlisting[language=sail]{sail_latex/sailupdateCE.tex}}

\newcommand{\sailmodCE}{\label{zzymodzyCE} \lstinputlisting[language=sail]{sail_latex/sailmodCE.tex}}

\newcommand{\sailgetCauseRegIV}{\label{zzygetzyCauseRegzyIV} \lstinputlisting[language=sail]{sail_latex/sailgetCauseRegIV.tex}}

\newcommand{\sailfngetCauseRegIV}{\label{zzygetzyCauseRegzyIV} \lstinputlisting[language=sail]{sail_latex/sailfngetCauseRegIV.tex}}

\newcommand{\sailsetCauseRegIV}{\label{zzysetzyCauseRegzyIV} \lstinputlisting[language=sail]{sail_latex/sailsetCauseRegIV.tex}}

\newcommand{\sailfnsetCauseRegIV}{\label{zzysetzyCauseRegzyIV} \lstinputlisting[language=sail]{sail_latex/sailfnsetCauseRegIV.tex}}

\newcommand{\sailupdateCauseRegIV}{\label{zzyupdatezyCauseRegzyIV} \lstinputlisting[language=sail]{sail_latex/sailupdateCauseRegIV.tex}}

\newcommand{\sailfnupdateCauseRegIV}{\label{zzyupdatezyCauseRegzyIV} \lstinputlisting[language=sail]{sail_latex/sailfnupdateCauseRegIV.tex}}

\newcommand{\sailupdateIV}{\label{zupdatezyIV} \lstinputlisting[language=sail]{sail_latex/sailupdateIV.tex}}

\newcommand{\sailmodIV}{\label{zzymodzyIV} \lstinputlisting[language=sail]{sail_latex/sailmodIV.tex}}

\newcommand{\sailgetCauseRegWP}{\label{zzygetzyCauseRegzyWP} \lstinputlisting[language=sail]{sail_latex/sailgetCauseRegWP.tex}}

\newcommand{\sailfngetCauseRegWP}{\label{zzygetzyCauseRegzyWP} \lstinputlisting[language=sail]{sail_latex/sailfngetCauseRegWP.tex}}

\newcommand{\sailsetCauseRegWP}{\label{zzysetzyCauseRegzyWP} \lstinputlisting[language=sail]{sail_latex/sailsetCauseRegWP.tex}}

\newcommand{\sailfnsetCauseRegWP}{\label{zzysetzyCauseRegzyWP} \lstinputlisting[language=sail]{sail_latex/sailfnsetCauseRegWP.tex}}

\newcommand{\sailupdateCauseRegWP}{\label{zzyupdatezyCauseRegzyWP} \lstinputlisting[language=sail]{sail_latex/sailupdateCauseRegWP.tex}}

\newcommand{\sailfnupdateCauseRegWP}{\label{zzyupdatezyCauseRegzyWP} \lstinputlisting[language=sail]{sail_latex/sailfnupdateCauseRegWP.tex}}

\newcommand{\sailupdateWP}{\label{zupdatezyWP} \lstinputlisting[language=sail]{sail_latex/sailupdateWP.tex}}

\newcommand{\sailmodWP}{\label{zzymodzyWP} \lstinputlisting[language=sail]{sail_latex/sailmodWP.tex}}

\newcommand{\sailgetCauseRegIP}{\label{zzygetzyCauseRegzyIP} \lstinputlisting[language=sail]{sail_latex/sailgetCauseRegIP.tex}}

\newcommand{\sailfngetCauseRegIP}{\label{zzygetzyCauseRegzyIP} \lstinputlisting[language=sail]{sail_latex/sailfngetCauseRegIP.tex}}

\newcommand{\sailsetCauseRegIP}{\label{zzysetzyCauseRegzyIP} \lstinputlisting[language=sail]{sail_latex/sailsetCauseRegIP.tex}}

\newcommand{\sailfnsetCauseRegIP}{\label{zzysetzyCauseRegzyIP} \lstinputlisting[language=sail]{sail_latex/sailfnsetCauseRegIP.tex}}

\newcommand{\sailupdateCauseRegIP}{\label{zzyupdatezyCauseRegzyIP} \lstinputlisting[language=sail]{sail_latex/sailupdateCauseRegIP.tex}}

\newcommand{\sailfnupdateCauseRegIP}{\label{zzyupdatezyCauseRegzyIP} \lstinputlisting[language=sail]{sail_latex/sailfnupdateCauseRegIP.tex}}

\newcommand{\sailupdateIP}{\label{zupdatezyIP} \lstinputlisting[language=sail]{sail_latex/sailupdateIP.tex}}

\newcommand{\sailmodIP}{\label{zzymodzyIP} \lstinputlisting[language=sail]{sail_latex/sailmodIP.tex}}

\newcommand{\sailgetCauseRegExcCode}{\label{zzygetzyCauseRegzyExcCode} \lstinputlisting[language=sail]{sail_latex/sailgetCauseRegExcCode.tex}}

\newcommand{\sailfngetCauseRegExcCode}{\label{zzygetzyCauseRegzyExcCode} \lstinputlisting[language=sail]{sail_latex/sailfngetCauseRegExcCode.tex}}

\newcommand{\sailsetCauseRegExcCode}{\label{zzysetzyCauseRegzyExcCode} \lstinputlisting[language=sail]{sail_latex/sailsetCauseRegExcCode.tex}}

\newcommand{\sailfnsetCauseRegExcCode}{\label{zzysetzyCauseRegzyExcCode} \lstinputlisting[language=sail]{sail_latex/sailfnsetCauseRegExcCode.tex}}

\newcommand{\sailupdateCauseRegExcCode}{\label{zzyupdatezyCauseRegzyExcCode} \lstinputlisting[language=sail]{sail_latex/sailupdateCauseRegExcCode.tex}}

\newcommand{\sailfnupdateCauseRegExcCode}{\label{zzyupdatezyCauseRegzyExcCode} \lstinputlisting[language=sail]{sail_latex/sailfnupdateCauseRegExcCode.tex}}

\newcommand{\sailsailupdateExcCodev}{\label{zupdatezyExcCode} \lstinputlisting[language=sail]{sail_latex/sailsailupdateExcCodev.tex}}

\newcommand{\sailsailmodExcCodev}{\label{zzymodzyExcCode} \lstinputlisting[language=sail]{sail_latex/sailsailmodExcCodev.tex}}

\newcommand{\sailTLBEntryLoReg}{\label{zTLBEntryLoReg} \lstinputlisting[language=sail]{sail_latex/sailTLBEntryLoReg.tex}}

\newcommand{\sailMkTLBEntryLoReg}{\label{zMkzyTLBEntryLoReg} \lstinputlisting[language=sail]{sail_latex/sailMkTLBEntryLoReg.tex}}

\newcommand{\sailfnMkTLBEntryLoReg}{\label{zMkzyTLBEntryLoReg} \lstinputlisting[language=sail]{sail_latex/sailfnMkTLBEntryLoReg.tex}}

\newcommand{\sailgetTLBEntryLoRegbits}{\label{zzygetzyTLBEntryLoRegzybits} \lstinputlisting[language=sail]{sail_latex/sailgetTLBEntryLoRegbits.tex}}

\newcommand{\sailfngetTLBEntryLoRegbits}{\label{zzygetzyTLBEntryLoRegzybits} \lstinputlisting[language=sail]{sail_latex/sailfngetTLBEntryLoRegbits.tex}}

\newcommand{\sailsetTLBEntryLoRegbits}{\label{zzysetzyTLBEntryLoRegzybits} \lstinputlisting[language=sail]{sail_latex/sailsetTLBEntryLoRegbits.tex}}

\newcommand{\sailfnsetTLBEntryLoRegbits}{\label{zzysetzyTLBEntryLoRegzybits} \lstinputlisting[language=sail]{sail_latex/sailfnsetTLBEntryLoRegbits.tex}}

\newcommand{\sailupdateTLBEntryLoRegbits}{\label{zzyupdatezyTLBEntryLoRegzybits} \lstinputlisting[language=sail]{sail_latex/sailupdateTLBEntryLoRegbits.tex}}

\newcommand{\sailfnupdateTLBEntryLoRegbits}{\label{zzyupdatezyTLBEntryLoRegzybits} \lstinputlisting[language=sail]{sail_latex/sailfnupdateTLBEntryLoRegbits.tex}}

\newcommand{\sailsailsailsailsailsailsailupdatebitsvvvvvv}{\label{zupdatezybits} \lstinputlisting[language=sail]{sail_latex/sailsailsailsailsailsailsailupdatebitsvvvvvv.tex}}

\newcommand{\sailsailsailsailsailsailsailmodbitsvvvvvv}{\label{zzymodzybits} \lstinputlisting[language=sail]{sail_latex/sailsailsailsailsailsailsailmodbitsvvvvvv.tex}}

\newcommand{\sailgetTLBEntryLoRegCapS}{\label{zzygetzyTLBEntryLoRegzyCapS} \lstinputlisting[language=sail]{sail_latex/sailgetTLBEntryLoRegCapS.tex}}

\newcommand{\sailfngetTLBEntryLoRegCapS}{\label{zzygetzyTLBEntryLoRegzyCapS} \lstinputlisting[language=sail]{sail_latex/sailfngetTLBEntryLoRegCapS.tex}}

\newcommand{\sailsetTLBEntryLoRegCapS}{\label{zzysetzyTLBEntryLoRegzyCapS} \lstinputlisting[language=sail]{sail_latex/sailsetTLBEntryLoRegCapS.tex}}

\newcommand{\sailfnsetTLBEntryLoRegCapS}{\label{zzysetzyTLBEntryLoRegzyCapS} \lstinputlisting[language=sail]{sail_latex/sailfnsetTLBEntryLoRegCapS.tex}}

\newcommand{\sailupdateTLBEntryLoRegCapS}{\label{zzyupdatezyTLBEntryLoRegzyCapS} \lstinputlisting[language=sail]{sail_latex/sailupdateTLBEntryLoRegCapS.tex}}

\newcommand{\sailfnupdateTLBEntryLoRegCapS}{\label{zzyupdatezyTLBEntryLoRegzyCapS} \lstinputlisting[language=sail]{sail_latex/sailfnupdateTLBEntryLoRegCapS.tex}}

\newcommand{\sailupdateCapS}{\label{zupdatezyCapS} \lstinputlisting[language=sail]{sail_latex/sailupdateCapS.tex}}

\newcommand{\sailmodCapS}{\label{zzymodzyCapS} \lstinputlisting[language=sail]{sail_latex/sailmodCapS.tex}}

\newcommand{\sailgetTLBEntryLoRegCapL}{\label{zzygetzyTLBEntryLoRegzyCapL} \lstinputlisting[language=sail]{sail_latex/sailgetTLBEntryLoRegCapL.tex}}

\newcommand{\sailfngetTLBEntryLoRegCapL}{\label{zzygetzyTLBEntryLoRegzyCapL} \lstinputlisting[language=sail]{sail_latex/sailfngetTLBEntryLoRegCapL.tex}}

\newcommand{\sailsetTLBEntryLoRegCapL}{\label{zzysetzyTLBEntryLoRegzyCapL} \lstinputlisting[language=sail]{sail_latex/sailsetTLBEntryLoRegCapL.tex}}

\newcommand{\sailfnsetTLBEntryLoRegCapL}{\label{zzysetzyTLBEntryLoRegzyCapL} \lstinputlisting[language=sail]{sail_latex/sailfnsetTLBEntryLoRegCapL.tex}}

\newcommand{\sailupdateTLBEntryLoRegCapL}{\label{zzyupdatezyTLBEntryLoRegzyCapL} \lstinputlisting[language=sail]{sail_latex/sailupdateTLBEntryLoRegCapL.tex}}

\newcommand{\sailfnupdateTLBEntryLoRegCapL}{\label{zzyupdatezyTLBEntryLoRegzyCapL} \lstinputlisting[language=sail]{sail_latex/sailfnupdateTLBEntryLoRegCapL.tex}}

\newcommand{\sailupdateCapL}{\label{zupdatezyCapL} \lstinputlisting[language=sail]{sail_latex/sailupdateCapL.tex}}

\newcommand{\sailmodCapL}{\label{zzymodzyCapL} \lstinputlisting[language=sail]{sail_latex/sailmodCapL.tex}}

\newcommand{\sailgetTLBEntryLoRegPFN}{\label{zzygetzyTLBEntryLoRegzyPFN} \lstinputlisting[language=sail]{sail_latex/sailgetTLBEntryLoRegPFN.tex}}

\newcommand{\sailfngetTLBEntryLoRegPFN}{\label{zzygetzyTLBEntryLoRegzyPFN} \lstinputlisting[language=sail]{sail_latex/sailfngetTLBEntryLoRegPFN.tex}}

\newcommand{\sailsetTLBEntryLoRegPFN}{\label{zzysetzyTLBEntryLoRegzyPFN} \lstinputlisting[language=sail]{sail_latex/sailsetTLBEntryLoRegPFN.tex}}

\newcommand{\sailfnsetTLBEntryLoRegPFN}{\label{zzysetzyTLBEntryLoRegzyPFN} \lstinputlisting[language=sail]{sail_latex/sailfnsetTLBEntryLoRegPFN.tex}}

\newcommand{\sailupdateTLBEntryLoRegPFN}{\label{zzyupdatezyTLBEntryLoRegzyPFN} \lstinputlisting[language=sail]{sail_latex/sailupdateTLBEntryLoRegPFN.tex}}

\newcommand{\sailfnupdateTLBEntryLoRegPFN}{\label{zzyupdatezyTLBEntryLoRegzyPFN} \lstinputlisting[language=sail]{sail_latex/sailfnupdateTLBEntryLoRegPFN.tex}}

\newcommand{\sailupdatePFN}{\label{zupdatezyPFN} \lstinputlisting[language=sail]{sail_latex/sailupdatePFN.tex}}

\newcommand{\sailmodPFN}{\label{zzymodzyPFN} \lstinputlisting[language=sail]{sail_latex/sailmodPFN.tex}}

\newcommand{\sailgetTLBEntryLoRegC}{\label{zzygetzyTLBEntryLoRegzyC} \lstinputlisting[language=sail]{sail_latex/sailgetTLBEntryLoRegC.tex}}

\newcommand{\sailfngetTLBEntryLoRegC}{\label{zzygetzyTLBEntryLoRegzyC} \lstinputlisting[language=sail]{sail_latex/sailfngetTLBEntryLoRegC.tex}}

\newcommand{\sailsetTLBEntryLoRegC}{\label{zzysetzyTLBEntryLoRegzyC} \lstinputlisting[language=sail]{sail_latex/sailsetTLBEntryLoRegC.tex}}

\newcommand{\sailfnsetTLBEntryLoRegC}{\label{zzysetzyTLBEntryLoRegzyC} \lstinputlisting[language=sail]{sail_latex/sailfnsetTLBEntryLoRegC.tex}}

\newcommand{\sailupdateTLBEntryLoRegC}{\label{zzyupdatezyTLBEntryLoRegzyC} \lstinputlisting[language=sail]{sail_latex/sailupdateTLBEntryLoRegC.tex}}

\newcommand{\sailfnupdateTLBEntryLoRegC}{\label{zzyupdatezyTLBEntryLoRegzyC} \lstinputlisting[language=sail]{sail_latex/sailfnupdateTLBEntryLoRegC.tex}}

\newcommand{\sailupdateC}{\label{zupdatezyC} \lstinputlisting[language=sail]{sail_latex/sailupdateC.tex}}

\newcommand{\sailmodC}{\label{zzymodzyC} \lstinputlisting[language=sail]{sail_latex/sailmodC.tex}}

\newcommand{\sailgetTLBEntryLoRegD}{\label{zzygetzyTLBEntryLoRegzyD} \lstinputlisting[language=sail]{sail_latex/sailgetTLBEntryLoRegD.tex}}

\newcommand{\sailfngetTLBEntryLoRegD}{\label{zzygetzyTLBEntryLoRegzyD} \lstinputlisting[language=sail]{sail_latex/sailfngetTLBEntryLoRegD.tex}}

\newcommand{\sailsetTLBEntryLoRegD}{\label{zzysetzyTLBEntryLoRegzyD} \lstinputlisting[language=sail]{sail_latex/sailsetTLBEntryLoRegD.tex}}

\newcommand{\sailfnsetTLBEntryLoRegD}{\label{zzysetzyTLBEntryLoRegzyD} \lstinputlisting[language=sail]{sail_latex/sailfnsetTLBEntryLoRegD.tex}}

\newcommand{\sailupdateTLBEntryLoRegD}{\label{zzyupdatezyTLBEntryLoRegzyD} \lstinputlisting[language=sail]{sail_latex/sailupdateTLBEntryLoRegD.tex}}

\newcommand{\sailfnupdateTLBEntryLoRegD}{\label{zzyupdatezyTLBEntryLoRegzyD} \lstinputlisting[language=sail]{sail_latex/sailfnupdateTLBEntryLoRegD.tex}}

\newcommand{\sailupdateD}{\label{zupdatezyD} \lstinputlisting[language=sail]{sail_latex/sailupdateD.tex}}

\newcommand{\sailmodD}{\label{zzymodzyD} \lstinputlisting[language=sail]{sail_latex/sailmodD.tex}}

\newcommand{\sailgetTLBEntryLoRegV}{\label{zzygetzyTLBEntryLoRegzyV} \lstinputlisting[language=sail]{sail_latex/sailgetTLBEntryLoRegV.tex}}

\newcommand{\sailfngetTLBEntryLoRegV}{\label{zzygetzyTLBEntryLoRegzyV} \lstinputlisting[language=sail]{sail_latex/sailfngetTLBEntryLoRegV.tex}}

\newcommand{\sailsetTLBEntryLoRegV}{\label{zzysetzyTLBEntryLoRegzyV} \lstinputlisting[language=sail]{sail_latex/sailsetTLBEntryLoRegV.tex}}

\newcommand{\sailfnsetTLBEntryLoRegV}{\label{zzysetzyTLBEntryLoRegzyV} \lstinputlisting[language=sail]{sail_latex/sailfnsetTLBEntryLoRegV.tex}}

\newcommand{\sailupdateTLBEntryLoRegV}{\label{zzyupdatezyTLBEntryLoRegzyV} \lstinputlisting[language=sail]{sail_latex/sailupdateTLBEntryLoRegV.tex}}

\newcommand{\sailfnupdateTLBEntryLoRegV}{\label{zzyupdatezyTLBEntryLoRegzyV} \lstinputlisting[language=sail]{sail_latex/sailfnupdateTLBEntryLoRegV.tex}}

\newcommand{\sailupdateV}{\label{zupdatezyV} \lstinputlisting[language=sail]{sail_latex/sailupdateV.tex}}

\newcommand{\sailmodV}{\label{zzymodzyV} \lstinputlisting[language=sail]{sail_latex/sailmodV.tex}}

\newcommand{\sailgetTLBEntryLoRegG}{\label{zzygetzyTLBEntryLoRegzyG} \lstinputlisting[language=sail]{sail_latex/sailgetTLBEntryLoRegG.tex}}

\newcommand{\sailfngetTLBEntryLoRegG}{\label{zzygetzyTLBEntryLoRegzyG} \lstinputlisting[language=sail]{sail_latex/sailfngetTLBEntryLoRegG.tex}}

\newcommand{\sailsetTLBEntryLoRegG}{\label{zzysetzyTLBEntryLoRegzyG} \lstinputlisting[language=sail]{sail_latex/sailsetTLBEntryLoRegG.tex}}

\newcommand{\sailfnsetTLBEntryLoRegG}{\label{zzysetzyTLBEntryLoRegzyG} \lstinputlisting[language=sail]{sail_latex/sailfnsetTLBEntryLoRegG.tex}}

\newcommand{\sailupdateTLBEntryLoRegG}{\label{zzyupdatezyTLBEntryLoRegzyG} \lstinputlisting[language=sail]{sail_latex/sailupdateTLBEntryLoRegG.tex}}

\newcommand{\sailfnupdateTLBEntryLoRegG}{\label{zzyupdatezyTLBEntryLoRegzyG} \lstinputlisting[language=sail]{sail_latex/sailfnupdateTLBEntryLoRegG.tex}}

\newcommand{\sailsailupdateGv}{\label{zupdatezyG} \lstinputlisting[language=sail]{sail_latex/sailsailupdateGv.tex}}

\newcommand{\sailsailmodGv}{\label{zzymodzyG} \lstinputlisting[language=sail]{sail_latex/sailsailmodGv.tex}}

\newcommand{\sailTLBEntryHiReg}{\label{zTLBEntryHiReg} \lstinputlisting[language=sail]{sail_latex/sailTLBEntryHiReg.tex}}

\newcommand{\sailMkTLBEntryHiReg}{\label{zMkzyTLBEntryHiReg} \lstinputlisting[language=sail]{sail_latex/sailMkTLBEntryHiReg.tex}}

\newcommand{\sailfnMkTLBEntryHiReg}{\label{zMkzyTLBEntryHiReg} \lstinputlisting[language=sail]{sail_latex/sailfnMkTLBEntryHiReg.tex}}

\newcommand{\sailgetTLBEntryHiRegbits}{\label{zzygetzyTLBEntryHiRegzybits} \lstinputlisting[language=sail]{sail_latex/sailgetTLBEntryHiRegbits.tex}}

\newcommand{\sailfngetTLBEntryHiRegbits}{\label{zzygetzyTLBEntryHiRegzybits} \lstinputlisting[language=sail]{sail_latex/sailfngetTLBEntryHiRegbits.tex}}

\newcommand{\sailsetTLBEntryHiRegbits}{\label{zzysetzyTLBEntryHiRegzybits} \lstinputlisting[language=sail]{sail_latex/sailsetTLBEntryHiRegbits.tex}}

\newcommand{\sailfnsetTLBEntryHiRegbits}{\label{zzysetzyTLBEntryHiRegzybits} \lstinputlisting[language=sail]{sail_latex/sailfnsetTLBEntryHiRegbits.tex}}

\newcommand{\sailupdateTLBEntryHiRegbits}{\label{zzyupdatezyTLBEntryHiRegzybits} \lstinputlisting[language=sail]{sail_latex/sailupdateTLBEntryHiRegbits.tex}}

\newcommand{\sailfnupdateTLBEntryHiRegbits}{\label{zzyupdatezyTLBEntryHiRegzybits} \lstinputlisting[language=sail]{sail_latex/sailfnupdateTLBEntryHiRegbits.tex}}

\newcommand{\sailsailsailsailsailsailupdatebitsvvvvv}{\label{zupdatezybits} \lstinputlisting[language=sail]{sail_latex/sailsailsailsailsailsailupdatebitsvvvvv.tex}}

\newcommand{\sailsailsailsailsailsailmodbitsvvvvv}{\label{zzymodzybits} \lstinputlisting[language=sail]{sail_latex/sailsailsailsailsailsailmodbitsvvvvv.tex}}

\newcommand{\sailgetTLBEntryHiRegR}{\label{zzygetzyTLBEntryHiRegzyR} \lstinputlisting[language=sail]{sail_latex/sailgetTLBEntryHiRegR.tex}}

\newcommand{\sailfngetTLBEntryHiRegR}{\label{zzygetzyTLBEntryHiRegzyR} \lstinputlisting[language=sail]{sail_latex/sailfngetTLBEntryHiRegR.tex}}

\newcommand{\sailsetTLBEntryHiRegR}{\label{zzysetzyTLBEntryHiRegzyR} \lstinputlisting[language=sail]{sail_latex/sailsetTLBEntryHiRegR.tex}}

\newcommand{\sailfnsetTLBEntryHiRegR}{\label{zzysetzyTLBEntryHiRegzyR} \lstinputlisting[language=sail]{sail_latex/sailfnsetTLBEntryHiRegR.tex}}

\newcommand{\sailupdateTLBEntryHiRegR}{\label{zzyupdatezyTLBEntryHiRegzyR} \lstinputlisting[language=sail]{sail_latex/sailupdateTLBEntryHiRegR.tex}}

\newcommand{\sailfnupdateTLBEntryHiRegR}{\label{zzyupdatezyTLBEntryHiRegzyR} \lstinputlisting[language=sail]{sail_latex/sailfnupdateTLBEntryHiRegR.tex}}

\newcommand{\sailsailupdateRv}{\label{zupdatezyR} \lstinputlisting[language=sail]{sail_latex/sailsailupdateRv.tex}}

\newcommand{\sailsailmodRv}{\label{zzymodzyR} \lstinputlisting[language=sail]{sail_latex/sailsailmodRv.tex}}

\newcommand{\sailgetTLBEntryHiRegVPNtwo}{\label{zzygetzyTLBEntryHiRegzyVPNtwo} \lstinputlisting[language=sail]{sail_latex/sailgetTLBEntryHiRegVPNtwo.tex}}

\newcommand{\sailfngetTLBEntryHiRegVPNtwo}{\label{zzygetzyTLBEntryHiRegzyVPNtwo} \lstinputlisting[language=sail]{sail_latex/sailfngetTLBEntryHiRegVPNtwo.tex}}

\newcommand{\sailsetTLBEntryHiRegVPNtwo}{\label{zzysetzyTLBEntryHiRegzyVPNtwo} \lstinputlisting[language=sail]{sail_latex/sailsetTLBEntryHiRegVPNtwo.tex}}

\newcommand{\sailfnsetTLBEntryHiRegVPNtwo}{\label{zzysetzyTLBEntryHiRegzyVPNtwo} \lstinputlisting[language=sail]{sail_latex/sailfnsetTLBEntryHiRegVPNtwo.tex}}

\newcommand{\sailupdateTLBEntryHiRegVPNtwo}{\label{zzyupdatezyTLBEntryHiRegzyVPNtwo} \lstinputlisting[language=sail]{sail_latex/sailupdateTLBEntryHiRegVPNtwo.tex}}

\newcommand{\sailfnupdateTLBEntryHiRegVPNtwo}{\label{zzyupdatezyTLBEntryHiRegzyVPNtwo} \lstinputlisting[language=sail]{sail_latex/sailfnupdateTLBEntryHiRegVPNtwo.tex}}

\newcommand{\sailsailupdateVPNtwov}{\label{zupdatezyVPNtwo} \lstinputlisting[language=sail]{sail_latex/sailsailupdateVPNtwov.tex}}

\newcommand{\sailsailmodVPNtwov}{\label{zzymodzyVPNtwo} \lstinputlisting[language=sail]{sail_latex/sailsailmodVPNtwov.tex}}

\newcommand{\sailgetTLBEntryHiRegASID}{\label{zzygetzyTLBEntryHiRegzyASID} \lstinputlisting[language=sail]{sail_latex/sailgetTLBEntryHiRegASID.tex}}

\newcommand{\sailfngetTLBEntryHiRegASID}{\label{zzygetzyTLBEntryHiRegzyASID} \lstinputlisting[language=sail]{sail_latex/sailfngetTLBEntryHiRegASID.tex}}

\newcommand{\sailsetTLBEntryHiRegASID}{\label{zzysetzyTLBEntryHiRegzyASID} \lstinputlisting[language=sail]{sail_latex/sailsetTLBEntryHiRegASID.tex}}

\newcommand{\sailfnsetTLBEntryHiRegASID}{\label{zzysetzyTLBEntryHiRegzyASID} \lstinputlisting[language=sail]{sail_latex/sailfnsetTLBEntryHiRegASID.tex}}

\newcommand{\sailupdateTLBEntryHiRegASID}{\label{zzyupdatezyTLBEntryHiRegzyASID} \lstinputlisting[language=sail]{sail_latex/sailupdateTLBEntryHiRegASID.tex}}

\newcommand{\sailfnupdateTLBEntryHiRegASID}{\label{zzyupdatezyTLBEntryHiRegzyASID} \lstinputlisting[language=sail]{sail_latex/sailfnupdateTLBEntryHiRegASID.tex}}

\newcommand{\sailsailupdateASIDv}{\label{zupdatezyASID} \lstinputlisting[language=sail]{sail_latex/sailsailupdateASIDv.tex}}

\newcommand{\sailsailmodASIDv}{\label{zzymodzyASID} \lstinputlisting[language=sail]{sail_latex/sailsailmodASIDv.tex}}

\newcommand{\sailContextReg}{\label{zContextReg} \lstinputlisting[language=sail]{sail_latex/sailContextReg.tex}}

\newcommand{\sailMkContextReg}{\label{zMkzyContextReg} \lstinputlisting[language=sail]{sail_latex/sailMkContextReg.tex}}

\newcommand{\sailfnMkContextReg}{\label{zMkzyContextReg} \lstinputlisting[language=sail]{sail_latex/sailfnMkContextReg.tex}}

\newcommand{\sailgetContextRegbits}{\label{zzygetzyContextRegzybits} \lstinputlisting[language=sail]{sail_latex/sailgetContextRegbits.tex}}

\newcommand{\sailfngetContextRegbits}{\label{zzygetzyContextRegzybits} \lstinputlisting[language=sail]{sail_latex/sailfngetContextRegbits.tex}}

\newcommand{\sailsetContextRegbits}{\label{zzysetzyContextRegzybits} \lstinputlisting[language=sail]{sail_latex/sailsetContextRegbits.tex}}

\newcommand{\sailfnsetContextRegbits}{\label{zzysetzyContextRegzybits} \lstinputlisting[language=sail]{sail_latex/sailfnsetContextRegbits.tex}}

\newcommand{\sailupdateContextRegbits}{\label{zzyupdatezyContextRegzybits} \lstinputlisting[language=sail]{sail_latex/sailupdateContextRegbits.tex}}

\newcommand{\sailfnupdateContextRegbits}{\label{zzyupdatezyContextRegzybits} \lstinputlisting[language=sail]{sail_latex/sailfnupdateContextRegbits.tex}}

\newcommand{\sailsailsailsailsailupdatebitsvvvv}{\label{zupdatezybits} \lstinputlisting[language=sail]{sail_latex/sailsailsailsailsailupdatebitsvvvv.tex}}

\newcommand{\sailsailsailsailsailmodbitsvvvv}{\label{zzymodzybits} \lstinputlisting[language=sail]{sail_latex/sailsailsailsailsailmodbitsvvvv.tex}}

\newcommand{\sailgetContextRegPTEBase}{\label{zzygetzyContextRegzyPTEBase} \lstinputlisting[language=sail]{sail_latex/sailgetContextRegPTEBase.tex}}

\newcommand{\sailfngetContextRegPTEBase}{\label{zzygetzyContextRegzyPTEBase} \lstinputlisting[language=sail]{sail_latex/sailfngetContextRegPTEBase.tex}}

\newcommand{\sailsetContextRegPTEBase}{\label{zzysetzyContextRegzyPTEBase} \lstinputlisting[language=sail]{sail_latex/sailsetContextRegPTEBase.tex}}

\newcommand{\sailfnsetContextRegPTEBase}{\label{zzysetzyContextRegzyPTEBase} \lstinputlisting[language=sail]{sail_latex/sailfnsetContextRegPTEBase.tex}}

\newcommand{\sailupdateContextRegPTEBase}{\label{zzyupdatezyContextRegzyPTEBase} \lstinputlisting[language=sail]{sail_latex/sailupdateContextRegPTEBase.tex}}

\newcommand{\sailfnupdateContextRegPTEBase}{\label{zzyupdatezyContextRegzyPTEBase} \lstinputlisting[language=sail]{sail_latex/sailfnupdateContextRegPTEBase.tex}}

\newcommand{\sailupdatePTEBase}{\label{zupdatezyPTEBase} \lstinputlisting[language=sail]{sail_latex/sailupdatePTEBase.tex}}

\newcommand{\sailmodPTEBase}{\label{zzymodzyPTEBase} \lstinputlisting[language=sail]{sail_latex/sailmodPTEBase.tex}}

\newcommand{\sailgetContextRegBadVPNtwo}{\label{zzygetzyContextRegzyBadVPNtwo} \lstinputlisting[language=sail]{sail_latex/sailgetContextRegBadVPNtwo.tex}}

\newcommand{\sailfngetContextRegBadVPNtwo}{\label{zzygetzyContextRegzyBadVPNtwo} \lstinputlisting[language=sail]{sail_latex/sailfngetContextRegBadVPNtwo.tex}}

\newcommand{\sailsetContextRegBadVPNtwo}{\label{zzysetzyContextRegzyBadVPNtwo} \lstinputlisting[language=sail]{sail_latex/sailsetContextRegBadVPNtwo.tex}}

\newcommand{\sailfnsetContextRegBadVPNtwo}{\label{zzysetzyContextRegzyBadVPNtwo} \lstinputlisting[language=sail]{sail_latex/sailfnsetContextRegBadVPNtwo.tex}}

\newcommand{\sailupdateContextRegBadVPNtwo}{\label{zzyupdatezyContextRegzyBadVPNtwo} \lstinputlisting[language=sail]{sail_latex/sailupdateContextRegBadVPNtwo.tex}}

\newcommand{\sailfnupdateContextRegBadVPNtwo}{\label{zzyupdatezyContextRegzyBadVPNtwo} \lstinputlisting[language=sail]{sail_latex/sailfnupdateContextRegBadVPNtwo.tex}}

\newcommand{\sailupdateBadVPNtwo}{\label{zupdatezyBadVPNtwo} \lstinputlisting[language=sail]{sail_latex/sailupdateBadVPNtwo.tex}}

\newcommand{\sailmodBadVPNtwo}{\label{zzymodzyBadVPNtwo} \lstinputlisting[language=sail]{sail_latex/sailmodBadVPNtwo.tex}}

\newcommand{\sailXContextReg}{\label{zXContextReg} \lstinputlisting[language=sail]{sail_latex/sailXContextReg.tex}}

\newcommand{\sailMkXContextReg}{\label{zMkzyXContextReg} \lstinputlisting[language=sail]{sail_latex/sailMkXContextReg.tex}}

\newcommand{\sailfnMkXContextReg}{\label{zMkzyXContextReg} \lstinputlisting[language=sail]{sail_latex/sailfnMkXContextReg.tex}}

\newcommand{\sailgetXContextRegbits}{\label{zzygetzyXContextRegzybits} \lstinputlisting[language=sail]{sail_latex/sailgetXContextRegbits.tex}}

\newcommand{\sailfngetXContextRegbits}{\label{zzygetzyXContextRegzybits} \lstinputlisting[language=sail]{sail_latex/sailfngetXContextRegbits.tex}}

\newcommand{\sailsetXContextRegbits}{\label{zzysetzyXContextRegzybits} \lstinputlisting[language=sail]{sail_latex/sailsetXContextRegbits.tex}}

\newcommand{\sailfnsetXContextRegbits}{\label{zzysetzyXContextRegzybits} \lstinputlisting[language=sail]{sail_latex/sailfnsetXContextRegbits.tex}}

\newcommand{\sailupdateXContextRegbits}{\label{zzyupdatezyXContextRegzybits} \lstinputlisting[language=sail]{sail_latex/sailupdateXContextRegbits.tex}}

\newcommand{\sailfnupdateXContextRegbits}{\label{zzyupdatezyXContextRegzybits} \lstinputlisting[language=sail]{sail_latex/sailfnupdateXContextRegbits.tex}}

\newcommand{\sailsailsailsailupdatebitsvvv}{\label{zupdatezybits} \lstinputlisting[language=sail]{sail_latex/sailsailsailsailupdatebitsvvv.tex}}

\newcommand{\sailsailsailsailmodbitsvvv}{\label{zzymodzybits} \lstinputlisting[language=sail]{sail_latex/sailsailsailsailmodbitsvvv.tex}}

\newcommand{\sailgetXContextRegXPTEBase}{\label{zzygetzyXContextRegzyXPTEBase} \lstinputlisting[language=sail]{sail_latex/sailgetXContextRegXPTEBase.tex}}

\newcommand{\sailfngetXContextRegXPTEBase}{\label{zzygetzyXContextRegzyXPTEBase} \lstinputlisting[language=sail]{sail_latex/sailfngetXContextRegXPTEBase.tex}}

\newcommand{\sailsetXContextRegXPTEBase}{\label{zzysetzyXContextRegzyXPTEBase} \lstinputlisting[language=sail]{sail_latex/sailsetXContextRegXPTEBase.tex}}

\newcommand{\sailfnsetXContextRegXPTEBase}{\label{zzysetzyXContextRegzyXPTEBase} \lstinputlisting[language=sail]{sail_latex/sailfnsetXContextRegXPTEBase.tex}}

\newcommand{\sailupdateXContextRegXPTEBase}{\label{zzyupdatezyXContextRegzyXPTEBase} \lstinputlisting[language=sail]{sail_latex/sailupdateXContextRegXPTEBase.tex}}

\newcommand{\sailfnupdateXContextRegXPTEBase}{\label{zzyupdatezyXContextRegzyXPTEBase} \lstinputlisting[language=sail]{sail_latex/sailfnupdateXContextRegXPTEBase.tex}}

\newcommand{\sailupdateXPTEBase}{\label{zupdatezyXPTEBase} \lstinputlisting[language=sail]{sail_latex/sailupdateXPTEBase.tex}}

\newcommand{\sailmodXPTEBase}{\label{zzymodzyXPTEBase} \lstinputlisting[language=sail]{sail_latex/sailmodXPTEBase.tex}}

\newcommand{\sailgetXContextRegXR}{\label{zzygetzyXContextRegzyXR} \lstinputlisting[language=sail]{sail_latex/sailgetXContextRegXR.tex}}

\newcommand{\sailfngetXContextRegXR}{\label{zzygetzyXContextRegzyXR} \lstinputlisting[language=sail]{sail_latex/sailfngetXContextRegXR.tex}}

\newcommand{\sailsetXContextRegXR}{\label{zzysetzyXContextRegzyXR} \lstinputlisting[language=sail]{sail_latex/sailsetXContextRegXR.tex}}

\newcommand{\sailfnsetXContextRegXR}{\label{zzysetzyXContextRegzyXR} \lstinputlisting[language=sail]{sail_latex/sailfnsetXContextRegXR.tex}}

\newcommand{\sailupdateXContextRegXR}{\label{zzyupdatezyXContextRegzyXR} \lstinputlisting[language=sail]{sail_latex/sailupdateXContextRegXR.tex}}

\newcommand{\sailfnupdateXContextRegXR}{\label{zzyupdatezyXContextRegzyXR} \lstinputlisting[language=sail]{sail_latex/sailfnupdateXContextRegXR.tex}}

\newcommand{\sailupdateXR}{\label{zupdatezyXR} \lstinputlisting[language=sail]{sail_latex/sailupdateXR.tex}}

\newcommand{\sailmodXR}{\label{zzymodzyXR} \lstinputlisting[language=sail]{sail_latex/sailmodXR.tex}}

\newcommand{\sailgetXContextRegXBadVPNtwo}{\label{zzygetzyXContextRegzyXBadVPNtwo} \lstinputlisting[language=sail]{sail_latex/sailgetXContextRegXBadVPNtwo.tex}}

\newcommand{\sailfngetXContextRegXBadVPNtwo}{\label{zzygetzyXContextRegzyXBadVPNtwo} \lstinputlisting[language=sail]{sail_latex/sailfngetXContextRegXBadVPNtwo.tex}}

\newcommand{\sailsetXContextRegXBadVPNtwo}{\label{zzysetzyXContextRegzyXBadVPNtwo} \lstinputlisting[language=sail]{sail_latex/sailsetXContextRegXBadVPNtwo.tex}}

\newcommand{\sailfnsetXContextRegXBadVPNtwo}{\label{zzysetzyXContextRegzyXBadVPNtwo} \lstinputlisting[language=sail]{sail_latex/sailfnsetXContextRegXBadVPNtwo.tex}}

\newcommand{\sailupdateXContextRegXBadVPNtwo}{\label{zzyupdatezyXContextRegzyXBadVPNtwo} \lstinputlisting[language=sail]{sail_latex/sailupdateXContextRegXBadVPNtwo.tex}}

\newcommand{\sailfnupdateXContextRegXBadVPNtwo}{\label{zzyupdatezyXContextRegzyXBadVPNtwo} \lstinputlisting[language=sail]{sail_latex/sailfnupdateXContextRegXBadVPNtwo.tex}}

\newcommand{\sailupdateXBadVPNtwo}{\label{zupdatezyXBadVPNtwo} \lstinputlisting[language=sail]{sail_latex/sailupdateXBadVPNtwo.tex}}

\newcommand{\sailmodXBadVPNtwo}{\label{zzymodzyXBadVPNtwo} \lstinputlisting[language=sail]{sail_latex/sailmodXBadVPNtwo.tex}}

\newcommand{\sailTLBIndexT}{\label{zTLBIndexT} \lstinputlisting[language=sail]{sail_latex/sailTLBIndexT.tex}}

\newcommand{\sailMAX}{\label{zMAX} \lstinputlisting[language=sail]{sail_latex/sailMAX.tex}}

\newcommand{\sailfnMAX}{\label{zMAX} \lstinputlisting[language=sail]{sail_latex/sailfnMAX.tex}}

\newcommand{\sailTLBEntry}{\label{zTLBEntry} \lstinputlisting[language=sail]{sail_latex/sailTLBEntry.tex}}

\newcommand{\sailMkTLBEntry}{\label{zMkzyTLBEntry} \lstinputlisting[language=sail]{sail_latex/sailMkTLBEntry.tex}}

\newcommand{\sailfnMkTLBEntry}{\label{zMkzyTLBEntry} \lstinputlisting[language=sail]{sail_latex/sailfnMkTLBEntry.tex}}

\newcommand{\sailgetTLBEntrybits}{\label{zzygetzyTLBEntryzybits} \lstinputlisting[language=sail]{sail_latex/sailgetTLBEntrybits.tex}}

\newcommand{\sailfngetTLBEntrybits}{\label{zzygetzyTLBEntryzybits} \lstinputlisting[language=sail]{sail_latex/sailfngetTLBEntrybits.tex}}

\newcommand{\sailsetTLBEntrybits}{\label{zzysetzyTLBEntryzybits} \lstinputlisting[language=sail]{sail_latex/sailsetTLBEntrybits.tex}}

\newcommand{\sailfnsetTLBEntrybits}{\label{zzysetzyTLBEntryzybits} \lstinputlisting[language=sail]{sail_latex/sailfnsetTLBEntrybits.tex}}

\newcommand{\sailupdateTLBEntrybits}{\label{zzyupdatezyTLBEntryzybits} \lstinputlisting[language=sail]{sail_latex/sailupdateTLBEntrybits.tex}}

\newcommand{\sailfnupdateTLBEntrybits}{\label{zzyupdatezyTLBEntryzybits} \lstinputlisting[language=sail]{sail_latex/sailfnupdateTLBEntrybits.tex}}

\newcommand{\sailsailsailupdatebitsvv}{\label{zupdatezybits} \lstinputlisting[language=sail]{sail_latex/sailsailsailupdatebitsvv.tex}}

\newcommand{\sailsailsailmodbitsvv}{\label{zzymodzybits} \lstinputlisting[language=sail]{sail_latex/sailsailsailmodbitsvv.tex}}

\newcommand{\sailgetTLBEntrypagemask}{\label{zzygetzyTLBEntryzypagemask} \lstinputlisting[language=sail]{sail_latex/sailgetTLBEntrypagemask.tex}}

\newcommand{\sailfngetTLBEntrypagemask}{\label{zzygetzyTLBEntryzypagemask} \lstinputlisting[language=sail]{sail_latex/sailfngetTLBEntrypagemask.tex}}

\newcommand{\sailsetTLBEntrypagemask}{\label{zzysetzyTLBEntryzypagemask} \lstinputlisting[language=sail]{sail_latex/sailsetTLBEntrypagemask.tex}}

\newcommand{\sailfnsetTLBEntrypagemask}{\label{zzysetzyTLBEntryzypagemask} \lstinputlisting[language=sail]{sail_latex/sailfnsetTLBEntrypagemask.tex}}

\newcommand{\sailupdateTLBEntrypagemask}{\label{zzyupdatezyTLBEntryzypagemask} \lstinputlisting[language=sail]{sail_latex/sailupdateTLBEntrypagemask.tex}}

\newcommand{\sailfnupdateTLBEntrypagemask}{\label{zzyupdatezyTLBEntryzypagemask} \lstinputlisting[language=sail]{sail_latex/sailfnupdateTLBEntrypagemask.tex}}

\newcommand{\sailupdatepagemask}{\label{zupdatezypagemask} \lstinputlisting[language=sail]{sail_latex/sailupdatepagemask.tex}}

\newcommand{\sailmodpagemask}{\label{zzymodzypagemask} \lstinputlisting[language=sail]{sail_latex/sailmodpagemask.tex}}

\newcommand{\sailgetTLBEntryr}{\label{zzygetzyTLBEntryzyr} \lstinputlisting[language=sail]{sail_latex/sailgetTLBEntryr.tex}}

\newcommand{\sailfngetTLBEntryr}{\label{zzygetzyTLBEntryzyr} \lstinputlisting[language=sail]{sail_latex/sailfngetTLBEntryr.tex}}

\newcommand{\sailsetTLBEntryr}{\label{zzysetzyTLBEntryzyr} \lstinputlisting[language=sail]{sail_latex/sailsetTLBEntryr.tex}}

\newcommand{\sailfnsetTLBEntryr}{\label{zzysetzyTLBEntryzyr} \lstinputlisting[language=sail]{sail_latex/sailfnsetTLBEntryr.tex}}

\newcommand{\sailupdateTLBEntryr}{\label{zzyupdatezyTLBEntryzyr} \lstinputlisting[language=sail]{sail_latex/sailupdateTLBEntryr.tex}}

\newcommand{\sailfnupdateTLBEntryr}{\label{zzyupdatezyTLBEntryzyr} \lstinputlisting[language=sail]{sail_latex/sailfnupdateTLBEntryr.tex}}

\newcommand{\sailupdater}{\label{zupdatezyr} \lstinputlisting[language=sail]{sail_latex/sailupdater.tex}}

\newcommand{\sailmodr}{\label{zzymodzyr} \lstinputlisting[language=sail]{sail_latex/sailmodr.tex}}

\newcommand{\sailgetTLBEntryvpntwo}{\label{zzygetzyTLBEntryzyvpntwo} \lstinputlisting[language=sail]{sail_latex/sailgetTLBEntryvpntwo.tex}}

\newcommand{\sailfngetTLBEntryvpntwo}{\label{zzygetzyTLBEntryzyvpntwo} \lstinputlisting[language=sail]{sail_latex/sailfngetTLBEntryvpntwo.tex}}

\newcommand{\sailsetTLBEntryvpntwo}{\label{zzysetzyTLBEntryzyvpntwo} \lstinputlisting[language=sail]{sail_latex/sailsetTLBEntryvpntwo.tex}}

\newcommand{\sailfnsetTLBEntryvpntwo}{\label{zzysetzyTLBEntryzyvpntwo} \lstinputlisting[language=sail]{sail_latex/sailfnsetTLBEntryvpntwo.tex}}

\newcommand{\sailupdateTLBEntryvpntwo}{\label{zzyupdatezyTLBEntryzyvpntwo} \lstinputlisting[language=sail]{sail_latex/sailupdateTLBEntryvpntwo.tex}}

\newcommand{\sailfnupdateTLBEntryvpntwo}{\label{zzyupdatezyTLBEntryzyvpntwo} \lstinputlisting[language=sail]{sail_latex/sailfnupdateTLBEntryvpntwo.tex}}

\newcommand{\sailupdatevpntwo}{\label{zupdatezyvpntwo} \lstinputlisting[language=sail]{sail_latex/sailupdatevpntwo.tex}}

\newcommand{\sailmodvpntwo}{\label{zzymodzyvpntwo} \lstinputlisting[language=sail]{sail_latex/sailmodvpntwo.tex}}

\newcommand{\sailgetTLBEntryasid}{\label{zzygetzyTLBEntryzyasid} \lstinputlisting[language=sail]{sail_latex/sailgetTLBEntryasid.tex}}

\newcommand{\sailfngetTLBEntryasid}{\label{zzygetzyTLBEntryzyasid} \lstinputlisting[language=sail]{sail_latex/sailfngetTLBEntryasid.tex}}

\newcommand{\sailsetTLBEntryasid}{\label{zzysetzyTLBEntryzyasid} \lstinputlisting[language=sail]{sail_latex/sailsetTLBEntryasid.tex}}

\newcommand{\sailfnsetTLBEntryasid}{\label{zzysetzyTLBEntryzyasid} \lstinputlisting[language=sail]{sail_latex/sailfnsetTLBEntryasid.tex}}

\newcommand{\sailupdateTLBEntryasid}{\label{zzyupdatezyTLBEntryzyasid} \lstinputlisting[language=sail]{sail_latex/sailupdateTLBEntryasid.tex}}

\newcommand{\sailfnupdateTLBEntryasid}{\label{zzyupdatezyTLBEntryzyasid} \lstinputlisting[language=sail]{sail_latex/sailfnupdateTLBEntryasid.tex}}

\newcommand{\sailupdateasid}{\label{zupdatezyasid} \lstinputlisting[language=sail]{sail_latex/sailupdateasid.tex}}

\newcommand{\sailmodasid}{\label{zzymodzyasid} \lstinputlisting[language=sail]{sail_latex/sailmodasid.tex}}

\newcommand{\sailgetTLBEntryg}{\label{zzygetzyTLBEntryzyg} \lstinputlisting[language=sail]{sail_latex/sailgetTLBEntryg.tex}}

\newcommand{\sailfngetTLBEntryg}{\label{zzygetzyTLBEntryzyg} \lstinputlisting[language=sail]{sail_latex/sailfngetTLBEntryg.tex}}

\newcommand{\sailsetTLBEntryg}{\label{zzysetzyTLBEntryzyg} \lstinputlisting[language=sail]{sail_latex/sailsetTLBEntryg.tex}}

\newcommand{\sailfnsetTLBEntryg}{\label{zzysetzyTLBEntryzyg} \lstinputlisting[language=sail]{sail_latex/sailfnsetTLBEntryg.tex}}

\newcommand{\sailupdateTLBEntryg}{\label{zzyupdatezyTLBEntryzyg} \lstinputlisting[language=sail]{sail_latex/sailupdateTLBEntryg.tex}}

\newcommand{\sailfnupdateTLBEntryg}{\label{zzyupdatezyTLBEntryzyg} \lstinputlisting[language=sail]{sail_latex/sailfnupdateTLBEntryg.tex}}

\newcommand{\sailupdateg}{\label{zupdatezyg} \lstinputlisting[language=sail]{sail_latex/sailupdateg.tex}}

\newcommand{\sailmodg}{\label{zzymodzyg} \lstinputlisting[language=sail]{sail_latex/sailmodg.tex}}

\newcommand{\sailgetTLBEntryvalid}{\label{zzygetzyTLBEntryzyvalid} \lstinputlisting[language=sail]{sail_latex/sailgetTLBEntryvalid.tex}}

\newcommand{\sailfngetTLBEntryvalid}{\label{zzygetzyTLBEntryzyvalid} \lstinputlisting[language=sail]{sail_latex/sailfngetTLBEntryvalid.tex}}

\newcommand{\sailsetTLBEntryvalid}{\label{zzysetzyTLBEntryzyvalid} \lstinputlisting[language=sail]{sail_latex/sailsetTLBEntryvalid.tex}}

\newcommand{\sailfnsetTLBEntryvalid}{\label{zzysetzyTLBEntryzyvalid} \lstinputlisting[language=sail]{sail_latex/sailfnsetTLBEntryvalid.tex}}

\newcommand{\sailupdateTLBEntryvalid}{\label{zzyupdatezyTLBEntryzyvalid} \lstinputlisting[language=sail]{sail_latex/sailupdateTLBEntryvalid.tex}}

\newcommand{\sailfnupdateTLBEntryvalid}{\label{zzyupdatezyTLBEntryzyvalid} \lstinputlisting[language=sail]{sail_latex/sailfnupdateTLBEntryvalid.tex}}

\newcommand{\sailupdatevalid}{\label{zupdatezyvalid} \lstinputlisting[language=sail]{sail_latex/sailupdatevalid.tex}}

\newcommand{\sailmodvalid}{\label{zzymodzyvalid} \lstinputlisting[language=sail]{sail_latex/sailmodvalid.tex}}

\newcommand{\sailgetTLBEntrycapsone}{\label{zzygetzyTLBEntryzycapsone} \lstinputlisting[language=sail]{sail_latex/sailgetTLBEntrycapsone.tex}}

\newcommand{\sailfngetTLBEntrycapsone}{\label{zzygetzyTLBEntryzycapsone} \lstinputlisting[language=sail]{sail_latex/sailfngetTLBEntrycapsone.tex}}

\newcommand{\sailsetTLBEntrycapsone}{\label{zzysetzyTLBEntryzycapsone} \lstinputlisting[language=sail]{sail_latex/sailsetTLBEntrycapsone.tex}}

\newcommand{\sailfnsetTLBEntrycapsone}{\label{zzysetzyTLBEntryzycapsone} \lstinputlisting[language=sail]{sail_latex/sailfnsetTLBEntrycapsone.tex}}

\newcommand{\sailupdateTLBEntrycapsone}{\label{zzyupdatezyTLBEntryzycapsone} \lstinputlisting[language=sail]{sail_latex/sailupdateTLBEntrycapsone.tex}}

\newcommand{\sailfnupdateTLBEntrycapsone}{\label{zzyupdatezyTLBEntryzycapsone} \lstinputlisting[language=sail]{sail_latex/sailfnupdateTLBEntrycapsone.tex}}

\newcommand{\sailupdatecapsone}{\label{zupdatezycapsone} \lstinputlisting[language=sail]{sail_latex/sailupdatecapsone.tex}}

\newcommand{\sailmodcapsone}{\label{zzymodzycapsone} \lstinputlisting[language=sail]{sail_latex/sailmodcapsone.tex}}

\newcommand{\sailgetTLBEntrycaplone}{\label{zzygetzyTLBEntryzycaplone} \lstinputlisting[language=sail]{sail_latex/sailgetTLBEntrycaplone.tex}}

\newcommand{\sailfngetTLBEntrycaplone}{\label{zzygetzyTLBEntryzycaplone} \lstinputlisting[language=sail]{sail_latex/sailfngetTLBEntrycaplone.tex}}

\newcommand{\sailsetTLBEntrycaplone}{\label{zzysetzyTLBEntryzycaplone} \lstinputlisting[language=sail]{sail_latex/sailsetTLBEntrycaplone.tex}}

\newcommand{\sailfnsetTLBEntrycaplone}{\label{zzysetzyTLBEntryzycaplone} \lstinputlisting[language=sail]{sail_latex/sailfnsetTLBEntrycaplone.tex}}

\newcommand{\sailupdateTLBEntrycaplone}{\label{zzyupdatezyTLBEntryzycaplone} \lstinputlisting[language=sail]{sail_latex/sailupdateTLBEntrycaplone.tex}}

\newcommand{\sailfnupdateTLBEntrycaplone}{\label{zzyupdatezyTLBEntryzycaplone} \lstinputlisting[language=sail]{sail_latex/sailfnupdateTLBEntrycaplone.tex}}

\newcommand{\sailupdatecaplone}{\label{zupdatezycaplone} \lstinputlisting[language=sail]{sail_latex/sailupdatecaplone.tex}}

\newcommand{\sailmodcaplone}{\label{zzymodzycaplone} \lstinputlisting[language=sail]{sail_latex/sailmodcaplone.tex}}

\newcommand{\sailgetTLBEntrypfnone}{\label{zzygetzyTLBEntryzypfnone} \lstinputlisting[language=sail]{sail_latex/sailgetTLBEntrypfnone.tex}}

\newcommand{\sailfngetTLBEntrypfnone}{\label{zzygetzyTLBEntryzypfnone} \lstinputlisting[language=sail]{sail_latex/sailfngetTLBEntrypfnone.tex}}

\newcommand{\sailsetTLBEntrypfnone}{\label{zzysetzyTLBEntryzypfnone} \lstinputlisting[language=sail]{sail_latex/sailsetTLBEntrypfnone.tex}}

\newcommand{\sailfnsetTLBEntrypfnone}{\label{zzysetzyTLBEntryzypfnone} \lstinputlisting[language=sail]{sail_latex/sailfnsetTLBEntrypfnone.tex}}

\newcommand{\sailupdateTLBEntrypfnone}{\label{zzyupdatezyTLBEntryzypfnone} \lstinputlisting[language=sail]{sail_latex/sailupdateTLBEntrypfnone.tex}}

\newcommand{\sailfnupdateTLBEntrypfnone}{\label{zzyupdatezyTLBEntryzypfnone} \lstinputlisting[language=sail]{sail_latex/sailfnupdateTLBEntrypfnone.tex}}

\newcommand{\sailupdatepfnone}{\label{zupdatezypfnone} \lstinputlisting[language=sail]{sail_latex/sailupdatepfnone.tex}}

\newcommand{\sailmodpfnone}{\label{zzymodzypfnone} \lstinputlisting[language=sail]{sail_latex/sailmodpfnone.tex}}

\newcommand{\sailgetTLBEntrycone}{\label{zzygetzyTLBEntryzycone} \lstinputlisting[language=sail]{sail_latex/sailgetTLBEntrycone.tex}}

\newcommand{\sailfngetTLBEntrycone}{\label{zzygetzyTLBEntryzycone} \lstinputlisting[language=sail]{sail_latex/sailfngetTLBEntrycone.tex}}

\newcommand{\sailsetTLBEntrycone}{\label{zzysetzyTLBEntryzycone} \lstinputlisting[language=sail]{sail_latex/sailsetTLBEntrycone.tex}}

\newcommand{\sailfnsetTLBEntrycone}{\label{zzysetzyTLBEntryzycone} \lstinputlisting[language=sail]{sail_latex/sailfnsetTLBEntrycone.tex}}

\newcommand{\sailupdateTLBEntrycone}{\label{zzyupdatezyTLBEntryzycone} \lstinputlisting[language=sail]{sail_latex/sailupdateTLBEntrycone.tex}}

\newcommand{\sailfnupdateTLBEntrycone}{\label{zzyupdatezyTLBEntryzycone} \lstinputlisting[language=sail]{sail_latex/sailfnupdateTLBEntrycone.tex}}

\newcommand{\sailupdatecone}{\label{zupdatezycone} \lstinputlisting[language=sail]{sail_latex/sailupdatecone.tex}}

\newcommand{\sailmodcone}{\label{zzymodzycone} \lstinputlisting[language=sail]{sail_latex/sailmodcone.tex}}

\newcommand{\sailgetTLBEntrydone}{\label{zzygetzyTLBEntryzydone} \lstinputlisting[language=sail]{sail_latex/sailgetTLBEntrydone.tex}}

\newcommand{\sailfngetTLBEntrydone}{\label{zzygetzyTLBEntryzydone} \lstinputlisting[language=sail]{sail_latex/sailfngetTLBEntrydone.tex}}

\newcommand{\sailsetTLBEntrydone}{\label{zzysetzyTLBEntryzydone} \lstinputlisting[language=sail]{sail_latex/sailsetTLBEntrydone.tex}}

\newcommand{\sailfnsetTLBEntrydone}{\label{zzysetzyTLBEntryzydone} \lstinputlisting[language=sail]{sail_latex/sailfnsetTLBEntrydone.tex}}

\newcommand{\sailupdateTLBEntrydone}{\label{zzyupdatezyTLBEntryzydone} \lstinputlisting[language=sail]{sail_latex/sailupdateTLBEntrydone.tex}}

\newcommand{\sailfnupdateTLBEntrydone}{\label{zzyupdatezyTLBEntryzydone} \lstinputlisting[language=sail]{sail_latex/sailfnupdateTLBEntrydone.tex}}

\newcommand{\sailupdatedone}{\label{zupdatezydone} \lstinputlisting[language=sail]{sail_latex/sailupdatedone.tex}}

\newcommand{\sailmoddone}{\label{zzymodzydone} \lstinputlisting[language=sail]{sail_latex/sailmoddone.tex}}

\newcommand{\sailgetTLBEntryvone}{\label{zzygetzyTLBEntryzyvone} \lstinputlisting[language=sail]{sail_latex/sailgetTLBEntryvone.tex}}

\newcommand{\sailfngetTLBEntryvone}{\label{zzygetzyTLBEntryzyvone} \lstinputlisting[language=sail]{sail_latex/sailfngetTLBEntryvone.tex}}

\newcommand{\sailsetTLBEntryvone}{\label{zzysetzyTLBEntryzyvone} \lstinputlisting[language=sail]{sail_latex/sailsetTLBEntryvone.tex}}

\newcommand{\sailfnsetTLBEntryvone}{\label{zzysetzyTLBEntryzyvone} \lstinputlisting[language=sail]{sail_latex/sailfnsetTLBEntryvone.tex}}

\newcommand{\sailupdateTLBEntryvone}{\label{zzyupdatezyTLBEntryzyvone} \lstinputlisting[language=sail]{sail_latex/sailupdateTLBEntryvone.tex}}

\newcommand{\sailfnupdateTLBEntryvone}{\label{zzyupdatezyTLBEntryzyvone} \lstinputlisting[language=sail]{sail_latex/sailfnupdateTLBEntryvone.tex}}

\newcommand{\sailupdatevone}{\label{zupdatezyvone} \lstinputlisting[language=sail]{sail_latex/sailupdatevone.tex}}

\newcommand{\sailmodvone}{\label{zzymodzyvone} \lstinputlisting[language=sail]{sail_latex/sailmodvone.tex}}

\newcommand{\sailgetTLBEntrycapszero}{\label{zzygetzyTLBEntryzycapszero} \lstinputlisting[language=sail]{sail_latex/sailgetTLBEntrycapszero.tex}}

\newcommand{\sailfngetTLBEntrycapszero}{\label{zzygetzyTLBEntryzycapszero} \lstinputlisting[language=sail]{sail_latex/sailfngetTLBEntrycapszero.tex}}

\newcommand{\sailsetTLBEntrycapszero}{\label{zzysetzyTLBEntryzycapszero} \lstinputlisting[language=sail]{sail_latex/sailsetTLBEntrycapszero.tex}}

\newcommand{\sailfnsetTLBEntrycapszero}{\label{zzysetzyTLBEntryzycapszero} \lstinputlisting[language=sail]{sail_latex/sailfnsetTLBEntrycapszero.tex}}

\newcommand{\sailupdateTLBEntrycapszero}{\label{zzyupdatezyTLBEntryzycapszero} \lstinputlisting[language=sail]{sail_latex/sailupdateTLBEntrycapszero.tex}}

\newcommand{\sailfnupdateTLBEntrycapszero}{\label{zzyupdatezyTLBEntryzycapszero} \lstinputlisting[language=sail]{sail_latex/sailfnupdateTLBEntrycapszero.tex}}

\newcommand{\sailupdatecapszero}{\label{zupdatezycapszero} \lstinputlisting[language=sail]{sail_latex/sailupdatecapszero.tex}}

\newcommand{\sailmodcapszero}{\label{zzymodzycapszero} \lstinputlisting[language=sail]{sail_latex/sailmodcapszero.tex}}

\newcommand{\sailgetTLBEntrycaplzero}{\label{zzygetzyTLBEntryzycaplzero} \lstinputlisting[language=sail]{sail_latex/sailgetTLBEntrycaplzero.tex}}

\newcommand{\sailfngetTLBEntrycaplzero}{\label{zzygetzyTLBEntryzycaplzero} \lstinputlisting[language=sail]{sail_latex/sailfngetTLBEntrycaplzero.tex}}

\newcommand{\sailsetTLBEntrycaplzero}{\label{zzysetzyTLBEntryzycaplzero} \lstinputlisting[language=sail]{sail_latex/sailsetTLBEntrycaplzero.tex}}

\newcommand{\sailfnsetTLBEntrycaplzero}{\label{zzysetzyTLBEntryzycaplzero} \lstinputlisting[language=sail]{sail_latex/sailfnsetTLBEntrycaplzero.tex}}

\newcommand{\sailupdateTLBEntrycaplzero}{\label{zzyupdatezyTLBEntryzycaplzero} \lstinputlisting[language=sail]{sail_latex/sailupdateTLBEntrycaplzero.tex}}

\newcommand{\sailfnupdateTLBEntrycaplzero}{\label{zzyupdatezyTLBEntryzycaplzero} \lstinputlisting[language=sail]{sail_latex/sailfnupdateTLBEntrycaplzero.tex}}

\newcommand{\sailupdatecaplzero}{\label{zupdatezycaplzero} \lstinputlisting[language=sail]{sail_latex/sailupdatecaplzero.tex}}

\newcommand{\sailmodcaplzero}{\label{zzymodzycaplzero} \lstinputlisting[language=sail]{sail_latex/sailmodcaplzero.tex}}

\newcommand{\sailgetTLBEntrypfnzero}{\label{zzygetzyTLBEntryzypfnzero} \lstinputlisting[language=sail]{sail_latex/sailgetTLBEntrypfnzero.tex}}

\newcommand{\sailfngetTLBEntrypfnzero}{\label{zzygetzyTLBEntryzypfnzero} \lstinputlisting[language=sail]{sail_latex/sailfngetTLBEntrypfnzero.tex}}

\newcommand{\sailsetTLBEntrypfnzero}{\label{zzysetzyTLBEntryzypfnzero} \lstinputlisting[language=sail]{sail_latex/sailsetTLBEntrypfnzero.tex}}

\newcommand{\sailfnsetTLBEntrypfnzero}{\label{zzysetzyTLBEntryzypfnzero} \lstinputlisting[language=sail]{sail_latex/sailfnsetTLBEntrypfnzero.tex}}

\newcommand{\sailupdateTLBEntrypfnzero}{\label{zzyupdatezyTLBEntryzypfnzero} \lstinputlisting[language=sail]{sail_latex/sailupdateTLBEntrypfnzero.tex}}

\newcommand{\sailfnupdateTLBEntrypfnzero}{\label{zzyupdatezyTLBEntryzypfnzero} \lstinputlisting[language=sail]{sail_latex/sailfnupdateTLBEntrypfnzero.tex}}

\newcommand{\sailupdatepfnzero}{\label{zupdatezypfnzero} \lstinputlisting[language=sail]{sail_latex/sailupdatepfnzero.tex}}

\newcommand{\sailmodpfnzero}{\label{zzymodzypfnzero} \lstinputlisting[language=sail]{sail_latex/sailmodpfnzero.tex}}

\newcommand{\sailgetTLBEntryczero}{\label{zzygetzyTLBEntryzyczero} \lstinputlisting[language=sail]{sail_latex/sailgetTLBEntryczero.tex}}

\newcommand{\sailfngetTLBEntryczero}{\label{zzygetzyTLBEntryzyczero} \lstinputlisting[language=sail]{sail_latex/sailfngetTLBEntryczero.tex}}

\newcommand{\sailsetTLBEntryczero}{\label{zzysetzyTLBEntryzyczero} \lstinputlisting[language=sail]{sail_latex/sailsetTLBEntryczero.tex}}

\newcommand{\sailfnsetTLBEntryczero}{\label{zzysetzyTLBEntryzyczero} \lstinputlisting[language=sail]{sail_latex/sailfnsetTLBEntryczero.tex}}

\newcommand{\sailupdateTLBEntryczero}{\label{zzyupdatezyTLBEntryzyczero} \lstinputlisting[language=sail]{sail_latex/sailupdateTLBEntryczero.tex}}

\newcommand{\sailfnupdateTLBEntryczero}{\label{zzyupdatezyTLBEntryzyczero} \lstinputlisting[language=sail]{sail_latex/sailfnupdateTLBEntryczero.tex}}

\newcommand{\sailupdateczero}{\label{zupdatezyczero} \lstinputlisting[language=sail]{sail_latex/sailupdateczero.tex}}

\newcommand{\sailmodczero}{\label{zzymodzyczero} \lstinputlisting[language=sail]{sail_latex/sailmodczero.tex}}

\newcommand{\sailgetTLBEntrydzero}{\label{zzygetzyTLBEntryzydzero} \lstinputlisting[language=sail]{sail_latex/sailgetTLBEntrydzero.tex}}

\newcommand{\sailfngetTLBEntrydzero}{\label{zzygetzyTLBEntryzydzero} \lstinputlisting[language=sail]{sail_latex/sailfngetTLBEntrydzero.tex}}

\newcommand{\sailsetTLBEntrydzero}{\label{zzysetzyTLBEntryzydzero} \lstinputlisting[language=sail]{sail_latex/sailsetTLBEntrydzero.tex}}

\newcommand{\sailfnsetTLBEntrydzero}{\label{zzysetzyTLBEntryzydzero} \lstinputlisting[language=sail]{sail_latex/sailfnsetTLBEntrydzero.tex}}

\newcommand{\sailupdateTLBEntrydzero}{\label{zzyupdatezyTLBEntryzydzero} \lstinputlisting[language=sail]{sail_latex/sailupdateTLBEntrydzero.tex}}

\newcommand{\sailfnupdateTLBEntrydzero}{\label{zzyupdatezyTLBEntryzydzero} \lstinputlisting[language=sail]{sail_latex/sailfnupdateTLBEntrydzero.tex}}

\newcommand{\sailupdatedzero}{\label{zupdatezydzero} \lstinputlisting[language=sail]{sail_latex/sailupdatedzero.tex}}

\newcommand{\sailmoddzero}{\label{zzymodzydzero} \lstinputlisting[language=sail]{sail_latex/sailmoddzero.tex}}

\newcommand{\sailgetTLBEntryvzero}{\label{zzygetzyTLBEntryzyvzero} \lstinputlisting[language=sail]{sail_latex/sailgetTLBEntryvzero.tex}}

\newcommand{\sailfngetTLBEntryvzero}{\label{zzygetzyTLBEntryzyvzero} \lstinputlisting[language=sail]{sail_latex/sailfngetTLBEntryvzero.tex}}

\newcommand{\sailsetTLBEntryvzero}{\label{zzysetzyTLBEntryzyvzero} \lstinputlisting[language=sail]{sail_latex/sailsetTLBEntryvzero.tex}}

\newcommand{\sailfnsetTLBEntryvzero}{\label{zzysetzyTLBEntryzyvzero} \lstinputlisting[language=sail]{sail_latex/sailfnsetTLBEntryvzero.tex}}

\newcommand{\sailupdateTLBEntryvzero}{\label{zzyupdatezyTLBEntryzyvzero} \lstinputlisting[language=sail]{sail_latex/sailupdateTLBEntryvzero.tex}}

\newcommand{\sailfnupdateTLBEntryvzero}{\label{zzyupdatezyTLBEntryzyvzero} \lstinputlisting[language=sail]{sail_latex/sailfnupdateTLBEntryvzero.tex}}

\newcommand{\sailupdatevzero}{\label{zupdatezyvzero} \lstinputlisting[language=sail]{sail_latex/sailupdatevzero.tex}}

\newcommand{\sailmodvzero}{\label{zzymodzyvzero} \lstinputlisting[language=sail]{sail_latex/sailmodvzero.tex}}

\newcommand{\sailStatusReg}{\label{zStatusReg} \lstinputlisting[language=sail]{sail_latex/sailStatusReg.tex}}

\newcommand{\sailMkStatusReg}{\label{zMkzyStatusReg} \lstinputlisting[language=sail]{sail_latex/sailMkStatusReg.tex}}

\newcommand{\sailfnMkStatusReg}{\label{zMkzyStatusReg} \lstinputlisting[language=sail]{sail_latex/sailfnMkStatusReg.tex}}

\newcommand{\sailgetStatusRegbits}{\label{zzygetzyStatusRegzybits} \lstinputlisting[language=sail]{sail_latex/sailgetStatusRegbits.tex}}

\newcommand{\sailfngetStatusRegbits}{\label{zzygetzyStatusRegzybits} \lstinputlisting[language=sail]{sail_latex/sailfngetStatusRegbits.tex}}

\newcommand{\sailsetStatusRegbits}{\label{zzysetzyStatusRegzybits} \lstinputlisting[language=sail]{sail_latex/sailsetStatusRegbits.tex}}

\newcommand{\sailfnsetStatusRegbits}{\label{zzysetzyStatusRegzybits} \lstinputlisting[language=sail]{sail_latex/sailfnsetStatusRegbits.tex}}

\newcommand{\sailupdateStatusRegbits}{\label{zzyupdatezyStatusRegzybits} \lstinputlisting[language=sail]{sail_latex/sailupdateStatusRegbits.tex}}

\newcommand{\sailfnupdateStatusRegbits}{\label{zzyupdatezyStatusRegzybits} \lstinputlisting[language=sail]{sail_latex/sailfnupdateStatusRegbits.tex}}

\newcommand{\sailsailupdatebitsv}{\label{zupdatezybits} \lstinputlisting[language=sail]{sail_latex/sailsailupdatebitsv.tex}}

\newcommand{\sailsailmodbitsv}{\label{zzymodzybits} \lstinputlisting[language=sail]{sail_latex/sailsailmodbitsv.tex}}

\newcommand{\sailgetStatusRegCU}{\label{zzygetzyStatusRegzyCU} \lstinputlisting[language=sail]{sail_latex/sailgetStatusRegCU.tex}}

\newcommand{\sailfngetStatusRegCU}{\label{zzygetzyStatusRegzyCU} \lstinputlisting[language=sail]{sail_latex/sailfngetStatusRegCU.tex}}

\newcommand{\sailsetStatusRegCU}{\label{zzysetzyStatusRegzyCU} \lstinputlisting[language=sail]{sail_latex/sailsetStatusRegCU.tex}}

\newcommand{\sailfnsetStatusRegCU}{\label{zzysetzyStatusRegzyCU} \lstinputlisting[language=sail]{sail_latex/sailfnsetStatusRegCU.tex}}

\newcommand{\sailupdateStatusRegCU}{\label{zzyupdatezyStatusRegzyCU} \lstinputlisting[language=sail]{sail_latex/sailupdateStatusRegCU.tex}}

\newcommand{\sailfnupdateStatusRegCU}{\label{zzyupdatezyStatusRegzyCU} \lstinputlisting[language=sail]{sail_latex/sailfnupdateStatusRegCU.tex}}

\newcommand{\sailupdateCU}{\label{zupdatezyCU} \lstinputlisting[language=sail]{sail_latex/sailupdateCU.tex}}

\newcommand{\sailmodCU}{\label{zzymodzyCU} \lstinputlisting[language=sail]{sail_latex/sailmodCU.tex}}

\newcommand{\sailgetStatusRegBEV}{\label{zzygetzyStatusRegzyBEV} \lstinputlisting[language=sail]{sail_latex/sailgetStatusRegBEV.tex}}

\newcommand{\sailfngetStatusRegBEV}{\label{zzygetzyStatusRegzyBEV} \lstinputlisting[language=sail]{sail_latex/sailfngetStatusRegBEV.tex}}

\newcommand{\sailsetStatusRegBEV}{\label{zzysetzyStatusRegzyBEV} \lstinputlisting[language=sail]{sail_latex/sailsetStatusRegBEV.tex}}

\newcommand{\sailfnsetStatusRegBEV}{\label{zzysetzyStatusRegzyBEV} \lstinputlisting[language=sail]{sail_latex/sailfnsetStatusRegBEV.tex}}

\newcommand{\sailupdateStatusRegBEV}{\label{zzyupdatezyStatusRegzyBEV} \lstinputlisting[language=sail]{sail_latex/sailupdateStatusRegBEV.tex}}

\newcommand{\sailfnupdateStatusRegBEV}{\label{zzyupdatezyStatusRegzyBEV} \lstinputlisting[language=sail]{sail_latex/sailfnupdateStatusRegBEV.tex}}

\newcommand{\sailupdateBEV}{\label{zupdatezyBEV} \lstinputlisting[language=sail]{sail_latex/sailupdateBEV.tex}}

\newcommand{\sailmodBEV}{\label{zzymodzyBEV} \lstinputlisting[language=sail]{sail_latex/sailmodBEV.tex}}

\newcommand{\sailgetStatusRegIM}{\label{zzygetzyStatusRegzyIM} \lstinputlisting[language=sail]{sail_latex/sailgetStatusRegIM.tex}}

\newcommand{\sailfngetStatusRegIM}{\label{zzygetzyStatusRegzyIM} \lstinputlisting[language=sail]{sail_latex/sailfngetStatusRegIM.tex}}

\newcommand{\sailsetStatusRegIM}{\label{zzysetzyStatusRegzyIM} \lstinputlisting[language=sail]{sail_latex/sailsetStatusRegIM.tex}}

\newcommand{\sailfnsetStatusRegIM}{\label{zzysetzyStatusRegzyIM} \lstinputlisting[language=sail]{sail_latex/sailfnsetStatusRegIM.tex}}

\newcommand{\sailupdateStatusRegIM}{\label{zzyupdatezyStatusRegzyIM} \lstinputlisting[language=sail]{sail_latex/sailupdateStatusRegIM.tex}}

\newcommand{\sailfnupdateStatusRegIM}{\label{zzyupdatezyStatusRegzyIM} \lstinputlisting[language=sail]{sail_latex/sailfnupdateStatusRegIM.tex}}

\newcommand{\sailupdateIM}{\label{zupdatezyIM} \lstinputlisting[language=sail]{sail_latex/sailupdateIM.tex}}

\newcommand{\sailmodIM}{\label{zzymodzyIM} \lstinputlisting[language=sail]{sail_latex/sailmodIM.tex}}

\newcommand{\sailgetStatusRegKX}{\label{zzygetzyStatusRegzyKX} \lstinputlisting[language=sail]{sail_latex/sailgetStatusRegKX.tex}}

\newcommand{\sailfngetStatusRegKX}{\label{zzygetzyStatusRegzyKX} \lstinputlisting[language=sail]{sail_latex/sailfngetStatusRegKX.tex}}

\newcommand{\sailsetStatusRegKX}{\label{zzysetzyStatusRegzyKX} \lstinputlisting[language=sail]{sail_latex/sailsetStatusRegKX.tex}}

\newcommand{\sailfnsetStatusRegKX}{\label{zzysetzyStatusRegzyKX} \lstinputlisting[language=sail]{sail_latex/sailfnsetStatusRegKX.tex}}

\newcommand{\sailupdateStatusRegKX}{\label{zzyupdatezyStatusRegzyKX} \lstinputlisting[language=sail]{sail_latex/sailupdateStatusRegKX.tex}}

\newcommand{\sailfnupdateStatusRegKX}{\label{zzyupdatezyStatusRegzyKX} \lstinputlisting[language=sail]{sail_latex/sailfnupdateStatusRegKX.tex}}

\newcommand{\sailupdateKX}{\label{zupdatezyKX} \lstinputlisting[language=sail]{sail_latex/sailupdateKX.tex}}

\newcommand{\sailmodKX}{\label{zzymodzyKX} \lstinputlisting[language=sail]{sail_latex/sailmodKX.tex}}

\newcommand{\sailgetStatusRegSX}{\label{zzygetzyStatusRegzySX} \lstinputlisting[language=sail]{sail_latex/sailgetStatusRegSX.tex}}

\newcommand{\sailfngetStatusRegSX}{\label{zzygetzyStatusRegzySX} \lstinputlisting[language=sail]{sail_latex/sailfngetStatusRegSX.tex}}

\newcommand{\sailsetStatusRegSX}{\label{zzysetzyStatusRegzySX} \lstinputlisting[language=sail]{sail_latex/sailsetStatusRegSX.tex}}

\newcommand{\sailfnsetStatusRegSX}{\label{zzysetzyStatusRegzySX} \lstinputlisting[language=sail]{sail_latex/sailfnsetStatusRegSX.tex}}

\newcommand{\sailupdateStatusRegSX}{\label{zzyupdatezyStatusRegzySX} \lstinputlisting[language=sail]{sail_latex/sailupdateStatusRegSX.tex}}

\newcommand{\sailfnupdateStatusRegSX}{\label{zzyupdatezyStatusRegzySX} \lstinputlisting[language=sail]{sail_latex/sailfnupdateStatusRegSX.tex}}

\newcommand{\sailupdateSX}{\label{zupdatezySX} \lstinputlisting[language=sail]{sail_latex/sailupdateSX.tex}}

\newcommand{\sailmodSX}{\label{zzymodzySX} \lstinputlisting[language=sail]{sail_latex/sailmodSX.tex}}

\newcommand{\sailgetStatusRegUX}{\label{zzygetzyStatusRegzyUX} \lstinputlisting[language=sail]{sail_latex/sailgetStatusRegUX.tex}}

\newcommand{\sailfngetStatusRegUX}{\label{zzygetzyStatusRegzyUX} \lstinputlisting[language=sail]{sail_latex/sailfngetStatusRegUX.tex}}

\newcommand{\sailsetStatusRegUX}{\label{zzysetzyStatusRegzyUX} \lstinputlisting[language=sail]{sail_latex/sailsetStatusRegUX.tex}}

\newcommand{\sailfnsetStatusRegUX}{\label{zzysetzyStatusRegzyUX} \lstinputlisting[language=sail]{sail_latex/sailfnsetStatusRegUX.tex}}

\newcommand{\sailupdateStatusRegUX}{\label{zzyupdatezyStatusRegzyUX} \lstinputlisting[language=sail]{sail_latex/sailupdateStatusRegUX.tex}}

\newcommand{\sailfnupdateStatusRegUX}{\label{zzyupdatezyStatusRegzyUX} \lstinputlisting[language=sail]{sail_latex/sailfnupdateStatusRegUX.tex}}

\newcommand{\sailupdateUX}{\label{zupdatezyUX} \lstinputlisting[language=sail]{sail_latex/sailupdateUX.tex}}

\newcommand{\sailmodUX}{\label{zzymodzyUX} \lstinputlisting[language=sail]{sail_latex/sailmodUX.tex}}

\newcommand{\sailgetStatusRegKSU}{\label{zzygetzyStatusRegzyKSU} \lstinputlisting[language=sail]{sail_latex/sailgetStatusRegKSU.tex}}

\newcommand{\sailfngetStatusRegKSU}{\label{zzygetzyStatusRegzyKSU} \lstinputlisting[language=sail]{sail_latex/sailfngetStatusRegKSU.tex}}

\newcommand{\sailsetStatusRegKSU}{\label{zzysetzyStatusRegzyKSU} \lstinputlisting[language=sail]{sail_latex/sailsetStatusRegKSU.tex}}

\newcommand{\sailfnsetStatusRegKSU}{\label{zzysetzyStatusRegzyKSU} \lstinputlisting[language=sail]{sail_latex/sailfnsetStatusRegKSU.tex}}

\newcommand{\sailupdateStatusRegKSU}{\label{zzyupdatezyStatusRegzyKSU} \lstinputlisting[language=sail]{sail_latex/sailupdateStatusRegKSU.tex}}

\newcommand{\sailfnupdateStatusRegKSU}{\label{zzyupdatezyStatusRegzyKSU} \lstinputlisting[language=sail]{sail_latex/sailfnupdateStatusRegKSU.tex}}

\newcommand{\sailupdateKSU}{\label{zupdatezyKSU} \lstinputlisting[language=sail]{sail_latex/sailupdateKSU.tex}}

\newcommand{\sailmodKSU}{\label{zzymodzyKSU} \lstinputlisting[language=sail]{sail_latex/sailmodKSU.tex}}

\newcommand{\sailgetStatusRegERL}{\label{zzygetzyStatusRegzyERL} \lstinputlisting[language=sail]{sail_latex/sailgetStatusRegERL.tex}}

\newcommand{\sailfngetStatusRegERL}{\label{zzygetzyStatusRegzyERL} \lstinputlisting[language=sail]{sail_latex/sailfngetStatusRegERL.tex}}

\newcommand{\sailsetStatusRegERL}{\label{zzysetzyStatusRegzyERL} \lstinputlisting[language=sail]{sail_latex/sailsetStatusRegERL.tex}}

\newcommand{\sailfnsetStatusRegERL}{\label{zzysetzyStatusRegzyERL} \lstinputlisting[language=sail]{sail_latex/sailfnsetStatusRegERL.tex}}

\newcommand{\sailupdateStatusRegERL}{\label{zzyupdatezyStatusRegzyERL} \lstinputlisting[language=sail]{sail_latex/sailupdateStatusRegERL.tex}}

\newcommand{\sailfnupdateStatusRegERL}{\label{zzyupdatezyStatusRegzyERL} \lstinputlisting[language=sail]{sail_latex/sailfnupdateStatusRegERL.tex}}

\newcommand{\sailupdateERL}{\label{zupdatezyERL} \lstinputlisting[language=sail]{sail_latex/sailupdateERL.tex}}

\newcommand{\sailmodERL}{\label{zzymodzyERL} \lstinputlisting[language=sail]{sail_latex/sailmodERL.tex}}

\newcommand{\sailgetStatusRegEXL}{\label{zzygetzyStatusRegzyEXL} \lstinputlisting[language=sail]{sail_latex/sailgetStatusRegEXL.tex}}

\newcommand{\sailfngetStatusRegEXL}{\label{zzygetzyStatusRegzyEXL} \lstinputlisting[language=sail]{sail_latex/sailfngetStatusRegEXL.tex}}

\newcommand{\sailsetStatusRegEXL}{\label{zzysetzyStatusRegzyEXL} \lstinputlisting[language=sail]{sail_latex/sailsetStatusRegEXL.tex}}

\newcommand{\sailfnsetStatusRegEXL}{\label{zzysetzyStatusRegzyEXL} \lstinputlisting[language=sail]{sail_latex/sailfnsetStatusRegEXL.tex}}

\newcommand{\sailupdateStatusRegEXL}{\label{zzyupdatezyStatusRegzyEXL} \lstinputlisting[language=sail]{sail_latex/sailupdateStatusRegEXL.tex}}

\newcommand{\sailfnupdateStatusRegEXL}{\label{zzyupdatezyStatusRegzyEXL} \lstinputlisting[language=sail]{sail_latex/sailfnupdateStatusRegEXL.tex}}

\newcommand{\sailupdateEXL}{\label{zupdatezyEXL} \lstinputlisting[language=sail]{sail_latex/sailupdateEXL.tex}}

\newcommand{\sailmodEXL}{\label{zzymodzyEXL} \lstinputlisting[language=sail]{sail_latex/sailmodEXL.tex}}

\newcommand{\sailgetStatusRegIE}{\label{zzygetzyStatusRegzyIE} \lstinputlisting[language=sail]{sail_latex/sailgetStatusRegIE.tex}}

\newcommand{\sailfngetStatusRegIE}{\label{zzygetzyStatusRegzyIE} \lstinputlisting[language=sail]{sail_latex/sailfngetStatusRegIE.tex}}

\newcommand{\sailsetStatusRegIE}{\label{zzysetzyStatusRegzyIE} \lstinputlisting[language=sail]{sail_latex/sailsetStatusRegIE.tex}}

\newcommand{\sailfnsetStatusRegIE}{\label{zzysetzyStatusRegzyIE} \lstinputlisting[language=sail]{sail_latex/sailfnsetStatusRegIE.tex}}

\newcommand{\sailupdateStatusRegIE}{\label{zzyupdatezyStatusRegzyIE} \lstinputlisting[language=sail]{sail_latex/sailupdateStatusRegIE.tex}}

\newcommand{\sailfnupdateStatusRegIE}{\label{zzyupdatezyStatusRegzyIE} \lstinputlisting[language=sail]{sail_latex/sailfnupdateStatusRegIE.tex}}

\newcommand{\sailupdateIE}{\label{zupdatezyIE} \lstinputlisting[language=sail]{sail_latex/sailupdateIE.tex}}

\newcommand{\sailmodIE}{\label{zzymodzyIE} \lstinputlisting[language=sail]{sail_latex/sailmodIE.tex}}

\newcommand{\sailexecutebranch}{\label{zexecutezybranch} \lstinputlisting[language=sail]{sail_latex/sailexecutebranch.tex}}

\newcommand{\sailfnexecutebranch}{\label{zexecutezybranch} \lstinputlisting[language=sail]{sail_latex/sailfnexecutebranch.tex}}

\newcommand{\sailNotWordVal}{\label{zNotWordVal} \lstinputlisting[language=sail]{sail_latex/sailNotWordVal.tex}}

\newcommand{\sailfnNotWordVal}{\label{zNotWordVal} \lstinputlisting[language=sail]{sail_latex/sailfnNotWordVal.tex}}

\newcommand{\sailrGPR}{\label{zrGPR} \lstinputlisting[language=sail]{sail_latex/sailrGPR.tex}}

\newcommand{\sailfnrGPR}{\label{zrGPR} \lstinputlisting[language=sail]{sail_latex/sailfnrGPR.tex}}

\newcommand{\sailwGPR}{\label{zwGPR} \lstinputlisting[language=sail]{sail_latex/sailwGPR.tex}}

\newcommand{\sailfnwGPR}{\label{zwGPR} \lstinputlisting[language=sail]{sail_latex/sailfnwGPR.tex}}

\newcommand{\sailMEMr}{\label{zMEMr} \lstinputlisting[language=sail]{sail_latex/sailMEMr.tex}}

\newcommand{\sailMEMrreserve}{\label{zMEMrzyreserve} \lstinputlisting[language=sail]{sail_latex/sailMEMrreserve.tex}}

\newcommand{\sailMEMsync}{\label{zMEMzysync} \lstinputlisting[language=sail]{sail_latex/sailMEMsync.tex}}

\newcommand{\sailMEMea}{\label{zMEMea} \lstinputlisting[language=sail]{sail_latex/sailMEMea.tex}}

\newcommand{\sailMEMeaconditional}{\label{zMEMeazyconditional} \lstinputlisting[language=sail]{sail_latex/sailMEMeaconditional.tex}}

\newcommand{\sailMEMval}{\label{zMEMval} \lstinputlisting[language=sail]{sail_latex/sailMEMval.tex}}

\newcommand{\sailMEMvalconditional}{\label{zMEMvalzyconditional} \lstinputlisting[language=sail]{sail_latex/sailMEMvalconditional.tex}}

\newcommand{\sailskipeamem}{\label{zskipzyeamem} \lstinputlisting[language=sail]{sail_latex/sailskipeamem.tex}}

\newcommand{\sailskipbarr}{\label{zskipzybarr} \lstinputlisting[language=sail]{sail_latex/sailskipbarr.tex}}

\newcommand{\sailskipwreg}{\label{zskipzywreg} \lstinputlisting[language=sail]{sail_latex/sailskipwreg.tex}}

\newcommand{\sailskiprreg}{\label{zskipzyrreg} \lstinputlisting[language=sail]{sail_latex/sailskiprreg.tex}}

\newcommand{\sailskipwmvt}{\label{zskipzywmvt} \lstinputlisting[language=sail]{sail_latex/sailskipwmvt.tex}}

\newcommand{\sailskiprmemt}{\label{zskipzyrmemt} \lstinputlisting[language=sail]{sail_latex/sailskiprmemt.tex}}

\newcommand{\sailskipescape}{\label{zskipzyescape} \lstinputlisting[language=sail]{sail_latex/sailskipescape.tex}}

\newcommand{\sailfnMEMr}{\label{zMEMr} \lstinputlisting[language=sail]{sail_latex/sailfnMEMr.tex}}

\newcommand{\sailfnMEMrreserve}{\label{zMEMrzyreserve} \lstinputlisting[language=sail]{sail_latex/sailfnMEMrreserve.tex}}

\newcommand{\sailfnMEMsync}{\label{zMEMzysync} \lstinputlisting[language=sail]{sail_latex/sailfnMEMsync.tex}}

\newcommand{\sailfnMEMea}{\label{zMEMea} \lstinputlisting[language=sail]{sail_latex/sailfnMEMea.tex}}

\newcommand{\sailfnMEMeaconditional}{\label{zMEMeazyconditional} \lstinputlisting[language=sail]{sail_latex/sailfnMEMeaconditional.tex}}

\newcommand{\sailfnMEMval}{\label{zMEMval} \lstinputlisting[language=sail]{sail_latex/sailfnMEMval.tex}}

\newcommand{\sailfnMEMvalconditional}{\label{zMEMvalzyconditional} \lstinputlisting[language=sail]{sail_latex/sailfnMEMvalconditional.tex}}

\newcommand{\sailException}{\label{zException} \lstinputlisting[language=sail]{sail_latex/sailException.tex}}

\newcommand{\sailExceptionofnum}{\label{zExceptionzyofzynum} \lstinputlisting[language=sail]{sail_latex/sailExceptionofnum.tex}}

\newcommand{\sailfnExceptionofnum}{\label{zExceptionzyofzynum} \lstinputlisting[language=sail]{sail_latex/sailfnExceptionofnum.tex}}

\newcommand{\sailnumofException}{\label{znumzyofzyException} \lstinputlisting[language=sail]{sail_latex/sailnumofException.tex}}

\newcommand{\sailfnnumofException}{\label{znumzyofzyException} \lstinputlisting[language=sail]{sail_latex/sailfnnumofException.tex}}

\newcommand{\sailExceptionCode}{\label{zExceptionCode} \lstinputlisting[language=sail]{sail_latex/sailExceptionCode.tex}}

\newcommand{\sailfnExceptionCode}{\label{zExceptionCode} \lstinputlisting[language=sail]{sail_latex/sailfnExceptionCode.tex}}

\newcommand{\sailSignalExceptionMIPS}{\label{zSignalExceptionMIPS} \lstinputlisting[language=sail]{sail_latex/sailSignalExceptionMIPS.tex}}

\newcommand{\sailfnSignalExceptionMIPS}{\label{zSignalExceptionMIPS} \lstinputlisting[language=sail]{sail_latex/sailfnSignalExceptionMIPS.tex}}

\newcommand{\sailSignalException}{\label{zSignalException} \lstinputlisting[language=sail]{sail_latex/sailSignalException.tex}}

\newcommand{\sailSignalExceptionBadAddr}{\label{zSignalExceptionBadAddr} \lstinputlisting[language=sail]{sail_latex/sailSignalExceptionBadAddr.tex}}

\newcommand{\sailfnSignalExceptionBadAddr}{\label{zSignalExceptionBadAddr} \lstinputlisting[language=sail]{sail_latex/sailfnSignalExceptionBadAddr.tex}}

\newcommand{\sailSignalExceptionTLB}{\label{zSignalExceptionTLB} \lstinputlisting[language=sail]{sail_latex/sailSignalExceptionTLB.tex}}

\newcommand{\sailfnSignalExceptionTLB}{\label{zSignalExceptionTLB} \lstinputlisting[language=sail]{sail_latex/sailfnSignalExceptionTLB.tex}}

\newcommand{\sailMemAccessType}{\label{zMemAccessType} \lstinputlisting[language=sail]{sail_latex/sailMemAccessType.tex}}

\newcommand{\sailMemAccessTypeofnum}{\label{zMemAccessTypezyofzynum} \lstinputlisting[language=sail]{sail_latex/sailMemAccessTypeofnum.tex}}

\newcommand{\sailfnMemAccessTypeofnum}{\label{zMemAccessTypezyofzynum} \lstinputlisting[language=sail]{sail_latex/sailfnMemAccessTypeofnum.tex}}

\newcommand{\sailnumofMemAccessType}{\label{znumzyofzyMemAccessType} \lstinputlisting[language=sail]{sail_latex/sailnumofMemAccessType.tex}}

\newcommand{\sailfnnumofMemAccessType}{\label{znumzyofzyMemAccessType} \lstinputlisting[language=sail]{sail_latex/sailfnnumofMemAccessType.tex}}

\newcommand{\sailAccessLevel}{\label{zAccessLevel} \lstinputlisting[language=sail]{sail_latex/sailAccessLevel.tex}}

\newcommand{\sailAccessLevelofnum}{\label{zAccessLevelzyofzynum} \lstinputlisting[language=sail]{sail_latex/sailAccessLevelofnum.tex}}

\newcommand{\sailfnAccessLevelofnum}{\label{zAccessLevelzyofzynum} \lstinputlisting[language=sail]{sail_latex/sailfnAccessLevelofnum.tex}}

\newcommand{\sailnumofAccessLevel}{\label{znumzyofzyAccessLevel} \lstinputlisting[language=sail]{sail_latex/sailnumofAccessLevel.tex}}

\newcommand{\sailfnnumofAccessLevel}{\label{znumzyofzyAccessLevel} \lstinputlisting[language=sail]{sail_latex/sailfnnumofAccessLevel.tex}}

\newcommand{\sailintofAccessLevel}{\label{zintzyofzyAccessLevel} \lstinputlisting[language=sail]{sail_latex/sailintofAccessLevel.tex}}

\newcommand{\sailfnintofAccessLevel}{\label{zintzyofzyAccessLevel} \lstinputlisting[language=sail]{sail_latex/sailfnintofAccessLevel.tex}}

\newcommand{\sailgrantsAccess}{\label{zgrantsAccess} 
Returns whether the first AccessLevel is sufficient to grant access at the second, required, access level.
 \lstinputlisting[language=sail]{sail_latex/sailgrantsAccess.tex}}

\newcommand{\sailfngrantsAccess}{\label{zgrantsAccess} \lstinputlisting[language=sail]{sail_latex/sailfngrantsAccess.tex}}

\newcommand{\sailgetAccessLevel}{\label{zgetAccessLevel} 
Returns the current effective access level determined by accessing the relevant parts of the MIPS status register.
 \lstinputlisting[language=sail]{sail_latex/sailgetAccessLevel.tex}}

\newcommand{\sailfngetAccessLevel}{\label{zgetAccessLevel} \lstinputlisting[language=sail]{sail_latex/sailfngetAccessLevel.tex}}

\newcommand{\sailcheckCPzeroAccess}{\label{zcheckCPzeroAccess} \lstinputlisting[language=sail]{sail_latex/sailcheckCPzeroAccess.tex}}

\newcommand{\sailfncheckCPzeroAccess}{\label{zcheckCPzeroAccess} \lstinputlisting[language=sail]{sail_latex/sailfncheckCPzeroAccess.tex}}

\newcommand{\sailincrementCPzeroCount}{\label{zincrementCPzeroCount} \lstinputlisting[language=sail]{sail_latex/sailincrementCPzeroCount.tex}}

\newcommand{\sailfnincrementCPzeroCount}{\label{zincrementCPzeroCount} \lstinputlisting[language=sail]{sail_latex/sailfnincrementCPzeroCount.tex}}

\newcommand{\sailregno}{\label{zregno} \lstinputlisting[language=sail]{sail_latex/sailregno.tex}}

\newcommand{\sailimmonesix}{\label{zimmonesix} \lstinputlisting[language=sail]{sail_latex/sailimmonesix.tex}}

\newcommand{\sailregregreg}{\label{zregregreg} \lstinputlisting[language=sail]{sail_latex/sailregregreg.tex}}

\newcommand{\sailregregimmonesix}{\label{zregregimmonesix} \lstinputlisting[language=sail]{sail_latex/sailregregimmonesix.tex}}

\newcommand{\saildecodefailure}{\label{zdecodezyfailure} \lstinputlisting[language=sail]{sail_latex/saildecodefailure.tex}}

\newcommand{\saildecodefailureofnum}{\label{zdecodezyfailurezyofzynum} \lstinputlisting[language=sail]{sail_latex/saildecodefailureofnum.tex}}

\newcommand{\sailfndecodefailureofnum}{\label{zdecodezyfailurezyofzynum} \lstinputlisting[language=sail]{sail_latex/sailfndecodefailureofnum.tex}}

\newcommand{\sailnumofdecodefailure}{\label{znumzyofzydecodezyfailure} \lstinputlisting[language=sail]{sail_latex/sailnumofdecodefailure.tex}}

\newcommand{\sailfnnumofdecodefailure}{\label{znumzyofzydecodezyfailure} \lstinputlisting[language=sail]{sail_latex/sailfnnumofdecodefailure.tex}}

\newcommand{\sailComparison}{\label{zComparison} \lstinputlisting[language=sail]{sail_latex/sailComparison.tex}}

\newcommand{\sailComparisonofnum}{\label{zComparisonzyofzynum} \lstinputlisting[language=sail]{sail_latex/sailComparisonofnum.tex}}

\newcommand{\sailfnComparisonofnum}{\label{zComparisonzyofzynum} \lstinputlisting[language=sail]{sail_latex/sailfnComparisonofnum.tex}}

\newcommand{\sailnumofComparison}{\label{znumzyofzyComparison} \lstinputlisting[language=sail]{sail_latex/sailnumofComparison.tex}}

\newcommand{\sailfnnumofComparison}{\label{znumzyofzyComparison} \lstinputlisting[language=sail]{sail_latex/sailfnnumofComparison.tex}}

\newcommand{\sailcompare}{\label{zcompare} \lstinputlisting[language=sail]{sail_latex/sailcompare.tex}}

\newcommand{\sailfncompare}{\label{zcompare} \lstinputlisting[language=sail]{sail_latex/sailfncompare.tex}}

\newcommand{\sailWordType}{\label{zWordType} \lstinputlisting[language=sail]{sail_latex/sailWordType.tex}}

\newcommand{\sailWordTypeofnum}{\label{zWordTypezyofzynum} \lstinputlisting[language=sail]{sail_latex/sailWordTypeofnum.tex}}

\newcommand{\sailfnWordTypeofnum}{\label{zWordTypezyofzynum} \lstinputlisting[language=sail]{sail_latex/sailfnWordTypeofnum.tex}}

\newcommand{\sailnumofWordType}{\label{znumzyofzyWordType} \lstinputlisting[language=sail]{sail_latex/sailnumofWordType.tex}}

\newcommand{\sailfnnumofWordType}{\label{znumzyofzyWordType} \lstinputlisting[language=sail]{sail_latex/sailfnnumofWordType.tex}}

\newcommand{\sailWordTypeUnaligned}{\label{zWordTypeUnaligned} \lstinputlisting[language=sail]{sail_latex/sailWordTypeUnaligned.tex}}

\newcommand{\sailWordTypeUnalignedofnum}{\label{zWordTypeUnalignedzyofzynum} \lstinputlisting[language=sail]{sail_latex/sailWordTypeUnalignedofnum.tex}}

\newcommand{\sailfnWordTypeUnalignedofnum}{\label{zWordTypeUnalignedzyofzynum} \lstinputlisting[language=sail]{sail_latex/sailfnWordTypeUnalignedofnum.tex}}

\newcommand{\sailnumofWordTypeUnaligned}{\label{znumzyofzyWordTypeUnaligned} \lstinputlisting[language=sail]{sail_latex/sailnumofWordTypeUnaligned.tex}}

\newcommand{\sailfnnumofWordTypeUnaligned}{\label{znumzyofzyWordTypeUnaligned} \lstinputlisting[language=sail]{sail_latex/sailfnnumofWordTypeUnaligned.tex}}

\newcommand{\sailwordWidthBytes}{\label{zwordWidthBytes} \lstinputlisting[language=sail]{sail_latex/sailwordWidthBytes.tex}}

\newcommand{\sailfnwordWidthBytes}{\label{zwordWidthBytes} \lstinputlisting[language=sail]{sail_latex/sailfnwordWidthBytes.tex}}

\newcommand{\sailisAddressAligned}{\label{zisAddressAligned} \lstinputlisting[language=sail]{sail_latex/sailisAddressAligned.tex}}

\newcommand{\sailfnisAddressAligned}{\label{zisAddressAligned} \lstinputlisting[language=sail]{sail_latex/sailfnisAddressAligned.tex}}

\newcommand{\sailreverseendianness}{\label{zreversezyendianness} \lstinputlisting[language=sail]{sail_latex/sailreverseendianness.tex}}

\newcommand{\sailMEMrwrapper}{\label{zMEMrzywrapper} \lstinputlisting[language=sail]{sail_latex/sailMEMrwrapper.tex}}

\newcommand{\sailfnMEMrwrapper}{\label{zMEMrzywrapper} \lstinputlisting[language=sail]{sail_latex/sailfnMEMrwrapper.tex}}

\newcommand{\sailMEMrreservewrapper}{\label{zMEMrzyreservezywrapper} \lstinputlisting[language=sail]{sail_latex/sailMEMrreservewrapper.tex}}

\newcommand{\sailfnMEMrreservewrapper}{\label{zMEMrzyreservezywrapper} \lstinputlisting[language=sail]{sail_latex/sailfnMEMrreservewrapper.tex}}

\newcommand{\sailinitcpzerostate}{\label{zinitzycpzerozystate} \lstinputlisting[language=sail]{sail_latex/sailinitcpzerostate.tex}}

\newcommand{\sailfninitcpzerostate}{\label{zinitzycpzerozystate} \lstinputlisting[language=sail]{sail_latex/sailfninitcpzerostate.tex}}

\newcommand{\sailinitcptwostate}{\label{zinitzycptwozystate} \lstinputlisting[language=sail]{sail_latex/sailinitcptwostate.tex}}

\newcommand{\sailcptwonextpc}{\label{zcptwozynextzypc} \lstinputlisting[language=sail]{sail_latex/sailcptwonextpc.tex}}

\newcommand{\saildumpcptwostate}{\label{zdumpzycptwozystate} \lstinputlisting[language=sail]{sail_latex/saildumpcptwostate.tex}}

\newcommand{\sailtlbEntryMatch}{\label{ztlbEntryMatch} \lstinputlisting[language=sail]{sail_latex/sailtlbEntryMatch.tex}}

\newcommand{\sailfntlbEntryMatch}{\label{ztlbEntryMatch} \lstinputlisting[language=sail]{sail_latex/sailfntlbEntryMatch.tex}}

\newcommand{\sailtlbSearch}{\label{ztlbSearch} \lstinputlisting[language=sail]{sail_latex/sailtlbSearch.tex}}

\newcommand{\sailfntlbSearch}{\label{ztlbSearch} \lstinputlisting[language=sail]{sail_latex/sailfntlbSearch.tex}}

\newcommand{\sailTLBTranslatetwo}{\label{zTLBTranslatetwo} \lstinputlisting[language=sail]{sail_latex/sailTLBTranslatetwo.tex}}

\newcommand{\sailfnTLBTranslatetwo}{\label{zTLBTranslatetwo} \lstinputlisting[language=sail]{sail_latex/sailfnTLBTranslatetwo.tex}}

\newcommand{\sailTLBTranslateC}{\label{zTLBTranslateC} \lstinputlisting[language=sail]{sail_latex/sailTLBTranslateC.tex}}

\newcommand{\sailfnTLBTranslateC}{\label{zTLBTranslateC} \lstinputlisting[language=sail]{sail_latex/sailfnTLBTranslateC.tex}}

\newcommand{\sailTLBTranslate}{\label{zTLBTranslate} \lstinputlisting[language=sail]{sail_latex/sailTLBTranslate.tex}}

\newcommand{\sailfnTLBTranslate}{\label{zTLBTranslate} \lstinputlisting[language=sail]{sail_latex/sailfnTLBTranslate.tex}}

\newcommand{\sailCapLen}{\label{zCapLen} \lstinputlisting[language=sail]{sail_latex/sailCapLen.tex}}

\newcommand{\sailuintsixfour}{\label{zuintsixfour} \lstinputlisting[language=sail]{sail_latex/sailuintsixfour.tex}}

\newcommand{\sailCPtrCmpOp}{\label{zCPtrCmpOp} \lstinputlisting[language=sail]{sail_latex/sailCPtrCmpOp.tex}}

\newcommand{\sailCPtrCmpOpofnum}{\label{zCPtrCmpOpzyofzynum} \lstinputlisting[language=sail]{sail_latex/sailCPtrCmpOpofnum.tex}}

\newcommand{\sailfnCPtrCmpOpofnum}{\label{zCPtrCmpOpzyofzynum} \lstinputlisting[language=sail]{sail_latex/sailfnCPtrCmpOpofnum.tex}}

\newcommand{\sailnumofCPtrCmpOp}{\label{znumzyofzyCPtrCmpOp} \lstinputlisting[language=sail]{sail_latex/sailnumofCPtrCmpOp.tex}}

\newcommand{\sailfnnumofCPtrCmpOp}{\label{znumzyofzyCPtrCmpOp} \lstinputlisting[language=sail]{sail_latex/sailfnnumofCPtrCmpOp.tex}}

\newcommand{\sailClearRegSet}{\label{zClearRegSet} \lstinputlisting[language=sail]{sail_latex/sailClearRegSet.tex}}

\newcommand{\sailClearRegSetofnum}{\label{zClearRegSetzyofzynum} \lstinputlisting[language=sail]{sail_latex/sailClearRegSetofnum.tex}}

\newcommand{\sailfnClearRegSetofnum}{\label{zClearRegSetzyofzynum} \lstinputlisting[language=sail]{sail_latex/sailfnClearRegSetofnum.tex}}

\newcommand{\sailnumofClearRegSet}{\label{znumzyofzyClearRegSet} \lstinputlisting[language=sail]{sail_latex/sailnumofClearRegSet.tex}}

\newcommand{\sailfnnumofClearRegSet}{\label{znumzyofzyClearRegSet} \lstinputlisting[language=sail]{sail_latex/sailfnnumofClearRegSet.tex}}

\newcommand{\sailCapReg}{\label{zCapReg} \lstinputlisting[language=sail]{sail_latex/sailCapReg.tex}}

\newcommand{\sailCapStruct}{\label{zCapStruct} \lstinputlisting[language=sail]{sail_latex/sailCapStruct.tex}}

\newcommand{\sailcapRegToCapStruct}{\label{zcapRegToCapStruct} \lstinputlisting[language=sail]{sail_latex/sailcapRegToCapStruct.tex}}

\newcommand{\sailfncapRegToCapStruct}{\label{zcapRegToCapStruct} \lstinputlisting[language=sail]{sail_latex/sailfncapRegToCapStruct.tex}}

\newcommand{\sailgetCapPerms}{\label{zgetCapPerms} \lstinputlisting[language=sail]{sail_latex/sailgetCapPerms.tex}}

\newcommand{\sailfngetCapPerms}{\label{zgetCapPerms} \lstinputlisting[language=sail]{sail_latex/sailfngetCapPerms.tex}}

\newcommand{\sailcapStructToMemBitstwofivesix}{\label{zcapStructToMemBitstwofivesix} \lstinputlisting[language=sail]{sail_latex/sailcapStructToMemBitstwofivesix.tex}}

\newcommand{\sailfncapStructToMemBitstwofivesix}{\label{zcapStructToMemBitstwofivesix} \lstinputlisting[language=sail]{sail_latex/sailfncapStructToMemBitstwofivesix.tex}}

\newcommand{\sailcapStructToMemBits}{\label{zcapStructToMemBits} \lstinputlisting[language=sail]{sail_latex/sailcapStructToMemBits.tex}}

\newcommand{\sailfncapStructToMemBits}{\label{zcapStructToMemBits} \lstinputlisting[language=sail]{sail_latex/sailfncapStructToMemBits.tex}}

\newcommand{\sailmemBitsToCapBits}{\label{zmemBitsToCapBits} \lstinputlisting[language=sail]{sail_latex/sailmemBitsToCapBits.tex}}

\newcommand{\sailfnmemBitsToCapBits}{\label{zmemBitsToCapBits} \lstinputlisting[language=sail]{sail_latex/sailfnmemBitsToCapBits.tex}}

\newcommand{\sailcapStructToCapReg}{\label{zcapStructToCapReg} \lstinputlisting[language=sail]{sail_latex/sailcapStructToCapReg.tex}}

\newcommand{\sailfncapStructToCapReg}{\label{zcapStructToCapReg} \lstinputlisting[language=sail]{sail_latex/sailfncapStructToCapReg.tex}}

\newcommand{\sailsetCapPerms}{\label{zsetCapPerms} \lstinputlisting[language=sail]{sail_latex/sailsetCapPerms.tex}}

\newcommand{\sailfnsetCapPerms}{\label{zsetCapPerms} \lstinputlisting[language=sail]{sail_latex/sailfnsetCapPerms.tex}}

\newcommand{\sailsealCap}{\label{zsealCap} \lstinputlisting[language=sail]{sail_latex/sailsealCap.tex}}

\newcommand{\sailfnsealCap}{\label{zsealCap} \lstinputlisting[language=sail]{sail_latex/sailfnsealCap.tex}}

\newcommand{\sailgetCapBase}{\label{zgetCapBase} \lstinputlisting[language=sail]{sail_latex/sailgetCapBase.tex}}

\newcommand{\sailfngetCapBase}{\label{zgetCapBase} \lstinputlisting[language=sail]{sail_latex/sailfngetCapBase.tex}}

\newcommand{\sailgetCapTop}{\label{zgetCapTop} \lstinputlisting[language=sail]{sail_latex/sailgetCapTop.tex}}

\newcommand{\sailfngetCapTop}{\label{zgetCapTop} \lstinputlisting[language=sail]{sail_latex/sailfngetCapTop.tex}}

\newcommand{\sailgetCapOffset}{\label{zgetCapOffset} \lstinputlisting[language=sail]{sail_latex/sailgetCapOffset.tex}}

\newcommand{\sailfngetCapOffset}{\label{zgetCapOffset} \lstinputlisting[language=sail]{sail_latex/sailfngetCapOffset.tex}}

\newcommand{\sailgetCapLength}{\label{zgetCapLength} \lstinputlisting[language=sail]{sail_latex/sailgetCapLength.tex}}

\newcommand{\sailfngetCapLength}{\label{zgetCapLength} \lstinputlisting[language=sail]{sail_latex/sailfngetCapLength.tex}}

\newcommand{\sailgetCapCursor}{\label{zgetCapCursor} \lstinputlisting[language=sail]{sail_latex/sailgetCapCursor.tex}}

\newcommand{\sailfngetCapCursor}{\label{zgetCapCursor} \lstinputlisting[language=sail]{sail_latex/sailfngetCapCursor.tex}}

\newcommand{\sailsetCapOffset}{\label{zsetCapOffset} 
Set the offset capability of the a capability to given value and return the result, along with a boolean indicating true if the operation preserved the existing bounds of the capability.  When using compressed capabilities, setting the offset far outside the capability bounds can cause the result to become unrepresentable (XXX mention guarantees). Additionally in some implementations a fast representablity check may be used that could cause the operation to return failure even though the capability would be representable (XXX provide details). 
 \lstinputlisting[language=sail]{sail_latex/sailsetCapOffset.tex}}

\newcommand{\sailfnsetCapOffset}{\label{zsetCapOffset} \lstinputlisting[language=sail]{sail_latex/sailfnsetCapOffset.tex}}

\newcommand{\sailincCapOffset}{\label{zincCapOffset} 
\function{incCapOffset} is the same as \function{setCapOffset} except that the 64-bit value is added to the current capability offset modulo $2^{64}$ (i.e. signed twos-complement arithemtic).
 \lstinputlisting[language=sail]{sail_latex/sailincCapOffset.tex}}

\newcommand{\sailfnincCapOffset}{\label{zincCapOffset} \lstinputlisting[language=sail]{sail_latex/sailfnincCapOffset.tex}}

\newcommand{\sailsetCapBounds}{\label{zsetCapBounds} 
Returns a capability derived from the given capability by setting the base and top to values provided.  The offset of the resulting capability is zero.  In case the requested bounds are not exactly representable the returned boolean is false and the returned capability has bounds at least including the region bounded by base and top but rounded to representable values.
 \lstinputlisting[language=sail]{sail_latex/sailsetCapBounds.tex}}

\newcommand{\sailfnsetCapBounds}{\label{zsetCapBounds} \lstinputlisting[language=sail]{sail_latex/sailfnsetCapBounds.tex}}

\newcommand{\sailinttocap}{\label{zintzytozycap} \lstinputlisting[language=sail]{sail_latex/sailinttocap.tex}}

\newcommand{\sailfninttocap}{\label{zintzytozycap} \lstinputlisting[language=sail]{sail_latex/sailfninttocap.tex}}

\newcommand{\sailexecute}{\label{zexecute} \lstinputlisting[language=sail]{sail_latex/sailexecute.tex}}

\newcommand{\saildecode}{\label{zdecode} \lstinputlisting[language=sail]{sail_latex/saildecode.tex}}

\newcommand{\sailreadCapReg}{\label{zreadCapReg} 
This function reads a given capability register and returns its contents converted to a CapStruct.
If the argument is zero then the null capability is returned.
\lstinputlisting[language=sail]{sail_latex/sailreadCapReg.tex}}

\newcommand{\sailfnreadCapReg}{\label{zreadCapReg} \lstinputlisting[language=sail]{sail_latex/sailfnreadCapReg.tex}}

\newcommand{\sailreadCapRegDDC}{\label{zreadCapRegDDC} 
This is the same as readCapReg except that when the argument is zero the value of DDC is returned 
instead of the null capability. This is used for instructions that expect an address, where using
null would always generate an exception.
\lstinputlisting[language=sail]{sail_latex/sailreadCapRegDDC.tex}}

\newcommand{\sailfnreadCapRegDDC}{\label{zreadCapRegDDC} \lstinputlisting[language=sail]{sail_latex/sailfnreadCapRegDDC.tex}}

\newcommand{\sailwriteCapReg}{\label{zwriteCapReg} \lstinputlisting[language=sail]{sail_latex/sailwriteCapReg.tex}}

\newcommand{\sailfnwriteCapReg}{\label{zwriteCapReg} \lstinputlisting[language=sail]{sail_latex/sailfnwriteCapReg.tex}}

\newcommand{\sailCapEx}{\label{zCapEx} \lstinputlisting[language=sail]{sail_latex/sailCapEx.tex}}

\newcommand{\sailCapExofnum}{\label{zCapExzyofzynum} \lstinputlisting[language=sail]{sail_latex/sailCapExofnum.tex}}

\newcommand{\sailfnCapExofnum}{\label{zCapExzyofzynum} \lstinputlisting[language=sail]{sail_latex/sailfnCapExofnum.tex}}

\newcommand{\sailnumofCapEx}{\label{znumzyofzyCapEx} \lstinputlisting[language=sail]{sail_latex/sailnumofCapEx.tex}}

\newcommand{\sailfnnumofCapEx}{\label{znumzyofzyCapEx} \lstinputlisting[language=sail]{sail_latex/sailfnnumofCapEx.tex}}

\newcommand{\sailCapExCode}{\label{zCapExCode} \lstinputlisting[language=sail]{sail_latex/sailCapExCode.tex}}

\newcommand{\sailfnCapExCode}{\label{zCapExCode} \lstinputlisting[language=sail]{sail_latex/sailfnCapExCode.tex}}

\newcommand{\sailCapCauseReg}{\label{zCapCauseReg} \lstinputlisting[language=sail]{sail_latex/sailCapCauseReg.tex}}

\newcommand{\sailMkCapCauseReg}{\label{zMkzyCapCauseReg} \lstinputlisting[language=sail]{sail_latex/sailMkCapCauseReg.tex}}

\newcommand{\sailfnMkCapCauseReg}{\label{zMkzyCapCauseReg} \lstinputlisting[language=sail]{sail_latex/sailfnMkCapCauseReg.tex}}

\newcommand{\sailgetCapCauseRegbits}{\label{zzygetzyCapCauseRegzybits} \lstinputlisting[language=sail]{sail_latex/sailgetCapCauseRegbits.tex}}

\newcommand{\sailfngetCapCauseRegbits}{\label{zzygetzyCapCauseRegzybits} \lstinputlisting[language=sail]{sail_latex/sailfngetCapCauseRegbits.tex}}

\newcommand{\sailsetCapCauseRegbits}{\label{zzysetzyCapCauseRegzybits} \lstinputlisting[language=sail]{sail_latex/sailsetCapCauseRegbits.tex}}

\newcommand{\sailfnsetCapCauseRegbits}{\label{zzysetzyCapCauseRegzybits} \lstinputlisting[language=sail]{sail_latex/sailfnsetCapCauseRegbits.tex}}

\newcommand{\sailupdateCapCauseRegbits}{\label{zzyupdatezyCapCauseRegzybits} \lstinputlisting[language=sail]{sail_latex/sailupdateCapCauseRegbits.tex}}

\newcommand{\sailfnupdateCapCauseRegbits}{\label{zzyupdatezyCapCauseRegzybits} \lstinputlisting[language=sail]{sail_latex/sailfnupdateCapCauseRegbits.tex}}

\newcommand{\sailupdatebits}{\label{zupdatezybits} \lstinputlisting[language=sail]{sail_latex/sailupdatebits.tex}}

\newcommand{\sailmodbits}{\label{zzymodzybits} \lstinputlisting[language=sail]{sail_latex/sailmodbits.tex}}

\newcommand{\sailgetCapCauseRegExcCode}{\label{zzygetzyCapCauseRegzyExcCode} \lstinputlisting[language=sail]{sail_latex/sailgetCapCauseRegExcCode.tex}}

\newcommand{\sailfngetCapCauseRegExcCode}{\label{zzygetzyCapCauseRegzyExcCode} \lstinputlisting[language=sail]{sail_latex/sailfngetCapCauseRegExcCode.tex}}

\newcommand{\sailsetCapCauseRegExcCode}{\label{zzysetzyCapCauseRegzyExcCode} \lstinputlisting[language=sail]{sail_latex/sailsetCapCauseRegExcCode.tex}}

\newcommand{\sailfnsetCapCauseRegExcCode}{\label{zzysetzyCapCauseRegzyExcCode} \lstinputlisting[language=sail]{sail_latex/sailfnsetCapCauseRegExcCode.tex}}

\newcommand{\sailupdateCapCauseRegExcCode}{\label{zzyupdatezyCapCauseRegzyExcCode} \lstinputlisting[language=sail]{sail_latex/sailupdateCapCauseRegExcCode.tex}}

\newcommand{\sailfnupdateCapCauseRegExcCode}{\label{zzyupdatezyCapCauseRegzyExcCode} \lstinputlisting[language=sail]{sail_latex/sailfnupdateCapCauseRegExcCode.tex}}

\newcommand{\sailupdateExcCode}{\label{zupdatezyExcCode} \lstinputlisting[language=sail]{sail_latex/sailupdateExcCode.tex}}

\newcommand{\sailmodExcCode}{\label{zzymodzyExcCode} \lstinputlisting[language=sail]{sail_latex/sailmodExcCode.tex}}

\newcommand{\sailgetCapCauseRegRegNum}{\label{zzygetzyCapCauseRegzyRegNum} \lstinputlisting[language=sail]{sail_latex/sailgetCapCauseRegRegNum.tex}}

\newcommand{\sailfngetCapCauseRegRegNum}{\label{zzygetzyCapCauseRegzyRegNum} \lstinputlisting[language=sail]{sail_latex/sailfngetCapCauseRegRegNum.tex}}

\newcommand{\sailsetCapCauseRegRegNum}{\label{zzysetzyCapCauseRegzyRegNum} \lstinputlisting[language=sail]{sail_latex/sailsetCapCauseRegRegNum.tex}}

\newcommand{\sailfnsetCapCauseRegRegNum}{\label{zzysetzyCapCauseRegzyRegNum} \lstinputlisting[language=sail]{sail_latex/sailfnsetCapCauseRegRegNum.tex}}

\newcommand{\sailupdateCapCauseRegRegNum}{\label{zzyupdatezyCapCauseRegzyRegNum} \lstinputlisting[language=sail]{sail_latex/sailupdateCapCauseRegRegNum.tex}}

\newcommand{\sailfnupdateCapCauseRegRegNum}{\label{zzyupdatezyCapCauseRegzyRegNum} \lstinputlisting[language=sail]{sail_latex/sailfnupdateCapCauseRegRegNum.tex}}

\newcommand{\sailupdateRegNum}{\label{zupdatezyRegNum} \lstinputlisting[language=sail]{sail_latex/sailupdateRegNum.tex}}

\newcommand{\sailmodRegNum}{\label{zzymodzyRegNum} \lstinputlisting[language=sail]{sail_latex/sailmodRegNum.tex}}

\newcommand{\sailexecutebranchpcc}{\label{zexecutezybranchzypcc} \lstinputlisting[language=sail]{sail_latex/sailexecutebranchpcc.tex}}

\newcommand{\sailfnexecutebranchpcc}{\label{zexecutezybranchzypcc} \lstinputlisting[language=sail]{sail_latex/sailfnexecutebranchpcc.tex}}

\newcommand{\sailfnSignalException}{\label{zSignalException} \lstinputlisting[language=sail]{sail_latex/sailfnSignalException.tex}}

\newcommand{\sailERETHook}{\label{zERETHook} \lstinputlisting[language=sail]{sail_latex/sailERETHook.tex}}

\newcommand{\sailfnERETHook}{\label{zERETHook} \lstinputlisting[language=sail]{sail_latex/sailfnERETHook.tex}}

\newcommand{\sailraisectwoexceptioneight}{\label{zraisezyctwozyexceptioneight} \lstinputlisting[language=sail]{sail_latex/sailraisectwoexceptioneight.tex}}

\newcommand{\sailfnraisectwoexceptioneight}{\label{zraisezyctwozyexceptioneight} \lstinputlisting[language=sail]{sail_latex/sailfnraisectwoexceptioneight.tex}}

\newcommand{\sailraisectwoexception}{\label{zraisezyctwozyexception} \lstinputlisting[language=sail]{sail_latex/sailraisectwoexception.tex}}

\newcommand{\sailfnraisectwoexception}{\label{zraisezyctwozyexception} \lstinputlisting[language=sail]{sail_latex/sailfnraisectwoexception.tex}}

\newcommand{\sailraisectwoexceptionnoreg}{\label{zraisezyctwozyexceptionzynoreg} \lstinputlisting[language=sail]{sail_latex/sailraisectwoexceptionnoreg.tex}}

\newcommand{\sailfnraisectwoexceptionnoreg}{\label{zraisezyctwozyexceptionzynoreg} \lstinputlisting[language=sail]{sail_latex/sailfnraisectwoexceptionnoreg.tex}}

\newcommand{\sailpccaccesssystemregs}{\label{zpcczyaccesszysystemzyregs} \lstinputlisting[language=sail]{sail_latex/sailpccaccesssystemregs.tex}}

\newcommand{\sailfnpccaccesssystemregs}{\label{zpcczyaccesszysystemzyregs} \lstinputlisting[language=sail]{sail_latex/sailfnpccaccesssystemregs.tex}}

\newcommand{\sailregisterinaccessible}{\label{zregisterzyinaccessible} 
The following function should be called before reading or writing any capability register to check whether it is one of the protected system capabilities. Although it is usually a general purpose capabilty the invoked data capabiltiy (IDC) is restricted in the branch delay slot of the CCall (selector one) instruction to protect the confidentiality and integrity of the invoked sandbox.
 \lstinputlisting[language=sail]{sail_latex/sailregisterinaccessible.tex}}

\newcommand{\sailfnregisterinaccessible}{\label{zregisterzyinaccessible} \lstinputlisting[language=sail]{sail_latex/sailfnregisterinaccessible.tex}}

\newcommand{\sailMEMrtag}{\label{zMEMrzytag} \lstinputlisting[language=sail]{sail_latex/sailMEMrtag.tex}}

\newcommand{\sailMEMwtag}{\label{zMEMwzytag} \lstinputlisting[language=sail]{sail_latex/sailMEMwtag.tex}}

\newcommand{\sailMEMrtagged}{\label{zMEMrzytagged} \lstinputlisting[language=sail]{sail_latex/sailMEMrtagged.tex}}

\newcommand{\sailfnMEMrtagged}{\label{zMEMrzytagged} \lstinputlisting[language=sail]{sail_latex/sailfnMEMrtagged.tex}}

\newcommand{\sailMEMrtaggedreserve}{\label{zMEMrzytaggedzyreserve} \lstinputlisting[language=sail]{sail_latex/sailMEMrtaggedreserve.tex}}

\newcommand{\sailfnMEMrtaggedreserve}{\label{zMEMrzytaggedzyreserve} \lstinputlisting[language=sail]{sail_latex/sailfnMEMrtaggedreserve.tex}}

\newcommand{\sailMEMwtagged}{\label{zMEMwzytagged} \lstinputlisting[language=sail]{sail_latex/sailMEMwtagged.tex}}

\newcommand{\sailfnMEMwtagged}{\label{zMEMwzytagged} \lstinputlisting[language=sail]{sail_latex/sailfnMEMwtagged.tex}}

\newcommand{\sailMEMwtaggedconditional}{\label{zMEMwzytaggedzyconditional} \lstinputlisting[language=sail]{sail_latex/sailMEMwtaggedconditional.tex}}

\newcommand{\sailfnMEMwtaggedconditional}{\label{zMEMwzytaggedzyconditional} \lstinputlisting[language=sail]{sail_latex/sailfnMEMwtaggedconditional.tex}}

\newcommand{\sailMEMwwrapper}{\label{zMEMwzywrapper} \lstinputlisting[language=sail]{sail_latex/sailMEMwwrapper.tex}}

\newcommand{\sailfnMEMwwrapper}{\label{zMEMwzywrapper} \lstinputlisting[language=sail]{sail_latex/sailfnMEMwwrapper.tex}}

\newcommand{\sailMEMwconditionalwrapper}{\label{zMEMwzyconditionalzywrapper} \lstinputlisting[language=sail]{sail_latex/sailMEMwconditionalwrapper.tex}}

\newcommand{\sailfnMEMwconditionalwrapper}{\label{zMEMwzyconditionalzywrapper} \lstinputlisting[language=sail]{sail_latex/sailfnMEMwconditionalwrapper.tex}}

\newcommand{\sailcheckDDCPerms}{\label{zcheckDDCPerms} \lstinputlisting[language=sail]{sail_latex/sailcheckDDCPerms.tex}}

\newcommand{\sailfncheckDDCPerms}{\label{zcheckDDCPerms} \lstinputlisting[language=sail]{sail_latex/sailfncheckDDCPerms.tex}}

\newcommand{\sailaddrWrapper}{\label{zaddrWrapper} \lstinputlisting[language=sail]{sail_latex/sailaddrWrapper.tex}}

\newcommand{\sailfnaddrWrapper}{\label{zaddrWrapper} \lstinputlisting[language=sail]{sail_latex/sailfnaddrWrapper.tex}}

\newcommand{\sailaddrWrapperUnaligned}{\label{zaddrWrapperUnaligned} \lstinputlisting[language=sail]{sail_latex/sailaddrWrapperUnaligned.tex}}

\newcommand{\sailfnaddrWrapperUnaligned}{\label{zaddrWrapperUnaligned} \lstinputlisting[language=sail]{sail_latex/sailfnaddrWrapperUnaligned.tex}}

\newcommand{\sailTranslatePC}{\label{zTranslatePC} \lstinputlisting[language=sail]{sail_latex/sailTranslatePC.tex}}

\newcommand{\sailfnTranslatePC}{\label{zTranslatePC} \lstinputlisting[language=sail]{sail_latex/sailfnTranslatePC.tex}}

\newcommand{\sailcheckCPtwousable}{\label{zcheckCPtwousable}   
All capability instrucitons must first check that the capability
co-processor is enabled using the following function that raises a
co-processor unusable exception if a CP0Status.CU2 is not set. This
allows the operating system to only save and restore the full
capability context for processes that use capabilities.
\lstinputlisting[language=sail]{sail_latex/sailcheckCPtwousable.tex}}

\newcommand{\sailfncheckCPtwousable}{\label{zcheckCPtwousable} \lstinputlisting[language=sail]{sail_latex/sailfncheckCPtwousable.tex}}

\newcommand{\sailfninitcptwostate}{\label{zinitzycptwozystate} \lstinputlisting[language=sail]{sail_latex/sailfninitcptwostate.tex}}

\newcommand{\sailfncptwonextpc}{\label{zcptwozynextzypc} \lstinputlisting[language=sail]{sail_latex/sailfncptwonextpc.tex}}

\newcommand{\sailcapToString}{\label{zcapToString} \lstinputlisting[language=sail]{sail_latex/sailcapToString.tex}}

\newcommand{\sailfncapToString}{\label{zcapToString} \lstinputlisting[language=sail]{sail_latex/sailfncapToString.tex}}

\newcommand{\sailfndumpcptwostate}{\label{zdumpzycptwozystate} \lstinputlisting[language=sail]{sail_latex/sailfndumpcptwostate.tex}}

\newcommand{\sailextendLoad}{\label{zextendLoad} \lstinputlisting[language=sail]{sail_latex/sailextendLoad.tex}}

\newcommand{\sailfnextendLoad}{\label{zextendLoad} \lstinputlisting[language=sail]{sail_latex/sailfnextendLoad.tex}}

\newcommand{\sailTLBWriteEntry}{\label{zTLBWriteEntry} \lstinputlisting[language=sail]{sail_latex/sailTLBWriteEntry.tex}}

\newcommand{\sailfnTLBWriteEntry}{\label{zTLBWriteEntry} \lstinputlisting[language=sail]{sail_latex/sailfnTLBWriteEntry.tex}}

\newcommand{\sailfndecodeSomeDADDIU}{ \lstinputlisting[language=sail]{sail_latex/sailfndecodeSomeDADDIU.tex}}

\newcommand{\sailfndecodeSomeDADDU}{ \lstinputlisting[language=sail]{sail_latex/sailfndecodeSomeDADDU.tex}}

\newcommand{\sailfndecodeSomeDADDI}{ \lstinputlisting[language=sail]{sail_latex/sailfndecodeSomeDADDI.tex}}

\newcommand{\sailfndecodeSomeDADD}{ \lstinputlisting[language=sail]{sail_latex/sailfndecodeSomeDADD.tex}}

\newcommand{\sailfndecodeSomeADD}{ \lstinputlisting[language=sail]{sail_latex/sailfndecodeSomeADD.tex}}

\newcommand{\sailfndecodeSomeADDI}{ \lstinputlisting[language=sail]{sail_latex/sailfndecodeSomeADDI.tex}}

\newcommand{\sailfndecodeSomeADDU}{ \lstinputlisting[language=sail]{sail_latex/sailfndecodeSomeADDU.tex}}

\newcommand{\sailfndecodeSomeADDIU}{ \lstinputlisting[language=sail]{sail_latex/sailfndecodeSomeADDIU.tex}}

\newcommand{\sailfndecodeSomeDSUBU}{ \lstinputlisting[language=sail]{sail_latex/sailfndecodeSomeDSUBU.tex}}

\newcommand{\sailfndecodeSomeDSUB}{ \lstinputlisting[language=sail]{sail_latex/sailfndecodeSomeDSUB.tex}}

\newcommand{\sailfndecodeSomeSUB}{ \lstinputlisting[language=sail]{sail_latex/sailfndecodeSomeSUB.tex}}

\newcommand{\sailfndecodeSomeSUBU}{ \lstinputlisting[language=sail]{sail_latex/sailfndecodeSomeSUBU.tex}}

\newcommand{\sailfndecodeSomeAND}{ \lstinputlisting[language=sail]{sail_latex/sailfndecodeSomeAND.tex}}

\newcommand{\sailfndecodeSomeANDI}{ \lstinputlisting[language=sail]{sail_latex/sailfndecodeSomeANDI.tex}}

\newcommand{\sailfndecodeSomeOR}{ \lstinputlisting[language=sail]{sail_latex/sailfndecodeSomeOR.tex}}

\newcommand{\sailfndecodeSomeORI}{ \lstinputlisting[language=sail]{sail_latex/sailfndecodeSomeORI.tex}}

\newcommand{\sailfndecodeSomeNOR}{ \lstinputlisting[language=sail]{sail_latex/sailfndecodeSomeNOR.tex}}

\newcommand{\sailfndecodeSomeXOR}{ \lstinputlisting[language=sail]{sail_latex/sailfndecodeSomeXOR.tex}}

\newcommand{\sailfndecodeSomeXORI}{ \lstinputlisting[language=sail]{sail_latex/sailfndecodeSomeXORI.tex}}

\newcommand{\sailfndecodeSomeLUI}{ \lstinputlisting[language=sail]{sail_latex/sailfndecodeSomeLUI.tex}}

\newcommand{\sailfndecodeSomeDSLL}{ \lstinputlisting[language=sail]{sail_latex/sailfndecodeSomeDSLL.tex}}

\newcommand{\sailfndecodeSomeDSLLthreetwo}{ \lstinputlisting[language=sail]{sail_latex/sailfndecodeSomeDSLLthreetwo.tex}}

\newcommand{\sailfndecodeSomeDSLLV}{ \lstinputlisting[language=sail]{sail_latex/sailfndecodeSomeDSLLV.tex}}

\newcommand{\sailfndecodeSomeDSRA}{ \lstinputlisting[language=sail]{sail_latex/sailfndecodeSomeDSRA.tex}}

\newcommand{\sailfndecodeSomeDSRAthreetwo}{ \lstinputlisting[language=sail]{sail_latex/sailfndecodeSomeDSRAthreetwo.tex}}

\newcommand{\sailfndecodeSomeDSRAV}{ \lstinputlisting[language=sail]{sail_latex/sailfndecodeSomeDSRAV.tex}}

\newcommand{\sailfndecodeSomeDSRL}{ \lstinputlisting[language=sail]{sail_latex/sailfndecodeSomeDSRL.tex}}

\newcommand{\sailfndecodeSomeDSRLthreetwo}{ \lstinputlisting[language=sail]{sail_latex/sailfndecodeSomeDSRLthreetwo.tex}}

\newcommand{\sailfndecodeSomeDSRLV}{ \lstinputlisting[language=sail]{sail_latex/sailfndecodeSomeDSRLV.tex}}

\newcommand{\sailfndecodeSomeSLL}{ \lstinputlisting[language=sail]{sail_latex/sailfndecodeSomeSLL.tex}}

\newcommand{\sailfndecodeSomeSLLV}{ \lstinputlisting[language=sail]{sail_latex/sailfndecodeSomeSLLV.tex}}

\newcommand{\sailfndecodeSomeSRA}{ \lstinputlisting[language=sail]{sail_latex/sailfndecodeSomeSRA.tex}}

\newcommand{\sailfndecodeSomeSRAV}{ \lstinputlisting[language=sail]{sail_latex/sailfndecodeSomeSRAV.tex}}

\newcommand{\sailfndecodeSomeSRL}{ \lstinputlisting[language=sail]{sail_latex/sailfndecodeSomeSRL.tex}}

\newcommand{\sailfndecodeSomeSRLV}{ \lstinputlisting[language=sail]{sail_latex/sailfndecodeSomeSRLV.tex}}

\newcommand{\sailfndecodeSomeSLT}{ \lstinputlisting[language=sail]{sail_latex/sailfndecodeSomeSLT.tex}}

\newcommand{\sailfndecodeSomeSLTI}{ \lstinputlisting[language=sail]{sail_latex/sailfndecodeSomeSLTI.tex}}

\newcommand{\sailfndecodeSomeSLTU}{ \lstinputlisting[language=sail]{sail_latex/sailfndecodeSomeSLTU.tex}}

\newcommand{\sailfndecodeSomeSLTIU}{ \lstinputlisting[language=sail]{sail_latex/sailfndecodeSomeSLTIU.tex}}

\newcommand{\sailfndecodeSomeMOVN}{ \lstinputlisting[language=sail]{sail_latex/sailfndecodeSomeMOVN.tex}}

\newcommand{\sailfndecodeSomeMOVZ}{ \lstinputlisting[language=sail]{sail_latex/sailfndecodeSomeMOVZ.tex}}

\newcommand{\sailfndecodeSomeMFHIrd}{ \lstinputlisting[language=sail]{sail_latex/sailfndecodeSomeMFHIrd.tex}}

\newcommand{\sailfndecodeSomeMFLOrd}{ \lstinputlisting[language=sail]{sail_latex/sailfndecodeSomeMFLOrd.tex}}

\newcommand{\sailfndecodeSomeMTHIrs}{ \lstinputlisting[language=sail]{sail_latex/sailfndecodeSomeMTHIrs.tex}}

\newcommand{\sailfndecodeSomeMTLOrs}{ \lstinputlisting[language=sail]{sail_latex/sailfndecodeSomeMTLOrs.tex}}

\newcommand{\sailfndecodeSomeMUL}{ \lstinputlisting[language=sail]{sail_latex/sailfndecodeSomeMUL.tex}}

\newcommand{\sailfndecodeSomeMULT}{ \lstinputlisting[language=sail]{sail_latex/sailfndecodeSomeMULT.tex}}

\newcommand{\sailfndecodeSomeMULTU}{ \lstinputlisting[language=sail]{sail_latex/sailfndecodeSomeMULTU.tex}}

\newcommand{\sailfndecodeSomeDMULT}{ \lstinputlisting[language=sail]{sail_latex/sailfndecodeSomeDMULT.tex}}

\newcommand{\sailfndecodeSomeDMULTU}{ \lstinputlisting[language=sail]{sail_latex/sailfndecodeSomeDMULTU.tex}}

\newcommand{\sailfndecodeSomeMADD}{ \lstinputlisting[language=sail]{sail_latex/sailfndecodeSomeMADD.tex}}

\newcommand{\sailfndecodeSomeMADDU}{ \lstinputlisting[language=sail]{sail_latex/sailfndecodeSomeMADDU.tex}}

\newcommand{\sailfndecodeSomeMSUB}{ \lstinputlisting[language=sail]{sail_latex/sailfndecodeSomeMSUB.tex}}

\newcommand{\sailfndecodeSomeMSUBU}{ \lstinputlisting[language=sail]{sail_latex/sailfndecodeSomeMSUBU.tex}}

\newcommand{\sailfndecodeSomeDIV}{ \lstinputlisting[language=sail]{sail_latex/sailfndecodeSomeDIV.tex}}

\newcommand{\sailfndecodeSomeDIVU}{ \lstinputlisting[language=sail]{sail_latex/sailfndecodeSomeDIVU.tex}}

\newcommand{\sailfndecodeSomeDDIV}{ \lstinputlisting[language=sail]{sail_latex/sailfndecodeSomeDDIV.tex}}

\newcommand{\sailfndecodeSomeDDIVU}{ \lstinputlisting[language=sail]{sail_latex/sailfndecodeSomeDDIVU.tex}}

\newcommand{\sailfndecodeSomeJoffset}{ \lstinputlisting[language=sail]{sail_latex/sailfndecodeSomeJoffset.tex}}

\newcommand{\sailfndecodeSomeJALoffset}{ \lstinputlisting[language=sail]{sail_latex/sailfndecodeSomeJALoffset.tex}}

\newcommand{\sailfndecodeSomeJRrs}{ \lstinputlisting[language=sail]{sail_latex/sailfndecodeSomeJRrs.tex}}

\newcommand{\sailfndecodeSomeJALR}{ \lstinputlisting[language=sail]{sail_latex/sailfndecodeSomeJALR.tex}}

\newcommand{\sailfndecodeSomeBEQ}{ \lstinputlisting[language=sail]{sail_latex/sailfndecodeSomeBEQ.tex}}

\newcommand{\sailsailfndecodeSomeBEQv}{ \lstinputlisting[language=sail]{sail_latex/sailsailfndecodeSomeBEQv.tex}}

\newcommand{\sailsailsailfndecodeSomeBEQvv}{ \lstinputlisting[language=sail]{sail_latex/sailsailsailfndecodeSomeBEQvv.tex}}

\newcommand{\sailsailsailsailfndecodeSomeBEQvvv}{ \lstinputlisting[language=sail]{sail_latex/sailsailsailsailfndecodeSomeBEQvvv.tex}}

\newcommand{\sailfndecodeSomeBCMPZ}{ \lstinputlisting[language=sail]{sail_latex/sailfndecodeSomeBCMPZ.tex}}

\newcommand{\sailsailfndecodeSomeBCMPZv}{ \lstinputlisting[language=sail]{sail_latex/sailsailfndecodeSomeBCMPZv.tex}}

\newcommand{\sailsailsailfndecodeSomeBCMPZvv}{ \lstinputlisting[language=sail]{sail_latex/sailsailsailfndecodeSomeBCMPZvv.tex}}

\newcommand{\sailsailsailsailfndecodeSomeBCMPZvvv}{ \lstinputlisting[language=sail]{sail_latex/sailsailsailsailfndecodeSomeBCMPZvvv.tex}}

\newcommand{\sailsailsailsailsailfndecodeSomeBCMPZvvvv}{ \lstinputlisting[language=sail]{sail_latex/sailsailsailsailsailfndecodeSomeBCMPZvvvv.tex}}

\newcommand{\sailsailsailsailsailsailfndecodeSomeBCMPZvvvvv}{ \lstinputlisting[language=sail]{sail_latex/sailsailsailsailsailsailfndecodeSomeBCMPZvvvvv.tex}}

\newcommand{\sailsailsailsailsailsailsailfndecodeSomeBCMPZvvvvvv}{ \lstinputlisting[language=sail]{sail_latex/sailsailsailsailsailsailsailfndecodeSomeBCMPZvvvvvv.tex}}

\newcommand{\sailsailsailsailsailsailsailsailfndecodeSomeBCMPZvvvvvvv}{ \lstinputlisting[language=sail]{sail_latex/sailsailsailsailsailsailsailsailfndecodeSomeBCMPZvvvvvvv.tex}}

\newcommand{\sailsailsailsailsailsailsailsailsailfndecodeSomeBCMPZvvvvvvvv}{ \lstinputlisting[language=sail]{sail_latex/sailsailsailsailsailsailsailsailsailfndecodeSomeBCMPZvvvvvvvv.tex}}

\newcommand{\sailsailsailsailsailsailsailsailsailsailfndecodeSomeBCMPZvvvvvvvvv}{ \lstinputlisting[language=sail]{sail_latex/sailsailsailsailsailsailsailsailsailsailfndecodeSomeBCMPZvvvvvvvvv.tex}}

\newcommand{\sailsailsailsailsailsailsailsailsailsailsailfndecodeSomeBCMPZvvvvvvvvvv}{ \lstinputlisting[language=sail]{sail_latex/sailsailsailsailsailsailsailsailsailsailsailfndecodeSomeBCMPZvvvvvvvvvv.tex}}

\newcommand{\sailsailsailsailsailsailsailsailsailsailsailsailfndecodeSomeBCMPZvvvvvvvvvvv}{ \lstinputlisting[language=sail]{sail_latex/sailsailsailsailsailsailsailsailsailsailsailsailfndecodeSomeBCMPZvvvvvvvvvvv.tex}}

\newcommand{\sailfndecodeSomeSYSCALL}{ \lstinputlisting[language=sail]{sail_latex/sailfndecodeSomeSYSCALL.tex}}

\newcommand{\sailfndecodeSomeBREAK}{ \lstinputlisting[language=sail]{sail_latex/sailfndecodeSomeBREAK.tex}}

\newcommand{\sailfndecodeSomeWAIT}{ \lstinputlisting[language=sail]{sail_latex/sailfndecodeSomeWAIT.tex}}

\newcommand{\sailfndecodeSomeTRAPREG}{ \lstinputlisting[language=sail]{sail_latex/sailfndecodeSomeTRAPREG.tex}}

\newcommand{\sailsailfndecodeSomeTRAPREGv}{ \lstinputlisting[language=sail]{sail_latex/sailsailfndecodeSomeTRAPREGv.tex}}

\newcommand{\sailsailsailfndecodeSomeTRAPREGvv}{ \lstinputlisting[language=sail]{sail_latex/sailsailsailfndecodeSomeTRAPREGvv.tex}}

\newcommand{\sailsailsailsailfndecodeSomeTRAPREGvvv}{ \lstinputlisting[language=sail]{sail_latex/sailsailsailsailfndecodeSomeTRAPREGvvv.tex}}

\newcommand{\sailsailsailsailsailfndecodeSomeTRAPREGvvvv}{ \lstinputlisting[language=sail]{sail_latex/sailsailsailsailsailfndecodeSomeTRAPREGvvvv.tex}}

\newcommand{\sailsailsailsailsailsailfndecodeSomeTRAPREGvvvvv}{ \lstinputlisting[language=sail]{sail_latex/sailsailsailsailsailsailfndecodeSomeTRAPREGvvvvv.tex}}

\newcommand{\sailfndecodeSomeTRAPIMM}{ \lstinputlisting[language=sail]{sail_latex/sailfndecodeSomeTRAPIMM.tex}}

\newcommand{\sailsailfndecodeSomeTRAPIMMv}{ \lstinputlisting[language=sail]{sail_latex/sailsailfndecodeSomeTRAPIMMv.tex}}

\newcommand{\sailsailsailfndecodeSomeTRAPIMMvv}{ \lstinputlisting[language=sail]{sail_latex/sailsailsailfndecodeSomeTRAPIMMvv.tex}}

\newcommand{\sailsailsailsailfndecodeSomeTRAPIMMvvv}{ \lstinputlisting[language=sail]{sail_latex/sailsailsailsailfndecodeSomeTRAPIMMvvv.tex}}

\newcommand{\sailsailsailsailsailfndecodeSomeTRAPIMMvvvv}{ \lstinputlisting[language=sail]{sail_latex/sailsailsailsailsailfndecodeSomeTRAPIMMvvvv.tex}}

\newcommand{\sailsailsailsailsailsailfndecodeSomeTRAPIMMvvvvv}{ \lstinputlisting[language=sail]{sail_latex/sailsailsailsailsailsailfndecodeSomeTRAPIMMvvvvv.tex}}

\newcommand{\sailfndecodeSomeLoad}{ \lstinputlisting[language=sail]{sail_latex/sailfndecodeSomeLoad.tex}}

\newcommand{\sailsailfndecodeSomeLoadv}{ \lstinputlisting[language=sail]{sail_latex/sailsailfndecodeSomeLoadv.tex}}

\newcommand{\sailsailsailfndecodeSomeLoadvv}{ \lstinputlisting[language=sail]{sail_latex/sailsailsailfndecodeSomeLoadvv.tex}}

\newcommand{\sailsailsailsailfndecodeSomeLoadvvv}{ \lstinputlisting[language=sail]{sail_latex/sailsailsailsailfndecodeSomeLoadvvv.tex}}

\newcommand{\sailsailsailsailsailfndecodeSomeLoadvvvv}{ \lstinputlisting[language=sail]{sail_latex/sailsailsailsailsailfndecodeSomeLoadvvvv.tex}}

\newcommand{\sailsailsailsailsailsailfndecodeSomeLoadvvvvv}{ \lstinputlisting[language=sail]{sail_latex/sailsailsailsailsailsailfndecodeSomeLoadvvvvv.tex}}

\newcommand{\sailsailsailsailsailsailsailfndecodeSomeLoadvvvvvv}{ \lstinputlisting[language=sail]{sail_latex/sailsailsailsailsailsailsailfndecodeSomeLoadvvvvvv.tex}}

\newcommand{\sailsailsailsailsailsailsailsailfndecodeSomeLoadvvvvvvv}{ \lstinputlisting[language=sail]{sail_latex/sailsailsailsailsailsailsailsailfndecodeSomeLoadvvvvvvv.tex}}

\newcommand{\sailsailsailsailsailsailsailsailsailfndecodeSomeLoadvvvvvvvv}{ \lstinputlisting[language=sail]{sail_latex/sailsailsailsailsailsailsailsailsailfndecodeSomeLoadvvvvvvvv.tex}}

\newcommand{\sailfndecodeSomeStore}{ \lstinputlisting[language=sail]{sail_latex/sailfndecodeSomeStore.tex}}

\newcommand{\sailsailfndecodeSomeStorev}{ \lstinputlisting[language=sail]{sail_latex/sailsailfndecodeSomeStorev.tex}}

\newcommand{\sailsailsailfndecodeSomeStorevv}{ \lstinputlisting[language=sail]{sail_latex/sailsailsailfndecodeSomeStorevv.tex}}

\newcommand{\sailsailsailsailfndecodeSomeStorevvv}{ \lstinputlisting[language=sail]{sail_latex/sailsailsailsailfndecodeSomeStorevvv.tex}}

\newcommand{\sailsailsailsailsailfndecodeSomeStorevvvv}{ \lstinputlisting[language=sail]{sail_latex/sailsailsailsailsailfndecodeSomeStorevvvv.tex}}

\newcommand{\sailsailsailsailsailsailfndecodeSomeStorevvvvv}{ \lstinputlisting[language=sail]{sail_latex/sailsailsailsailsailsailfndecodeSomeStorevvvvv.tex}}

\newcommand{\sailfndecodeSomeLWL}{ \lstinputlisting[language=sail]{sail_latex/sailfndecodeSomeLWL.tex}}

\newcommand{\sailfndecodeSomeLWR}{ \lstinputlisting[language=sail]{sail_latex/sailfndecodeSomeLWR.tex}}

\newcommand{\sailfndecodeSomeSWL}{ \lstinputlisting[language=sail]{sail_latex/sailfndecodeSomeSWL.tex}}

\newcommand{\sailfndecodeSomeSWR}{ \lstinputlisting[language=sail]{sail_latex/sailfndecodeSomeSWR.tex}}

\newcommand{\sailfndecodeSomeLDL}{ \lstinputlisting[language=sail]{sail_latex/sailfndecodeSomeLDL.tex}}

\newcommand{\sailfndecodeSomeLDR}{ \lstinputlisting[language=sail]{sail_latex/sailfndecodeSomeLDR.tex}}

\newcommand{\sailfndecodeSomeSDL}{ \lstinputlisting[language=sail]{sail_latex/sailfndecodeSomeSDL.tex}}

\newcommand{\sailfndecodeSomeSDR}{ \lstinputlisting[language=sail]{sail_latex/sailfndecodeSomeSDR.tex}}

\newcommand{\sailfndecodeSomeCACHE}{ \lstinputlisting[language=sail]{sail_latex/sailfndecodeSomeCACHE.tex}}

\newcommand{\sailfndecodeSomeSYNC}{ \lstinputlisting[language=sail]{sail_latex/sailfndecodeSomeSYNC.tex}}

\newcommand{\sailfndecodeSomeMFCzero}{ \lstinputlisting[language=sail]{sail_latex/sailfndecodeSomeMFCzero.tex}}

\newcommand{\sailsailfndecodeSomeMFCzerov}{ \lstinputlisting[language=sail]{sail_latex/sailsailfndecodeSomeMFCzerov.tex}}

\newcommand{\sailfndecodeSomeHCF}{ \lstinputlisting[language=sail]{sail_latex/sailfndecodeSomeHCF.tex}}

\newcommand{\sailsailfndecodeSomeHCFv}{ \lstinputlisting[language=sail]{sail_latex/sailsailfndecodeSomeHCFv.tex}}

\newcommand{\sailfndecodeSomeMTCzero}{ \lstinputlisting[language=sail]{sail_latex/sailfndecodeSomeMTCzero.tex}}

\newcommand{\sailsailfndecodeSomeMTCzerov}{ \lstinputlisting[language=sail]{sail_latex/sailsailfndecodeSomeMTCzerov.tex}}

\newcommand{\sailfndecodeSome}{ \lstinputlisting[language=sail]{sail_latex/sailfndecodeSome.tex}}

\newcommand{\sailsailfndecodeSomev}{ \lstinputlisting[language=sail]{sail_latex/sailsailfndecodeSomev.tex}}

\newcommand{\sailsailsailfndecodeSomevv}{ \lstinputlisting[language=sail]{sail_latex/sailsailsailfndecodeSomevv.tex}}

\newcommand{\sailsailsailsailfndecodeSomevvv}{ \lstinputlisting[language=sail]{sail_latex/sailsailsailsailfndecodeSomevvv.tex}}

\newcommand{\sailfndecodeSomeRDHWR}{ \lstinputlisting[language=sail]{sail_latex/sailfndecodeSomeRDHWR.tex}}

\newcommand{\sailfndecodeSomeERET}{ \lstinputlisting[language=sail]{sail_latex/sailfndecodeSomeERET.tex}}

\newcommand{\sailfndecodeSomeCGetPerm}{ \lstinputlisting[language=sail]{sail_latex/sailfndecodeSomeCGetPerm.tex}}

\newcommand{\sailfndecodeSomeCGetType}{ \lstinputlisting[language=sail]{sail_latex/sailfndecodeSomeCGetType.tex}}

\newcommand{\sailfndecodeSomeCGetBase}{ \lstinputlisting[language=sail]{sail_latex/sailfndecodeSomeCGetBase.tex}}

\newcommand{\sailfndecodeSomeCGetLen}{ \lstinputlisting[language=sail]{sail_latex/sailfndecodeSomeCGetLen.tex}}

\newcommand{\sailfndecodeSomeCGetTag}{ \lstinputlisting[language=sail]{sail_latex/sailfndecodeSomeCGetTag.tex}}

\newcommand{\sailfndecodeSomeCGetSealed}{ \lstinputlisting[language=sail]{sail_latex/sailfndecodeSomeCGetSealed.tex}}

\newcommand{\sailfndecodeSomeCGetCauserd}{ \lstinputlisting[language=sail]{sail_latex/sailfndecodeSomeCGetCauserd.tex}}

\newcommand{\sailfndecodeSomeCReturn}{ \lstinputlisting[language=sail]{sail_latex/sailfndecodeSomeCReturn.tex}}

\newcommand{\sailfndecodeSomeCGetOffset}{ \lstinputlisting[language=sail]{sail_latex/sailfndecodeSomeCGetOffset.tex}}

\newcommand{\sailfndecodeSomeCSetCausert}{ \lstinputlisting[language=sail]{sail_latex/sailfndecodeSomeCSetCausert.tex}}

\newcommand{\sailfndecodeSomeCAndPerm}{ \lstinputlisting[language=sail]{sail_latex/sailfndecodeSomeCAndPerm.tex}}

\newcommand{\sailfndecodeSomeCToPtr}{ \lstinputlisting[language=sail]{sail_latex/sailfndecodeSomeCToPtr.tex}}

\newcommand{\sailfndecodeSomeCPtrCmp}{ \lstinputlisting[language=sail]{sail_latex/sailfndecodeSomeCPtrCmp.tex}}

\newcommand{\sailsailfndecodeSomeCPtrCmpv}{ \lstinputlisting[language=sail]{sail_latex/sailsailfndecodeSomeCPtrCmpv.tex}}

\newcommand{\sailsailsailfndecodeSomeCPtrCmpvv}{ \lstinputlisting[language=sail]{sail_latex/sailsailsailfndecodeSomeCPtrCmpvv.tex}}

\newcommand{\sailsailsailsailfndecodeSomeCPtrCmpvvv}{ \lstinputlisting[language=sail]{sail_latex/sailsailsailsailfndecodeSomeCPtrCmpvvv.tex}}

\newcommand{\sailsailsailsailsailfndecodeSomeCPtrCmpvvvv}{ \lstinputlisting[language=sail]{sail_latex/sailsailsailsailsailfndecodeSomeCPtrCmpvvvv.tex}}

\newcommand{\sailsailsailsailsailsailfndecodeSomeCPtrCmpvvvvv}{ \lstinputlisting[language=sail]{sail_latex/sailsailsailsailsailsailfndecodeSomeCPtrCmpvvvvv.tex}}

\newcommand{\sailsailsailsailsailsailsailfndecodeSomeCPtrCmpvvvvvv}{ \lstinputlisting[language=sail]{sail_latex/sailsailsailsailsailsailsailfndecodeSomeCPtrCmpvvvvvv.tex}}

\newcommand{\sailsailsailsailsailsailsailsailfndecodeSomeCPtrCmpvvvvvvv}{ \lstinputlisting[language=sail]{sail_latex/sailsailsailsailsailsailsailsailfndecodeSomeCPtrCmpvvvvvvv.tex}}

\newcommand{\sailfndecodeSomeCIncOffset}{ \lstinputlisting[language=sail]{sail_latex/sailfndecodeSomeCIncOffset.tex}}

\newcommand{\sailfndecodeSomeCSetOffset}{ \lstinputlisting[language=sail]{sail_latex/sailfndecodeSomeCSetOffset.tex}}

\newcommand{\sailfndecodeSomeCSetBounds}{ \lstinputlisting[language=sail]{sail_latex/sailfndecodeSomeCSetBounds.tex}}

\newcommand{\sailfndecodeSomeCClearTag}{ \lstinputlisting[language=sail]{sail_latex/sailfndecodeSomeCClearTag.tex}}

\newcommand{\sailfndecodeSomeCFromPtr}{ \lstinputlisting[language=sail]{sail_latex/sailfndecodeSomeCFromPtr.tex}}

\newcommand{\sailfndecodeSomeCCheckPerm}{ \lstinputlisting[language=sail]{sail_latex/sailfndecodeSomeCCheckPerm.tex}}

\newcommand{\sailfndecodeSomeCCheckType}{ \lstinputlisting[language=sail]{sail_latex/sailfndecodeSomeCCheckType.tex}}

\newcommand{\sailfndecodeSomeCSeal}{ \lstinputlisting[language=sail]{sail_latex/sailfndecodeSomeCSeal.tex}}

\newcommand{\sailfndecodeSomeCUnseal}{ \lstinputlisting[language=sail]{sail_latex/sailfndecodeSomeCUnseal.tex}}

\newcommand{\sailfndecodeSomeCJALR}{ \lstinputlisting[language=sail]{sail_latex/sailfndecodeSomeCJALR.tex}}

\newcommand{\sailsailfndecodeSomeCJALRv}{ \lstinputlisting[language=sail]{sail_latex/sailsailfndecodeSomeCJALRv.tex}}

\newcommand{\sailsailfndecodeSomeCGetCauserdv}{ \lstinputlisting[language=sail]{sail_latex/sailsailfndecodeSomeCGetCauserdv.tex}}

\newcommand{\sailfndecodeSomeCSetCausers}{ \lstinputlisting[language=sail]{sail_latex/sailfndecodeSomeCSetCausers.tex}}

\newcommand{\sailfndecodeSomeCGetPCCcd}{ \lstinputlisting[language=sail]{sail_latex/sailfndecodeSomeCGetPCCcd.tex}}

\newcommand{\sailsailsailfndecodeSomeCJALRvv}{ \lstinputlisting[language=sail]{sail_latex/sailsailsailfndecodeSomeCJALRvv.tex}}

\newcommand{\sailsailfndecodeSomeCCheckPermv}{ \lstinputlisting[language=sail]{sail_latex/sailsailfndecodeSomeCCheckPermv.tex}}

\newcommand{\sailsailfndecodeSomeCCheckTypev}{ \lstinputlisting[language=sail]{sail_latex/sailsailfndecodeSomeCCheckTypev.tex}}

\newcommand{\sailsailfndecodeSomeCClearTagv}{ \lstinputlisting[language=sail]{sail_latex/sailsailfndecodeSomeCClearTagv.tex}}

\newcommand{\sailfndecodeSomeCMOVX}{ \lstinputlisting[language=sail]{sail_latex/sailfndecodeSomeCMOVX.tex}}

\newcommand{\sailsailsailsailfndecodeSomeCJALRvvv}{ \lstinputlisting[language=sail]{sail_latex/sailsailsailsailfndecodeSomeCJALRvvv.tex}}

\newcommand{\sailsailfndecodeSomeCGetPermv}{ \lstinputlisting[language=sail]{sail_latex/sailsailfndecodeSomeCGetPermv.tex}}

\newcommand{\sailsailfndecodeSomeCGetTypev}{ \lstinputlisting[language=sail]{sail_latex/sailsailfndecodeSomeCGetTypev.tex}}

\newcommand{\sailsailfndecodeSomeCGetBasev}{ \lstinputlisting[language=sail]{sail_latex/sailsailfndecodeSomeCGetBasev.tex}}

\newcommand{\sailsailfndecodeSomeCGetLenv}{ \lstinputlisting[language=sail]{sail_latex/sailsailfndecodeSomeCGetLenv.tex}}

\newcommand{\sailsailfndecodeSomeCGetTagv}{ \lstinputlisting[language=sail]{sail_latex/sailsailfndecodeSomeCGetTagv.tex}}

\newcommand{\sailsailfndecodeSomeCGetSealedv}{ \lstinputlisting[language=sail]{sail_latex/sailsailfndecodeSomeCGetSealedv.tex}}

\newcommand{\sailsailfndecodeSomeCGetOffsetv}{ \lstinputlisting[language=sail]{sail_latex/sailsailfndecodeSomeCGetOffsetv.tex}}

\newcommand{\sailfndecodeSomeCGetPCCSetOffset}{ \lstinputlisting[language=sail]{sail_latex/sailfndecodeSomeCGetPCCSetOffset.tex}}

\newcommand{\sailfndecodeSomeCReadHwr}{ \lstinputlisting[language=sail]{sail_latex/sailfndecodeSomeCReadHwr.tex}}

\newcommand{\sailfndecodeSomeCWriteHwr}{ \lstinputlisting[language=sail]{sail_latex/sailfndecodeSomeCWriteHwr.tex}}

\newcommand{\sailfndecodeSomeCGetAddr}{ \lstinputlisting[language=sail]{sail_latex/sailfndecodeSomeCGetAddr.tex}}

\newcommand{\sailsailfndecodeSomeCSealv}{ \lstinputlisting[language=sail]{sail_latex/sailsailfndecodeSomeCSealv.tex}}

\newcommand{\sailsailfndecodeSomeCUnsealv}{ \lstinputlisting[language=sail]{sail_latex/sailsailfndecodeSomeCUnsealv.tex}}

\newcommand{\sailsailfndecodeSomeCAndPermv}{ \lstinputlisting[language=sail]{sail_latex/sailsailfndecodeSomeCAndPermv.tex}}

\newcommand{\sailsailfndecodeSomeCSetOffsetv}{ \lstinputlisting[language=sail]{sail_latex/sailsailfndecodeSomeCSetOffsetv.tex}}

\newcommand{\sailsailfndecodeSomeCSetBoundsv}{ \lstinputlisting[language=sail]{sail_latex/sailsailfndecodeSomeCSetBoundsv.tex}}

\newcommand{\sailfndecodeSomeCSetBoundsExact}{ \lstinputlisting[language=sail]{sail_latex/sailfndecodeSomeCSetBoundsExact.tex}}

\newcommand{\sailsailfndecodeSomeCIncOffsetv}{ \lstinputlisting[language=sail]{sail_latex/sailsailfndecodeSomeCIncOffsetv.tex}}

\newcommand{\sailfndecodeSomeCBuildCap}{ \lstinputlisting[language=sail]{sail_latex/sailfndecodeSomeCBuildCap.tex}}

\newcommand{\sailfndecodeSomeCCopyType}{ \lstinputlisting[language=sail]{sail_latex/sailfndecodeSomeCCopyType.tex}}

\newcommand{\sailfndecodeSomeCCSeal}{ \lstinputlisting[language=sail]{sail_latex/sailfndecodeSomeCCSeal.tex}}

\newcommand{\sailsailfndecodeSomeCToPtrv}{ \lstinputlisting[language=sail]{sail_latex/sailsailfndecodeSomeCToPtrv.tex}}

\newcommand{\sailsailfndecodeSomeCFromPtrv}{ \lstinputlisting[language=sail]{sail_latex/sailsailfndecodeSomeCFromPtrv.tex}}

\newcommand{\sailfndecodeSomeCSub}{ \lstinputlisting[language=sail]{sail_latex/sailfndecodeSomeCSub.tex}}

\newcommand{\sailsailfndecodeSomeCMOVXv}{ \lstinputlisting[language=sail]{sail_latex/sailsailfndecodeSomeCMOVXv.tex}}

\newcommand{\sailsailsailfndecodeSomeCMOVXvv}{ \lstinputlisting[language=sail]{sail_latex/sailsailsailfndecodeSomeCMOVXvv.tex}}

\newcommand{\sailsailsailsailsailsailsailsailsailfndecodeSomeCPtrCmpvvvvvvvv}{ \lstinputlisting[language=sail]{sail_latex/sailsailsailsailsailsailsailsailsailfndecodeSomeCPtrCmpvvvvvvvv.tex}}

\newcommand{\sailsailsailsailsailsailsailsailsailsailfndecodeSomeCPtrCmpvvvvvvvvv}{ \lstinputlisting[language=sail]{sail_latex/sailsailsailsailsailsailsailsailsailsailfndecodeSomeCPtrCmpvvvvvvvvv.tex}}

\newcommand{\sailsailsailsailsailsailsailsailsailsailsailfndecodeSomeCPtrCmpvvvvvvvvvv}{ \lstinputlisting[language=sail]{sail_latex/sailsailsailsailsailsailsailsailsailsailsailfndecodeSomeCPtrCmpvvvvvvvvvv.tex}}

\newcommand{\sailsailsailsailsailsailsailsailsailsailsailsailfndecodeSomeCPtrCmpvvvvvvvvvvv}{ \lstinputlisting[language=sail]{sail_latex/sailsailsailsailsailsailsailsailsailsailsailsailfndecodeSomeCPtrCmpvvvvvvvvvvv.tex}}

\newcommand{\sailsailsailsailsailsailsailsailsailsailsailsailsailfndecodeSomeCPtrCmpvvvvvvvvvvvv}{ \lstinputlisting[language=sail]{sail_latex/sailsailsailsailsailsailsailsailsailsailsailsailsailfndecodeSomeCPtrCmpvvvvvvvvvvvv.tex}}

\newcommand{\sailsailsailsailsailsailsailsailsailsailsailsailsailsailfndecodeSomeCPtrCmpvvvvvvvvvvvvv}{ \lstinputlisting[language=sail]{sail_latex/sailsailsailsailsailsailsailsailsailsailsailsailsailsailfndecodeSomeCPtrCmpvvvvvvvvvvvvv.tex}}

\newcommand{\sailsailsailsailsailsailsailsailsailsailsailsailsailsailsailfndecodeSomeCPtrCmpvvvvvvvvvvvvvv}{ \lstinputlisting[language=sail]{sail_latex/sailsailsailsailsailsailsailsailsailsailsailsailsailsailsailfndecodeSomeCPtrCmpvvvvvvvvvvvvvv.tex}}

\newcommand{\sailsailsailsailsailsailsailsailsailsailsailsailsailsailsailsailfndecodeSomeCPtrCmpvvvvvvvvvvvvvvv}{ \lstinputlisting[language=sail]{sail_latex/sailsailsailsailsailsailsailsailsailsailsailsailsailsailsailsailfndecodeSomeCPtrCmpvvvvvvvvvvvvvvv.tex}}

\newcommand{\sailfndecodeSomeCTestSubset}{ \lstinputlisting[language=sail]{sail_latex/sailfndecodeSomeCTestSubset.tex}}

\newcommand{\sailfndecodeSomeCBX}{ \lstinputlisting[language=sail]{sail_latex/sailfndecodeSomeCBX.tex}}

\newcommand{\sailsailfndecodeSomeCBXv}{ \lstinputlisting[language=sail]{sail_latex/sailsailfndecodeSomeCBXv.tex}}

\newcommand{\sailfndecodeSomeCBZ}{ \lstinputlisting[language=sail]{sail_latex/sailfndecodeSomeCBZ.tex}}

\newcommand{\sailsailfndecodeSomeCBZv}{ \lstinputlisting[language=sail]{sail_latex/sailsailfndecodeSomeCBZv.tex}}

\newcommand{\sailsailfndecodeSomeCReturnv}{ \lstinputlisting[language=sail]{sail_latex/sailsailfndecodeSomeCReturnv.tex}}

\newcommand{\sailfndecodeSomeCCall}{ \lstinputlisting[language=sail]{sail_latex/sailfndecodeSomeCCall.tex}}

\newcommand{\sailfndecodeSomeClearRegs}{ \lstinputlisting[language=sail]{sail_latex/sailfndecodeSomeClearRegs.tex}}

\newcommand{\sailsailfndecodeSomeClearRegsv}{ \lstinputlisting[language=sail]{sail_latex/sailsailfndecodeSomeClearRegsv.tex}}

\newcommand{\sailsailsailfndecodeSomeClearRegsvv}{ \lstinputlisting[language=sail]{sail_latex/sailsailsailfndecodeSomeClearRegsvv.tex}}

\newcommand{\sailsailsailsailfndecodeSomeClearRegsvvv}{ \lstinputlisting[language=sail]{sail_latex/sailsailsailsailfndecodeSomeClearRegsvvv.tex}}

\newcommand{\sailfndecodeSomeCIncOffsetImmediate}{ \lstinputlisting[language=sail]{sail_latex/sailfndecodeSomeCIncOffsetImmediate.tex}}

\newcommand{\sailfndecodeSomeCSetBoundsImmediate}{ \lstinputlisting[language=sail]{sail_latex/sailfndecodeSomeCSetBoundsImmediate.tex}}

\newcommand{\sailfndecodeSomeCLoad}{ \lstinputlisting[language=sail]{sail_latex/sailfndecodeSomeCLoad.tex}}

\newcommand{\sailsailfndecodeSomeCLoadv}{ \lstinputlisting[language=sail]{sail_latex/sailsailfndecodeSomeCLoadv.tex}}

\newcommand{\sailsailsailfndecodeSomeCLoadvv}{ \lstinputlisting[language=sail]{sail_latex/sailsailsailfndecodeSomeCLoadvv.tex}}

\newcommand{\sailsailsailsailfndecodeSomeCLoadvvv}{ \lstinputlisting[language=sail]{sail_latex/sailsailsailsailfndecodeSomeCLoadvvv.tex}}

\newcommand{\sailsailsailsailsailfndecodeSomeCLoadvvvv}{ \lstinputlisting[language=sail]{sail_latex/sailsailsailsailsailfndecodeSomeCLoadvvvv.tex}}

\newcommand{\sailsailsailsailsailsailfndecodeSomeCLoadvvvvv}{ \lstinputlisting[language=sail]{sail_latex/sailsailsailsailsailsailfndecodeSomeCLoadvvvvv.tex}}

\newcommand{\sailsailsailsailsailsailsailfndecodeSomeCLoadvvvvvv}{ \lstinputlisting[language=sail]{sail_latex/sailsailsailsailsailsailsailfndecodeSomeCLoadvvvvvv.tex}}

\newcommand{\sailsailsailsailsailsailsailsailfndecodeSomeCLoadvvvvvvv}{ \lstinputlisting[language=sail]{sail_latex/sailsailsailsailsailsailsailsailfndecodeSomeCLoadvvvvvvv.tex}}

\newcommand{\sailsailsailsailsailsailsailsailsailfndecodeSomeCLoadvvvvvvvv}{ \lstinputlisting[language=sail]{sail_latex/sailsailsailsailsailsailsailsailsailfndecodeSomeCLoadvvvvvvvv.tex}}

\newcommand{\sailsailsailsailsailsailsailsailsailsailfndecodeSomeCLoadvvvvvvvvv}{ \lstinputlisting[language=sail]{sail_latex/sailsailsailsailsailsailsailsailsailsailfndecodeSomeCLoadvvvvvvvvv.tex}}

\newcommand{\sailsailsailsailsailsailsailsailsailsailsailfndecodeSomeCLoadvvvvvvvvvv}{ \lstinputlisting[language=sail]{sail_latex/sailsailsailsailsailsailsailsailsailsailsailfndecodeSomeCLoadvvvvvvvvvv.tex}}

\newcommand{\sailsailsailsailsailsailsailsailsailsailsailsailfndecodeSomeCLoadvvvvvvvvvvv}{ \lstinputlisting[language=sail]{sail_latex/sailsailsailsailsailsailsailsailsailsailsailsailfndecodeSomeCLoadvvvvvvvvvvv.tex}}

\newcommand{\sailsailsailsailsailsailsailsailsailsailsailsailsailfndecodeSomeCLoadvvvvvvvvvvvv}{ \lstinputlisting[language=sail]{sail_latex/sailsailsailsailsailsailsailsailsailsailsailsailsailfndecodeSomeCLoadvvvvvvvvvvvv.tex}}

\newcommand{\sailsailsailsailsailsailsailsailsailsailsailsailsailsailfndecodeSomeCLoadvvvvvvvvvvvvv}{ \lstinputlisting[language=sail]{sail_latex/sailsailsailsailsailsailsailsailsailsailsailsailsailsailfndecodeSomeCLoadvvvvvvvvvvvvv.tex}}

\newcommand{\sailfndecodeSomeCStore}{ \lstinputlisting[language=sail]{sail_latex/sailfndecodeSomeCStore.tex}}

\newcommand{\sailsailfndecodeSomeCStorev}{ \lstinputlisting[language=sail]{sail_latex/sailsailfndecodeSomeCStorev.tex}}

\newcommand{\sailsailsailfndecodeSomeCStorevv}{ \lstinputlisting[language=sail]{sail_latex/sailsailsailfndecodeSomeCStorevv.tex}}

\newcommand{\sailsailsailsailfndecodeSomeCStorevvv}{ \lstinputlisting[language=sail]{sail_latex/sailsailsailsailfndecodeSomeCStorevvv.tex}}

\newcommand{\sailsailsailsailsailfndecodeSomeCStorevvvv}{ \lstinputlisting[language=sail]{sail_latex/sailsailsailsailsailfndecodeSomeCStorevvvv.tex}}

\newcommand{\sailsailsailsailsailsailfndecodeSomeCStorevvvvv}{ \lstinputlisting[language=sail]{sail_latex/sailsailsailsailsailsailfndecodeSomeCStorevvvvv.tex}}

\newcommand{\sailsailsailsailsailsailsailfndecodeSomeCStorevvvvvv}{ \lstinputlisting[language=sail]{sail_latex/sailsailsailsailsailsailsailfndecodeSomeCStorevvvvvv.tex}}

\newcommand{\sailsailsailsailsailsailsailsailfndecodeSomeCStorevvvvvvv}{ \lstinputlisting[language=sail]{sail_latex/sailsailsailsailsailsailsailsailfndecodeSomeCStorevvvvvvv.tex}}

\newcommand{\sailfndecodeSomeCSC}{ \lstinputlisting[language=sail]{sail_latex/sailfndecodeSomeCSC.tex}}

\newcommand{\sailsailfndecodeSomeCSCv}{ \lstinputlisting[language=sail]{sail_latex/sailsailfndecodeSomeCSCv.tex}}

\newcommand{\sailfndecodeSomeCLC}{ \lstinputlisting[language=sail]{sail_latex/sailfndecodeSomeCLC.tex}}

\newcommand{\sailsailfndecodeSomeCLCv}{ \lstinputlisting[language=sail]{sail_latex/sailsailfndecodeSomeCLCv.tex}}

\newcommand{\sailsailsailfndecodeSomeCLCvv}{ \lstinputlisting[language=sail]{sail_latex/sailsailsailfndecodeSomeCLCvv.tex}}

\newcommand{\sailfndecodeSomeCtwoDumprt}{ \lstinputlisting[language=sail]{sail_latex/sailfndecodeSomeCtwoDumprt.tex}}

\newcommand{\sailfndecodeSomeRI}{ \lstinputlisting[language=sail]{sail_latex/sailfndecodeSomeRI.tex}}



\newcommand{\sailfnexecuteDADDIU}{ \lstinputlisting[language=sail]{sail_latex/sailfnexecuteDADDIU.tex}}

\newcommand{\sailfnexecuteDADDU}{ \lstinputlisting[language=sail]{sail_latex/sailfnexecuteDADDU.tex}}

\newcommand{\sailfnexecuteDADDI}{ \lstinputlisting[language=sail]{sail_latex/sailfnexecuteDADDI.tex}}

\newcommand{\sailfnexecuteDADD}{ \lstinputlisting[language=sail]{sail_latex/sailfnexecuteDADD.tex}}

\newcommand{\sailfnexecuteADD}{ \lstinputlisting[language=sail]{sail_latex/sailfnexecuteADD.tex}}

\newcommand{\sailfnexecuteADDI}{ \lstinputlisting[language=sail]{sail_latex/sailfnexecuteADDI.tex}}

\newcommand{\sailfnexecuteADDU}{ \lstinputlisting[language=sail]{sail_latex/sailfnexecuteADDU.tex}}

\newcommand{\sailfnexecuteADDIU}{ \lstinputlisting[language=sail]{sail_latex/sailfnexecuteADDIU.tex}}

\newcommand{\sailfnexecuteDSUBU}{ \lstinputlisting[language=sail]{sail_latex/sailfnexecuteDSUBU.tex}}

\newcommand{\sailfnexecuteDSUB}{ \lstinputlisting[language=sail]{sail_latex/sailfnexecuteDSUB.tex}}

\newcommand{\sailfnexecuteSUB}{ \lstinputlisting[language=sail]{sail_latex/sailfnexecuteSUB.tex}}

\newcommand{\sailfnexecuteSUBU}{ \lstinputlisting[language=sail]{sail_latex/sailfnexecuteSUBU.tex}}

\newcommand{\sailfnexecuteAND}{ \lstinputlisting[language=sail]{sail_latex/sailfnexecuteAND.tex}}

\newcommand{\sailfnexecuteANDI}{ \lstinputlisting[language=sail]{sail_latex/sailfnexecuteANDI.tex}}

\newcommand{\sailfnexecuteOR}{ \lstinputlisting[language=sail]{sail_latex/sailfnexecuteOR.tex}}

\newcommand{\sailfnexecuteORI}{ \lstinputlisting[language=sail]{sail_latex/sailfnexecuteORI.tex}}

\newcommand{\sailfnexecuteNOR}{ \lstinputlisting[language=sail]{sail_latex/sailfnexecuteNOR.tex}}

\newcommand{\sailfnexecuteXOR}{ \lstinputlisting[language=sail]{sail_latex/sailfnexecuteXOR.tex}}

\newcommand{\sailfnexecuteXORI}{ \lstinputlisting[language=sail]{sail_latex/sailfnexecuteXORI.tex}}

\newcommand{\sailfnexecuteLUI}{ \lstinputlisting[language=sail]{sail_latex/sailfnexecuteLUI.tex}}

\newcommand{\sailfnexecuteDSLL}{ \lstinputlisting[language=sail]{sail_latex/sailfnexecuteDSLL.tex}}

\newcommand{\sailfnexecuteDSLLthreetwo}{ \lstinputlisting[language=sail]{sail_latex/sailfnexecuteDSLLthreetwo.tex}}

\newcommand{\sailfnexecuteDSLLV}{ \lstinputlisting[language=sail]{sail_latex/sailfnexecuteDSLLV.tex}}

\newcommand{\sailfnexecuteDSRA}{ \lstinputlisting[language=sail]{sail_latex/sailfnexecuteDSRA.tex}}

\newcommand{\sailfnexecuteDSRAthreetwo}{ \lstinputlisting[language=sail]{sail_latex/sailfnexecuteDSRAthreetwo.tex}}

\newcommand{\sailfnexecuteDSRAV}{ \lstinputlisting[language=sail]{sail_latex/sailfnexecuteDSRAV.tex}}

\newcommand{\sailfnexecuteDSRL}{ \lstinputlisting[language=sail]{sail_latex/sailfnexecuteDSRL.tex}}

\newcommand{\sailfnexecuteDSRLthreetwo}{ \lstinputlisting[language=sail]{sail_latex/sailfnexecuteDSRLthreetwo.tex}}

\newcommand{\sailfnexecuteDSRLV}{ \lstinputlisting[language=sail]{sail_latex/sailfnexecuteDSRLV.tex}}

\newcommand{\sailfnexecuteSLL}{ \lstinputlisting[language=sail]{sail_latex/sailfnexecuteSLL.tex}}

\newcommand{\sailfnexecuteSLLV}{ \lstinputlisting[language=sail]{sail_latex/sailfnexecuteSLLV.tex}}

\newcommand{\sailfnexecuteSRA}{ \lstinputlisting[language=sail]{sail_latex/sailfnexecuteSRA.tex}}

\newcommand{\sailfnexecuteSRAV}{ \lstinputlisting[language=sail]{sail_latex/sailfnexecuteSRAV.tex}}

\newcommand{\sailfnexecuteSRL}{ \lstinputlisting[language=sail]{sail_latex/sailfnexecuteSRL.tex}}

\newcommand{\sailfnexecuteSRLV}{ \lstinputlisting[language=sail]{sail_latex/sailfnexecuteSRLV.tex}}

\newcommand{\sailfnexecuteSLT}{ \lstinputlisting[language=sail]{sail_latex/sailfnexecuteSLT.tex}}

\newcommand{\sailfnexecuteSLTI}{ \lstinputlisting[language=sail]{sail_latex/sailfnexecuteSLTI.tex}}

\newcommand{\sailfnexecuteSLTU}{ \lstinputlisting[language=sail]{sail_latex/sailfnexecuteSLTU.tex}}

\newcommand{\sailfnexecuteSLTIU}{ \lstinputlisting[language=sail]{sail_latex/sailfnexecuteSLTIU.tex}}

\newcommand{\sailfnexecuteMOVN}{ \lstinputlisting[language=sail]{sail_latex/sailfnexecuteMOVN.tex}}

\newcommand{\sailfnexecuteMOVZ}{ \lstinputlisting[language=sail]{sail_latex/sailfnexecuteMOVZ.tex}}

\newcommand{\sailfnexecuteMFHI}{ \lstinputlisting[language=sail]{sail_latex/sailfnexecuteMFHI.tex}}

\newcommand{\sailfnexecuteMFLO}{ \lstinputlisting[language=sail]{sail_latex/sailfnexecuteMFLO.tex}}

\newcommand{\sailfnexecuteMTHI}{ \lstinputlisting[language=sail]{sail_latex/sailfnexecuteMTHI.tex}}

\newcommand{\sailfnexecuteMTLO}{ \lstinputlisting[language=sail]{sail_latex/sailfnexecuteMTLO.tex}}

\newcommand{\sailfnexecuteMUL}{ \lstinputlisting[language=sail]{sail_latex/sailfnexecuteMUL.tex}}

\newcommand{\sailfnexecuteMULT}{ \lstinputlisting[language=sail]{sail_latex/sailfnexecuteMULT.tex}}

\newcommand{\sailfnexecuteMULTU}{ \lstinputlisting[language=sail]{sail_latex/sailfnexecuteMULTU.tex}}

\newcommand{\sailfnexecuteDMULT}{ \lstinputlisting[language=sail]{sail_latex/sailfnexecuteDMULT.tex}}

\newcommand{\sailfnexecuteDMULTU}{ \lstinputlisting[language=sail]{sail_latex/sailfnexecuteDMULTU.tex}}

\newcommand{\sailfnexecuteMADD}{ \lstinputlisting[language=sail]{sail_latex/sailfnexecuteMADD.tex}}

\newcommand{\sailfnexecuteMADDU}{ \lstinputlisting[language=sail]{sail_latex/sailfnexecuteMADDU.tex}}

\newcommand{\sailfnexecuteMSUB}{ \lstinputlisting[language=sail]{sail_latex/sailfnexecuteMSUB.tex}}

\newcommand{\sailfnexecuteMSUBU}{ \lstinputlisting[language=sail]{sail_latex/sailfnexecuteMSUBU.tex}}

\newcommand{\sailfnexecuteDIV}{ \lstinputlisting[language=sail]{sail_latex/sailfnexecuteDIV.tex}}

\newcommand{\sailfnexecuteDIVU}{ \lstinputlisting[language=sail]{sail_latex/sailfnexecuteDIVU.tex}}

\newcommand{\sailfnexecuteDDIV}{ \lstinputlisting[language=sail]{sail_latex/sailfnexecuteDDIV.tex}}

\newcommand{\sailfnexecuteDDIVU}{ \lstinputlisting[language=sail]{sail_latex/sailfnexecuteDDIVU.tex}}

\newcommand{\sailfnexecuteJ}{ \lstinputlisting[language=sail]{sail_latex/sailfnexecuteJ.tex}}

\newcommand{\sailfnexecuteJAL}{ \lstinputlisting[language=sail]{sail_latex/sailfnexecuteJAL.tex}}

\newcommand{\sailfnexecuteJR}{ \lstinputlisting[language=sail]{sail_latex/sailfnexecuteJR.tex}}

\newcommand{\sailfnexecuteJALR}{ \lstinputlisting[language=sail]{sail_latex/sailfnexecuteJALR.tex}}

\newcommand{\sailfnexecuteBEQ}{ \lstinputlisting[language=sail]{sail_latex/sailfnexecuteBEQ.tex}}

\newcommand{\sailfnexecuteBCMPZ}{ \lstinputlisting[language=sail]{sail_latex/sailfnexecuteBCMPZ.tex}}

\newcommand{\sailfnexecuteSYSCALL}{ \lstinputlisting[language=sail]{sail_latex/sailfnexecuteSYSCALL.tex}}

\newcommand{\sailfnexecuteBREAK}{ \lstinputlisting[language=sail]{sail_latex/sailfnexecuteBREAK.tex}}

\newcommand{\sailfnexecuteWAIT}{ \lstinputlisting[language=sail]{sail_latex/sailfnexecuteWAIT.tex}}

\newcommand{\sailfnexecuteTRAPREG}{ \lstinputlisting[language=sail]{sail_latex/sailfnexecuteTRAPREG.tex}}

\newcommand{\sailfnexecuteTRAPIMM}{ \lstinputlisting[language=sail]{sail_latex/sailfnexecuteTRAPIMM.tex}}

\newcommand{\sailfnexecuteLoad}{ \lstinputlisting[language=sail]{sail_latex/sailfnexecuteLoad.tex}}

\newcommand{\sailfnexecuteStore}{ \lstinputlisting[language=sail]{sail_latex/sailfnexecuteStore.tex}}

\newcommand{\sailfnexecuteLWL}{ \lstinputlisting[language=sail]{sail_latex/sailfnexecuteLWL.tex}}

\newcommand{\sailfnexecuteLWR}{ \lstinputlisting[language=sail]{sail_latex/sailfnexecuteLWR.tex}}

\newcommand{\sailfnexecuteSWL}{ \lstinputlisting[language=sail]{sail_latex/sailfnexecuteSWL.tex}}

\newcommand{\sailfnexecuteSWR}{ \lstinputlisting[language=sail]{sail_latex/sailfnexecuteSWR.tex}}

\newcommand{\sailfnexecuteLDL}{ \lstinputlisting[language=sail]{sail_latex/sailfnexecuteLDL.tex}}

\newcommand{\sailfnexecuteLDR}{ \lstinputlisting[language=sail]{sail_latex/sailfnexecuteLDR.tex}}

\newcommand{\sailfnexecuteSDL}{ \lstinputlisting[language=sail]{sail_latex/sailfnexecuteSDL.tex}}

\newcommand{\sailfnexecuteSDR}{ \lstinputlisting[language=sail]{sail_latex/sailfnexecuteSDR.tex}}

\newcommand{\sailfnexecuteCACHE}{ \lstinputlisting[language=sail]{sail_latex/sailfnexecuteCACHE.tex}}

\newcommand{\sailfnexecuteSYNC}{ \lstinputlisting[language=sail]{sail_latex/sailfnexecuteSYNC.tex}}

\newcommand{\sailfnexecuteMFCzero}{ \lstinputlisting[language=sail]{sail_latex/sailfnexecuteMFCzero.tex}}

\newcommand{\sailfnexecuteHCF}{ \lstinputlisting[language=sail]{sail_latex/sailfnexecuteHCF.tex}}

\newcommand{\sailfnexecuteMTCzero}{ \lstinputlisting[language=sail]{sail_latex/sailfnexecuteMTCzero.tex}}

\newcommand{\sailfnexecuteTLBWI}{ \lstinputlisting[language=sail]{sail_latex/sailfnexecuteTLBWI.tex}}

\newcommand{\sailfnexecuteTLBWR}{ \lstinputlisting[language=sail]{sail_latex/sailfnexecuteTLBWR.tex}}

\newcommand{\sailfnexecuteTLBR}{ \lstinputlisting[language=sail]{sail_latex/sailfnexecuteTLBR.tex}}

\newcommand{\sailfnexecuteTLBP}{ \lstinputlisting[language=sail]{sail_latex/sailfnexecuteTLBP.tex}}

\newcommand{\sailfnexecuteRDHWR}{ \lstinputlisting[language=sail]{sail_latex/sailfnexecuteRDHWR.tex}}

\newcommand{\sailfnexecuteERET}{ \lstinputlisting[language=sail]{sail_latex/sailfnexecuteERET.tex}}

\newcommand{\sailfnexecuteCGetPerm}{ \lstinputlisting[language=sail]{sail_latex/sailfnexecuteCGetPerm.tex}}

\newcommand{\sailfnexecuteCGetType}{ \lstinputlisting[language=sail]{sail_latex/sailfnexecuteCGetType.tex}}

\newcommand{\sailfnexecuteCGetBase}{ \lstinputlisting[language=sail]{sail_latex/sailfnexecuteCGetBase.tex}}

\newcommand{\sailfnexecuteCGetOffset}{ \lstinputlisting[language=sail]{sail_latex/sailfnexecuteCGetOffset.tex}}

\newcommand{\sailfnexecuteCGetLen}{ \lstinputlisting[language=sail]{sail_latex/sailfnexecuteCGetLen.tex}}

\newcommand{\sailfnexecuteCGetTag}{ \lstinputlisting[language=sail]{sail_latex/sailfnexecuteCGetTag.tex}}

\newcommand{\sailfnexecuteCGetSealed}{ \lstinputlisting[language=sail]{sail_latex/sailfnexecuteCGetSealed.tex}}

\newcommand{\sailfnexecuteCGetAddr}{ \lstinputlisting[language=sail]{sail_latex/sailfnexecuteCGetAddr.tex}}

\newcommand{\sailfnexecuteCGetPCC}{ \lstinputlisting[language=sail]{sail_latex/sailfnexecuteCGetPCC.tex}}

\newcommand{\sailfnexecuteCGetPCCSetOffset}{ \lstinputlisting[language=sail]{sail_latex/sailfnexecuteCGetPCCSetOffset.tex}}

\newcommand{\sailfnexecuteCGetCause}{ \lstinputlisting[language=sail]{sail_latex/sailfnexecuteCGetCause.tex}}

\newcommand{\sailfnexecuteCSetCause}{ \lstinputlisting[language=sail]{sail_latex/sailfnexecuteCSetCause.tex}}

\newcommand{\sailfnexecuteCReadHwr}{ \lstinputlisting[language=sail]{sail_latex/sailfnexecuteCReadHwr.tex}}

\newcommand{\sailfnexecuteCWriteHwr}{ \lstinputlisting[language=sail]{sail_latex/sailfnexecuteCWriteHwr.tex}}

\newcommand{\sailfnexecuteCAndPerm}{ \lstinputlisting[language=sail]{sail_latex/sailfnexecuteCAndPerm.tex}}

\newcommand{\sailfnexecuteCToPtr}{ \lstinputlisting[language=sail]{sail_latex/sailfnexecuteCToPtr.tex}}

\newcommand{\sailfnexecuteCSub}{ \lstinputlisting[language=sail]{sail_latex/sailfnexecuteCSub.tex}}

\newcommand{\sailfnexecuteCPtrCmp}{ \lstinputlisting[language=sail]{sail_latex/sailfnexecuteCPtrCmp.tex}}

\newcommand{\sailfnexecuteCIncOffset}{ \lstinputlisting[language=sail]{sail_latex/sailfnexecuteCIncOffset.tex}}

\newcommand{\sailfnexecuteCIncOffsetImmediate}{ \lstinputlisting[language=sail]{sail_latex/sailfnexecuteCIncOffsetImmediate.tex}}

\newcommand{\sailfnexecuteCSetOffset}{ \lstinputlisting[language=sail]{sail_latex/sailfnexecuteCSetOffset.tex}}

\newcommand{\sailfnexecuteCSetBounds}{ \lstinputlisting[language=sail]{sail_latex/sailfnexecuteCSetBounds.tex}}

\newcommand{\sailfnexecuteCSetBoundsImmediate}{ \lstinputlisting[language=sail]{sail_latex/sailfnexecuteCSetBoundsImmediate.tex}}

\newcommand{\sailfnexecuteCSetBoundsExact}{ \lstinputlisting[language=sail]{sail_latex/sailfnexecuteCSetBoundsExact.tex}}

\newcommand{\sailfnexecuteCClearTag}{ \lstinputlisting[language=sail]{sail_latex/sailfnexecuteCClearTag.tex}}

\newcommand{\sailfnexecuteCMOVX}{ \lstinputlisting[language=sail]{sail_latex/sailfnexecuteCMOVX.tex}}

\newcommand{\sailfnexecuteClearRegs}{ \lstinputlisting[language=sail]{sail_latex/sailfnexecuteClearRegs.tex}}

\newcommand{\sailfnexecuteCFromPtr}{ \lstinputlisting[language=sail]{sail_latex/sailfnexecuteCFromPtr.tex}}

\newcommand{\sailfnexecuteCBuildCap}{ \lstinputlisting[language=sail]{sail_latex/sailfnexecuteCBuildCap.tex}}

\newcommand{\sailfnexecuteCCopyType}{ \lstinputlisting[language=sail]{sail_latex/sailfnexecuteCCopyType.tex}}

\newcommand{\sailfnexecuteCCheckPerm}{ \lstinputlisting[language=sail]{sail_latex/sailfnexecuteCCheckPerm.tex}}

\newcommand{\sailfnexecuteCCheckType}{ \lstinputlisting[language=sail]{sail_latex/sailfnexecuteCCheckType.tex}}

\newcommand{\sailfnexecuteCTestSubset}{ \lstinputlisting[language=sail]{sail_latex/sailfnexecuteCTestSubset.tex}}

\newcommand{\sailfnexecuteCSeal}{ \lstinputlisting[language=sail]{sail_latex/sailfnexecuteCSeal.tex}}

\newcommand{\sailfnexecuteCCSeal}{ \lstinputlisting[language=sail]{sail_latex/sailfnexecuteCCSeal.tex}}

\newcommand{\sailfnexecuteCUnseal}{ \lstinputlisting[language=sail]{sail_latex/sailfnexecuteCUnseal.tex}}

\newcommand{\sailfnexecuteCCall}{ \lstinputlisting[language=sail]{sail_latex/sailfnexecuteCCall.tex}}

\newcommand{\sailsailfnexecuteCCallv}{ \lstinputlisting[language=sail]{sail_latex/sailsailfnexecuteCCallv.tex}}

\newcommand{\sailfnexecuteCReturn}{ \lstinputlisting[language=sail]{sail_latex/sailfnexecuteCReturn.tex}}

\newcommand{\sailfnexecuteCBX}{ \lstinputlisting[language=sail]{sail_latex/sailfnexecuteCBX.tex}}

\newcommand{\sailfnexecuteCBZ}{ \lstinputlisting[language=sail]{sail_latex/sailfnexecuteCBZ.tex}}

\newcommand{\sailfnexecuteCJALR}{ \lstinputlisting[language=sail]{sail_latex/sailfnexecuteCJALR.tex}}

\newcommand{\sailfnexecuteCLoad}{ \lstinputlisting[language=sail]{sail_latex/sailfnexecuteCLoad.tex}}

\newcommand{\sailfnexecuteCStore}{ \lstinputlisting[language=sail]{sail_latex/sailfnexecuteCStore.tex}}

\newcommand{\sailfnexecuteCSC}{ \lstinputlisting[language=sail]{sail_latex/sailfnexecuteCSC.tex}}

\newcommand{\sailfnexecuteCLC}{ \lstinputlisting[language=sail]{sail_latex/sailfnexecuteCLC.tex}}

\newcommand{\sailfnexecuteCtwoDump}{ \lstinputlisting[language=sail]{sail_latex/sailfnexecuteCtwoDump.tex}}

\newcommand{\sailfnexecuteRI}{ \lstinputlisting[language=sail]{sail_latex/sailfnexecuteRI.tex}}



\newcommand{\sailsupportedinstructions}{\label{zsupportedzyinstructions} \lstinputlisting[language=sail]{sail_latex/sailsupportedinstructions.tex}}

\newcommand{\sailfnsupportedinstructions}{\label{zsupportedzyinstructions} \lstinputlisting[language=sail]{sail_latex/sailfnsupportedinstructions.tex}}


  \sailval{my_function}
  \sailfn{my_function}
\end{lstlisting}
which would enclude the type declaration (\verb+\sailval+) for
\verb+my_function+ as well as type body of that function
(\verb+\sailfn+).

It is sometimes useful to include multiple versions of the same Sail
definitions in a latex document. In this case the \verb+-latex_prefix+
option can be used. For example if we used \verb+-latex_prefix prefix+
then the above example would become:
\begin{lstlisting}[language=TeX]
  \newcommand{\sailsailregderefv}{\label{zregzyderef} \lstinputlisting[language=sail]{sail_latex/sailsailregderefv.tex}}

\newcommand{\sailregderef}{\label{zzyregzyderef} \lstinputlisting[language=sail]{sail_latex/sailregderef.tex}}

\newcommand{\saileqbittwo}{\label{zeqzybittwo} \lstinputlisting[language=sail]{sail_latex/saileqbittwo.tex}}

\newcommand{\sailsailsailsailzeightoperatorzzerozJzJzninevvv}{\label{zzeightoperatorzzerozJzJznine} \lstinputlisting[language=sail]{sail_latex/sailsailsailsailzeightoperatorzzerozJzJzninevvv.tex}}

\newcommand{\saildiv}{\label{zdiv} \lstinputlisting[language=sail]{sail_latex/saildiv.tex}}

\newcommand{\sailsailsailzeightoperatorzzerozFzninevv}{\label{zzeightoperatorzzerozFznine} \lstinputlisting[language=sail]{sail_latex/sailsailsailzeightoperatorzzerozFzninevv.tex}}

\newcommand{\sailmod}{\label{zmod} \lstinputlisting[language=sail]{sail_latex/sailmod.tex}}

\newcommand{\sailsailsailzeightoperatorzzerozfivezninevv}{\label{zzeightoperatorzzerozfiveznine} \lstinputlisting[language=sail]{sail_latex/sailsailsailzeightoperatorzzerozfivezninevv.tex}}

\newcommand{\sailabsatom}{\label{zabszyatom} \lstinputlisting[language=sail]{sail_latex/sailabsatom.tex}}

\newcommand{\sailnotbool}{\label{znotzybool} \lstinputlisting[language=sail]{sail_latex/sailnotbool.tex}}

\newcommand{\sailandbool}{\label{zandzybool} \lstinputlisting[language=sail]{sail_latex/sailandbool.tex}}

\newcommand{\sailorbool}{\label{zorzybool} \lstinputlisting[language=sail]{sail_latex/sailorbool.tex}}

\newcommand{\saileqatom}{\label{zeqzyatom} \lstinputlisting[language=sail]{sail_latex/saileqatom.tex}}

\newcommand{\sailneqatom}{\label{zneqzyatom} \lstinputlisting[language=sail]{sail_latex/sailneqatom.tex}}

\newcommand{\sailfnneqatom}{\label{zneqzyatom} \lstinputlisting[language=sail]{sail_latex/sailfnneqatom.tex}}

\newcommand{\saillteqatom}{\label{zlteqzyatom} \lstinputlisting[language=sail]{sail_latex/saillteqatom.tex}}

\newcommand{\sailgteqatom}{\label{zgteqzyatom} \lstinputlisting[language=sail]{sail_latex/sailgteqatom.tex}}

\newcommand{\sailltatom}{\label{zltzyatom} \lstinputlisting[language=sail]{sail_latex/sailltatom.tex}}

\newcommand{\sailgtatom}{\label{zgtzyatom} \lstinputlisting[language=sail]{sail_latex/sailgtatom.tex}}

\newcommand{\sailltrangeatom}{\label{zltzyrangezyatom} \lstinputlisting[language=sail]{sail_latex/sailltrangeatom.tex}}

\newcommand{\saillteqrangeatom}{\label{zlteqzyrangezyatom} \lstinputlisting[language=sail]{sail_latex/saillteqrangeatom.tex}}

\newcommand{\sailgtrangeatom}{\label{zgtzyrangezyatom} \lstinputlisting[language=sail]{sail_latex/sailgtrangeatom.tex}}

\newcommand{\sailgteqrangeatom}{\label{zgteqzyrangezyatom} \lstinputlisting[language=sail]{sail_latex/sailgteqrangeatom.tex}}

\newcommand{\sailltatomrange}{\label{zltzyatomzyrange} \lstinputlisting[language=sail]{sail_latex/sailltatomrange.tex}}

\newcommand{\saillteqatomrange}{\label{zlteqzyatomzyrange} \lstinputlisting[language=sail]{sail_latex/saillteqatomrange.tex}}

\newcommand{\sailgtatomrange}{\label{zgtzyatomzyrange} \lstinputlisting[language=sail]{sail_latex/sailgtatomrange.tex}}

\newcommand{\sailgteqatomrange}{\label{zgteqzyatomzyrange} \lstinputlisting[language=sail]{sail_latex/sailgteqatomrange.tex}}

\newcommand{\saileqrange}{\label{zeqzyrange} \lstinputlisting[language=sail]{sail_latex/saileqrange.tex}}

\newcommand{\saileqint}{\label{zeqzyint} \lstinputlisting[language=sail]{sail_latex/saileqint.tex}}

\newcommand{\saileqbool}{\label{zeqzybool} \lstinputlisting[language=sail]{sail_latex/saileqbool.tex}}

\newcommand{\sailneqrange}{\label{zneqzyrange} \lstinputlisting[language=sail]{sail_latex/sailneqrange.tex}}

\newcommand{\sailfnneqrange}{\label{zneqzyrange} \lstinputlisting[language=sail]{sail_latex/sailfnneqrange.tex}}

\newcommand{\sailneqint}{\label{zneqzyint} \lstinputlisting[language=sail]{sail_latex/sailneqint.tex}}

\newcommand{\sailfnneqint}{\label{zneqzyint} \lstinputlisting[language=sail]{sail_latex/sailfnneqint.tex}}

\newcommand{\sailneqbool}{\label{zneqzybool} \lstinputlisting[language=sail]{sail_latex/sailneqbool.tex}}

\newcommand{\sailfnneqbool}{\label{zneqzybool} \lstinputlisting[language=sail]{sail_latex/sailfnneqbool.tex}}

\newcommand{\saillteqint}{\label{zlteqzyint} \lstinputlisting[language=sail]{sail_latex/saillteqint.tex}}

\newcommand{\sailgteqint}{\label{zgteqzyint} \lstinputlisting[language=sail]{sail_latex/sailgteqint.tex}}

\newcommand{\sailltint}{\label{zltzyint} \lstinputlisting[language=sail]{sail_latex/sailltint.tex}}

\newcommand{\sailgtint}{\label{zgtzyint} \lstinputlisting[language=sail]{sail_latex/sailgtint.tex}}

\newcommand{\sailsailsailzeightoperatorzzerozJzJzninevv}{\label{zzeightoperatorzzerozJzJznine} \lstinputlisting[language=sail]{sail_latex/sailsailsailzeightoperatorzzerozJzJzninevv.tex}}

\newcommand{\sailsailzeightoperatorzzerozonezJzninev}{\label{zzeightoperatorzzerozonezJznine} \lstinputlisting[language=sail]{sail_latex/sailsailzeightoperatorzzerozonezJzninev.tex}}

\newcommand{\sailsailzeightoperatorzzerozUzninev}{\label{zzeightoperatorzzerozUznine} \lstinputlisting[language=sail]{sail_latex/sailsailzeightoperatorzzerozUzninev.tex}}

\newcommand{\sailsailzeightoperatorzzerozsixzninev}{\label{zzeightoperatorzzerozsixznine} \lstinputlisting[language=sail]{sail_latex/sailsailzeightoperatorzzerozsixzninev.tex}}

\newcommand{\sailzeightoperatorzzerozIzJznine}{\label{zzeightoperatorzzerozIzJznine} \lstinputlisting[language=sail]{sail_latex/sailzeightoperatorzzerozIzJznine.tex}}

\newcommand{\sailzeightoperatorzzerozIznine}{\label{zzeightoperatorzzerozIznine} \lstinputlisting[language=sail]{sail_latex/sailzeightoperatorzzerozIznine.tex}}

\newcommand{\sailzeightoperatorzzerozKzJznine}{\label{zzeightoperatorzzerozKzJznine} \lstinputlisting[language=sail]{sail_latex/sailzeightoperatorzzerozKzJznine.tex}}

\newcommand{\sailzeightoperatorzzerozKznine}{\label{zzeightoperatorzzerozKznine} \lstinputlisting[language=sail]{sail_latex/sailzeightoperatorzzerozKznine.tex}}

\newcommand{\sailaddatom}{\label{zaddzyatom} \lstinputlisting[language=sail]{sail_latex/sailaddatom.tex}}

\newcommand{\sailaddint}{\label{zaddzyint} \lstinputlisting[language=sail]{sail_latex/sailaddint.tex}}

\newcommand{\sailsailsailzeightoperatorzzerozBzninevv}{\label{zzeightoperatorzzerozBznine} \lstinputlisting[language=sail]{sail_latex/sailsailsailzeightoperatorzzerozBzninevv.tex}}

\newcommand{\sailsubatom}{\label{zsubzyatom} \lstinputlisting[language=sail]{sail_latex/sailsubatom.tex}}

\newcommand{\sailsubint}{\label{zsubzyint} \lstinputlisting[language=sail]{sail_latex/sailsubint.tex}}

\newcommand{\sailsailzeightoperatorzzerozDzninev}{\label{zzeightoperatorzzerozDznine} \lstinputlisting[language=sail]{sail_latex/sailsailzeightoperatorzzerozDzninev.tex}}

\newcommand{\sailnegateatom}{\label{znegatezyatom} \lstinputlisting[language=sail]{sail_latex/sailnegateatom.tex}}

\newcommand{\sailnegateint}{\label{znegatezyint} \lstinputlisting[language=sail]{sail_latex/sailnegateint.tex}}

\newcommand{\sailsailnegatev}{\label{znegate} \lstinputlisting[language=sail]{sail_latex/sailsailnegatev.tex}}

\newcommand{\sailmultatom}{\label{zmultzyatom} \lstinputlisting[language=sail]{sail_latex/sailmultatom.tex}}

\newcommand{\sailmultint}{\label{zmultzyint} \lstinputlisting[language=sail]{sail_latex/sailmultint.tex}}

\newcommand{\sailsailzeightoperatorzzerozAzninev}{\label{zzeightoperatorzzerozAznine} \lstinputlisting[language=sail]{sail_latex/sailsailzeightoperatorzzerozAzninev.tex}}

\newcommand{\sailprintint}{\label{zprintzyint} \lstinputlisting[language=sail]{sail_latex/sailprintint.tex}}

\newcommand{\sailprerrint}{\label{zprerrzyint} \lstinputlisting[language=sail]{sail_latex/sailprerrint.tex}}

\newcommand{\sailshlint}{\label{zshlzyint} \lstinputlisting[language=sail]{sail_latex/sailshlint.tex}}

\newcommand{\sailshrint}{\label{zshrzyint} \lstinputlisting[language=sail]{sail_latex/sailshrint.tex}}

\newcommand{\saildivint}{\label{zdivzyint} \lstinputlisting[language=sail]{sail_latex/saildivint.tex}}

\newcommand{\sailsailzeightoperatorzzerozFzninev}{\label{zzeightoperatorzzerozFznine} \lstinputlisting[language=sail]{sail_latex/sailsailzeightoperatorzzerozFzninev.tex}}

\newcommand{\sailmodint}{\label{zmodzyint} \lstinputlisting[language=sail]{sail_latex/sailmodint.tex}}

\newcommand{\sailsailzeightoperatorzzerozfivezninev}{\label{zzeightoperatorzzerozfiveznine} \lstinputlisting[language=sail]{sail_latex/sailsailzeightoperatorzzerozfivezninev.tex}}

\newcommand{\sailabsint}{\label{zabszyint} \lstinputlisting[language=sail]{sail_latex/sailabsint.tex}}

\newcommand{\sailisnone}{\label{ziszynone} \lstinputlisting[language=sail]{sail_latex/sailisnone.tex}}

\newcommand{\sailfnisnone}{\label{ziszynone} \lstinputlisting[language=sail]{sail_latex/sailfnisnone.tex}}

\newcommand{\sailissome}{\label{ziszysome} \lstinputlisting[language=sail]{sail_latex/sailissome.tex}}

\newcommand{\sailfnissome}{\label{ziszysome} \lstinputlisting[language=sail]{sail_latex/sailfnissome.tex}}

\newcommand{\sailbits}{\label{zbits} \lstinputlisting[language=sail]{sail_latex/sailbits.tex}}

\newcommand{\saileqbit}{\label{zeqzybit} \lstinputlisting[language=sail]{sail_latex/saileqbit.tex}}

\newcommand{\saileqbits}{\label{zeqzybits} \lstinputlisting[language=sail]{sail_latex/saileqbits.tex}}

\newcommand{\sailsailzeightoperatorzzerozJzJzninev}{\label{zzeightoperatorzzerozJzJznine} \lstinputlisting[language=sail]{sail_latex/sailsailzeightoperatorzzerozJzJzninev.tex}}

\newcommand{\sailbitvectorlength}{\label{zbitvectorzylength} \lstinputlisting[language=sail]{sail_latex/sailbitvectorlength.tex}}

\newcommand{\sailvectorlength}{\label{zvectorzylength} \lstinputlisting[language=sail]{sail_latex/sailvectorlength.tex}}

\newcommand{\saillength}{\label{zlength} \lstinputlisting[language=sail]{sail_latex/saillength.tex}}

\newcommand{\sailsailzzeros}{\label{zsailzyzzeros} \lstinputlisting[language=sail]{sail_latex/sailsailzzeros.tex}}

\newcommand{\sailprintbits}{\label{zprintzybits} \lstinputlisting[language=sail]{sail_latex/sailprintbits.tex}}

\newcommand{\sailprerrbits}{\label{zprerrzybits} \lstinputlisting[language=sail]{sail_latex/sailprerrbits.tex}}

\newcommand{\sailsailsignextend}{\label{zsailzysignzyextend} \lstinputlisting[language=sail]{sail_latex/sailsailsignextend.tex}}

\newcommand{\sailsailzzeroextend}{\label{zsailzyzzerozyextend} \lstinputlisting[language=sail]{sail_latex/sailsailzzeroextend.tex}}

\newcommand{\sailtruncate}{\label{ztruncate} \lstinputlisting[language=sail]{sail_latex/sailtruncate.tex}}

\newcommand{\sailsailmask}{\label{zsailzymask} \lstinputlisting[language=sail]{sail_latex/sailsailmask.tex}}

\newcommand{\sailfnsailmask}{\label{zsailzymask} \lstinputlisting[language=sail]{sail_latex/sailfnsailmask.tex}}

\newcommand{\sailsailzeightoperatorzzerozQzninev}{\label{zzeightoperatorzzerozQznine} \lstinputlisting[language=sail]{sail_latex/sailsailzeightoperatorzzerozQzninev.tex}}

\newcommand{\sailbitvectorconcat}{\label{zbitvectorzyconcat} \lstinputlisting[language=sail]{sail_latex/sailbitvectorconcat.tex}}

\newcommand{\sailappend}{\label{zappend} \lstinputlisting[language=sail]{sail_latex/sailappend.tex}}

\newcommand{\sailappendsixfour}{\label{zappendzysixfour} \lstinputlisting[language=sail]{sail_latex/sailappendsixfour.tex}}

\newcommand{\sailbitvectoraccess}{\label{zbitvectorzyaccess} \lstinputlisting[language=sail]{sail_latex/sailbitvectoraccess.tex}}

\newcommand{\sailplainvectoraccess}{\label{zplainzyvectorzyaccess} \lstinputlisting[language=sail]{sail_latex/sailplainvectoraccess.tex}}

\newcommand{\sailvectoraccess}{\label{zvectorzyaccess} \lstinputlisting[language=sail]{sail_latex/sailvectoraccess.tex}}

\newcommand{\sailbitvectorupdate}{\label{zbitvectorzyupdate} \lstinputlisting[language=sail]{sail_latex/sailbitvectorupdate.tex}}

\newcommand{\sailplainvectorupdate}{\label{zplainzyvectorzyupdate} \lstinputlisting[language=sail]{sail_latex/sailplainvectorupdate.tex}}

\newcommand{\sailvectorupdate}{\label{zvectorzyupdate} \lstinputlisting[language=sail]{sail_latex/sailvectorupdate.tex}}

\newcommand{\sailaddbits}{\label{zaddzybits} \lstinputlisting[language=sail]{sail_latex/sailaddbits.tex}}

\newcommand{\sailaddbitsint}{\label{zaddzybitszyint} \lstinputlisting[language=sail]{sail_latex/sailaddbitsint.tex}}

\newcommand{\sailsailzeightoperatorzzerozBzninev}{\label{zzeightoperatorzzerozBznine} \lstinputlisting[language=sail]{sail_latex/sailsailzeightoperatorzzerozBzninev.tex}}

\newcommand{\sailvectorsubrange}{\label{zvectorzysubrange} \lstinputlisting[language=sail]{sail_latex/sailvectorsubrange.tex}}

\newcommand{\sailvectorupdatesubrange}{\label{zvectorzyupdatezysubrange} \lstinputlisting[language=sail]{sail_latex/sailvectorupdatesubrange.tex}}

\newcommand{\sailgetsliceint}{\label{zgetzyslicezyint} \lstinputlisting[language=sail]{sail_latex/sailgetsliceint.tex}}

\newcommand{\sailsetsliceint}{\label{zsetzyslicezyint} \lstinputlisting[language=sail]{sail_latex/sailsetsliceint.tex}}

\newcommand{\sailsetslicebits}{\label{zsetzyslicezybits} \lstinputlisting[language=sail]{sail_latex/sailsetslicebits.tex}}

\newcommand{\sailslice}{\label{zslice} \lstinputlisting[language=sail]{sail_latex/sailslice.tex}}

\newcommand{\sailreplicatebits}{\label{zreplicatezybits} \lstinputlisting[language=sail]{sail_latex/sailreplicatebits.tex}}

\newcommand{\sailunsigned}{\label{zunsigned} \lstinputlisting[language=sail]{sail_latex/sailunsigned.tex}}

\newcommand{\sailsigned}{\label{zsigned} \lstinputlisting[language=sail]{sail_latex/sailsigned.tex}}

\newcommand{\saileqanything}{\label{zeqzyanything} \lstinputlisting[language=sail]{sail_latex/saileqanything.tex}}

\newcommand{\sailzeightoperatorzzerozJzJznine}{\label{zzeightoperatorzzerozJzJznine} \lstinputlisting[language=sail]{sail_latex/sailzeightoperatorzzerozJzJznine.tex}}

\newcommand{\sailnotvec}{\label{znotzyvec} \lstinputlisting[language=sail]{sail_latex/sailnotvec.tex}}

\newcommand{\sailzW}{\label{zzW} \lstinputlisting[language=sail]{sail_latex/sailzW.tex}}

\newcommand{\sailnot}{\label{znot} \lstinputlisting[language=sail]{sail_latex/sailnot.tex}}

\newcommand{\sailneqvec}{\label{zneqzyvec} \lstinputlisting[language=sail]{sail_latex/sailneqvec.tex}}

\newcommand{\sailfnneqvec}{\label{zneqzyvec} \lstinputlisting[language=sail]{sail_latex/sailfnneqvec.tex}}

\newcommand{\sailneqanything}{\label{zneqzyanything} \lstinputlisting[language=sail]{sail_latex/sailneqanything.tex}}

\newcommand{\sailfnneqanything}{\label{zneqzyanything} \lstinputlisting[language=sail]{sail_latex/sailfnneqanything.tex}}

\newcommand{\sailzeightoperatorzzerozonezJznine}{\label{zzeightoperatorzzerozonezJznine} \lstinputlisting[language=sail]{sail_latex/sailzeightoperatorzzerozonezJznine.tex}}

\newcommand{\sailandbits}{\label{zandzybits} \lstinputlisting[language=sail]{sail_latex/sailandbits.tex}}

\newcommand{\sailzeightoperatorzzerozsixznine}{\label{zzeightoperatorzzerozsixznine} \lstinputlisting[language=sail]{sail_latex/sailzeightoperatorzzerozsixznine.tex}}

\newcommand{\sailorbits}{\label{zorzybits} \lstinputlisting[language=sail]{sail_latex/sailorbits.tex}}

\newcommand{\sailzeightoperatorzzerozUznine}{\label{zzeightoperatorzzerozUznine} \lstinputlisting[language=sail]{sail_latex/sailzeightoperatorzzerozUznine.tex}}

\newcommand{\sailcastunitvec}{\label{zcastzyunitzyvec} \lstinputlisting[language=sail]{sail_latex/sailcastunitvec.tex}}

\newcommand{\sailfncastunitvec}{\label{zcastzyunitzyvec} \lstinputlisting[language=sail]{sail_latex/sailfncastunitvec.tex}}

\newcommand{\sailprint}{\label{zprint} \lstinputlisting[language=sail]{sail_latex/sailprint.tex}}

\newcommand{\sailprerrendline}{\label{zprerrzyendline} \lstinputlisting[language=sail]{sail_latex/sailprerrendline.tex}}

\newcommand{\sailprerrstring}{\label{zprerrzystring} \lstinputlisting[language=sail]{sail_latex/sailprerrstring.tex}}

\newcommand{\sailputchar}{\label{zputchar} \lstinputlisting[language=sail]{sail_latex/sailputchar.tex}}

\newcommand{\sailconcatstr}{\label{zconcatzystr} \lstinputlisting[language=sail]{sail_latex/sailconcatstr.tex}}

\newcommand{\sailstringofint}{\label{zstringzyofzyint} \lstinputlisting[language=sail]{sail_latex/sailstringofint.tex}}

\newcommand{\sailBitStr}{\label{zBitStr} \lstinputlisting[language=sail]{sail_latex/sailBitStr.tex}}

\newcommand{\sailxorvec}{\label{zxorzyvec} \lstinputlisting[language=sail]{sail_latex/sailxorvec.tex}}

\newcommand{\sailintpower}{\label{zintzypower} \lstinputlisting[language=sail]{sail_latex/sailintpower.tex}}

\newcommand{\sailzeightoperatorzzerozQznine}{\label{zzeightoperatorzzerozQznine} \lstinputlisting[language=sail]{sail_latex/sailzeightoperatorzzerozQznine.tex}}

\newcommand{\sailaddrange}{\label{zaddzyrange} \lstinputlisting[language=sail]{sail_latex/sailaddrange.tex}}

\newcommand{\sailaddvec}{\label{zaddzyvec} \lstinputlisting[language=sail]{sail_latex/sailaddvec.tex}}

\newcommand{\sailaddvecint}{\label{zaddzyveczyint} \lstinputlisting[language=sail]{sail_latex/sailaddvecint.tex}}

\newcommand{\sailzeightoperatorzzerozBznine}{\label{zzeightoperatorzzerozBznine} \lstinputlisting[language=sail]{sail_latex/sailzeightoperatorzzerozBznine.tex}}

\newcommand{\sailsubrange}{\label{zsubzyrange} \lstinputlisting[language=sail]{sail_latex/sailsubrange.tex}}

\newcommand{\sailsubvec}{\label{zsubzyvec} \lstinputlisting[language=sail]{sail_latex/sailsubvec.tex}}

\newcommand{\sailsubvecint}{\label{zsubzyveczyint} \lstinputlisting[language=sail]{sail_latex/sailsubvecint.tex}}

\newcommand{\sailnegaterange}{\label{znegatezyrange} \lstinputlisting[language=sail]{sail_latex/sailnegaterange.tex}}

\newcommand{\sailzeightoperatorzzerozDznine}{\label{zzeightoperatorzzerozDznine} \lstinputlisting[language=sail]{sail_latex/sailzeightoperatorzzerozDznine.tex}}

\newcommand{\sailnegate}{\label{znegate} \lstinputlisting[language=sail]{sail_latex/sailnegate.tex}}

\newcommand{\sailzeightoperatorzzerozAznine}{\label{zzeightoperatorzzerozAznine} \lstinputlisting[language=sail]{sail_latex/sailzeightoperatorzzerozAznine.tex}}

\newcommand{\sailquotientnat}{\label{zquotientzynat} \lstinputlisting[language=sail]{sail_latex/sailquotientnat.tex}}

\newcommand{\sailquotient}{\label{zquotient} \lstinputlisting[language=sail]{sail_latex/sailquotient.tex}}

\newcommand{\sailzeightoperatorzzerozFznine}{\label{zzeightoperatorzzerozFznine} \lstinputlisting[language=sail]{sail_latex/sailzeightoperatorzzerozFznine.tex}}

\newcommand{\sailquotroundzzero}{\label{zquotzyroundzyzzero} \lstinputlisting[language=sail]{sail_latex/sailquotroundzzero.tex}}

\newcommand{\sailremroundzzero}{\label{zremzyroundzyzzero} \lstinputlisting[language=sail]{sail_latex/sailremroundzzero.tex}}

\newcommand{\sailmodulus}{\label{zmodulus} \lstinputlisting[language=sail]{sail_latex/sailmodulus.tex}}

\newcommand{\sailzeightoperatorzzerozfiveznine}{\label{zzeightoperatorzzerozfiveznine} \lstinputlisting[language=sail]{sail_latex/sailzeightoperatorzzerozfiveznine.tex}}

\newcommand{\sailminnat}{\label{zminzynat} \lstinputlisting[language=sail]{sail_latex/sailminnat.tex}}

\newcommand{\sailminint}{\label{zminzyint} \lstinputlisting[language=sail]{sail_latex/sailminint.tex}}

\newcommand{\sailmaxnat}{\label{zmaxzynat} \lstinputlisting[language=sail]{sail_latex/sailmaxnat.tex}}

\newcommand{\sailmaxint}{\label{zmaxzyint} \lstinputlisting[language=sail]{sail_latex/sailmaxint.tex}}

\newcommand{\sailminatom}{\label{zminzyatom} \lstinputlisting[language=sail]{sail_latex/sailminatom.tex}}

\newcommand{\sailmaxatom}{\label{zmaxzyatom} \lstinputlisting[language=sail]{sail_latex/sailmaxatom.tex}}

\newcommand{\sailmin}{\label{zmin} \lstinputlisting[language=sail]{sail_latex/sailmin.tex}}

\newcommand{\sailsailmaxv}{\label{zmax} \lstinputlisting[language=sail]{sail_latex/sailsailmaxv.tex}}

\newcommand{\sailWriteRAM}{\label{zzyzyWriteRAM} \lstinputlisting[language=sail]{sail_latex/sailWriteRAM.tex}}

\newcommand{\sailMIPSwrite}{\label{zzyzyMIPSzywrite} \lstinputlisting[language=sail]{sail_latex/sailMIPSwrite.tex}}

\newcommand{\sailfnMIPSwrite}{\label{zzyzyMIPSzywrite} \lstinputlisting[language=sail]{sail_latex/sailfnMIPSwrite.tex}}

\newcommand{\sailReadRAM}{\label{zzyzyReadRAM} \lstinputlisting[language=sail]{sail_latex/sailReadRAM.tex}}

\newcommand{\sailMIPSread}{\label{zzyzyMIPSzyread} \lstinputlisting[language=sail]{sail_latex/sailMIPSread.tex}}

\newcommand{\sailfnMIPSread}{\label{zzyzyMIPSzyread} \lstinputlisting[language=sail]{sail_latex/sailfnMIPSread.tex}}

\newcommand{\sailzeightoperatorzzerozQzQznine}{\label{zzeightoperatorzzerozQzQznine} \lstinputlisting[language=sail]{sail_latex/sailzeightoperatorzzerozQzQznine.tex}}

\newcommand{\sailfnzeightoperatorzzerozQzQznine}{\label{zzeightoperatorzzerozQzQznine} \lstinputlisting[language=sail]{sail_latex/sailfnzeightoperatorzzerozQzQznine.tex}}

\newcommand{\sailpowtwo}{\label{zpowtwo} \lstinputlisting[language=sail]{sail_latex/sailpowtwo.tex}}

\newcommand{\sailmipssignextend}{\label{zmipszysignzyextend} \lstinputlisting[language=sail]{sail_latex/sailmipssignextend.tex}}

\newcommand{\sailmipszzeroextend}{\label{zmipszyzzerozyextend} \lstinputlisting[language=sail]{sail_latex/sailmipszzeroextend.tex}}

\newcommand{\sailfnmipssignextend}{\label{zmipszysignzyextend} \lstinputlisting[language=sail]{sail_latex/sailfnmipssignextend.tex}}

\newcommand{\sailfnmipszzeroextend}{\label{zmipszyzzerozyextend} \lstinputlisting[language=sail]{sail_latex/sailfnmipszzeroextend.tex}}

\newcommand{\sailsignextend}{\label{zsignzyextend} \lstinputlisting[language=sail]{sail_latex/sailsignextend.tex}}

\newcommand{\sailzzeroextend}{\label{zzzerozyextend} \lstinputlisting[language=sail]{sail_latex/sailzzeroextend.tex}}

\newcommand{\sailzzeros}{\label{zzzeros} \lstinputlisting[language=sail]{sail_latex/sailzzeros.tex}}

\newcommand{\sailfnzzeros}{\label{zzzeros} \lstinputlisting[language=sail]{sail_latex/sailfnzzeros.tex}}

\newcommand{\sailones}{\label{zones} \lstinputlisting[language=sail]{sail_latex/sailones.tex}}

\newcommand{\sailfnones}{\label{zones} \lstinputlisting[language=sail]{sail_latex/sailfnones.tex}}

\newcommand{\sailzeightoperatorzzerozIsznine}{\label{zzeightoperatorzzerozIzysznine} \lstinputlisting[language=sail]{sail_latex/sailzeightoperatorzzerozIsznine.tex}}

\newcommand{\sailzeightoperatorzzerozKzJsznine}{\label{zzeightoperatorzzerozKzJzysznine} \lstinputlisting[language=sail]{sail_latex/sailzeightoperatorzzerozKzJsznine.tex}}

\newcommand{\sailzeightoperatorzzerozIuznine}{\label{zzeightoperatorzzerozIzyuznine} \lstinputlisting[language=sail]{sail_latex/sailzeightoperatorzzerozIuznine.tex}}

\newcommand{\sailzeightoperatorzzerozKzJuznine}{\label{zzeightoperatorzzerozKzJzyuznine} \lstinputlisting[language=sail]{sail_latex/sailzeightoperatorzzerozKzJuznine.tex}}

\newcommand{\sailfnzeightoperatorzzerozIsznine}{\label{zzeightoperatorzzerozIzysznine} \lstinputlisting[language=sail]{sail_latex/sailfnzeightoperatorzzerozIsznine.tex}}

\newcommand{\sailfnzeightoperatorzzerozKzJsznine}{\label{zzeightoperatorzzerozKzJzysznine} \lstinputlisting[language=sail]{sail_latex/sailfnzeightoperatorzzerozKzJsznine.tex}}

\newcommand{\sailfnzeightoperatorzzerozIuznine}{\label{zzeightoperatorzzerozIzyuznine} \lstinputlisting[language=sail]{sail_latex/sailfnzeightoperatorzzerozIuznine.tex}}

\newcommand{\sailfnzeightoperatorzzerozKzJuznine}{\label{zzeightoperatorzzerozKzJzyuznine} \lstinputlisting[language=sail]{sail_latex/sailfnzeightoperatorzzerozKzJuznine.tex}}

\newcommand{\sailbooltobits}{\label{zboolzytozybits} \lstinputlisting[language=sail]{sail_latex/sailbooltobits.tex}}

\newcommand{\sailfnbooltobits}{\label{zboolzytozybits} \lstinputlisting[language=sail]{sail_latex/sailfnbooltobits.tex}}

\newcommand{\sailbittobool}{\label{zbitzytozybool} \lstinputlisting[language=sail]{sail_latex/sailbittobool.tex}}

\newcommand{\sailfnbittobool}{\label{zbitzytozybool} \lstinputlisting[language=sail]{sail_latex/sailfnbittobool.tex}}

\newcommand{\sailbitstobool}{\label{zbitszytozybool} \lstinputlisting[language=sail]{sail_latex/sailbitstobool.tex}}

\newcommand{\sailfnbitstobool}{\label{zbitszytozybool} \lstinputlisting[language=sail]{sail_latex/sailfnbitstobool.tex}}

\newcommand{\sailshiftbitsright}{\label{zshiftzybitszyright} \lstinputlisting[language=sail]{sail_latex/sailshiftbitsright.tex}}

\newcommand{\sailshiftbitsleft}{\label{zshiftzybitszyleft} \lstinputlisting[language=sail]{sail_latex/sailshiftbitsleft.tex}}

\newcommand{\sailshiftl}{\label{zshiftl} \lstinputlisting[language=sail]{sail_latex/sailshiftl.tex}}

\newcommand{\sailshiftr}{\label{zshiftr} \lstinputlisting[language=sail]{sail_latex/sailshiftr.tex}}

\newcommand{\sailzeightoperatorzzerozKzKznine}{\label{zzeightoperatorzzerozKzKznine} \lstinputlisting[language=sail]{sail_latex/sailzeightoperatorzzerozKzKznine.tex}}

\newcommand{\sailzeightoperatorzzerozIzIznine}{\label{zzeightoperatorzzerozIzIznine} \lstinputlisting[language=sail]{sail_latex/sailzeightoperatorzzerozIzIznine.tex}}

\newcommand{\sailzeightoperatorzzerozKzKsznine}{\label{zzeightoperatorzzerozKzKzysznine} \lstinputlisting[language=sail]{sail_latex/sailzeightoperatorzzerozKzKsznine.tex}}

\newcommand{\sailzeightoperatorzzerozAsznine}{\label{zzeightoperatorzzerozAzysznine} \lstinputlisting[language=sail]{sail_latex/sailzeightoperatorzzerozAsznine.tex}}

\newcommand{\sailzeightoperatorzzerozAuznine}{\label{zzeightoperatorzzerozAzyuznine} \lstinputlisting[language=sail]{sail_latex/sailzeightoperatorzzerozAuznine.tex}}

\newcommand{\sailtobits}{\label{ztozybits} 
\function{to\_bits} converts an integer to a bit vector of given length. If the integer is negative a twos-complement representation is used. If the integer is too large (or too negative) to fit in the requested length then it is truncated to the least significant bits.
\lstinputlisting[language=sail]{sail_latex/sailtobits.tex}}

\newcommand{\sailfntobits}{\label{ztozybits} \lstinputlisting[language=sail]{sail_latex/sailfntobits.tex}}

\newcommand{\sailmask}{\label{zmask} \lstinputlisting[language=sail]{sail_latex/sailmask.tex}}

\newcommand{\sailfnmask}{\label{zmask} \lstinputlisting[language=sail]{sail_latex/sailfnmask.tex}}

\newcommand{\sailgettimens}{\label{zgetzytimezyns} \lstinputlisting[language=sail]{sail_latex/sailgettimens.tex}}

\newcommand{\sailCauseReg}{\label{zCauseReg} \lstinputlisting[language=sail]{sail_latex/sailCauseReg.tex}}

\newcommand{\sailMkCauseReg}{\label{zMkzyCauseReg} \lstinputlisting[language=sail]{sail_latex/sailMkCauseReg.tex}}

\newcommand{\sailfnMkCauseReg}{\label{zMkzyCauseReg} \lstinputlisting[language=sail]{sail_latex/sailfnMkCauseReg.tex}}

\newcommand{\sailgetCauseRegbits}{\label{zzygetzyCauseRegzybits} \lstinputlisting[language=sail]{sail_latex/sailgetCauseRegbits.tex}}

\newcommand{\sailfngetCauseRegbits}{\label{zzygetzyCauseRegzybits} \lstinputlisting[language=sail]{sail_latex/sailfngetCauseRegbits.tex}}

\newcommand{\sailsetCauseRegbits}{\label{zzysetzyCauseRegzybits} \lstinputlisting[language=sail]{sail_latex/sailsetCauseRegbits.tex}}

\newcommand{\sailfnsetCauseRegbits}{\label{zzysetzyCauseRegzybits} \lstinputlisting[language=sail]{sail_latex/sailfnsetCauseRegbits.tex}}

\newcommand{\sailupdateCauseRegbits}{\label{zzyupdatezyCauseRegzybits} \lstinputlisting[language=sail]{sail_latex/sailupdateCauseRegbits.tex}}

\newcommand{\sailfnupdateCauseRegbits}{\label{zzyupdatezyCauseRegzybits} \lstinputlisting[language=sail]{sail_latex/sailfnupdateCauseRegbits.tex}}

\newcommand{\sailsailsailsailsailsailsailsailupdatebitsvvvvvvv}{\label{zupdatezybits} \lstinputlisting[language=sail]{sail_latex/sailsailsailsailsailsailsailsailupdatebitsvvvvvvv.tex}}

\newcommand{\sailsailsailsailsailsailsailsailmodbitsvvvvvvv}{\label{zzymodzybits} \lstinputlisting[language=sail]{sail_latex/sailsailsailsailsailsailsailsailmodbitsvvvvvvv.tex}}

\newcommand{\sailgetCauseRegBD}{\label{zzygetzyCauseRegzyBD} \lstinputlisting[language=sail]{sail_latex/sailgetCauseRegBD.tex}}

\newcommand{\sailfngetCauseRegBD}{\label{zzygetzyCauseRegzyBD} \lstinputlisting[language=sail]{sail_latex/sailfngetCauseRegBD.tex}}

\newcommand{\sailsetCauseRegBD}{\label{zzysetzyCauseRegzyBD} \lstinputlisting[language=sail]{sail_latex/sailsetCauseRegBD.tex}}

\newcommand{\sailfnsetCauseRegBD}{\label{zzysetzyCauseRegzyBD} \lstinputlisting[language=sail]{sail_latex/sailfnsetCauseRegBD.tex}}

\newcommand{\sailupdateCauseRegBD}{\label{zzyupdatezyCauseRegzyBD} \lstinputlisting[language=sail]{sail_latex/sailupdateCauseRegBD.tex}}

\newcommand{\sailfnupdateCauseRegBD}{\label{zzyupdatezyCauseRegzyBD} \lstinputlisting[language=sail]{sail_latex/sailfnupdateCauseRegBD.tex}}

\newcommand{\sailupdateBD}{\label{zupdatezyBD} \lstinputlisting[language=sail]{sail_latex/sailupdateBD.tex}}

\newcommand{\sailmodBD}{\label{zzymodzyBD} \lstinputlisting[language=sail]{sail_latex/sailmodBD.tex}}

\newcommand{\sailgetCauseRegCE}{\label{zzygetzyCauseRegzyCE} \lstinputlisting[language=sail]{sail_latex/sailgetCauseRegCE.tex}}

\newcommand{\sailfngetCauseRegCE}{\label{zzygetzyCauseRegzyCE} \lstinputlisting[language=sail]{sail_latex/sailfngetCauseRegCE.tex}}

\newcommand{\sailsetCauseRegCE}{\label{zzysetzyCauseRegzyCE} \lstinputlisting[language=sail]{sail_latex/sailsetCauseRegCE.tex}}

\newcommand{\sailfnsetCauseRegCE}{\label{zzysetzyCauseRegzyCE} \lstinputlisting[language=sail]{sail_latex/sailfnsetCauseRegCE.tex}}

\newcommand{\sailupdateCauseRegCE}{\label{zzyupdatezyCauseRegzyCE} \lstinputlisting[language=sail]{sail_latex/sailupdateCauseRegCE.tex}}

\newcommand{\sailfnupdateCauseRegCE}{\label{zzyupdatezyCauseRegzyCE} \lstinputlisting[language=sail]{sail_latex/sailfnupdateCauseRegCE.tex}}

\newcommand{\sailupdateCE}{\label{zupdatezyCE} \lstinputlisting[language=sail]{sail_latex/sailupdateCE.tex}}

\newcommand{\sailmodCE}{\label{zzymodzyCE} \lstinputlisting[language=sail]{sail_latex/sailmodCE.tex}}

\newcommand{\sailgetCauseRegIV}{\label{zzygetzyCauseRegzyIV} \lstinputlisting[language=sail]{sail_latex/sailgetCauseRegIV.tex}}

\newcommand{\sailfngetCauseRegIV}{\label{zzygetzyCauseRegzyIV} \lstinputlisting[language=sail]{sail_latex/sailfngetCauseRegIV.tex}}

\newcommand{\sailsetCauseRegIV}{\label{zzysetzyCauseRegzyIV} \lstinputlisting[language=sail]{sail_latex/sailsetCauseRegIV.tex}}

\newcommand{\sailfnsetCauseRegIV}{\label{zzysetzyCauseRegzyIV} \lstinputlisting[language=sail]{sail_latex/sailfnsetCauseRegIV.tex}}

\newcommand{\sailupdateCauseRegIV}{\label{zzyupdatezyCauseRegzyIV} \lstinputlisting[language=sail]{sail_latex/sailupdateCauseRegIV.tex}}

\newcommand{\sailfnupdateCauseRegIV}{\label{zzyupdatezyCauseRegzyIV} \lstinputlisting[language=sail]{sail_latex/sailfnupdateCauseRegIV.tex}}

\newcommand{\sailupdateIV}{\label{zupdatezyIV} \lstinputlisting[language=sail]{sail_latex/sailupdateIV.tex}}

\newcommand{\sailmodIV}{\label{zzymodzyIV} \lstinputlisting[language=sail]{sail_latex/sailmodIV.tex}}

\newcommand{\sailgetCauseRegWP}{\label{zzygetzyCauseRegzyWP} \lstinputlisting[language=sail]{sail_latex/sailgetCauseRegWP.tex}}

\newcommand{\sailfngetCauseRegWP}{\label{zzygetzyCauseRegzyWP} \lstinputlisting[language=sail]{sail_latex/sailfngetCauseRegWP.tex}}

\newcommand{\sailsetCauseRegWP}{\label{zzysetzyCauseRegzyWP} \lstinputlisting[language=sail]{sail_latex/sailsetCauseRegWP.tex}}

\newcommand{\sailfnsetCauseRegWP}{\label{zzysetzyCauseRegzyWP} \lstinputlisting[language=sail]{sail_latex/sailfnsetCauseRegWP.tex}}

\newcommand{\sailupdateCauseRegWP}{\label{zzyupdatezyCauseRegzyWP} \lstinputlisting[language=sail]{sail_latex/sailupdateCauseRegWP.tex}}

\newcommand{\sailfnupdateCauseRegWP}{\label{zzyupdatezyCauseRegzyWP} \lstinputlisting[language=sail]{sail_latex/sailfnupdateCauseRegWP.tex}}

\newcommand{\sailupdateWP}{\label{zupdatezyWP} \lstinputlisting[language=sail]{sail_latex/sailupdateWP.tex}}

\newcommand{\sailmodWP}{\label{zzymodzyWP} \lstinputlisting[language=sail]{sail_latex/sailmodWP.tex}}

\newcommand{\sailgetCauseRegIP}{\label{zzygetzyCauseRegzyIP} \lstinputlisting[language=sail]{sail_latex/sailgetCauseRegIP.tex}}

\newcommand{\sailfngetCauseRegIP}{\label{zzygetzyCauseRegzyIP} \lstinputlisting[language=sail]{sail_latex/sailfngetCauseRegIP.tex}}

\newcommand{\sailsetCauseRegIP}{\label{zzysetzyCauseRegzyIP} \lstinputlisting[language=sail]{sail_latex/sailsetCauseRegIP.tex}}

\newcommand{\sailfnsetCauseRegIP}{\label{zzysetzyCauseRegzyIP} \lstinputlisting[language=sail]{sail_latex/sailfnsetCauseRegIP.tex}}

\newcommand{\sailupdateCauseRegIP}{\label{zzyupdatezyCauseRegzyIP} \lstinputlisting[language=sail]{sail_latex/sailupdateCauseRegIP.tex}}

\newcommand{\sailfnupdateCauseRegIP}{\label{zzyupdatezyCauseRegzyIP} \lstinputlisting[language=sail]{sail_latex/sailfnupdateCauseRegIP.tex}}

\newcommand{\sailupdateIP}{\label{zupdatezyIP} \lstinputlisting[language=sail]{sail_latex/sailupdateIP.tex}}

\newcommand{\sailmodIP}{\label{zzymodzyIP} \lstinputlisting[language=sail]{sail_latex/sailmodIP.tex}}

\newcommand{\sailgetCauseRegExcCode}{\label{zzygetzyCauseRegzyExcCode} \lstinputlisting[language=sail]{sail_latex/sailgetCauseRegExcCode.tex}}

\newcommand{\sailfngetCauseRegExcCode}{\label{zzygetzyCauseRegzyExcCode} \lstinputlisting[language=sail]{sail_latex/sailfngetCauseRegExcCode.tex}}

\newcommand{\sailsetCauseRegExcCode}{\label{zzysetzyCauseRegzyExcCode} \lstinputlisting[language=sail]{sail_latex/sailsetCauseRegExcCode.tex}}

\newcommand{\sailfnsetCauseRegExcCode}{\label{zzysetzyCauseRegzyExcCode} \lstinputlisting[language=sail]{sail_latex/sailfnsetCauseRegExcCode.tex}}

\newcommand{\sailupdateCauseRegExcCode}{\label{zzyupdatezyCauseRegzyExcCode} \lstinputlisting[language=sail]{sail_latex/sailupdateCauseRegExcCode.tex}}

\newcommand{\sailfnupdateCauseRegExcCode}{\label{zzyupdatezyCauseRegzyExcCode} \lstinputlisting[language=sail]{sail_latex/sailfnupdateCauseRegExcCode.tex}}

\newcommand{\sailsailupdateExcCodev}{\label{zupdatezyExcCode} \lstinputlisting[language=sail]{sail_latex/sailsailupdateExcCodev.tex}}

\newcommand{\sailsailmodExcCodev}{\label{zzymodzyExcCode} \lstinputlisting[language=sail]{sail_latex/sailsailmodExcCodev.tex}}

\newcommand{\sailTLBEntryLoReg}{\label{zTLBEntryLoReg} \lstinputlisting[language=sail]{sail_latex/sailTLBEntryLoReg.tex}}

\newcommand{\sailMkTLBEntryLoReg}{\label{zMkzyTLBEntryLoReg} \lstinputlisting[language=sail]{sail_latex/sailMkTLBEntryLoReg.tex}}

\newcommand{\sailfnMkTLBEntryLoReg}{\label{zMkzyTLBEntryLoReg} \lstinputlisting[language=sail]{sail_latex/sailfnMkTLBEntryLoReg.tex}}

\newcommand{\sailgetTLBEntryLoRegbits}{\label{zzygetzyTLBEntryLoRegzybits} \lstinputlisting[language=sail]{sail_latex/sailgetTLBEntryLoRegbits.tex}}

\newcommand{\sailfngetTLBEntryLoRegbits}{\label{zzygetzyTLBEntryLoRegzybits} \lstinputlisting[language=sail]{sail_latex/sailfngetTLBEntryLoRegbits.tex}}

\newcommand{\sailsetTLBEntryLoRegbits}{\label{zzysetzyTLBEntryLoRegzybits} \lstinputlisting[language=sail]{sail_latex/sailsetTLBEntryLoRegbits.tex}}

\newcommand{\sailfnsetTLBEntryLoRegbits}{\label{zzysetzyTLBEntryLoRegzybits} \lstinputlisting[language=sail]{sail_latex/sailfnsetTLBEntryLoRegbits.tex}}

\newcommand{\sailupdateTLBEntryLoRegbits}{\label{zzyupdatezyTLBEntryLoRegzybits} \lstinputlisting[language=sail]{sail_latex/sailupdateTLBEntryLoRegbits.tex}}

\newcommand{\sailfnupdateTLBEntryLoRegbits}{\label{zzyupdatezyTLBEntryLoRegzybits} \lstinputlisting[language=sail]{sail_latex/sailfnupdateTLBEntryLoRegbits.tex}}

\newcommand{\sailsailsailsailsailsailsailupdatebitsvvvvvv}{\label{zupdatezybits} \lstinputlisting[language=sail]{sail_latex/sailsailsailsailsailsailsailupdatebitsvvvvvv.tex}}

\newcommand{\sailsailsailsailsailsailsailmodbitsvvvvvv}{\label{zzymodzybits} \lstinputlisting[language=sail]{sail_latex/sailsailsailsailsailsailsailmodbitsvvvvvv.tex}}

\newcommand{\sailgetTLBEntryLoRegCapS}{\label{zzygetzyTLBEntryLoRegzyCapS} \lstinputlisting[language=sail]{sail_latex/sailgetTLBEntryLoRegCapS.tex}}

\newcommand{\sailfngetTLBEntryLoRegCapS}{\label{zzygetzyTLBEntryLoRegzyCapS} \lstinputlisting[language=sail]{sail_latex/sailfngetTLBEntryLoRegCapS.tex}}

\newcommand{\sailsetTLBEntryLoRegCapS}{\label{zzysetzyTLBEntryLoRegzyCapS} \lstinputlisting[language=sail]{sail_latex/sailsetTLBEntryLoRegCapS.tex}}

\newcommand{\sailfnsetTLBEntryLoRegCapS}{\label{zzysetzyTLBEntryLoRegzyCapS} \lstinputlisting[language=sail]{sail_latex/sailfnsetTLBEntryLoRegCapS.tex}}

\newcommand{\sailupdateTLBEntryLoRegCapS}{\label{zzyupdatezyTLBEntryLoRegzyCapS} \lstinputlisting[language=sail]{sail_latex/sailupdateTLBEntryLoRegCapS.tex}}

\newcommand{\sailfnupdateTLBEntryLoRegCapS}{\label{zzyupdatezyTLBEntryLoRegzyCapS} \lstinputlisting[language=sail]{sail_latex/sailfnupdateTLBEntryLoRegCapS.tex}}

\newcommand{\sailupdateCapS}{\label{zupdatezyCapS} \lstinputlisting[language=sail]{sail_latex/sailupdateCapS.tex}}

\newcommand{\sailmodCapS}{\label{zzymodzyCapS} \lstinputlisting[language=sail]{sail_latex/sailmodCapS.tex}}

\newcommand{\sailgetTLBEntryLoRegCapL}{\label{zzygetzyTLBEntryLoRegzyCapL} \lstinputlisting[language=sail]{sail_latex/sailgetTLBEntryLoRegCapL.tex}}

\newcommand{\sailfngetTLBEntryLoRegCapL}{\label{zzygetzyTLBEntryLoRegzyCapL} \lstinputlisting[language=sail]{sail_latex/sailfngetTLBEntryLoRegCapL.tex}}

\newcommand{\sailsetTLBEntryLoRegCapL}{\label{zzysetzyTLBEntryLoRegzyCapL} \lstinputlisting[language=sail]{sail_latex/sailsetTLBEntryLoRegCapL.tex}}

\newcommand{\sailfnsetTLBEntryLoRegCapL}{\label{zzysetzyTLBEntryLoRegzyCapL} \lstinputlisting[language=sail]{sail_latex/sailfnsetTLBEntryLoRegCapL.tex}}

\newcommand{\sailupdateTLBEntryLoRegCapL}{\label{zzyupdatezyTLBEntryLoRegzyCapL} \lstinputlisting[language=sail]{sail_latex/sailupdateTLBEntryLoRegCapL.tex}}

\newcommand{\sailfnupdateTLBEntryLoRegCapL}{\label{zzyupdatezyTLBEntryLoRegzyCapL} \lstinputlisting[language=sail]{sail_latex/sailfnupdateTLBEntryLoRegCapL.tex}}

\newcommand{\sailupdateCapL}{\label{zupdatezyCapL} \lstinputlisting[language=sail]{sail_latex/sailupdateCapL.tex}}

\newcommand{\sailmodCapL}{\label{zzymodzyCapL} \lstinputlisting[language=sail]{sail_latex/sailmodCapL.tex}}

\newcommand{\sailgetTLBEntryLoRegPFN}{\label{zzygetzyTLBEntryLoRegzyPFN} \lstinputlisting[language=sail]{sail_latex/sailgetTLBEntryLoRegPFN.tex}}

\newcommand{\sailfngetTLBEntryLoRegPFN}{\label{zzygetzyTLBEntryLoRegzyPFN} \lstinputlisting[language=sail]{sail_latex/sailfngetTLBEntryLoRegPFN.tex}}

\newcommand{\sailsetTLBEntryLoRegPFN}{\label{zzysetzyTLBEntryLoRegzyPFN} \lstinputlisting[language=sail]{sail_latex/sailsetTLBEntryLoRegPFN.tex}}

\newcommand{\sailfnsetTLBEntryLoRegPFN}{\label{zzysetzyTLBEntryLoRegzyPFN} \lstinputlisting[language=sail]{sail_latex/sailfnsetTLBEntryLoRegPFN.tex}}

\newcommand{\sailupdateTLBEntryLoRegPFN}{\label{zzyupdatezyTLBEntryLoRegzyPFN} \lstinputlisting[language=sail]{sail_latex/sailupdateTLBEntryLoRegPFN.tex}}

\newcommand{\sailfnupdateTLBEntryLoRegPFN}{\label{zzyupdatezyTLBEntryLoRegzyPFN} \lstinputlisting[language=sail]{sail_latex/sailfnupdateTLBEntryLoRegPFN.tex}}

\newcommand{\sailupdatePFN}{\label{zupdatezyPFN} \lstinputlisting[language=sail]{sail_latex/sailupdatePFN.tex}}

\newcommand{\sailmodPFN}{\label{zzymodzyPFN} \lstinputlisting[language=sail]{sail_latex/sailmodPFN.tex}}

\newcommand{\sailgetTLBEntryLoRegC}{\label{zzygetzyTLBEntryLoRegzyC} \lstinputlisting[language=sail]{sail_latex/sailgetTLBEntryLoRegC.tex}}

\newcommand{\sailfngetTLBEntryLoRegC}{\label{zzygetzyTLBEntryLoRegzyC} \lstinputlisting[language=sail]{sail_latex/sailfngetTLBEntryLoRegC.tex}}

\newcommand{\sailsetTLBEntryLoRegC}{\label{zzysetzyTLBEntryLoRegzyC} \lstinputlisting[language=sail]{sail_latex/sailsetTLBEntryLoRegC.tex}}

\newcommand{\sailfnsetTLBEntryLoRegC}{\label{zzysetzyTLBEntryLoRegzyC} \lstinputlisting[language=sail]{sail_latex/sailfnsetTLBEntryLoRegC.tex}}

\newcommand{\sailupdateTLBEntryLoRegC}{\label{zzyupdatezyTLBEntryLoRegzyC} \lstinputlisting[language=sail]{sail_latex/sailupdateTLBEntryLoRegC.tex}}

\newcommand{\sailfnupdateTLBEntryLoRegC}{\label{zzyupdatezyTLBEntryLoRegzyC} \lstinputlisting[language=sail]{sail_latex/sailfnupdateTLBEntryLoRegC.tex}}

\newcommand{\sailupdateC}{\label{zupdatezyC} \lstinputlisting[language=sail]{sail_latex/sailupdateC.tex}}

\newcommand{\sailmodC}{\label{zzymodzyC} \lstinputlisting[language=sail]{sail_latex/sailmodC.tex}}

\newcommand{\sailgetTLBEntryLoRegD}{\label{zzygetzyTLBEntryLoRegzyD} \lstinputlisting[language=sail]{sail_latex/sailgetTLBEntryLoRegD.tex}}

\newcommand{\sailfngetTLBEntryLoRegD}{\label{zzygetzyTLBEntryLoRegzyD} \lstinputlisting[language=sail]{sail_latex/sailfngetTLBEntryLoRegD.tex}}

\newcommand{\sailsetTLBEntryLoRegD}{\label{zzysetzyTLBEntryLoRegzyD} \lstinputlisting[language=sail]{sail_latex/sailsetTLBEntryLoRegD.tex}}

\newcommand{\sailfnsetTLBEntryLoRegD}{\label{zzysetzyTLBEntryLoRegzyD} \lstinputlisting[language=sail]{sail_latex/sailfnsetTLBEntryLoRegD.tex}}

\newcommand{\sailupdateTLBEntryLoRegD}{\label{zzyupdatezyTLBEntryLoRegzyD} \lstinputlisting[language=sail]{sail_latex/sailupdateTLBEntryLoRegD.tex}}

\newcommand{\sailfnupdateTLBEntryLoRegD}{\label{zzyupdatezyTLBEntryLoRegzyD} \lstinputlisting[language=sail]{sail_latex/sailfnupdateTLBEntryLoRegD.tex}}

\newcommand{\sailupdateD}{\label{zupdatezyD} \lstinputlisting[language=sail]{sail_latex/sailupdateD.tex}}

\newcommand{\sailmodD}{\label{zzymodzyD} \lstinputlisting[language=sail]{sail_latex/sailmodD.tex}}

\newcommand{\sailgetTLBEntryLoRegV}{\label{zzygetzyTLBEntryLoRegzyV} \lstinputlisting[language=sail]{sail_latex/sailgetTLBEntryLoRegV.tex}}

\newcommand{\sailfngetTLBEntryLoRegV}{\label{zzygetzyTLBEntryLoRegzyV} \lstinputlisting[language=sail]{sail_latex/sailfngetTLBEntryLoRegV.tex}}

\newcommand{\sailsetTLBEntryLoRegV}{\label{zzysetzyTLBEntryLoRegzyV} \lstinputlisting[language=sail]{sail_latex/sailsetTLBEntryLoRegV.tex}}

\newcommand{\sailfnsetTLBEntryLoRegV}{\label{zzysetzyTLBEntryLoRegzyV} \lstinputlisting[language=sail]{sail_latex/sailfnsetTLBEntryLoRegV.tex}}

\newcommand{\sailupdateTLBEntryLoRegV}{\label{zzyupdatezyTLBEntryLoRegzyV} \lstinputlisting[language=sail]{sail_latex/sailupdateTLBEntryLoRegV.tex}}

\newcommand{\sailfnupdateTLBEntryLoRegV}{\label{zzyupdatezyTLBEntryLoRegzyV} \lstinputlisting[language=sail]{sail_latex/sailfnupdateTLBEntryLoRegV.tex}}

\newcommand{\sailupdateV}{\label{zupdatezyV} \lstinputlisting[language=sail]{sail_latex/sailupdateV.tex}}

\newcommand{\sailmodV}{\label{zzymodzyV} \lstinputlisting[language=sail]{sail_latex/sailmodV.tex}}

\newcommand{\sailgetTLBEntryLoRegG}{\label{zzygetzyTLBEntryLoRegzyG} \lstinputlisting[language=sail]{sail_latex/sailgetTLBEntryLoRegG.tex}}

\newcommand{\sailfngetTLBEntryLoRegG}{\label{zzygetzyTLBEntryLoRegzyG} \lstinputlisting[language=sail]{sail_latex/sailfngetTLBEntryLoRegG.tex}}

\newcommand{\sailsetTLBEntryLoRegG}{\label{zzysetzyTLBEntryLoRegzyG} \lstinputlisting[language=sail]{sail_latex/sailsetTLBEntryLoRegG.tex}}

\newcommand{\sailfnsetTLBEntryLoRegG}{\label{zzysetzyTLBEntryLoRegzyG} \lstinputlisting[language=sail]{sail_latex/sailfnsetTLBEntryLoRegG.tex}}

\newcommand{\sailupdateTLBEntryLoRegG}{\label{zzyupdatezyTLBEntryLoRegzyG} \lstinputlisting[language=sail]{sail_latex/sailupdateTLBEntryLoRegG.tex}}

\newcommand{\sailfnupdateTLBEntryLoRegG}{\label{zzyupdatezyTLBEntryLoRegzyG} \lstinputlisting[language=sail]{sail_latex/sailfnupdateTLBEntryLoRegG.tex}}

\newcommand{\sailsailupdateGv}{\label{zupdatezyG} \lstinputlisting[language=sail]{sail_latex/sailsailupdateGv.tex}}

\newcommand{\sailsailmodGv}{\label{zzymodzyG} \lstinputlisting[language=sail]{sail_latex/sailsailmodGv.tex}}

\newcommand{\sailTLBEntryHiReg}{\label{zTLBEntryHiReg} \lstinputlisting[language=sail]{sail_latex/sailTLBEntryHiReg.tex}}

\newcommand{\sailMkTLBEntryHiReg}{\label{zMkzyTLBEntryHiReg} \lstinputlisting[language=sail]{sail_latex/sailMkTLBEntryHiReg.tex}}

\newcommand{\sailfnMkTLBEntryHiReg}{\label{zMkzyTLBEntryHiReg} \lstinputlisting[language=sail]{sail_latex/sailfnMkTLBEntryHiReg.tex}}

\newcommand{\sailgetTLBEntryHiRegbits}{\label{zzygetzyTLBEntryHiRegzybits} \lstinputlisting[language=sail]{sail_latex/sailgetTLBEntryHiRegbits.tex}}

\newcommand{\sailfngetTLBEntryHiRegbits}{\label{zzygetzyTLBEntryHiRegzybits} \lstinputlisting[language=sail]{sail_latex/sailfngetTLBEntryHiRegbits.tex}}

\newcommand{\sailsetTLBEntryHiRegbits}{\label{zzysetzyTLBEntryHiRegzybits} \lstinputlisting[language=sail]{sail_latex/sailsetTLBEntryHiRegbits.tex}}

\newcommand{\sailfnsetTLBEntryHiRegbits}{\label{zzysetzyTLBEntryHiRegzybits} \lstinputlisting[language=sail]{sail_latex/sailfnsetTLBEntryHiRegbits.tex}}

\newcommand{\sailupdateTLBEntryHiRegbits}{\label{zzyupdatezyTLBEntryHiRegzybits} \lstinputlisting[language=sail]{sail_latex/sailupdateTLBEntryHiRegbits.tex}}

\newcommand{\sailfnupdateTLBEntryHiRegbits}{\label{zzyupdatezyTLBEntryHiRegzybits} \lstinputlisting[language=sail]{sail_latex/sailfnupdateTLBEntryHiRegbits.tex}}

\newcommand{\sailsailsailsailsailsailupdatebitsvvvvv}{\label{zupdatezybits} \lstinputlisting[language=sail]{sail_latex/sailsailsailsailsailsailupdatebitsvvvvv.tex}}

\newcommand{\sailsailsailsailsailsailmodbitsvvvvv}{\label{zzymodzybits} \lstinputlisting[language=sail]{sail_latex/sailsailsailsailsailsailmodbitsvvvvv.tex}}

\newcommand{\sailgetTLBEntryHiRegR}{\label{zzygetzyTLBEntryHiRegzyR} \lstinputlisting[language=sail]{sail_latex/sailgetTLBEntryHiRegR.tex}}

\newcommand{\sailfngetTLBEntryHiRegR}{\label{zzygetzyTLBEntryHiRegzyR} \lstinputlisting[language=sail]{sail_latex/sailfngetTLBEntryHiRegR.tex}}

\newcommand{\sailsetTLBEntryHiRegR}{\label{zzysetzyTLBEntryHiRegzyR} \lstinputlisting[language=sail]{sail_latex/sailsetTLBEntryHiRegR.tex}}

\newcommand{\sailfnsetTLBEntryHiRegR}{\label{zzysetzyTLBEntryHiRegzyR} \lstinputlisting[language=sail]{sail_latex/sailfnsetTLBEntryHiRegR.tex}}

\newcommand{\sailupdateTLBEntryHiRegR}{\label{zzyupdatezyTLBEntryHiRegzyR} \lstinputlisting[language=sail]{sail_latex/sailupdateTLBEntryHiRegR.tex}}

\newcommand{\sailfnupdateTLBEntryHiRegR}{\label{zzyupdatezyTLBEntryHiRegzyR} \lstinputlisting[language=sail]{sail_latex/sailfnupdateTLBEntryHiRegR.tex}}

\newcommand{\sailsailupdateRv}{\label{zupdatezyR} \lstinputlisting[language=sail]{sail_latex/sailsailupdateRv.tex}}

\newcommand{\sailsailmodRv}{\label{zzymodzyR} \lstinputlisting[language=sail]{sail_latex/sailsailmodRv.tex}}

\newcommand{\sailgetTLBEntryHiRegVPNtwo}{\label{zzygetzyTLBEntryHiRegzyVPNtwo} \lstinputlisting[language=sail]{sail_latex/sailgetTLBEntryHiRegVPNtwo.tex}}

\newcommand{\sailfngetTLBEntryHiRegVPNtwo}{\label{zzygetzyTLBEntryHiRegzyVPNtwo} \lstinputlisting[language=sail]{sail_latex/sailfngetTLBEntryHiRegVPNtwo.tex}}

\newcommand{\sailsetTLBEntryHiRegVPNtwo}{\label{zzysetzyTLBEntryHiRegzyVPNtwo} \lstinputlisting[language=sail]{sail_latex/sailsetTLBEntryHiRegVPNtwo.tex}}

\newcommand{\sailfnsetTLBEntryHiRegVPNtwo}{\label{zzysetzyTLBEntryHiRegzyVPNtwo} \lstinputlisting[language=sail]{sail_latex/sailfnsetTLBEntryHiRegVPNtwo.tex}}

\newcommand{\sailupdateTLBEntryHiRegVPNtwo}{\label{zzyupdatezyTLBEntryHiRegzyVPNtwo} \lstinputlisting[language=sail]{sail_latex/sailupdateTLBEntryHiRegVPNtwo.tex}}

\newcommand{\sailfnupdateTLBEntryHiRegVPNtwo}{\label{zzyupdatezyTLBEntryHiRegzyVPNtwo} \lstinputlisting[language=sail]{sail_latex/sailfnupdateTLBEntryHiRegVPNtwo.tex}}

\newcommand{\sailsailupdateVPNtwov}{\label{zupdatezyVPNtwo} \lstinputlisting[language=sail]{sail_latex/sailsailupdateVPNtwov.tex}}

\newcommand{\sailsailmodVPNtwov}{\label{zzymodzyVPNtwo} \lstinputlisting[language=sail]{sail_latex/sailsailmodVPNtwov.tex}}

\newcommand{\sailgetTLBEntryHiRegASID}{\label{zzygetzyTLBEntryHiRegzyASID} \lstinputlisting[language=sail]{sail_latex/sailgetTLBEntryHiRegASID.tex}}

\newcommand{\sailfngetTLBEntryHiRegASID}{\label{zzygetzyTLBEntryHiRegzyASID} \lstinputlisting[language=sail]{sail_latex/sailfngetTLBEntryHiRegASID.tex}}

\newcommand{\sailsetTLBEntryHiRegASID}{\label{zzysetzyTLBEntryHiRegzyASID} \lstinputlisting[language=sail]{sail_latex/sailsetTLBEntryHiRegASID.tex}}

\newcommand{\sailfnsetTLBEntryHiRegASID}{\label{zzysetzyTLBEntryHiRegzyASID} \lstinputlisting[language=sail]{sail_latex/sailfnsetTLBEntryHiRegASID.tex}}

\newcommand{\sailupdateTLBEntryHiRegASID}{\label{zzyupdatezyTLBEntryHiRegzyASID} \lstinputlisting[language=sail]{sail_latex/sailupdateTLBEntryHiRegASID.tex}}

\newcommand{\sailfnupdateTLBEntryHiRegASID}{\label{zzyupdatezyTLBEntryHiRegzyASID} \lstinputlisting[language=sail]{sail_latex/sailfnupdateTLBEntryHiRegASID.tex}}

\newcommand{\sailsailupdateASIDv}{\label{zupdatezyASID} \lstinputlisting[language=sail]{sail_latex/sailsailupdateASIDv.tex}}

\newcommand{\sailsailmodASIDv}{\label{zzymodzyASID} \lstinputlisting[language=sail]{sail_latex/sailsailmodASIDv.tex}}

\newcommand{\sailContextReg}{\label{zContextReg} \lstinputlisting[language=sail]{sail_latex/sailContextReg.tex}}

\newcommand{\sailMkContextReg}{\label{zMkzyContextReg} \lstinputlisting[language=sail]{sail_latex/sailMkContextReg.tex}}

\newcommand{\sailfnMkContextReg}{\label{zMkzyContextReg} \lstinputlisting[language=sail]{sail_latex/sailfnMkContextReg.tex}}

\newcommand{\sailgetContextRegbits}{\label{zzygetzyContextRegzybits} \lstinputlisting[language=sail]{sail_latex/sailgetContextRegbits.tex}}

\newcommand{\sailfngetContextRegbits}{\label{zzygetzyContextRegzybits} \lstinputlisting[language=sail]{sail_latex/sailfngetContextRegbits.tex}}

\newcommand{\sailsetContextRegbits}{\label{zzysetzyContextRegzybits} \lstinputlisting[language=sail]{sail_latex/sailsetContextRegbits.tex}}

\newcommand{\sailfnsetContextRegbits}{\label{zzysetzyContextRegzybits} \lstinputlisting[language=sail]{sail_latex/sailfnsetContextRegbits.tex}}

\newcommand{\sailupdateContextRegbits}{\label{zzyupdatezyContextRegzybits} \lstinputlisting[language=sail]{sail_latex/sailupdateContextRegbits.tex}}

\newcommand{\sailfnupdateContextRegbits}{\label{zzyupdatezyContextRegzybits} \lstinputlisting[language=sail]{sail_latex/sailfnupdateContextRegbits.tex}}

\newcommand{\sailsailsailsailsailupdatebitsvvvv}{\label{zupdatezybits} \lstinputlisting[language=sail]{sail_latex/sailsailsailsailsailupdatebitsvvvv.tex}}

\newcommand{\sailsailsailsailsailmodbitsvvvv}{\label{zzymodzybits} \lstinputlisting[language=sail]{sail_latex/sailsailsailsailsailmodbitsvvvv.tex}}

\newcommand{\sailgetContextRegPTEBase}{\label{zzygetzyContextRegzyPTEBase} \lstinputlisting[language=sail]{sail_latex/sailgetContextRegPTEBase.tex}}

\newcommand{\sailfngetContextRegPTEBase}{\label{zzygetzyContextRegzyPTEBase} \lstinputlisting[language=sail]{sail_latex/sailfngetContextRegPTEBase.tex}}

\newcommand{\sailsetContextRegPTEBase}{\label{zzysetzyContextRegzyPTEBase} \lstinputlisting[language=sail]{sail_latex/sailsetContextRegPTEBase.tex}}

\newcommand{\sailfnsetContextRegPTEBase}{\label{zzysetzyContextRegzyPTEBase} \lstinputlisting[language=sail]{sail_latex/sailfnsetContextRegPTEBase.tex}}

\newcommand{\sailupdateContextRegPTEBase}{\label{zzyupdatezyContextRegzyPTEBase} \lstinputlisting[language=sail]{sail_latex/sailupdateContextRegPTEBase.tex}}

\newcommand{\sailfnupdateContextRegPTEBase}{\label{zzyupdatezyContextRegzyPTEBase} \lstinputlisting[language=sail]{sail_latex/sailfnupdateContextRegPTEBase.tex}}

\newcommand{\sailupdatePTEBase}{\label{zupdatezyPTEBase} \lstinputlisting[language=sail]{sail_latex/sailupdatePTEBase.tex}}

\newcommand{\sailmodPTEBase}{\label{zzymodzyPTEBase} \lstinputlisting[language=sail]{sail_latex/sailmodPTEBase.tex}}

\newcommand{\sailgetContextRegBadVPNtwo}{\label{zzygetzyContextRegzyBadVPNtwo} \lstinputlisting[language=sail]{sail_latex/sailgetContextRegBadVPNtwo.tex}}

\newcommand{\sailfngetContextRegBadVPNtwo}{\label{zzygetzyContextRegzyBadVPNtwo} \lstinputlisting[language=sail]{sail_latex/sailfngetContextRegBadVPNtwo.tex}}

\newcommand{\sailsetContextRegBadVPNtwo}{\label{zzysetzyContextRegzyBadVPNtwo} \lstinputlisting[language=sail]{sail_latex/sailsetContextRegBadVPNtwo.tex}}

\newcommand{\sailfnsetContextRegBadVPNtwo}{\label{zzysetzyContextRegzyBadVPNtwo} \lstinputlisting[language=sail]{sail_latex/sailfnsetContextRegBadVPNtwo.tex}}

\newcommand{\sailupdateContextRegBadVPNtwo}{\label{zzyupdatezyContextRegzyBadVPNtwo} \lstinputlisting[language=sail]{sail_latex/sailupdateContextRegBadVPNtwo.tex}}

\newcommand{\sailfnupdateContextRegBadVPNtwo}{\label{zzyupdatezyContextRegzyBadVPNtwo} \lstinputlisting[language=sail]{sail_latex/sailfnupdateContextRegBadVPNtwo.tex}}

\newcommand{\sailupdateBadVPNtwo}{\label{zupdatezyBadVPNtwo} \lstinputlisting[language=sail]{sail_latex/sailupdateBadVPNtwo.tex}}

\newcommand{\sailmodBadVPNtwo}{\label{zzymodzyBadVPNtwo} \lstinputlisting[language=sail]{sail_latex/sailmodBadVPNtwo.tex}}

\newcommand{\sailXContextReg}{\label{zXContextReg} \lstinputlisting[language=sail]{sail_latex/sailXContextReg.tex}}

\newcommand{\sailMkXContextReg}{\label{zMkzyXContextReg} \lstinputlisting[language=sail]{sail_latex/sailMkXContextReg.tex}}

\newcommand{\sailfnMkXContextReg}{\label{zMkzyXContextReg} \lstinputlisting[language=sail]{sail_latex/sailfnMkXContextReg.tex}}

\newcommand{\sailgetXContextRegbits}{\label{zzygetzyXContextRegzybits} \lstinputlisting[language=sail]{sail_latex/sailgetXContextRegbits.tex}}

\newcommand{\sailfngetXContextRegbits}{\label{zzygetzyXContextRegzybits} \lstinputlisting[language=sail]{sail_latex/sailfngetXContextRegbits.tex}}

\newcommand{\sailsetXContextRegbits}{\label{zzysetzyXContextRegzybits} \lstinputlisting[language=sail]{sail_latex/sailsetXContextRegbits.tex}}

\newcommand{\sailfnsetXContextRegbits}{\label{zzysetzyXContextRegzybits} \lstinputlisting[language=sail]{sail_latex/sailfnsetXContextRegbits.tex}}

\newcommand{\sailupdateXContextRegbits}{\label{zzyupdatezyXContextRegzybits} \lstinputlisting[language=sail]{sail_latex/sailupdateXContextRegbits.tex}}

\newcommand{\sailfnupdateXContextRegbits}{\label{zzyupdatezyXContextRegzybits} \lstinputlisting[language=sail]{sail_latex/sailfnupdateXContextRegbits.tex}}

\newcommand{\sailsailsailsailupdatebitsvvv}{\label{zupdatezybits} \lstinputlisting[language=sail]{sail_latex/sailsailsailsailupdatebitsvvv.tex}}

\newcommand{\sailsailsailsailmodbitsvvv}{\label{zzymodzybits} \lstinputlisting[language=sail]{sail_latex/sailsailsailsailmodbitsvvv.tex}}

\newcommand{\sailgetXContextRegXPTEBase}{\label{zzygetzyXContextRegzyXPTEBase} \lstinputlisting[language=sail]{sail_latex/sailgetXContextRegXPTEBase.tex}}

\newcommand{\sailfngetXContextRegXPTEBase}{\label{zzygetzyXContextRegzyXPTEBase} \lstinputlisting[language=sail]{sail_latex/sailfngetXContextRegXPTEBase.tex}}

\newcommand{\sailsetXContextRegXPTEBase}{\label{zzysetzyXContextRegzyXPTEBase} \lstinputlisting[language=sail]{sail_latex/sailsetXContextRegXPTEBase.tex}}

\newcommand{\sailfnsetXContextRegXPTEBase}{\label{zzysetzyXContextRegzyXPTEBase} \lstinputlisting[language=sail]{sail_latex/sailfnsetXContextRegXPTEBase.tex}}

\newcommand{\sailupdateXContextRegXPTEBase}{\label{zzyupdatezyXContextRegzyXPTEBase} \lstinputlisting[language=sail]{sail_latex/sailupdateXContextRegXPTEBase.tex}}

\newcommand{\sailfnupdateXContextRegXPTEBase}{\label{zzyupdatezyXContextRegzyXPTEBase} \lstinputlisting[language=sail]{sail_latex/sailfnupdateXContextRegXPTEBase.tex}}

\newcommand{\sailupdateXPTEBase}{\label{zupdatezyXPTEBase} \lstinputlisting[language=sail]{sail_latex/sailupdateXPTEBase.tex}}

\newcommand{\sailmodXPTEBase}{\label{zzymodzyXPTEBase} \lstinputlisting[language=sail]{sail_latex/sailmodXPTEBase.tex}}

\newcommand{\sailgetXContextRegXR}{\label{zzygetzyXContextRegzyXR} \lstinputlisting[language=sail]{sail_latex/sailgetXContextRegXR.tex}}

\newcommand{\sailfngetXContextRegXR}{\label{zzygetzyXContextRegzyXR} \lstinputlisting[language=sail]{sail_latex/sailfngetXContextRegXR.tex}}

\newcommand{\sailsetXContextRegXR}{\label{zzysetzyXContextRegzyXR} \lstinputlisting[language=sail]{sail_latex/sailsetXContextRegXR.tex}}

\newcommand{\sailfnsetXContextRegXR}{\label{zzysetzyXContextRegzyXR} \lstinputlisting[language=sail]{sail_latex/sailfnsetXContextRegXR.tex}}

\newcommand{\sailupdateXContextRegXR}{\label{zzyupdatezyXContextRegzyXR} \lstinputlisting[language=sail]{sail_latex/sailupdateXContextRegXR.tex}}

\newcommand{\sailfnupdateXContextRegXR}{\label{zzyupdatezyXContextRegzyXR} \lstinputlisting[language=sail]{sail_latex/sailfnupdateXContextRegXR.tex}}

\newcommand{\sailupdateXR}{\label{zupdatezyXR} \lstinputlisting[language=sail]{sail_latex/sailupdateXR.tex}}

\newcommand{\sailmodXR}{\label{zzymodzyXR} \lstinputlisting[language=sail]{sail_latex/sailmodXR.tex}}

\newcommand{\sailgetXContextRegXBadVPNtwo}{\label{zzygetzyXContextRegzyXBadVPNtwo} \lstinputlisting[language=sail]{sail_latex/sailgetXContextRegXBadVPNtwo.tex}}

\newcommand{\sailfngetXContextRegXBadVPNtwo}{\label{zzygetzyXContextRegzyXBadVPNtwo} \lstinputlisting[language=sail]{sail_latex/sailfngetXContextRegXBadVPNtwo.tex}}

\newcommand{\sailsetXContextRegXBadVPNtwo}{\label{zzysetzyXContextRegzyXBadVPNtwo} \lstinputlisting[language=sail]{sail_latex/sailsetXContextRegXBadVPNtwo.tex}}

\newcommand{\sailfnsetXContextRegXBadVPNtwo}{\label{zzysetzyXContextRegzyXBadVPNtwo} \lstinputlisting[language=sail]{sail_latex/sailfnsetXContextRegXBadVPNtwo.tex}}

\newcommand{\sailupdateXContextRegXBadVPNtwo}{\label{zzyupdatezyXContextRegzyXBadVPNtwo} \lstinputlisting[language=sail]{sail_latex/sailupdateXContextRegXBadVPNtwo.tex}}

\newcommand{\sailfnupdateXContextRegXBadVPNtwo}{\label{zzyupdatezyXContextRegzyXBadVPNtwo} \lstinputlisting[language=sail]{sail_latex/sailfnupdateXContextRegXBadVPNtwo.tex}}

\newcommand{\sailupdateXBadVPNtwo}{\label{zupdatezyXBadVPNtwo} \lstinputlisting[language=sail]{sail_latex/sailupdateXBadVPNtwo.tex}}

\newcommand{\sailmodXBadVPNtwo}{\label{zzymodzyXBadVPNtwo} \lstinputlisting[language=sail]{sail_latex/sailmodXBadVPNtwo.tex}}

\newcommand{\sailTLBIndexT}{\label{zTLBIndexT} \lstinputlisting[language=sail]{sail_latex/sailTLBIndexT.tex}}

\newcommand{\sailMAX}{\label{zMAX} \lstinputlisting[language=sail]{sail_latex/sailMAX.tex}}

\newcommand{\sailfnMAX}{\label{zMAX} \lstinputlisting[language=sail]{sail_latex/sailfnMAX.tex}}

\newcommand{\sailTLBEntry}{\label{zTLBEntry} \lstinputlisting[language=sail]{sail_latex/sailTLBEntry.tex}}

\newcommand{\sailMkTLBEntry}{\label{zMkzyTLBEntry} \lstinputlisting[language=sail]{sail_latex/sailMkTLBEntry.tex}}

\newcommand{\sailfnMkTLBEntry}{\label{zMkzyTLBEntry} \lstinputlisting[language=sail]{sail_latex/sailfnMkTLBEntry.tex}}

\newcommand{\sailgetTLBEntrybits}{\label{zzygetzyTLBEntryzybits} \lstinputlisting[language=sail]{sail_latex/sailgetTLBEntrybits.tex}}

\newcommand{\sailfngetTLBEntrybits}{\label{zzygetzyTLBEntryzybits} \lstinputlisting[language=sail]{sail_latex/sailfngetTLBEntrybits.tex}}

\newcommand{\sailsetTLBEntrybits}{\label{zzysetzyTLBEntryzybits} \lstinputlisting[language=sail]{sail_latex/sailsetTLBEntrybits.tex}}

\newcommand{\sailfnsetTLBEntrybits}{\label{zzysetzyTLBEntryzybits} \lstinputlisting[language=sail]{sail_latex/sailfnsetTLBEntrybits.tex}}

\newcommand{\sailupdateTLBEntrybits}{\label{zzyupdatezyTLBEntryzybits} \lstinputlisting[language=sail]{sail_latex/sailupdateTLBEntrybits.tex}}

\newcommand{\sailfnupdateTLBEntrybits}{\label{zzyupdatezyTLBEntryzybits} \lstinputlisting[language=sail]{sail_latex/sailfnupdateTLBEntrybits.tex}}

\newcommand{\sailsailsailupdatebitsvv}{\label{zupdatezybits} \lstinputlisting[language=sail]{sail_latex/sailsailsailupdatebitsvv.tex}}

\newcommand{\sailsailsailmodbitsvv}{\label{zzymodzybits} \lstinputlisting[language=sail]{sail_latex/sailsailsailmodbitsvv.tex}}

\newcommand{\sailgetTLBEntrypagemask}{\label{zzygetzyTLBEntryzypagemask} \lstinputlisting[language=sail]{sail_latex/sailgetTLBEntrypagemask.tex}}

\newcommand{\sailfngetTLBEntrypagemask}{\label{zzygetzyTLBEntryzypagemask} \lstinputlisting[language=sail]{sail_latex/sailfngetTLBEntrypagemask.tex}}

\newcommand{\sailsetTLBEntrypagemask}{\label{zzysetzyTLBEntryzypagemask} \lstinputlisting[language=sail]{sail_latex/sailsetTLBEntrypagemask.tex}}

\newcommand{\sailfnsetTLBEntrypagemask}{\label{zzysetzyTLBEntryzypagemask} \lstinputlisting[language=sail]{sail_latex/sailfnsetTLBEntrypagemask.tex}}

\newcommand{\sailupdateTLBEntrypagemask}{\label{zzyupdatezyTLBEntryzypagemask} \lstinputlisting[language=sail]{sail_latex/sailupdateTLBEntrypagemask.tex}}

\newcommand{\sailfnupdateTLBEntrypagemask}{\label{zzyupdatezyTLBEntryzypagemask} \lstinputlisting[language=sail]{sail_latex/sailfnupdateTLBEntrypagemask.tex}}

\newcommand{\sailupdatepagemask}{\label{zupdatezypagemask} \lstinputlisting[language=sail]{sail_latex/sailupdatepagemask.tex}}

\newcommand{\sailmodpagemask}{\label{zzymodzypagemask} \lstinputlisting[language=sail]{sail_latex/sailmodpagemask.tex}}

\newcommand{\sailgetTLBEntryr}{\label{zzygetzyTLBEntryzyr} \lstinputlisting[language=sail]{sail_latex/sailgetTLBEntryr.tex}}

\newcommand{\sailfngetTLBEntryr}{\label{zzygetzyTLBEntryzyr} \lstinputlisting[language=sail]{sail_latex/sailfngetTLBEntryr.tex}}

\newcommand{\sailsetTLBEntryr}{\label{zzysetzyTLBEntryzyr} \lstinputlisting[language=sail]{sail_latex/sailsetTLBEntryr.tex}}

\newcommand{\sailfnsetTLBEntryr}{\label{zzysetzyTLBEntryzyr} \lstinputlisting[language=sail]{sail_latex/sailfnsetTLBEntryr.tex}}

\newcommand{\sailupdateTLBEntryr}{\label{zzyupdatezyTLBEntryzyr} \lstinputlisting[language=sail]{sail_latex/sailupdateTLBEntryr.tex}}

\newcommand{\sailfnupdateTLBEntryr}{\label{zzyupdatezyTLBEntryzyr} \lstinputlisting[language=sail]{sail_latex/sailfnupdateTLBEntryr.tex}}

\newcommand{\sailupdater}{\label{zupdatezyr} \lstinputlisting[language=sail]{sail_latex/sailupdater.tex}}

\newcommand{\sailmodr}{\label{zzymodzyr} \lstinputlisting[language=sail]{sail_latex/sailmodr.tex}}

\newcommand{\sailgetTLBEntryvpntwo}{\label{zzygetzyTLBEntryzyvpntwo} \lstinputlisting[language=sail]{sail_latex/sailgetTLBEntryvpntwo.tex}}

\newcommand{\sailfngetTLBEntryvpntwo}{\label{zzygetzyTLBEntryzyvpntwo} \lstinputlisting[language=sail]{sail_latex/sailfngetTLBEntryvpntwo.tex}}

\newcommand{\sailsetTLBEntryvpntwo}{\label{zzysetzyTLBEntryzyvpntwo} \lstinputlisting[language=sail]{sail_latex/sailsetTLBEntryvpntwo.tex}}

\newcommand{\sailfnsetTLBEntryvpntwo}{\label{zzysetzyTLBEntryzyvpntwo} \lstinputlisting[language=sail]{sail_latex/sailfnsetTLBEntryvpntwo.tex}}

\newcommand{\sailupdateTLBEntryvpntwo}{\label{zzyupdatezyTLBEntryzyvpntwo} \lstinputlisting[language=sail]{sail_latex/sailupdateTLBEntryvpntwo.tex}}

\newcommand{\sailfnupdateTLBEntryvpntwo}{\label{zzyupdatezyTLBEntryzyvpntwo} \lstinputlisting[language=sail]{sail_latex/sailfnupdateTLBEntryvpntwo.tex}}

\newcommand{\sailupdatevpntwo}{\label{zupdatezyvpntwo} \lstinputlisting[language=sail]{sail_latex/sailupdatevpntwo.tex}}

\newcommand{\sailmodvpntwo}{\label{zzymodzyvpntwo} \lstinputlisting[language=sail]{sail_latex/sailmodvpntwo.tex}}

\newcommand{\sailgetTLBEntryasid}{\label{zzygetzyTLBEntryzyasid} \lstinputlisting[language=sail]{sail_latex/sailgetTLBEntryasid.tex}}

\newcommand{\sailfngetTLBEntryasid}{\label{zzygetzyTLBEntryzyasid} \lstinputlisting[language=sail]{sail_latex/sailfngetTLBEntryasid.tex}}

\newcommand{\sailsetTLBEntryasid}{\label{zzysetzyTLBEntryzyasid} \lstinputlisting[language=sail]{sail_latex/sailsetTLBEntryasid.tex}}

\newcommand{\sailfnsetTLBEntryasid}{\label{zzysetzyTLBEntryzyasid} \lstinputlisting[language=sail]{sail_latex/sailfnsetTLBEntryasid.tex}}

\newcommand{\sailupdateTLBEntryasid}{\label{zzyupdatezyTLBEntryzyasid} \lstinputlisting[language=sail]{sail_latex/sailupdateTLBEntryasid.tex}}

\newcommand{\sailfnupdateTLBEntryasid}{\label{zzyupdatezyTLBEntryzyasid} \lstinputlisting[language=sail]{sail_latex/sailfnupdateTLBEntryasid.tex}}

\newcommand{\sailupdateasid}{\label{zupdatezyasid} \lstinputlisting[language=sail]{sail_latex/sailupdateasid.tex}}

\newcommand{\sailmodasid}{\label{zzymodzyasid} \lstinputlisting[language=sail]{sail_latex/sailmodasid.tex}}

\newcommand{\sailgetTLBEntryg}{\label{zzygetzyTLBEntryzyg} \lstinputlisting[language=sail]{sail_latex/sailgetTLBEntryg.tex}}

\newcommand{\sailfngetTLBEntryg}{\label{zzygetzyTLBEntryzyg} \lstinputlisting[language=sail]{sail_latex/sailfngetTLBEntryg.tex}}

\newcommand{\sailsetTLBEntryg}{\label{zzysetzyTLBEntryzyg} \lstinputlisting[language=sail]{sail_latex/sailsetTLBEntryg.tex}}

\newcommand{\sailfnsetTLBEntryg}{\label{zzysetzyTLBEntryzyg} \lstinputlisting[language=sail]{sail_latex/sailfnsetTLBEntryg.tex}}

\newcommand{\sailupdateTLBEntryg}{\label{zzyupdatezyTLBEntryzyg} \lstinputlisting[language=sail]{sail_latex/sailupdateTLBEntryg.tex}}

\newcommand{\sailfnupdateTLBEntryg}{\label{zzyupdatezyTLBEntryzyg} \lstinputlisting[language=sail]{sail_latex/sailfnupdateTLBEntryg.tex}}

\newcommand{\sailupdateg}{\label{zupdatezyg} \lstinputlisting[language=sail]{sail_latex/sailupdateg.tex}}

\newcommand{\sailmodg}{\label{zzymodzyg} \lstinputlisting[language=sail]{sail_latex/sailmodg.tex}}

\newcommand{\sailgetTLBEntryvalid}{\label{zzygetzyTLBEntryzyvalid} \lstinputlisting[language=sail]{sail_latex/sailgetTLBEntryvalid.tex}}

\newcommand{\sailfngetTLBEntryvalid}{\label{zzygetzyTLBEntryzyvalid} \lstinputlisting[language=sail]{sail_latex/sailfngetTLBEntryvalid.tex}}

\newcommand{\sailsetTLBEntryvalid}{\label{zzysetzyTLBEntryzyvalid} \lstinputlisting[language=sail]{sail_latex/sailsetTLBEntryvalid.tex}}

\newcommand{\sailfnsetTLBEntryvalid}{\label{zzysetzyTLBEntryzyvalid} \lstinputlisting[language=sail]{sail_latex/sailfnsetTLBEntryvalid.tex}}

\newcommand{\sailupdateTLBEntryvalid}{\label{zzyupdatezyTLBEntryzyvalid} \lstinputlisting[language=sail]{sail_latex/sailupdateTLBEntryvalid.tex}}

\newcommand{\sailfnupdateTLBEntryvalid}{\label{zzyupdatezyTLBEntryzyvalid} \lstinputlisting[language=sail]{sail_latex/sailfnupdateTLBEntryvalid.tex}}

\newcommand{\sailupdatevalid}{\label{zupdatezyvalid} \lstinputlisting[language=sail]{sail_latex/sailupdatevalid.tex}}

\newcommand{\sailmodvalid}{\label{zzymodzyvalid} \lstinputlisting[language=sail]{sail_latex/sailmodvalid.tex}}

\newcommand{\sailgetTLBEntrycapsone}{\label{zzygetzyTLBEntryzycapsone} \lstinputlisting[language=sail]{sail_latex/sailgetTLBEntrycapsone.tex}}

\newcommand{\sailfngetTLBEntrycapsone}{\label{zzygetzyTLBEntryzycapsone} \lstinputlisting[language=sail]{sail_latex/sailfngetTLBEntrycapsone.tex}}

\newcommand{\sailsetTLBEntrycapsone}{\label{zzysetzyTLBEntryzycapsone} \lstinputlisting[language=sail]{sail_latex/sailsetTLBEntrycapsone.tex}}

\newcommand{\sailfnsetTLBEntrycapsone}{\label{zzysetzyTLBEntryzycapsone} \lstinputlisting[language=sail]{sail_latex/sailfnsetTLBEntrycapsone.tex}}

\newcommand{\sailupdateTLBEntrycapsone}{\label{zzyupdatezyTLBEntryzycapsone} \lstinputlisting[language=sail]{sail_latex/sailupdateTLBEntrycapsone.tex}}

\newcommand{\sailfnupdateTLBEntrycapsone}{\label{zzyupdatezyTLBEntryzycapsone} \lstinputlisting[language=sail]{sail_latex/sailfnupdateTLBEntrycapsone.tex}}

\newcommand{\sailupdatecapsone}{\label{zupdatezycapsone} \lstinputlisting[language=sail]{sail_latex/sailupdatecapsone.tex}}

\newcommand{\sailmodcapsone}{\label{zzymodzycapsone} \lstinputlisting[language=sail]{sail_latex/sailmodcapsone.tex}}

\newcommand{\sailgetTLBEntrycaplone}{\label{zzygetzyTLBEntryzycaplone} \lstinputlisting[language=sail]{sail_latex/sailgetTLBEntrycaplone.tex}}

\newcommand{\sailfngetTLBEntrycaplone}{\label{zzygetzyTLBEntryzycaplone} \lstinputlisting[language=sail]{sail_latex/sailfngetTLBEntrycaplone.tex}}

\newcommand{\sailsetTLBEntrycaplone}{\label{zzysetzyTLBEntryzycaplone} \lstinputlisting[language=sail]{sail_latex/sailsetTLBEntrycaplone.tex}}

\newcommand{\sailfnsetTLBEntrycaplone}{\label{zzysetzyTLBEntryzycaplone} \lstinputlisting[language=sail]{sail_latex/sailfnsetTLBEntrycaplone.tex}}

\newcommand{\sailupdateTLBEntrycaplone}{\label{zzyupdatezyTLBEntryzycaplone} \lstinputlisting[language=sail]{sail_latex/sailupdateTLBEntrycaplone.tex}}

\newcommand{\sailfnupdateTLBEntrycaplone}{\label{zzyupdatezyTLBEntryzycaplone} \lstinputlisting[language=sail]{sail_latex/sailfnupdateTLBEntrycaplone.tex}}

\newcommand{\sailupdatecaplone}{\label{zupdatezycaplone} \lstinputlisting[language=sail]{sail_latex/sailupdatecaplone.tex}}

\newcommand{\sailmodcaplone}{\label{zzymodzycaplone} \lstinputlisting[language=sail]{sail_latex/sailmodcaplone.tex}}

\newcommand{\sailgetTLBEntrypfnone}{\label{zzygetzyTLBEntryzypfnone} \lstinputlisting[language=sail]{sail_latex/sailgetTLBEntrypfnone.tex}}

\newcommand{\sailfngetTLBEntrypfnone}{\label{zzygetzyTLBEntryzypfnone} \lstinputlisting[language=sail]{sail_latex/sailfngetTLBEntrypfnone.tex}}

\newcommand{\sailsetTLBEntrypfnone}{\label{zzysetzyTLBEntryzypfnone} \lstinputlisting[language=sail]{sail_latex/sailsetTLBEntrypfnone.tex}}

\newcommand{\sailfnsetTLBEntrypfnone}{\label{zzysetzyTLBEntryzypfnone} \lstinputlisting[language=sail]{sail_latex/sailfnsetTLBEntrypfnone.tex}}

\newcommand{\sailupdateTLBEntrypfnone}{\label{zzyupdatezyTLBEntryzypfnone} \lstinputlisting[language=sail]{sail_latex/sailupdateTLBEntrypfnone.tex}}

\newcommand{\sailfnupdateTLBEntrypfnone}{\label{zzyupdatezyTLBEntryzypfnone} \lstinputlisting[language=sail]{sail_latex/sailfnupdateTLBEntrypfnone.tex}}

\newcommand{\sailupdatepfnone}{\label{zupdatezypfnone} \lstinputlisting[language=sail]{sail_latex/sailupdatepfnone.tex}}

\newcommand{\sailmodpfnone}{\label{zzymodzypfnone} \lstinputlisting[language=sail]{sail_latex/sailmodpfnone.tex}}

\newcommand{\sailgetTLBEntrycone}{\label{zzygetzyTLBEntryzycone} \lstinputlisting[language=sail]{sail_latex/sailgetTLBEntrycone.tex}}

\newcommand{\sailfngetTLBEntrycone}{\label{zzygetzyTLBEntryzycone} \lstinputlisting[language=sail]{sail_latex/sailfngetTLBEntrycone.tex}}

\newcommand{\sailsetTLBEntrycone}{\label{zzysetzyTLBEntryzycone} \lstinputlisting[language=sail]{sail_latex/sailsetTLBEntrycone.tex}}

\newcommand{\sailfnsetTLBEntrycone}{\label{zzysetzyTLBEntryzycone} \lstinputlisting[language=sail]{sail_latex/sailfnsetTLBEntrycone.tex}}

\newcommand{\sailupdateTLBEntrycone}{\label{zzyupdatezyTLBEntryzycone} \lstinputlisting[language=sail]{sail_latex/sailupdateTLBEntrycone.tex}}

\newcommand{\sailfnupdateTLBEntrycone}{\label{zzyupdatezyTLBEntryzycone} \lstinputlisting[language=sail]{sail_latex/sailfnupdateTLBEntrycone.tex}}

\newcommand{\sailupdatecone}{\label{zupdatezycone} \lstinputlisting[language=sail]{sail_latex/sailupdatecone.tex}}

\newcommand{\sailmodcone}{\label{zzymodzycone} \lstinputlisting[language=sail]{sail_latex/sailmodcone.tex}}

\newcommand{\sailgetTLBEntrydone}{\label{zzygetzyTLBEntryzydone} \lstinputlisting[language=sail]{sail_latex/sailgetTLBEntrydone.tex}}

\newcommand{\sailfngetTLBEntrydone}{\label{zzygetzyTLBEntryzydone} \lstinputlisting[language=sail]{sail_latex/sailfngetTLBEntrydone.tex}}

\newcommand{\sailsetTLBEntrydone}{\label{zzysetzyTLBEntryzydone} \lstinputlisting[language=sail]{sail_latex/sailsetTLBEntrydone.tex}}

\newcommand{\sailfnsetTLBEntrydone}{\label{zzysetzyTLBEntryzydone} \lstinputlisting[language=sail]{sail_latex/sailfnsetTLBEntrydone.tex}}

\newcommand{\sailupdateTLBEntrydone}{\label{zzyupdatezyTLBEntryzydone} \lstinputlisting[language=sail]{sail_latex/sailupdateTLBEntrydone.tex}}

\newcommand{\sailfnupdateTLBEntrydone}{\label{zzyupdatezyTLBEntryzydone} \lstinputlisting[language=sail]{sail_latex/sailfnupdateTLBEntrydone.tex}}

\newcommand{\sailupdatedone}{\label{zupdatezydone} \lstinputlisting[language=sail]{sail_latex/sailupdatedone.tex}}

\newcommand{\sailmoddone}{\label{zzymodzydone} \lstinputlisting[language=sail]{sail_latex/sailmoddone.tex}}

\newcommand{\sailgetTLBEntryvone}{\label{zzygetzyTLBEntryzyvone} \lstinputlisting[language=sail]{sail_latex/sailgetTLBEntryvone.tex}}

\newcommand{\sailfngetTLBEntryvone}{\label{zzygetzyTLBEntryzyvone} \lstinputlisting[language=sail]{sail_latex/sailfngetTLBEntryvone.tex}}

\newcommand{\sailsetTLBEntryvone}{\label{zzysetzyTLBEntryzyvone} \lstinputlisting[language=sail]{sail_latex/sailsetTLBEntryvone.tex}}

\newcommand{\sailfnsetTLBEntryvone}{\label{zzysetzyTLBEntryzyvone} \lstinputlisting[language=sail]{sail_latex/sailfnsetTLBEntryvone.tex}}

\newcommand{\sailupdateTLBEntryvone}{\label{zzyupdatezyTLBEntryzyvone} \lstinputlisting[language=sail]{sail_latex/sailupdateTLBEntryvone.tex}}

\newcommand{\sailfnupdateTLBEntryvone}{\label{zzyupdatezyTLBEntryzyvone} \lstinputlisting[language=sail]{sail_latex/sailfnupdateTLBEntryvone.tex}}

\newcommand{\sailupdatevone}{\label{zupdatezyvone} \lstinputlisting[language=sail]{sail_latex/sailupdatevone.tex}}

\newcommand{\sailmodvone}{\label{zzymodzyvone} \lstinputlisting[language=sail]{sail_latex/sailmodvone.tex}}

\newcommand{\sailgetTLBEntrycapszero}{\label{zzygetzyTLBEntryzycapszero} \lstinputlisting[language=sail]{sail_latex/sailgetTLBEntrycapszero.tex}}

\newcommand{\sailfngetTLBEntrycapszero}{\label{zzygetzyTLBEntryzycapszero} \lstinputlisting[language=sail]{sail_latex/sailfngetTLBEntrycapszero.tex}}

\newcommand{\sailsetTLBEntrycapszero}{\label{zzysetzyTLBEntryzycapszero} \lstinputlisting[language=sail]{sail_latex/sailsetTLBEntrycapszero.tex}}

\newcommand{\sailfnsetTLBEntrycapszero}{\label{zzysetzyTLBEntryzycapszero} \lstinputlisting[language=sail]{sail_latex/sailfnsetTLBEntrycapszero.tex}}

\newcommand{\sailupdateTLBEntrycapszero}{\label{zzyupdatezyTLBEntryzycapszero} \lstinputlisting[language=sail]{sail_latex/sailupdateTLBEntrycapszero.tex}}

\newcommand{\sailfnupdateTLBEntrycapszero}{\label{zzyupdatezyTLBEntryzycapszero} \lstinputlisting[language=sail]{sail_latex/sailfnupdateTLBEntrycapszero.tex}}

\newcommand{\sailupdatecapszero}{\label{zupdatezycapszero} \lstinputlisting[language=sail]{sail_latex/sailupdatecapszero.tex}}

\newcommand{\sailmodcapszero}{\label{zzymodzycapszero} \lstinputlisting[language=sail]{sail_latex/sailmodcapszero.tex}}

\newcommand{\sailgetTLBEntrycaplzero}{\label{zzygetzyTLBEntryzycaplzero} \lstinputlisting[language=sail]{sail_latex/sailgetTLBEntrycaplzero.tex}}

\newcommand{\sailfngetTLBEntrycaplzero}{\label{zzygetzyTLBEntryzycaplzero} \lstinputlisting[language=sail]{sail_latex/sailfngetTLBEntrycaplzero.tex}}

\newcommand{\sailsetTLBEntrycaplzero}{\label{zzysetzyTLBEntryzycaplzero} \lstinputlisting[language=sail]{sail_latex/sailsetTLBEntrycaplzero.tex}}

\newcommand{\sailfnsetTLBEntrycaplzero}{\label{zzysetzyTLBEntryzycaplzero} \lstinputlisting[language=sail]{sail_latex/sailfnsetTLBEntrycaplzero.tex}}

\newcommand{\sailupdateTLBEntrycaplzero}{\label{zzyupdatezyTLBEntryzycaplzero} \lstinputlisting[language=sail]{sail_latex/sailupdateTLBEntrycaplzero.tex}}

\newcommand{\sailfnupdateTLBEntrycaplzero}{\label{zzyupdatezyTLBEntryzycaplzero} \lstinputlisting[language=sail]{sail_latex/sailfnupdateTLBEntrycaplzero.tex}}

\newcommand{\sailupdatecaplzero}{\label{zupdatezycaplzero} \lstinputlisting[language=sail]{sail_latex/sailupdatecaplzero.tex}}

\newcommand{\sailmodcaplzero}{\label{zzymodzycaplzero} \lstinputlisting[language=sail]{sail_latex/sailmodcaplzero.tex}}

\newcommand{\sailgetTLBEntrypfnzero}{\label{zzygetzyTLBEntryzypfnzero} \lstinputlisting[language=sail]{sail_latex/sailgetTLBEntrypfnzero.tex}}

\newcommand{\sailfngetTLBEntrypfnzero}{\label{zzygetzyTLBEntryzypfnzero} \lstinputlisting[language=sail]{sail_latex/sailfngetTLBEntrypfnzero.tex}}

\newcommand{\sailsetTLBEntrypfnzero}{\label{zzysetzyTLBEntryzypfnzero} \lstinputlisting[language=sail]{sail_latex/sailsetTLBEntrypfnzero.tex}}

\newcommand{\sailfnsetTLBEntrypfnzero}{\label{zzysetzyTLBEntryzypfnzero} \lstinputlisting[language=sail]{sail_latex/sailfnsetTLBEntrypfnzero.tex}}

\newcommand{\sailupdateTLBEntrypfnzero}{\label{zzyupdatezyTLBEntryzypfnzero} \lstinputlisting[language=sail]{sail_latex/sailupdateTLBEntrypfnzero.tex}}

\newcommand{\sailfnupdateTLBEntrypfnzero}{\label{zzyupdatezyTLBEntryzypfnzero} \lstinputlisting[language=sail]{sail_latex/sailfnupdateTLBEntrypfnzero.tex}}

\newcommand{\sailupdatepfnzero}{\label{zupdatezypfnzero} \lstinputlisting[language=sail]{sail_latex/sailupdatepfnzero.tex}}

\newcommand{\sailmodpfnzero}{\label{zzymodzypfnzero} \lstinputlisting[language=sail]{sail_latex/sailmodpfnzero.tex}}

\newcommand{\sailgetTLBEntryczero}{\label{zzygetzyTLBEntryzyczero} \lstinputlisting[language=sail]{sail_latex/sailgetTLBEntryczero.tex}}

\newcommand{\sailfngetTLBEntryczero}{\label{zzygetzyTLBEntryzyczero} \lstinputlisting[language=sail]{sail_latex/sailfngetTLBEntryczero.tex}}

\newcommand{\sailsetTLBEntryczero}{\label{zzysetzyTLBEntryzyczero} \lstinputlisting[language=sail]{sail_latex/sailsetTLBEntryczero.tex}}

\newcommand{\sailfnsetTLBEntryczero}{\label{zzysetzyTLBEntryzyczero} \lstinputlisting[language=sail]{sail_latex/sailfnsetTLBEntryczero.tex}}

\newcommand{\sailupdateTLBEntryczero}{\label{zzyupdatezyTLBEntryzyczero} \lstinputlisting[language=sail]{sail_latex/sailupdateTLBEntryczero.tex}}

\newcommand{\sailfnupdateTLBEntryczero}{\label{zzyupdatezyTLBEntryzyczero} \lstinputlisting[language=sail]{sail_latex/sailfnupdateTLBEntryczero.tex}}

\newcommand{\sailupdateczero}{\label{zupdatezyczero} \lstinputlisting[language=sail]{sail_latex/sailupdateczero.tex}}

\newcommand{\sailmodczero}{\label{zzymodzyczero} \lstinputlisting[language=sail]{sail_latex/sailmodczero.tex}}

\newcommand{\sailgetTLBEntrydzero}{\label{zzygetzyTLBEntryzydzero} \lstinputlisting[language=sail]{sail_latex/sailgetTLBEntrydzero.tex}}

\newcommand{\sailfngetTLBEntrydzero}{\label{zzygetzyTLBEntryzydzero} \lstinputlisting[language=sail]{sail_latex/sailfngetTLBEntrydzero.tex}}

\newcommand{\sailsetTLBEntrydzero}{\label{zzysetzyTLBEntryzydzero} \lstinputlisting[language=sail]{sail_latex/sailsetTLBEntrydzero.tex}}

\newcommand{\sailfnsetTLBEntrydzero}{\label{zzysetzyTLBEntryzydzero} \lstinputlisting[language=sail]{sail_latex/sailfnsetTLBEntrydzero.tex}}

\newcommand{\sailupdateTLBEntrydzero}{\label{zzyupdatezyTLBEntryzydzero} \lstinputlisting[language=sail]{sail_latex/sailupdateTLBEntrydzero.tex}}

\newcommand{\sailfnupdateTLBEntrydzero}{\label{zzyupdatezyTLBEntryzydzero} \lstinputlisting[language=sail]{sail_latex/sailfnupdateTLBEntrydzero.tex}}

\newcommand{\sailupdatedzero}{\label{zupdatezydzero} \lstinputlisting[language=sail]{sail_latex/sailupdatedzero.tex}}

\newcommand{\sailmoddzero}{\label{zzymodzydzero} \lstinputlisting[language=sail]{sail_latex/sailmoddzero.tex}}

\newcommand{\sailgetTLBEntryvzero}{\label{zzygetzyTLBEntryzyvzero} \lstinputlisting[language=sail]{sail_latex/sailgetTLBEntryvzero.tex}}

\newcommand{\sailfngetTLBEntryvzero}{\label{zzygetzyTLBEntryzyvzero} \lstinputlisting[language=sail]{sail_latex/sailfngetTLBEntryvzero.tex}}

\newcommand{\sailsetTLBEntryvzero}{\label{zzysetzyTLBEntryzyvzero} \lstinputlisting[language=sail]{sail_latex/sailsetTLBEntryvzero.tex}}

\newcommand{\sailfnsetTLBEntryvzero}{\label{zzysetzyTLBEntryzyvzero} \lstinputlisting[language=sail]{sail_latex/sailfnsetTLBEntryvzero.tex}}

\newcommand{\sailupdateTLBEntryvzero}{\label{zzyupdatezyTLBEntryzyvzero} \lstinputlisting[language=sail]{sail_latex/sailupdateTLBEntryvzero.tex}}

\newcommand{\sailfnupdateTLBEntryvzero}{\label{zzyupdatezyTLBEntryzyvzero} \lstinputlisting[language=sail]{sail_latex/sailfnupdateTLBEntryvzero.tex}}

\newcommand{\sailupdatevzero}{\label{zupdatezyvzero} \lstinputlisting[language=sail]{sail_latex/sailupdatevzero.tex}}

\newcommand{\sailmodvzero}{\label{zzymodzyvzero} \lstinputlisting[language=sail]{sail_latex/sailmodvzero.tex}}

\newcommand{\sailStatusReg}{\label{zStatusReg} \lstinputlisting[language=sail]{sail_latex/sailStatusReg.tex}}

\newcommand{\sailMkStatusReg}{\label{zMkzyStatusReg} \lstinputlisting[language=sail]{sail_latex/sailMkStatusReg.tex}}

\newcommand{\sailfnMkStatusReg}{\label{zMkzyStatusReg} \lstinputlisting[language=sail]{sail_latex/sailfnMkStatusReg.tex}}

\newcommand{\sailgetStatusRegbits}{\label{zzygetzyStatusRegzybits} \lstinputlisting[language=sail]{sail_latex/sailgetStatusRegbits.tex}}

\newcommand{\sailfngetStatusRegbits}{\label{zzygetzyStatusRegzybits} \lstinputlisting[language=sail]{sail_latex/sailfngetStatusRegbits.tex}}

\newcommand{\sailsetStatusRegbits}{\label{zzysetzyStatusRegzybits} \lstinputlisting[language=sail]{sail_latex/sailsetStatusRegbits.tex}}

\newcommand{\sailfnsetStatusRegbits}{\label{zzysetzyStatusRegzybits} \lstinputlisting[language=sail]{sail_latex/sailfnsetStatusRegbits.tex}}

\newcommand{\sailupdateStatusRegbits}{\label{zzyupdatezyStatusRegzybits} \lstinputlisting[language=sail]{sail_latex/sailupdateStatusRegbits.tex}}

\newcommand{\sailfnupdateStatusRegbits}{\label{zzyupdatezyStatusRegzybits} \lstinputlisting[language=sail]{sail_latex/sailfnupdateStatusRegbits.tex}}

\newcommand{\sailsailupdatebitsv}{\label{zupdatezybits} \lstinputlisting[language=sail]{sail_latex/sailsailupdatebitsv.tex}}

\newcommand{\sailsailmodbitsv}{\label{zzymodzybits} \lstinputlisting[language=sail]{sail_latex/sailsailmodbitsv.tex}}

\newcommand{\sailgetStatusRegCU}{\label{zzygetzyStatusRegzyCU} \lstinputlisting[language=sail]{sail_latex/sailgetStatusRegCU.tex}}

\newcommand{\sailfngetStatusRegCU}{\label{zzygetzyStatusRegzyCU} \lstinputlisting[language=sail]{sail_latex/sailfngetStatusRegCU.tex}}

\newcommand{\sailsetStatusRegCU}{\label{zzysetzyStatusRegzyCU} \lstinputlisting[language=sail]{sail_latex/sailsetStatusRegCU.tex}}

\newcommand{\sailfnsetStatusRegCU}{\label{zzysetzyStatusRegzyCU} \lstinputlisting[language=sail]{sail_latex/sailfnsetStatusRegCU.tex}}

\newcommand{\sailupdateStatusRegCU}{\label{zzyupdatezyStatusRegzyCU} \lstinputlisting[language=sail]{sail_latex/sailupdateStatusRegCU.tex}}

\newcommand{\sailfnupdateStatusRegCU}{\label{zzyupdatezyStatusRegzyCU} \lstinputlisting[language=sail]{sail_latex/sailfnupdateStatusRegCU.tex}}

\newcommand{\sailupdateCU}{\label{zupdatezyCU} \lstinputlisting[language=sail]{sail_latex/sailupdateCU.tex}}

\newcommand{\sailmodCU}{\label{zzymodzyCU} \lstinputlisting[language=sail]{sail_latex/sailmodCU.tex}}

\newcommand{\sailgetStatusRegBEV}{\label{zzygetzyStatusRegzyBEV} \lstinputlisting[language=sail]{sail_latex/sailgetStatusRegBEV.tex}}

\newcommand{\sailfngetStatusRegBEV}{\label{zzygetzyStatusRegzyBEV} \lstinputlisting[language=sail]{sail_latex/sailfngetStatusRegBEV.tex}}

\newcommand{\sailsetStatusRegBEV}{\label{zzysetzyStatusRegzyBEV} \lstinputlisting[language=sail]{sail_latex/sailsetStatusRegBEV.tex}}

\newcommand{\sailfnsetStatusRegBEV}{\label{zzysetzyStatusRegzyBEV} \lstinputlisting[language=sail]{sail_latex/sailfnsetStatusRegBEV.tex}}

\newcommand{\sailupdateStatusRegBEV}{\label{zzyupdatezyStatusRegzyBEV} \lstinputlisting[language=sail]{sail_latex/sailupdateStatusRegBEV.tex}}

\newcommand{\sailfnupdateStatusRegBEV}{\label{zzyupdatezyStatusRegzyBEV} \lstinputlisting[language=sail]{sail_latex/sailfnupdateStatusRegBEV.tex}}

\newcommand{\sailupdateBEV}{\label{zupdatezyBEV} \lstinputlisting[language=sail]{sail_latex/sailupdateBEV.tex}}

\newcommand{\sailmodBEV}{\label{zzymodzyBEV} \lstinputlisting[language=sail]{sail_latex/sailmodBEV.tex}}

\newcommand{\sailgetStatusRegIM}{\label{zzygetzyStatusRegzyIM} \lstinputlisting[language=sail]{sail_latex/sailgetStatusRegIM.tex}}

\newcommand{\sailfngetStatusRegIM}{\label{zzygetzyStatusRegzyIM} \lstinputlisting[language=sail]{sail_latex/sailfngetStatusRegIM.tex}}

\newcommand{\sailsetStatusRegIM}{\label{zzysetzyStatusRegzyIM} \lstinputlisting[language=sail]{sail_latex/sailsetStatusRegIM.tex}}

\newcommand{\sailfnsetStatusRegIM}{\label{zzysetzyStatusRegzyIM} \lstinputlisting[language=sail]{sail_latex/sailfnsetStatusRegIM.tex}}

\newcommand{\sailupdateStatusRegIM}{\label{zzyupdatezyStatusRegzyIM} \lstinputlisting[language=sail]{sail_latex/sailupdateStatusRegIM.tex}}

\newcommand{\sailfnupdateStatusRegIM}{\label{zzyupdatezyStatusRegzyIM} \lstinputlisting[language=sail]{sail_latex/sailfnupdateStatusRegIM.tex}}

\newcommand{\sailupdateIM}{\label{zupdatezyIM} \lstinputlisting[language=sail]{sail_latex/sailupdateIM.tex}}

\newcommand{\sailmodIM}{\label{zzymodzyIM} \lstinputlisting[language=sail]{sail_latex/sailmodIM.tex}}

\newcommand{\sailgetStatusRegKX}{\label{zzygetzyStatusRegzyKX} \lstinputlisting[language=sail]{sail_latex/sailgetStatusRegKX.tex}}

\newcommand{\sailfngetStatusRegKX}{\label{zzygetzyStatusRegzyKX} \lstinputlisting[language=sail]{sail_latex/sailfngetStatusRegKX.tex}}

\newcommand{\sailsetStatusRegKX}{\label{zzysetzyStatusRegzyKX} \lstinputlisting[language=sail]{sail_latex/sailsetStatusRegKX.tex}}

\newcommand{\sailfnsetStatusRegKX}{\label{zzysetzyStatusRegzyKX} \lstinputlisting[language=sail]{sail_latex/sailfnsetStatusRegKX.tex}}

\newcommand{\sailupdateStatusRegKX}{\label{zzyupdatezyStatusRegzyKX} \lstinputlisting[language=sail]{sail_latex/sailupdateStatusRegKX.tex}}

\newcommand{\sailfnupdateStatusRegKX}{\label{zzyupdatezyStatusRegzyKX} \lstinputlisting[language=sail]{sail_latex/sailfnupdateStatusRegKX.tex}}

\newcommand{\sailupdateKX}{\label{zupdatezyKX} \lstinputlisting[language=sail]{sail_latex/sailupdateKX.tex}}

\newcommand{\sailmodKX}{\label{zzymodzyKX} \lstinputlisting[language=sail]{sail_latex/sailmodKX.tex}}

\newcommand{\sailgetStatusRegSX}{\label{zzygetzyStatusRegzySX} \lstinputlisting[language=sail]{sail_latex/sailgetStatusRegSX.tex}}

\newcommand{\sailfngetStatusRegSX}{\label{zzygetzyStatusRegzySX} \lstinputlisting[language=sail]{sail_latex/sailfngetStatusRegSX.tex}}

\newcommand{\sailsetStatusRegSX}{\label{zzysetzyStatusRegzySX} \lstinputlisting[language=sail]{sail_latex/sailsetStatusRegSX.tex}}

\newcommand{\sailfnsetStatusRegSX}{\label{zzysetzyStatusRegzySX} \lstinputlisting[language=sail]{sail_latex/sailfnsetStatusRegSX.tex}}

\newcommand{\sailupdateStatusRegSX}{\label{zzyupdatezyStatusRegzySX} \lstinputlisting[language=sail]{sail_latex/sailupdateStatusRegSX.tex}}

\newcommand{\sailfnupdateStatusRegSX}{\label{zzyupdatezyStatusRegzySX} \lstinputlisting[language=sail]{sail_latex/sailfnupdateStatusRegSX.tex}}

\newcommand{\sailupdateSX}{\label{zupdatezySX} \lstinputlisting[language=sail]{sail_latex/sailupdateSX.tex}}

\newcommand{\sailmodSX}{\label{zzymodzySX} \lstinputlisting[language=sail]{sail_latex/sailmodSX.tex}}

\newcommand{\sailgetStatusRegUX}{\label{zzygetzyStatusRegzyUX} \lstinputlisting[language=sail]{sail_latex/sailgetStatusRegUX.tex}}

\newcommand{\sailfngetStatusRegUX}{\label{zzygetzyStatusRegzyUX} \lstinputlisting[language=sail]{sail_latex/sailfngetStatusRegUX.tex}}

\newcommand{\sailsetStatusRegUX}{\label{zzysetzyStatusRegzyUX} \lstinputlisting[language=sail]{sail_latex/sailsetStatusRegUX.tex}}

\newcommand{\sailfnsetStatusRegUX}{\label{zzysetzyStatusRegzyUX} \lstinputlisting[language=sail]{sail_latex/sailfnsetStatusRegUX.tex}}

\newcommand{\sailupdateStatusRegUX}{\label{zzyupdatezyStatusRegzyUX} \lstinputlisting[language=sail]{sail_latex/sailupdateStatusRegUX.tex}}

\newcommand{\sailfnupdateStatusRegUX}{\label{zzyupdatezyStatusRegzyUX} \lstinputlisting[language=sail]{sail_latex/sailfnupdateStatusRegUX.tex}}

\newcommand{\sailupdateUX}{\label{zupdatezyUX} \lstinputlisting[language=sail]{sail_latex/sailupdateUX.tex}}

\newcommand{\sailmodUX}{\label{zzymodzyUX} \lstinputlisting[language=sail]{sail_latex/sailmodUX.tex}}

\newcommand{\sailgetStatusRegKSU}{\label{zzygetzyStatusRegzyKSU} \lstinputlisting[language=sail]{sail_latex/sailgetStatusRegKSU.tex}}

\newcommand{\sailfngetStatusRegKSU}{\label{zzygetzyStatusRegzyKSU} \lstinputlisting[language=sail]{sail_latex/sailfngetStatusRegKSU.tex}}

\newcommand{\sailsetStatusRegKSU}{\label{zzysetzyStatusRegzyKSU} \lstinputlisting[language=sail]{sail_latex/sailsetStatusRegKSU.tex}}

\newcommand{\sailfnsetStatusRegKSU}{\label{zzysetzyStatusRegzyKSU} \lstinputlisting[language=sail]{sail_latex/sailfnsetStatusRegKSU.tex}}

\newcommand{\sailupdateStatusRegKSU}{\label{zzyupdatezyStatusRegzyKSU} \lstinputlisting[language=sail]{sail_latex/sailupdateStatusRegKSU.tex}}

\newcommand{\sailfnupdateStatusRegKSU}{\label{zzyupdatezyStatusRegzyKSU} \lstinputlisting[language=sail]{sail_latex/sailfnupdateStatusRegKSU.tex}}

\newcommand{\sailupdateKSU}{\label{zupdatezyKSU} \lstinputlisting[language=sail]{sail_latex/sailupdateKSU.tex}}

\newcommand{\sailmodKSU}{\label{zzymodzyKSU} \lstinputlisting[language=sail]{sail_latex/sailmodKSU.tex}}

\newcommand{\sailgetStatusRegERL}{\label{zzygetzyStatusRegzyERL} \lstinputlisting[language=sail]{sail_latex/sailgetStatusRegERL.tex}}

\newcommand{\sailfngetStatusRegERL}{\label{zzygetzyStatusRegzyERL} \lstinputlisting[language=sail]{sail_latex/sailfngetStatusRegERL.tex}}

\newcommand{\sailsetStatusRegERL}{\label{zzysetzyStatusRegzyERL} \lstinputlisting[language=sail]{sail_latex/sailsetStatusRegERL.tex}}

\newcommand{\sailfnsetStatusRegERL}{\label{zzysetzyStatusRegzyERL} \lstinputlisting[language=sail]{sail_latex/sailfnsetStatusRegERL.tex}}

\newcommand{\sailupdateStatusRegERL}{\label{zzyupdatezyStatusRegzyERL} \lstinputlisting[language=sail]{sail_latex/sailupdateStatusRegERL.tex}}

\newcommand{\sailfnupdateStatusRegERL}{\label{zzyupdatezyStatusRegzyERL} \lstinputlisting[language=sail]{sail_latex/sailfnupdateStatusRegERL.tex}}

\newcommand{\sailupdateERL}{\label{zupdatezyERL} \lstinputlisting[language=sail]{sail_latex/sailupdateERL.tex}}

\newcommand{\sailmodERL}{\label{zzymodzyERL} \lstinputlisting[language=sail]{sail_latex/sailmodERL.tex}}

\newcommand{\sailgetStatusRegEXL}{\label{zzygetzyStatusRegzyEXL} \lstinputlisting[language=sail]{sail_latex/sailgetStatusRegEXL.tex}}

\newcommand{\sailfngetStatusRegEXL}{\label{zzygetzyStatusRegzyEXL} \lstinputlisting[language=sail]{sail_latex/sailfngetStatusRegEXL.tex}}

\newcommand{\sailsetStatusRegEXL}{\label{zzysetzyStatusRegzyEXL} \lstinputlisting[language=sail]{sail_latex/sailsetStatusRegEXL.tex}}

\newcommand{\sailfnsetStatusRegEXL}{\label{zzysetzyStatusRegzyEXL} \lstinputlisting[language=sail]{sail_latex/sailfnsetStatusRegEXL.tex}}

\newcommand{\sailupdateStatusRegEXL}{\label{zzyupdatezyStatusRegzyEXL} \lstinputlisting[language=sail]{sail_latex/sailupdateStatusRegEXL.tex}}

\newcommand{\sailfnupdateStatusRegEXL}{\label{zzyupdatezyStatusRegzyEXL} \lstinputlisting[language=sail]{sail_latex/sailfnupdateStatusRegEXL.tex}}

\newcommand{\sailupdateEXL}{\label{zupdatezyEXL} \lstinputlisting[language=sail]{sail_latex/sailupdateEXL.tex}}

\newcommand{\sailmodEXL}{\label{zzymodzyEXL} \lstinputlisting[language=sail]{sail_latex/sailmodEXL.tex}}

\newcommand{\sailgetStatusRegIE}{\label{zzygetzyStatusRegzyIE} \lstinputlisting[language=sail]{sail_latex/sailgetStatusRegIE.tex}}

\newcommand{\sailfngetStatusRegIE}{\label{zzygetzyStatusRegzyIE} \lstinputlisting[language=sail]{sail_latex/sailfngetStatusRegIE.tex}}

\newcommand{\sailsetStatusRegIE}{\label{zzysetzyStatusRegzyIE} \lstinputlisting[language=sail]{sail_latex/sailsetStatusRegIE.tex}}

\newcommand{\sailfnsetStatusRegIE}{\label{zzysetzyStatusRegzyIE} \lstinputlisting[language=sail]{sail_latex/sailfnsetStatusRegIE.tex}}

\newcommand{\sailupdateStatusRegIE}{\label{zzyupdatezyStatusRegzyIE} \lstinputlisting[language=sail]{sail_latex/sailupdateStatusRegIE.tex}}

\newcommand{\sailfnupdateStatusRegIE}{\label{zzyupdatezyStatusRegzyIE} \lstinputlisting[language=sail]{sail_latex/sailfnupdateStatusRegIE.tex}}

\newcommand{\sailupdateIE}{\label{zupdatezyIE} \lstinputlisting[language=sail]{sail_latex/sailupdateIE.tex}}

\newcommand{\sailmodIE}{\label{zzymodzyIE} \lstinputlisting[language=sail]{sail_latex/sailmodIE.tex}}

\newcommand{\sailexecutebranch}{\label{zexecutezybranch} \lstinputlisting[language=sail]{sail_latex/sailexecutebranch.tex}}

\newcommand{\sailfnexecutebranch}{\label{zexecutezybranch} \lstinputlisting[language=sail]{sail_latex/sailfnexecutebranch.tex}}

\newcommand{\sailNotWordVal}{\label{zNotWordVal} \lstinputlisting[language=sail]{sail_latex/sailNotWordVal.tex}}

\newcommand{\sailfnNotWordVal}{\label{zNotWordVal} \lstinputlisting[language=sail]{sail_latex/sailfnNotWordVal.tex}}

\newcommand{\sailrGPR}{\label{zrGPR} \lstinputlisting[language=sail]{sail_latex/sailrGPR.tex}}

\newcommand{\sailfnrGPR}{\label{zrGPR} \lstinputlisting[language=sail]{sail_latex/sailfnrGPR.tex}}

\newcommand{\sailwGPR}{\label{zwGPR} \lstinputlisting[language=sail]{sail_latex/sailwGPR.tex}}

\newcommand{\sailfnwGPR}{\label{zwGPR} \lstinputlisting[language=sail]{sail_latex/sailfnwGPR.tex}}

\newcommand{\sailMEMr}{\label{zMEMr} \lstinputlisting[language=sail]{sail_latex/sailMEMr.tex}}

\newcommand{\sailMEMrreserve}{\label{zMEMrzyreserve} \lstinputlisting[language=sail]{sail_latex/sailMEMrreserve.tex}}

\newcommand{\sailMEMsync}{\label{zMEMzysync} \lstinputlisting[language=sail]{sail_latex/sailMEMsync.tex}}

\newcommand{\sailMEMea}{\label{zMEMea} \lstinputlisting[language=sail]{sail_latex/sailMEMea.tex}}

\newcommand{\sailMEMeaconditional}{\label{zMEMeazyconditional} \lstinputlisting[language=sail]{sail_latex/sailMEMeaconditional.tex}}

\newcommand{\sailMEMval}{\label{zMEMval} \lstinputlisting[language=sail]{sail_latex/sailMEMval.tex}}

\newcommand{\sailMEMvalconditional}{\label{zMEMvalzyconditional} \lstinputlisting[language=sail]{sail_latex/sailMEMvalconditional.tex}}

\newcommand{\sailskipeamem}{\label{zskipzyeamem} \lstinputlisting[language=sail]{sail_latex/sailskipeamem.tex}}

\newcommand{\sailskipbarr}{\label{zskipzybarr} \lstinputlisting[language=sail]{sail_latex/sailskipbarr.tex}}

\newcommand{\sailskipwreg}{\label{zskipzywreg} \lstinputlisting[language=sail]{sail_latex/sailskipwreg.tex}}

\newcommand{\sailskiprreg}{\label{zskipzyrreg} \lstinputlisting[language=sail]{sail_latex/sailskiprreg.tex}}

\newcommand{\sailskipwmvt}{\label{zskipzywmvt} \lstinputlisting[language=sail]{sail_latex/sailskipwmvt.tex}}

\newcommand{\sailskiprmemt}{\label{zskipzyrmemt} \lstinputlisting[language=sail]{sail_latex/sailskiprmemt.tex}}

\newcommand{\sailskipescape}{\label{zskipzyescape} \lstinputlisting[language=sail]{sail_latex/sailskipescape.tex}}

\newcommand{\sailfnMEMr}{\label{zMEMr} \lstinputlisting[language=sail]{sail_latex/sailfnMEMr.tex}}

\newcommand{\sailfnMEMrreserve}{\label{zMEMrzyreserve} \lstinputlisting[language=sail]{sail_latex/sailfnMEMrreserve.tex}}

\newcommand{\sailfnMEMsync}{\label{zMEMzysync} \lstinputlisting[language=sail]{sail_latex/sailfnMEMsync.tex}}

\newcommand{\sailfnMEMea}{\label{zMEMea} \lstinputlisting[language=sail]{sail_latex/sailfnMEMea.tex}}

\newcommand{\sailfnMEMeaconditional}{\label{zMEMeazyconditional} \lstinputlisting[language=sail]{sail_latex/sailfnMEMeaconditional.tex}}

\newcommand{\sailfnMEMval}{\label{zMEMval} \lstinputlisting[language=sail]{sail_latex/sailfnMEMval.tex}}

\newcommand{\sailfnMEMvalconditional}{\label{zMEMvalzyconditional} \lstinputlisting[language=sail]{sail_latex/sailfnMEMvalconditional.tex}}

\newcommand{\sailException}{\label{zException} \lstinputlisting[language=sail]{sail_latex/sailException.tex}}

\newcommand{\sailExceptionofnum}{\label{zExceptionzyofzynum} \lstinputlisting[language=sail]{sail_latex/sailExceptionofnum.tex}}

\newcommand{\sailfnExceptionofnum}{\label{zExceptionzyofzynum} \lstinputlisting[language=sail]{sail_latex/sailfnExceptionofnum.tex}}

\newcommand{\sailnumofException}{\label{znumzyofzyException} \lstinputlisting[language=sail]{sail_latex/sailnumofException.tex}}

\newcommand{\sailfnnumofException}{\label{znumzyofzyException} \lstinputlisting[language=sail]{sail_latex/sailfnnumofException.tex}}

\newcommand{\sailExceptionCode}{\label{zExceptionCode} \lstinputlisting[language=sail]{sail_latex/sailExceptionCode.tex}}

\newcommand{\sailfnExceptionCode}{\label{zExceptionCode} \lstinputlisting[language=sail]{sail_latex/sailfnExceptionCode.tex}}

\newcommand{\sailSignalExceptionMIPS}{\label{zSignalExceptionMIPS} \lstinputlisting[language=sail]{sail_latex/sailSignalExceptionMIPS.tex}}

\newcommand{\sailfnSignalExceptionMIPS}{\label{zSignalExceptionMIPS} \lstinputlisting[language=sail]{sail_latex/sailfnSignalExceptionMIPS.tex}}

\newcommand{\sailSignalException}{\label{zSignalException} \lstinputlisting[language=sail]{sail_latex/sailSignalException.tex}}

\newcommand{\sailSignalExceptionBadAddr}{\label{zSignalExceptionBadAddr} \lstinputlisting[language=sail]{sail_latex/sailSignalExceptionBadAddr.tex}}

\newcommand{\sailfnSignalExceptionBadAddr}{\label{zSignalExceptionBadAddr} \lstinputlisting[language=sail]{sail_latex/sailfnSignalExceptionBadAddr.tex}}

\newcommand{\sailSignalExceptionTLB}{\label{zSignalExceptionTLB} \lstinputlisting[language=sail]{sail_latex/sailSignalExceptionTLB.tex}}

\newcommand{\sailfnSignalExceptionTLB}{\label{zSignalExceptionTLB} \lstinputlisting[language=sail]{sail_latex/sailfnSignalExceptionTLB.tex}}

\newcommand{\sailMemAccessType}{\label{zMemAccessType} \lstinputlisting[language=sail]{sail_latex/sailMemAccessType.tex}}

\newcommand{\sailMemAccessTypeofnum}{\label{zMemAccessTypezyofzynum} \lstinputlisting[language=sail]{sail_latex/sailMemAccessTypeofnum.tex}}

\newcommand{\sailfnMemAccessTypeofnum}{\label{zMemAccessTypezyofzynum} \lstinputlisting[language=sail]{sail_latex/sailfnMemAccessTypeofnum.tex}}

\newcommand{\sailnumofMemAccessType}{\label{znumzyofzyMemAccessType} \lstinputlisting[language=sail]{sail_latex/sailnumofMemAccessType.tex}}

\newcommand{\sailfnnumofMemAccessType}{\label{znumzyofzyMemAccessType} \lstinputlisting[language=sail]{sail_latex/sailfnnumofMemAccessType.tex}}

\newcommand{\sailAccessLevel}{\label{zAccessLevel} \lstinputlisting[language=sail]{sail_latex/sailAccessLevel.tex}}

\newcommand{\sailAccessLevelofnum}{\label{zAccessLevelzyofzynum} \lstinputlisting[language=sail]{sail_latex/sailAccessLevelofnum.tex}}

\newcommand{\sailfnAccessLevelofnum}{\label{zAccessLevelzyofzynum} \lstinputlisting[language=sail]{sail_latex/sailfnAccessLevelofnum.tex}}

\newcommand{\sailnumofAccessLevel}{\label{znumzyofzyAccessLevel} \lstinputlisting[language=sail]{sail_latex/sailnumofAccessLevel.tex}}

\newcommand{\sailfnnumofAccessLevel}{\label{znumzyofzyAccessLevel} \lstinputlisting[language=sail]{sail_latex/sailfnnumofAccessLevel.tex}}

\newcommand{\sailintofAccessLevel}{\label{zintzyofzyAccessLevel} \lstinputlisting[language=sail]{sail_latex/sailintofAccessLevel.tex}}

\newcommand{\sailfnintofAccessLevel}{\label{zintzyofzyAccessLevel} \lstinputlisting[language=sail]{sail_latex/sailfnintofAccessLevel.tex}}

\newcommand{\sailgrantsAccess}{\label{zgrantsAccess} 
Returns whether the first AccessLevel is sufficient to grant access at the second, required, access level.
 \lstinputlisting[language=sail]{sail_latex/sailgrantsAccess.tex}}

\newcommand{\sailfngrantsAccess}{\label{zgrantsAccess} \lstinputlisting[language=sail]{sail_latex/sailfngrantsAccess.tex}}

\newcommand{\sailgetAccessLevel}{\label{zgetAccessLevel} 
Returns the current effective access level determined by accessing the relevant parts of the MIPS status register.
 \lstinputlisting[language=sail]{sail_latex/sailgetAccessLevel.tex}}

\newcommand{\sailfngetAccessLevel}{\label{zgetAccessLevel} \lstinputlisting[language=sail]{sail_latex/sailfngetAccessLevel.tex}}

\newcommand{\sailcheckCPzeroAccess}{\label{zcheckCPzeroAccess} \lstinputlisting[language=sail]{sail_latex/sailcheckCPzeroAccess.tex}}

\newcommand{\sailfncheckCPzeroAccess}{\label{zcheckCPzeroAccess} \lstinputlisting[language=sail]{sail_latex/sailfncheckCPzeroAccess.tex}}

\newcommand{\sailincrementCPzeroCount}{\label{zincrementCPzeroCount} \lstinputlisting[language=sail]{sail_latex/sailincrementCPzeroCount.tex}}

\newcommand{\sailfnincrementCPzeroCount}{\label{zincrementCPzeroCount} \lstinputlisting[language=sail]{sail_latex/sailfnincrementCPzeroCount.tex}}

\newcommand{\sailregno}{\label{zregno} \lstinputlisting[language=sail]{sail_latex/sailregno.tex}}

\newcommand{\sailimmonesix}{\label{zimmonesix} \lstinputlisting[language=sail]{sail_latex/sailimmonesix.tex}}

\newcommand{\sailregregreg}{\label{zregregreg} \lstinputlisting[language=sail]{sail_latex/sailregregreg.tex}}

\newcommand{\sailregregimmonesix}{\label{zregregimmonesix} \lstinputlisting[language=sail]{sail_latex/sailregregimmonesix.tex}}

\newcommand{\saildecodefailure}{\label{zdecodezyfailure} \lstinputlisting[language=sail]{sail_latex/saildecodefailure.tex}}

\newcommand{\saildecodefailureofnum}{\label{zdecodezyfailurezyofzynum} \lstinputlisting[language=sail]{sail_latex/saildecodefailureofnum.tex}}

\newcommand{\sailfndecodefailureofnum}{\label{zdecodezyfailurezyofzynum} \lstinputlisting[language=sail]{sail_latex/sailfndecodefailureofnum.tex}}

\newcommand{\sailnumofdecodefailure}{\label{znumzyofzydecodezyfailure} \lstinputlisting[language=sail]{sail_latex/sailnumofdecodefailure.tex}}

\newcommand{\sailfnnumofdecodefailure}{\label{znumzyofzydecodezyfailure} \lstinputlisting[language=sail]{sail_latex/sailfnnumofdecodefailure.tex}}

\newcommand{\sailComparison}{\label{zComparison} \lstinputlisting[language=sail]{sail_latex/sailComparison.tex}}

\newcommand{\sailComparisonofnum}{\label{zComparisonzyofzynum} \lstinputlisting[language=sail]{sail_latex/sailComparisonofnum.tex}}

\newcommand{\sailfnComparisonofnum}{\label{zComparisonzyofzynum} \lstinputlisting[language=sail]{sail_latex/sailfnComparisonofnum.tex}}

\newcommand{\sailnumofComparison}{\label{znumzyofzyComparison} \lstinputlisting[language=sail]{sail_latex/sailnumofComparison.tex}}

\newcommand{\sailfnnumofComparison}{\label{znumzyofzyComparison} \lstinputlisting[language=sail]{sail_latex/sailfnnumofComparison.tex}}

\newcommand{\sailcompare}{\label{zcompare} \lstinputlisting[language=sail]{sail_latex/sailcompare.tex}}

\newcommand{\sailfncompare}{\label{zcompare} \lstinputlisting[language=sail]{sail_latex/sailfncompare.tex}}

\newcommand{\sailWordType}{\label{zWordType} \lstinputlisting[language=sail]{sail_latex/sailWordType.tex}}

\newcommand{\sailWordTypeofnum}{\label{zWordTypezyofzynum} \lstinputlisting[language=sail]{sail_latex/sailWordTypeofnum.tex}}

\newcommand{\sailfnWordTypeofnum}{\label{zWordTypezyofzynum} \lstinputlisting[language=sail]{sail_latex/sailfnWordTypeofnum.tex}}

\newcommand{\sailnumofWordType}{\label{znumzyofzyWordType} \lstinputlisting[language=sail]{sail_latex/sailnumofWordType.tex}}

\newcommand{\sailfnnumofWordType}{\label{znumzyofzyWordType} \lstinputlisting[language=sail]{sail_latex/sailfnnumofWordType.tex}}

\newcommand{\sailWordTypeUnaligned}{\label{zWordTypeUnaligned} \lstinputlisting[language=sail]{sail_latex/sailWordTypeUnaligned.tex}}

\newcommand{\sailWordTypeUnalignedofnum}{\label{zWordTypeUnalignedzyofzynum} \lstinputlisting[language=sail]{sail_latex/sailWordTypeUnalignedofnum.tex}}

\newcommand{\sailfnWordTypeUnalignedofnum}{\label{zWordTypeUnalignedzyofzynum} \lstinputlisting[language=sail]{sail_latex/sailfnWordTypeUnalignedofnum.tex}}

\newcommand{\sailnumofWordTypeUnaligned}{\label{znumzyofzyWordTypeUnaligned} \lstinputlisting[language=sail]{sail_latex/sailnumofWordTypeUnaligned.tex}}

\newcommand{\sailfnnumofWordTypeUnaligned}{\label{znumzyofzyWordTypeUnaligned} \lstinputlisting[language=sail]{sail_latex/sailfnnumofWordTypeUnaligned.tex}}

\newcommand{\sailwordWidthBytes}{\label{zwordWidthBytes} \lstinputlisting[language=sail]{sail_latex/sailwordWidthBytes.tex}}

\newcommand{\sailfnwordWidthBytes}{\label{zwordWidthBytes} \lstinputlisting[language=sail]{sail_latex/sailfnwordWidthBytes.tex}}

\newcommand{\sailisAddressAligned}{\label{zisAddressAligned} \lstinputlisting[language=sail]{sail_latex/sailisAddressAligned.tex}}

\newcommand{\sailfnisAddressAligned}{\label{zisAddressAligned} \lstinputlisting[language=sail]{sail_latex/sailfnisAddressAligned.tex}}

\newcommand{\sailreverseendianness}{\label{zreversezyendianness} \lstinputlisting[language=sail]{sail_latex/sailreverseendianness.tex}}

\newcommand{\sailMEMrwrapper}{\label{zMEMrzywrapper} \lstinputlisting[language=sail]{sail_latex/sailMEMrwrapper.tex}}

\newcommand{\sailfnMEMrwrapper}{\label{zMEMrzywrapper} \lstinputlisting[language=sail]{sail_latex/sailfnMEMrwrapper.tex}}

\newcommand{\sailMEMrreservewrapper}{\label{zMEMrzyreservezywrapper} \lstinputlisting[language=sail]{sail_latex/sailMEMrreservewrapper.tex}}

\newcommand{\sailfnMEMrreservewrapper}{\label{zMEMrzyreservezywrapper} \lstinputlisting[language=sail]{sail_latex/sailfnMEMrreservewrapper.tex}}

\newcommand{\sailinitcpzerostate}{\label{zinitzycpzerozystate} \lstinputlisting[language=sail]{sail_latex/sailinitcpzerostate.tex}}

\newcommand{\sailfninitcpzerostate}{\label{zinitzycpzerozystate} \lstinputlisting[language=sail]{sail_latex/sailfninitcpzerostate.tex}}

\newcommand{\sailinitcptwostate}{\label{zinitzycptwozystate} \lstinputlisting[language=sail]{sail_latex/sailinitcptwostate.tex}}

\newcommand{\sailcptwonextpc}{\label{zcptwozynextzypc} \lstinputlisting[language=sail]{sail_latex/sailcptwonextpc.tex}}

\newcommand{\saildumpcptwostate}{\label{zdumpzycptwozystate} \lstinputlisting[language=sail]{sail_latex/saildumpcptwostate.tex}}

\newcommand{\sailtlbEntryMatch}{\label{ztlbEntryMatch} \lstinputlisting[language=sail]{sail_latex/sailtlbEntryMatch.tex}}

\newcommand{\sailfntlbEntryMatch}{\label{ztlbEntryMatch} \lstinputlisting[language=sail]{sail_latex/sailfntlbEntryMatch.tex}}

\newcommand{\sailtlbSearch}{\label{ztlbSearch} \lstinputlisting[language=sail]{sail_latex/sailtlbSearch.tex}}

\newcommand{\sailfntlbSearch}{\label{ztlbSearch} \lstinputlisting[language=sail]{sail_latex/sailfntlbSearch.tex}}

\newcommand{\sailTLBTranslatetwo}{\label{zTLBTranslatetwo} \lstinputlisting[language=sail]{sail_latex/sailTLBTranslatetwo.tex}}

\newcommand{\sailfnTLBTranslatetwo}{\label{zTLBTranslatetwo} \lstinputlisting[language=sail]{sail_latex/sailfnTLBTranslatetwo.tex}}

\newcommand{\sailTLBTranslateC}{\label{zTLBTranslateC} \lstinputlisting[language=sail]{sail_latex/sailTLBTranslateC.tex}}

\newcommand{\sailfnTLBTranslateC}{\label{zTLBTranslateC} \lstinputlisting[language=sail]{sail_latex/sailfnTLBTranslateC.tex}}

\newcommand{\sailTLBTranslate}{\label{zTLBTranslate} \lstinputlisting[language=sail]{sail_latex/sailTLBTranslate.tex}}

\newcommand{\sailfnTLBTranslate}{\label{zTLBTranslate} \lstinputlisting[language=sail]{sail_latex/sailfnTLBTranslate.tex}}

\newcommand{\sailCapLen}{\label{zCapLen} \lstinputlisting[language=sail]{sail_latex/sailCapLen.tex}}

\newcommand{\sailuintsixfour}{\label{zuintsixfour} \lstinputlisting[language=sail]{sail_latex/sailuintsixfour.tex}}

\newcommand{\sailCPtrCmpOp}{\label{zCPtrCmpOp} \lstinputlisting[language=sail]{sail_latex/sailCPtrCmpOp.tex}}

\newcommand{\sailCPtrCmpOpofnum}{\label{zCPtrCmpOpzyofzynum} \lstinputlisting[language=sail]{sail_latex/sailCPtrCmpOpofnum.tex}}

\newcommand{\sailfnCPtrCmpOpofnum}{\label{zCPtrCmpOpzyofzynum} \lstinputlisting[language=sail]{sail_latex/sailfnCPtrCmpOpofnum.tex}}

\newcommand{\sailnumofCPtrCmpOp}{\label{znumzyofzyCPtrCmpOp} \lstinputlisting[language=sail]{sail_latex/sailnumofCPtrCmpOp.tex}}

\newcommand{\sailfnnumofCPtrCmpOp}{\label{znumzyofzyCPtrCmpOp} \lstinputlisting[language=sail]{sail_latex/sailfnnumofCPtrCmpOp.tex}}

\newcommand{\sailClearRegSet}{\label{zClearRegSet} \lstinputlisting[language=sail]{sail_latex/sailClearRegSet.tex}}

\newcommand{\sailClearRegSetofnum}{\label{zClearRegSetzyofzynum} \lstinputlisting[language=sail]{sail_latex/sailClearRegSetofnum.tex}}

\newcommand{\sailfnClearRegSetofnum}{\label{zClearRegSetzyofzynum} \lstinputlisting[language=sail]{sail_latex/sailfnClearRegSetofnum.tex}}

\newcommand{\sailnumofClearRegSet}{\label{znumzyofzyClearRegSet} \lstinputlisting[language=sail]{sail_latex/sailnumofClearRegSet.tex}}

\newcommand{\sailfnnumofClearRegSet}{\label{znumzyofzyClearRegSet} \lstinputlisting[language=sail]{sail_latex/sailfnnumofClearRegSet.tex}}

\newcommand{\sailCapReg}{\label{zCapReg} \lstinputlisting[language=sail]{sail_latex/sailCapReg.tex}}

\newcommand{\sailCapStruct}{\label{zCapStruct} \lstinputlisting[language=sail]{sail_latex/sailCapStruct.tex}}

\newcommand{\sailcapRegToCapStruct}{\label{zcapRegToCapStruct} \lstinputlisting[language=sail]{sail_latex/sailcapRegToCapStruct.tex}}

\newcommand{\sailfncapRegToCapStruct}{\label{zcapRegToCapStruct} \lstinputlisting[language=sail]{sail_latex/sailfncapRegToCapStruct.tex}}

\newcommand{\sailgetCapPerms}{\label{zgetCapPerms} \lstinputlisting[language=sail]{sail_latex/sailgetCapPerms.tex}}

\newcommand{\sailfngetCapPerms}{\label{zgetCapPerms} \lstinputlisting[language=sail]{sail_latex/sailfngetCapPerms.tex}}

\newcommand{\sailcapStructToMemBitstwofivesix}{\label{zcapStructToMemBitstwofivesix} \lstinputlisting[language=sail]{sail_latex/sailcapStructToMemBitstwofivesix.tex}}

\newcommand{\sailfncapStructToMemBitstwofivesix}{\label{zcapStructToMemBitstwofivesix} \lstinputlisting[language=sail]{sail_latex/sailfncapStructToMemBitstwofivesix.tex}}

\newcommand{\sailcapStructToMemBits}{\label{zcapStructToMemBits} \lstinputlisting[language=sail]{sail_latex/sailcapStructToMemBits.tex}}

\newcommand{\sailfncapStructToMemBits}{\label{zcapStructToMemBits} \lstinputlisting[language=sail]{sail_latex/sailfncapStructToMemBits.tex}}

\newcommand{\sailmemBitsToCapBits}{\label{zmemBitsToCapBits} \lstinputlisting[language=sail]{sail_latex/sailmemBitsToCapBits.tex}}

\newcommand{\sailfnmemBitsToCapBits}{\label{zmemBitsToCapBits} \lstinputlisting[language=sail]{sail_latex/sailfnmemBitsToCapBits.tex}}

\newcommand{\sailcapStructToCapReg}{\label{zcapStructToCapReg} \lstinputlisting[language=sail]{sail_latex/sailcapStructToCapReg.tex}}

\newcommand{\sailfncapStructToCapReg}{\label{zcapStructToCapReg} \lstinputlisting[language=sail]{sail_latex/sailfncapStructToCapReg.tex}}

\newcommand{\sailsetCapPerms}{\label{zsetCapPerms} \lstinputlisting[language=sail]{sail_latex/sailsetCapPerms.tex}}

\newcommand{\sailfnsetCapPerms}{\label{zsetCapPerms} \lstinputlisting[language=sail]{sail_latex/sailfnsetCapPerms.tex}}

\newcommand{\sailsealCap}{\label{zsealCap} \lstinputlisting[language=sail]{sail_latex/sailsealCap.tex}}

\newcommand{\sailfnsealCap}{\label{zsealCap} \lstinputlisting[language=sail]{sail_latex/sailfnsealCap.tex}}

\newcommand{\sailgetCapBase}{\label{zgetCapBase} \lstinputlisting[language=sail]{sail_latex/sailgetCapBase.tex}}

\newcommand{\sailfngetCapBase}{\label{zgetCapBase} \lstinputlisting[language=sail]{sail_latex/sailfngetCapBase.tex}}

\newcommand{\sailgetCapTop}{\label{zgetCapTop} \lstinputlisting[language=sail]{sail_latex/sailgetCapTop.tex}}

\newcommand{\sailfngetCapTop}{\label{zgetCapTop} \lstinputlisting[language=sail]{sail_latex/sailfngetCapTop.tex}}

\newcommand{\sailgetCapOffset}{\label{zgetCapOffset} \lstinputlisting[language=sail]{sail_latex/sailgetCapOffset.tex}}

\newcommand{\sailfngetCapOffset}{\label{zgetCapOffset} \lstinputlisting[language=sail]{sail_latex/sailfngetCapOffset.tex}}

\newcommand{\sailgetCapLength}{\label{zgetCapLength} \lstinputlisting[language=sail]{sail_latex/sailgetCapLength.tex}}

\newcommand{\sailfngetCapLength}{\label{zgetCapLength} \lstinputlisting[language=sail]{sail_latex/sailfngetCapLength.tex}}

\newcommand{\sailgetCapCursor}{\label{zgetCapCursor} \lstinputlisting[language=sail]{sail_latex/sailgetCapCursor.tex}}

\newcommand{\sailfngetCapCursor}{\label{zgetCapCursor} \lstinputlisting[language=sail]{sail_latex/sailfngetCapCursor.tex}}

\newcommand{\sailsetCapOffset}{\label{zsetCapOffset} 
Set the offset capability of the a capability to given value and return the result, along with a boolean indicating true if the operation preserved the existing bounds of the capability.  When using compressed capabilities, setting the offset far outside the capability bounds can cause the result to become unrepresentable (XXX mention guarantees). Additionally in some implementations a fast representablity check may be used that could cause the operation to return failure even though the capability would be representable (XXX provide details). 
 \lstinputlisting[language=sail]{sail_latex/sailsetCapOffset.tex}}

\newcommand{\sailfnsetCapOffset}{\label{zsetCapOffset} \lstinputlisting[language=sail]{sail_latex/sailfnsetCapOffset.tex}}

\newcommand{\sailincCapOffset}{\label{zincCapOffset} 
\function{incCapOffset} is the same as \function{setCapOffset} except that the 64-bit value is added to the current capability offset modulo $2^{64}$ (i.e. signed twos-complement arithemtic).
 \lstinputlisting[language=sail]{sail_latex/sailincCapOffset.tex}}

\newcommand{\sailfnincCapOffset}{\label{zincCapOffset} \lstinputlisting[language=sail]{sail_latex/sailfnincCapOffset.tex}}

\newcommand{\sailsetCapBounds}{\label{zsetCapBounds} 
Returns a capability derived from the given capability by setting the base and top to values provided.  The offset of the resulting capability is zero.  In case the requested bounds are not exactly representable the returned boolean is false and the returned capability has bounds at least including the region bounded by base and top but rounded to representable values.
 \lstinputlisting[language=sail]{sail_latex/sailsetCapBounds.tex}}

\newcommand{\sailfnsetCapBounds}{\label{zsetCapBounds} \lstinputlisting[language=sail]{sail_latex/sailfnsetCapBounds.tex}}

\newcommand{\sailinttocap}{\label{zintzytozycap} \lstinputlisting[language=sail]{sail_latex/sailinttocap.tex}}

\newcommand{\sailfninttocap}{\label{zintzytozycap} \lstinputlisting[language=sail]{sail_latex/sailfninttocap.tex}}

\newcommand{\sailexecute}{\label{zexecute} \lstinputlisting[language=sail]{sail_latex/sailexecute.tex}}

\newcommand{\saildecode}{\label{zdecode} \lstinputlisting[language=sail]{sail_latex/saildecode.tex}}

\newcommand{\sailreadCapReg}{\label{zreadCapReg} 
This function reads a given capability register and returns its contents converted to a CapStruct.
If the argument is zero then the null capability is returned.
\lstinputlisting[language=sail]{sail_latex/sailreadCapReg.tex}}

\newcommand{\sailfnreadCapReg}{\label{zreadCapReg} \lstinputlisting[language=sail]{sail_latex/sailfnreadCapReg.tex}}

\newcommand{\sailreadCapRegDDC}{\label{zreadCapRegDDC} 
This is the same as readCapReg except that when the argument is zero the value of DDC is returned 
instead of the null capability. This is used for instructions that expect an address, where using
null would always generate an exception.
\lstinputlisting[language=sail]{sail_latex/sailreadCapRegDDC.tex}}

\newcommand{\sailfnreadCapRegDDC}{\label{zreadCapRegDDC} \lstinputlisting[language=sail]{sail_latex/sailfnreadCapRegDDC.tex}}

\newcommand{\sailwriteCapReg}{\label{zwriteCapReg} \lstinputlisting[language=sail]{sail_latex/sailwriteCapReg.tex}}

\newcommand{\sailfnwriteCapReg}{\label{zwriteCapReg} \lstinputlisting[language=sail]{sail_latex/sailfnwriteCapReg.tex}}

\newcommand{\sailCapEx}{\label{zCapEx} \lstinputlisting[language=sail]{sail_latex/sailCapEx.tex}}

\newcommand{\sailCapExofnum}{\label{zCapExzyofzynum} \lstinputlisting[language=sail]{sail_latex/sailCapExofnum.tex}}

\newcommand{\sailfnCapExofnum}{\label{zCapExzyofzynum} \lstinputlisting[language=sail]{sail_latex/sailfnCapExofnum.tex}}

\newcommand{\sailnumofCapEx}{\label{znumzyofzyCapEx} \lstinputlisting[language=sail]{sail_latex/sailnumofCapEx.tex}}

\newcommand{\sailfnnumofCapEx}{\label{znumzyofzyCapEx} \lstinputlisting[language=sail]{sail_latex/sailfnnumofCapEx.tex}}

\newcommand{\sailCapExCode}{\label{zCapExCode} \lstinputlisting[language=sail]{sail_latex/sailCapExCode.tex}}

\newcommand{\sailfnCapExCode}{\label{zCapExCode} \lstinputlisting[language=sail]{sail_latex/sailfnCapExCode.tex}}

\newcommand{\sailCapCauseReg}{\label{zCapCauseReg} \lstinputlisting[language=sail]{sail_latex/sailCapCauseReg.tex}}

\newcommand{\sailMkCapCauseReg}{\label{zMkzyCapCauseReg} \lstinputlisting[language=sail]{sail_latex/sailMkCapCauseReg.tex}}

\newcommand{\sailfnMkCapCauseReg}{\label{zMkzyCapCauseReg} \lstinputlisting[language=sail]{sail_latex/sailfnMkCapCauseReg.tex}}

\newcommand{\sailgetCapCauseRegbits}{\label{zzygetzyCapCauseRegzybits} \lstinputlisting[language=sail]{sail_latex/sailgetCapCauseRegbits.tex}}

\newcommand{\sailfngetCapCauseRegbits}{\label{zzygetzyCapCauseRegzybits} \lstinputlisting[language=sail]{sail_latex/sailfngetCapCauseRegbits.tex}}

\newcommand{\sailsetCapCauseRegbits}{\label{zzysetzyCapCauseRegzybits} \lstinputlisting[language=sail]{sail_latex/sailsetCapCauseRegbits.tex}}

\newcommand{\sailfnsetCapCauseRegbits}{\label{zzysetzyCapCauseRegzybits} \lstinputlisting[language=sail]{sail_latex/sailfnsetCapCauseRegbits.tex}}

\newcommand{\sailupdateCapCauseRegbits}{\label{zzyupdatezyCapCauseRegzybits} \lstinputlisting[language=sail]{sail_latex/sailupdateCapCauseRegbits.tex}}

\newcommand{\sailfnupdateCapCauseRegbits}{\label{zzyupdatezyCapCauseRegzybits} \lstinputlisting[language=sail]{sail_latex/sailfnupdateCapCauseRegbits.tex}}

\newcommand{\sailupdatebits}{\label{zupdatezybits} \lstinputlisting[language=sail]{sail_latex/sailupdatebits.tex}}

\newcommand{\sailmodbits}{\label{zzymodzybits} \lstinputlisting[language=sail]{sail_latex/sailmodbits.tex}}

\newcommand{\sailgetCapCauseRegExcCode}{\label{zzygetzyCapCauseRegzyExcCode} \lstinputlisting[language=sail]{sail_latex/sailgetCapCauseRegExcCode.tex}}

\newcommand{\sailfngetCapCauseRegExcCode}{\label{zzygetzyCapCauseRegzyExcCode} \lstinputlisting[language=sail]{sail_latex/sailfngetCapCauseRegExcCode.tex}}

\newcommand{\sailsetCapCauseRegExcCode}{\label{zzysetzyCapCauseRegzyExcCode} \lstinputlisting[language=sail]{sail_latex/sailsetCapCauseRegExcCode.tex}}

\newcommand{\sailfnsetCapCauseRegExcCode}{\label{zzysetzyCapCauseRegzyExcCode} \lstinputlisting[language=sail]{sail_latex/sailfnsetCapCauseRegExcCode.tex}}

\newcommand{\sailupdateCapCauseRegExcCode}{\label{zzyupdatezyCapCauseRegzyExcCode} \lstinputlisting[language=sail]{sail_latex/sailupdateCapCauseRegExcCode.tex}}

\newcommand{\sailfnupdateCapCauseRegExcCode}{\label{zzyupdatezyCapCauseRegzyExcCode} \lstinputlisting[language=sail]{sail_latex/sailfnupdateCapCauseRegExcCode.tex}}

\newcommand{\sailupdateExcCode}{\label{zupdatezyExcCode} \lstinputlisting[language=sail]{sail_latex/sailupdateExcCode.tex}}

\newcommand{\sailmodExcCode}{\label{zzymodzyExcCode} \lstinputlisting[language=sail]{sail_latex/sailmodExcCode.tex}}

\newcommand{\sailgetCapCauseRegRegNum}{\label{zzygetzyCapCauseRegzyRegNum} \lstinputlisting[language=sail]{sail_latex/sailgetCapCauseRegRegNum.tex}}

\newcommand{\sailfngetCapCauseRegRegNum}{\label{zzygetzyCapCauseRegzyRegNum} \lstinputlisting[language=sail]{sail_latex/sailfngetCapCauseRegRegNum.tex}}

\newcommand{\sailsetCapCauseRegRegNum}{\label{zzysetzyCapCauseRegzyRegNum} \lstinputlisting[language=sail]{sail_latex/sailsetCapCauseRegRegNum.tex}}

\newcommand{\sailfnsetCapCauseRegRegNum}{\label{zzysetzyCapCauseRegzyRegNum} \lstinputlisting[language=sail]{sail_latex/sailfnsetCapCauseRegRegNum.tex}}

\newcommand{\sailupdateCapCauseRegRegNum}{\label{zzyupdatezyCapCauseRegzyRegNum} \lstinputlisting[language=sail]{sail_latex/sailupdateCapCauseRegRegNum.tex}}

\newcommand{\sailfnupdateCapCauseRegRegNum}{\label{zzyupdatezyCapCauseRegzyRegNum} \lstinputlisting[language=sail]{sail_latex/sailfnupdateCapCauseRegRegNum.tex}}

\newcommand{\sailupdateRegNum}{\label{zupdatezyRegNum} \lstinputlisting[language=sail]{sail_latex/sailupdateRegNum.tex}}

\newcommand{\sailmodRegNum}{\label{zzymodzyRegNum} \lstinputlisting[language=sail]{sail_latex/sailmodRegNum.tex}}

\newcommand{\sailexecutebranchpcc}{\label{zexecutezybranchzypcc} \lstinputlisting[language=sail]{sail_latex/sailexecutebranchpcc.tex}}

\newcommand{\sailfnexecutebranchpcc}{\label{zexecutezybranchzypcc} \lstinputlisting[language=sail]{sail_latex/sailfnexecutebranchpcc.tex}}

\newcommand{\sailfnSignalException}{\label{zSignalException} \lstinputlisting[language=sail]{sail_latex/sailfnSignalException.tex}}

\newcommand{\sailERETHook}{\label{zERETHook} \lstinputlisting[language=sail]{sail_latex/sailERETHook.tex}}

\newcommand{\sailfnERETHook}{\label{zERETHook} \lstinputlisting[language=sail]{sail_latex/sailfnERETHook.tex}}

\newcommand{\sailraisectwoexceptioneight}{\label{zraisezyctwozyexceptioneight} \lstinputlisting[language=sail]{sail_latex/sailraisectwoexceptioneight.tex}}

\newcommand{\sailfnraisectwoexceptioneight}{\label{zraisezyctwozyexceptioneight} \lstinputlisting[language=sail]{sail_latex/sailfnraisectwoexceptioneight.tex}}

\newcommand{\sailraisectwoexception}{\label{zraisezyctwozyexception} \lstinputlisting[language=sail]{sail_latex/sailraisectwoexception.tex}}

\newcommand{\sailfnraisectwoexception}{\label{zraisezyctwozyexception} \lstinputlisting[language=sail]{sail_latex/sailfnraisectwoexception.tex}}

\newcommand{\sailraisectwoexceptionnoreg}{\label{zraisezyctwozyexceptionzynoreg} \lstinputlisting[language=sail]{sail_latex/sailraisectwoexceptionnoreg.tex}}

\newcommand{\sailfnraisectwoexceptionnoreg}{\label{zraisezyctwozyexceptionzynoreg} \lstinputlisting[language=sail]{sail_latex/sailfnraisectwoexceptionnoreg.tex}}

\newcommand{\sailpccaccesssystemregs}{\label{zpcczyaccesszysystemzyregs} \lstinputlisting[language=sail]{sail_latex/sailpccaccesssystemregs.tex}}

\newcommand{\sailfnpccaccesssystemregs}{\label{zpcczyaccesszysystemzyregs} \lstinputlisting[language=sail]{sail_latex/sailfnpccaccesssystemregs.tex}}

\newcommand{\sailregisterinaccessible}{\label{zregisterzyinaccessible} 
The following function should be called before reading or writing any capability register to check whether it is one of the protected system capabilities. Although it is usually a general purpose capabilty the invoked data capabiltiy (IDC) is restricted in the branch delay slot of the CCall (selector one) instruction to protect the confidentiality and integrity of the invoked sandbox.
 \lstinputlisting[language=sail]{sail_latex/sailregisterinaccessible.tex}}

\newcommand{\sailfnregisterinaccessible}{\label{zregisterzyinaccessible} \lstinputlisting[language=sail]{sail_latex/sailfnregisterinaccessible.tex}}

\newcommand{\sailMEMrtag}{\label{zMEMrzytag} \lstinputlisting[language=sail]{sail_latex/sailMEMrtag.tex}}

\newcommand{\sailMEMwtag}{\label{zMEMwzytag} \lstinputlisting[language=sail]{sail_latex/sailMEMwtag.tex}}

\newcommand{\sailMEMrtagged}{\label{zMEMrzytagged} \lstinputlisting[language=sail]{sail_latex/sailMEMrtagged.tex}}

\newcommand{\sailfnMEMrtagged}{\label{zMEMrzytagged} \lstinputlisting[language=sail]{sail_latex/sailfnMEMrtagged.tex}}

\newcommand{\sailMEMrtaggedreserve}{\label{zMEMrzytaggedzyreserve} \lstinputlisting[language=sail]{sail_latex/sailMEMrtaggedreserve.tex}}

\newcommand{\sailfnMEMrtaggedreserve}{\label{zMEMrzytaggedzyreserve} \lstinputlisting[language=sail]{sail_latex/sailfnMEMrtaggedreserve.tex}}

\newcommand{\sailMEMwtagged}{\label{zMEMwzytagged} \lstinputlisting[language=sail]{sail_latex/sailMEMwtagged.tex}}

\newcommand{\sailfnMEMwtagged}{\label{zMEMwzytagged} \lstinputlisting[language=sail]{sail_latex/sailfnMEMwtagged.tex}}

\newcommand{\sailMEMwtaggedconditional}{\label{zMEMwzytaggedzyconditional} \lstinputlisting[language=sail]{sail_latex/sailMEMwtaggedconditional.tex}}

\newcommand{\sailfnMEMwtaggedconditional}{\label{zMEMwzytaggedzyconditional} \lstinputlisting[language=sail]{sail_latex/sailfnMEMwtaggedconditional.tex}}

\newcommand{\sailMEMwwrapper}{\label{zMEMwzywrapper} \lstinputlisting[language=sail]{sail_latex/sailMEMwwrapper.tex}}

\newcommand{\sailfnMEMwwrapper}{\label{zMEMwzywrapper} \lstinputlisting[language=sail]{sail_latex/sailfnMEMwwrapper.tex}}

\newcommand{\sailMEMwconditionalwrapper}{\label{zMEMwzyconditionalzywrapper} \lstinputlisting[language=sail]{sail_latex/sailMEMwconditionalwrapper.tex}}

\newcommand{\sailfnMEMwconditionalwrapper}{\label{zMEMwzyconditionalzywrapper} \lstinputlisting[language=sail]{sail_latex/sailfnMEMwconditionalwrapper.tex}}

\newcommand{\sailcheckDDCPerms}{\label{zcheckDDCPerms} \lstinputlisting[language=sail]{sail_latex/sailcheckDDCPerms.tex}}

\newcommand{\sailfncheckDDCPerms}{\label{zcheckDDCPerms} \lstinputlisting[language=sail]{sail_latex/sailfncheckDDCPerms.tex}}

\newcommand{\sailaddrWrapper}{\label{zaddrWrapper} \lstinputlisting[language=sail]{sail_latex/sailaddrWrapper.tex}}

\newcommand{\sailfnaddrWrapper}{\label{zaddrWrapper} \lstinputlisting[language=sail]{sail_latex/sailfnaddrWrapper.tex}}

\newcommand{\sailaddrWrapperUnaligned}{\label{zaddrWrapperUnaligned} \lstinputlisting[language=sail]{sail_latex/sailaddrWrapperUnaligned.tex}}

\newcommand{\sailfnaddrWrapperUnaligned}{\label{zaddrWrapperUnaligned} \lstinputlisting[language=sail]{sail_latex/sailfnaddrWrapperUnaligned.tex}}

\newcommand{\sailTranslatePC}{\label{zTranslatePC} \lstinputlisting[language=sail]{sail_latex/sailTranslatePC.tex}}

\newcommand{\sailfnTranslatePC}{\label{zTranslatePC} \lstinputlisting[language=sail]{sail_latex/sailfnTranslatePC.tex}}

\newcommand{\sailcheckCPtwousable}{\label{zcheckCPtwousable}   
All capability instrucitons must first check that the capability
co-processor is enabled using the following function that raises a
co-processor unusable exception if a CP0Status.CU2 is not set. This
allows the operating system to only save and restore the full
capability context for processes that use capabilities.
\lstinputlisting[language=sail]{sail_latex/sailcheckCPtwousable.tex}}

\newcommand{\sailfncheckCPtwousable}{\label{zcheckCPtwousable} \lstinputlisting[language=sail]{sail_latex/sailfncheckCPtwousable.tex}}

\newcommand{\sailfninitcptwostate}{\label{zinitzycptwozystate} \lstinputlisting[language=sail]{sail_latex/sailfninitcptwostate.tex}}

\newcommand{\sailfncptwonextpc}{\label{zcptwozynextzypc} \lstinputlisting[language=sail]{sail_latex/sailfncptwonextpc.tex}}

\newcommand{\sailcapToString}{\label{zcapToString} \lstinputlisting[language=sail]{sail_latex/sailcapToString.tex}}

\newcommand{\sailfncapToString}{\label{zcapToString} \lstinputlisting[language=sail]{sail_latex/sailfncapToString.tex}}

\newcommand{\sailfndumpcptwostate}{\label{zdumpzycptwozystate} \lstinputlisting[language=sail]{sail_latex/sailfndumpcptwostate.tex}}

\newcommand{\sailextendLoad}{\label{zextendLoad} \lstinputlisting[language=sail]{sail_latex/sailextendLoad.tex}}

\newcommand{\sailfnextendLoad}{\label{zextendLoad} \lstinputlisting[language=sail]{sail_latex/sailfnextendLoad.tex}}

\newcommand{\sailTLBWriteEntry}{\label{zTLBWriteEntry} \lstinputlisting[language=sail]{sail_latex/sailTLBWriteEntry.tex}}

\newcommand{\sailfnTLBWriteEntry}{\label{zTLBWriteEntry} \lstinputlisting[language=sail]{sail_latex/sailfnTLBWriteEntry.tex}}

\newcommand{\sailfndecodeSomeDADDIU}{ \lstinputlisting[language=sail]{sail_latex/sailfndecodeSomeDADDIU.tex}}

\newcommand{\sailfndecodeSomeDADDU}{ \lstinputlisting[language=sail]{sail_latex/sailfndecodeSomeDADDU.tex}}

\newcommand{\sailfndecodeSomeDADDI}{ \lstinputlisting[language=sail]{sail_latex/sailfndecodeSomeDADDI.tex}}

\newcommand{\sailfndecodeSomeDADD}{ \lstinputlisting[language=sail]{sail_latex/sailfndecodeSomeDADD.tex}}

\newcommand{\sailfndecodeSomeADD}{ \lstinputlisting[language=sail]{sail_latex/sailfndecodeSomeADD.tex}}

\newcommand{\sailfndecodeSomeADDI}{ \lstinputlisting[language=sail]{sail_latex/sailfndecodeSomeADDI.tex}}

\newcommand{\sailfndecodeSomeADDU}{ \lstinputlisting[language=sail]{sail_latex/sailfndecodeSomeADDU.tex}}

\newcommand{\sailfndecodeSomeADDIU}{ \lstinputlisting[language=sail]{sail_latex/sailfndecodeSomeADDIU.tex}}

\newcommand{\sailfndecodeSomeDSUBU}{ \lstinputlisting[language=sail]{sail_latex/sailfndecodeSomeDSUBU.tex}}

\newcommand{\sailfndecodeSomeDSUB}{ \lstinputlisting[language=sail]{sail_latex/sailfndecodeSomeDSUB.tex}}

\newcommand{\sailfndecodeSomeSUB}{ \lstinputlisting[language=sail]{sail_latex/sailfndecodeSomeSUB.tex}}

\newcommand{\sailfndecodeSomeSUBU}{ \lstinputlisting[language=sail]{sail_latex/sailfndecodeSomeSUBU.tex}}

\newcommand{\sailfndecodeSomeAND}{ \lstinputlisting[language=sail]{sail_latex/sailfndecodeSomeAND.tex}}

\newcommand{\sailfndecodeSomeANDI}{ \lstinputlisting[language=sail]{sail_latex/sailfndecodeSomeANDI.tex}}

\newcommand{\sailfndecodeSomeOR}{ \lstinputlisting[language=sail]{sail_latex/sailfndecodeSomeOR.tex}}

\newcommand{\sailfndecodeSomeORI}{ \lstinputlisting[language=sail]{sail_latex/sailfndecodeSomeORI.tex}}

\newcommand{\sailfndecodeSomeNOR}{ \lstinputlisting[language=sail]{sail_latex/sailfndecodeSomeNOR.tex}}

\newcommand{\sailfndecodeSomeXOR}{ \lstinputlisting[language=sail]{sail_latex/sailfndecodeSomeXOR.tex}}

\newcommand{\sailfndecodeSomeXORI}{ \lstinputlisting[language=sail]{sail_latex/sailfndecodeSomeXORI.tex}}

\newcommand{\sailfndecodeSomeLUI}{ \lstinputlisting[language=sail]{sail_latex/sailfndecodeSomeLUI.tex}}

\newcommand{\sailfndecodeSomeDSLL}{ \lstinputlisting[language=sail]{sail_latex/sailfndecodeSomeDSLL.tex}}

\newcommand{\sailfndecodeSomeDSLLthreetwo}{ \lstinputlisting[language=sail]{sail_latex/sailfndecodeSomeDSLLthreetwo.tex}}

\newcommand{\sailfndecodeSomeDSLLV}{ \lstinputlisting[language=sail]{sail_latex/sailfndecodeSomeDSLLV.tex}}

\newcommand{\sailfndecodeSomeDSRA}{ \lstinputlisting[language=sail]{sail_latex/sailfndecodeSomeDSRA.tex}}

\newcommand{\sailfndecodeSomeDSRAthreetwo}{ \lstinputlisting[language=sail]{sail_latex/sailfndecodeSomeDSRAthreetwo.tex}}

\newcommand{\sailfndecodeSomeDSRAV}{ \lstinputlisting[language=sail]{sail_latex/sailfndecodeSomeDSRAV.tex}}

\newcommand{\sailfndecodeSomeDSRL}{ \lstinputlisting[language=sail]{sail_latex/sailfndecodeSomeDSRL.tex}}

\newcommand{\sailfndecodeSomeDSRLthreetwo}{ \lstinputlisting[language=sail]{sail_latex/sailfndecodeSomeDSRLthreetwo.tex}}

\newcommand{\sailfndecodeSomeDSRLV}{ \lstinputlisting[language=sail]{sail_latex/sailfndecodeSomeDSRLV.tex}}

\newcommand{\sailfndecodeSomeSLL}{ \lstinputlisting[language=sail]{sail_latex/sailfndecodeSomeSLL.tex}}

\newcommand{\sailfndecodeSomeSLLV}{ \lstinputlisting[language=sail]{sail_latex/sailfndecodeSomeSLLV.tex}}

\newcommand{\sailfndecodeSomeSRA}{ \lstinputlisting[language=sail]{sail_latex/sailfndecodeSomeSRA.tex}}

\newcommand{\sailfndecodeSomeSRAV}{ \lstinputlisting[language=sail]{sail_latex/sailfndecodeSomeSRAV.tex}}

\newcommand{\sailfndecodeSomeSRL}{ \lstinputlisting[language=sail]{sail_latex/sailfndecodeSomeSRL.tex}}

\newcommand{\sailfndecodeSomeSRLV}{ \lstinputlisting[language=sail]{sail_latex/sailfndecodeSomeSRLV.tex}}

\newcommand{\sailfndecodeSomeSLT}{ \lstinputlisting[language=sail]{sail_latex/sailfndecodeSomeSLT.tex}}

\newcommand{\sailfndecodeSomeSLTI}{ \lstinputlisting[language=sail]{sail_latex/sailfndecodeSomeSLTI.tex}}

\newcommand{\sailfndecodeSomeSLTU}{ \lstinputlisting[language=sail]{sail_latex/sailfndecodeSomeSLTU.tex}}

\newcommand{\sailfndecodeSomeSLTIU}{ \lstinputlisting[language=sail]{sail_latex/sailfndecodeSomeSLTIU.tex}}

\newcommand{\sailfndecodeSomeMOVN}{ \lstinputlisting[language=sail]{sail_latex/sailfndecodeSomeMOVN.tex}}

\newcommand{\sailfndecodeSomeMOVZ}{ \lstinputlisting[language=sail]{sail_latex/sailfndecodeSomeMOVZ.tex}}

\newcommand{\sailfndecodeSomeMFHIrd}{ \lstinputlisting[language=sail]{sail_latex/sailfndecodeSomeMFHIrd.tex}}

\newcommand{\sailfndecodeSomeMFLOrd}{ \lstinputlisting[language=sail]{sail_latex/sailfndecodeSomeMFLOrd.tex}}

\newcommand{\sailfndecodeSomeMTHIrs}{ \lstinputlisting[language=sail]{sail_latex/sailfndecodeSomeMTHIrs.tex}}

\newcommand{\sailfndecodeSomeMTLOrs}{ \lstinputlisting[language=sail]{sail_latex/sailfndecodeSomeMTLOrs.tex}}

\newcommand{\sailfndecodeSomeMUL}{ \lstinputlisting[language=sail]{sail_latex/sailfndecodeSomeMUL.tex}}

\newcommand{\sailfndecodeSomeMULT}{ \lstinputlisting[language=sail]{sail_latex/sailfndecodeSomeMULT.tex}}

\newcommand{\sailfndecodeSomeMULTU}{ \lstinputlisting[language=sail]{sail_latex/sailfndecodeSomeMULTU.tex}}

\newcommand{\sailfndecodeSomeDMULT}{ \lstinputlisting[language=sail]{sail_latex/sailfndecodeSomeDMULT.tex}}

\newcommand{\sailfndecodeSomeDMULTU}{ \lstinputlisting[language=sail]{sail_latex/sailfndecodeSomeDMULTU.tex}}

\newcommand{\sailfndecodeSomeMADD}{ \lstinputlisting[language=sail]{sail_latex/sailfndecodeSomeMADD.tex}}

\newcommand{\sailfndecodeSomeMADDU}{ \lstinputlisting[language=sail]{sail_latex/sailfndecodeSomeMADDU.tex}}

\newcommand{\sailfndecodeSomeMSUB}{ \lstinputlisting[language=sail]{sail_latex/sailfndecodeSomeMSUB.tex}}

\newcommand{\sailfndecodeSomeMSUBU}{ \lstinputlisting[language=sail]{sail_latex/sailfndecodeSomeMSUBU.tex}}

\newcommand{\sailfndecodeSomeDIV}{ \lstinputlisting[language=sail]{sail_latex/sailfndecodeSomeDIV.tex}}

\newcommand{\sailfndecodeSomeDIVU}{ \lstinputlisting[language=sail]{sail_latex/sailfndecodeSomeDIVU.tex}}

\newcommand{\sailfndecodeSomeDDIV}{ \lstinputlisting[language=sail]{sail_latex/sailfndecodeSomeDDIV.tex}}

\newcommand{\sailfndecodeSomeDDIVU}{ \lstinputlisting[language=sail]{sail_latex/sailfndecodeSomeDDIVU.tex}}

\newcommand{\sailfndecodeSomeJoffset}{ \lstinputlisting[language=sail]{sail_latex/sailfndecodeSomeJoffset.tex}}

\newcommand{\sailfndecodeSomeJALoffset}{ \lstinputlisting[language=sail]{sail_latex/sailfndecodeSomeJALoffset.tex}}

\newcommand{\sailfndecodeSomeJRrs}{ \lstinputlisting[language=sail]{sail_latex/sailfndecodeSomeJRrs.tex}}

\newcommand{\sailfndecodeSomeJALR}{ \lstinputlisting[language=sail]{sail_latex/sailfndecodeSomeJALR.tex}}

\newcommand{\sailfndecodeSomeBEQ}{ \lstinputlisting[language=sail]{sail_latex/sailfndecodeSomeBEQ.tex}}

\newcommand{\sailsailfndecodeSomeBEQv}{ \lstinputlisting[language=sail]{sail_latex/sailsailfndecodeSomeBEQv.tex}}

\newcommand{\sailsailsailfndecodeSomeBEQvv}{ \lstinputlisting[language=sail]{sail_latex/sailsailsailfndecodeSomeBEQvv.tex}}

\newcommand{\sailsailsailsailfndecodeSomeBEQvvv}{ \lstinputlisting[language=sail]{sail_latex/sailsailsailsailfndecodeSomeBEQvvv.tex}}

\newcommand{\sailfndecodeSomeBCMPZ}{ \lstinputlisting[language=sail]{sail_latex/sailfndecodeSomeBCMPZ.tex}}

\newcommand{\sailsailfndecodeSomeBCMPZv}{ \lstinputlisting[language=sail]{sail_latex/sailsailfndecodeSomeBCMPZv.tex}}

\newcommand{\sailsailsailfndecodeSomeBCMPZvv}{ \lstinputlisting[language=sail]{sail_latex/sailsailsailfndecodeSomeBCMPZvv.tex}}

\newcommand{\sailsailsailsailfndecodeSomeBCMPZvvv}{ \lstinputlisting[language=sail]{sail_latex/sailsailsailsailfndecodeSomeBCMPZvvv.tex}}

\newcommand{\sailsailsailsailsailfndecodeSomeBCMPZvvvv}{ \lstinputlisting[language=sail]{sail_latex/sailsailsailsailsailfndecodeSomeBCMPZvvvv.tex}}

\newcommand{\sailsailsailsailsailsailfndecodeSomeBCMPZvvvvv}{ \lstinputlisting[language=sail]{sail_latex/sailsailsailsailsailsailfndecodeSomeBCMPZvvvvv.tex}}

\newcommand{\sailsailsailsailsailsailsailfndecodeSomeBCMPZvvvvvv}{ \lstinputlisting[language=sail]{sail_latex/sailsailsailsailsailsailsailfndecodeSomeBCMPZvvvvvv.tex}}

\newcommand{\sailsailsailsailsailsailsailsailfndecodeSomeBCMPZvvvvvvv}{ \lstinputlisting[language=sail]{sail_latex/sailsailsailsailsailsailsailsailfndecodeSomeBCMPZvvvvvvv.tex}}

\newcommand{\sailsailsailsailsailsailsailsailsailfndecodeSomeBCMPZvvvvvvvv}{ \lstinputlisting[language=sail]{sail_latex/sailsailsailsailsailsailsailsailsailfndecodeSomeBCMPZvvvvvvvv.tex}}

\newcommand{\sailsailsailsailsailsailsailsailsailsailfndecodeSomeBCMPZvvvvvvvvv}{ \lstinputlisting[language=sail]{sail_latex/sailsailsailsailsailsailsailsailsailsailfndecodeSomeBCMPZvvvvvvvvv.tex}}

\newcommand{\sailsailsailsailsailsailsailsailsailsailsailfndecodeSomeBCMPZvvvvvvvvvv}{ \lstinputlisting[language=sail]{sail_latex/sailsailsailsailsailsailsailsailsailsailsailfndecodeSomeBCMPZvvvvvvvvvv.tex}}

\newcommand{\sailsailsailsailsailsailsailsailsailsailsailsailfndecodeSomeBCMPZvvvvvvvvvvv}{ \lstinputlisting[language=sail]{sail_latex/sailsailsailsailsailsailsailsailsailsailsailsailfndecodeSomeBCMPZvvvvvvvvvvv.tex}}

\newcommand{\sailfndecodeSomeSYSCALL}{ \lstinputlisting[language=sail]{sail_latex/sailfndecodeSomeSYSCALL.tex}}

\newcommand{\sailfndecodeSomeBREAK}{ \lstinputlisting[language=sail]{sail_latex/sailfndecodeSomeBREAK.tex}}

\newcommand{\sailfndecodeSomeWAIT}{ \lstinputlisting[language=sail]{sail_latex/sailfndecodeSomeWAIT.tex}}

\newcommand{\sailfndecodeSomeTRAPREG}{ \lstinputlisting[language=sail]{sail_latex/sailfndecodeSomeTRAPREG.tex}}

\newcommand{\sailsailfndecodeSomeTRAPREGv}{ \lstinputlisting[language=sail]{sail_latex/sailsailfndecodeSomeTRAPREGv.tex}}

\newcommand{\sailsailsailfndecodeSomeTRAPREGvv}{ \lstinputlisting[language=sail]{sail_latex/sailsailsailfndecodeSomeTRAPREGvv.tex}}

\newcommand{\sailsailsailsailfndecodeSomeTRAPREGvvv}{ \lstinputlisting[language=sail]{sail_latex/sailsailsailsailfndecodeSomeTRAPREGvvv.tex}}

\newcommand{\sailsailsailsailsailfndecodeSomeTRAPREGvvvv}{ \lstinputlisting[language=sail]{sail_latex/sailsailsailsailsailfndecodeSomeTRAPREGvvvv.tex}}

\newcommand{\sailsailsailsailsailsailfndecodeSomeTRAPREGvvvvv}{ \lstinputlisting[language=sail]{sail_latex/sailsailsailsailsailsailfndecodeSomeTRAPREGvvvvv.tex}}

\newcommand{\sailfndecodeSomeTRAPIMM}{ \lstinputlisting[language=sail]{sail_latex/sailfndecodeSomeTRAPIMM.tex}}

\newcommand{\sailsailfndecodeSomeTRAPIMMv}{ \lstinputlisting[language=sail]{sail_latex/sailsailfndecodeSomeTRAPIMMv.tex}}

\newcommand{\sailsailsailfndecodeSomeTRAPIMMvv}{ \lstinputlisting[language=sail]{sail_latex/sailsailsailfndecodeSomeTRAPIMMvv.tex}}

\newcommand{\sailsailsailsailfndecodeSomeTRAPIMMvvv}{ \lstinputlisting[language=sail]{sail_latex/sailsailsailsailfndecodeSomeTRAPIMMvvv.tex}}

\newcommand{\sailsailsailsailsailfndecodeSomeTRAPIMMvvvv}{ \lstinputlisting[language=sail]{sail_latex/sailsailsailsailsailfndecodeSomeTRAPIMMvvvv.tex}}

\newcommand{\sailsailsailsailsailsailfndecodeSomeTRAPIMMvvvvv}{ \lstinputlisting[language=sail]{sail_latex/sailsailsailsailsailsailfndecodeSomeTRAPIMMvvvvv.tex}}

\newcommand{\sailfndecodeSomeLoad}{ \lstinputlisting[language=sail]{sail_latex/sailfndecodeSomeLoad.tex}}

\newcommand{\sailsailfndecodeSomeLoadv}{ \lstinputlisting[language=sail]{sail_latex/sailsailfndecodeSomeLoadv.tex}}

\newcommand{\sailsailsailfndecodeSomeLoadvv}{ \lstinputlisting[language=sail]{sail_latex/sailsailsailfndecodeSomeLoadvv.tex}}

\newcommand{\sailsailsailsailfndecodeSomeLoadvvv}{ \lstinputlisting[language=sail]{sail_latex/sailsailsailsailfndecodeSomeLoadvvv.tex}}

\newcommand{\sailsailsailsailsailfndecodeSomeLoadvvvv}{ \lstinputlisting[language=sail]{sail_latex/sailsailsailsailsailfndecodeSomeLoadvvvv.tex}}

\newcommand{\sailsailsailsailsailsailfndecodeSomeLoadvvvvv}{ \lstinputlisting[language=sail]{sail_latex/sailsailsailsailsailsailfndecodeSomeLoadvvvvv.tex}}

\newcommand{\sailsailsailsailsailsailsailfndecodeSomeLoadvvvvvv}{ \lstinputlisting[language=sail]{sail_latex/sailsailsailsailsailsailsailfndecodeSomeLoadvvvvvv.tex}}

\newcommand{\sailsailsailsailsailsailsailsailfndecodeSomeLoadvvvvvvv}{ \lstinputlisting[language=sail]{sail_latex/sailsailsailsailsailsailsailsailfndecodeSomeLoadvvvvvvv.tex}}

\newcommand{\sailsailsailsailsailsailsailsailsailfndecodeSomeLoadvvvvvvvv}{ \lstinputlisting[language=sail]{sail_latex/sailsailsailsailsailsailsailsailsailfndecodeSomeLoadvvvvvvvv.tex}}

\newcommand{\sailfndecodeSomeStore}{ \lstinputlisting[language=sail]{sail_latex/sailfndecodeSomeStore.tex}}

\newcommand{\sailsailfndecodeSomeStorev}{ \lstinputlisting[language=sail]{sail_latex/sailsailfndecodeSomeStorev.tex}}

\newcommand{\sailsailsailfndecodeSomeStorevv}{ \lstinputlisting[language=sail]{sail_latex/sailsailsailfndecodeSomeStorevv.tex}}

\newcommand{\sailsailsailsailfndecodeSomeStorevvv}{ \lstinputlisting[language=sail]{sail_latex/sailsailsailsailfndecodeSomeStorevvv.tex}}

\newcommand{\sailsailsailsailsailfndecodeSomeStorevvvv}{ \lstinputlisting[language=sail]{sail_latex/sailsailsailsailsailfndecodeSomeStorevvvv.tex}}

\newcommand{\sailsailsailsailsailsailfndecodeSomeStorevvvvv}{ \lstinputlisting[language=sail]{sail_latex/sailsailsailsailsailsailfndecodeSomeStorevvvvv.tex}}

\newcommand{\sailfndecodeSomeLWL}{ \lstinputlisting[language=sail]{sail_latex/sailfndecodeSomeLWL.tex}}

\newcommand{\sailfndecodeSomeLWR}{ \lstinputlisting[language=sail]{sail_latex/sailfndecodeSomeLWR.tex}}

\newcommand{\sailfndecodeSomeSWL}{ \lstinputlisting[language=sail]{sail_latex/sailfndecodeSomeSWL.tex}}

\newcommand{\sailfndecodeSomeSWR}{ \lstinputlisting[language=sail]{sail_latex/sailfndecodeSomeSWR.tex}}

\newcommand{\sailfndecodeSomeLDL}{ \lstinputlisting[language=sail]{sail_latex/sailfndecodeSomeLDL.tex}}

\newcommand{\sailfndecodeSomeLDR}{ \lstinputlisting[language=sail]{sail_latex/sailfndecodeSomeLDR.tex}}

\newcommand{\sailfndecodeSomeSDL}{ \lstinputlisting[language=sail]{sail_latex/sailfndecodeSomeSDL.tex}}

\newcommand{\sailfndecodeSomeSDR}{ \lstinputlisting[language=sail]{sail_latex/sailfndecodeSomeSDR.tex}}

\newcommand{\sailfndecodeSomeCACHE}{ \lstinputlisting[language=sail]{sail_latex/sailfndecodeSomeCACHE.tex}}

\newcommand{\sailfndecodeSomeSYNC}{ \lstinputlisting[language=sail]{sail_latex/sailfndecodeSomeSYNC.tex}}

\newcommand{\sailfndecodeSomeMFCzero}{ \lstinputlisting[language=sail]{sail_latex/sailfndecodeSomeMFCzero.tex}}

\newcommand{\sailsailfndecodeSomeMFCzerov}{ \lstinputlisting[language=sail]{sail_latex/sailsailfndecodeSomeMFCzerov.tex}}

\newcommand{\sailfndecodeSomeHCF}{ \lstinputlisting[language=sail]{sail_latex/sailfndecodeSomeHCF.tex}}

\newcommand{\sailsailfndecodeSomeHCFv}{ \lstinputlisting[language=sail]{sail_latex/sailsailfndecodeSomeHCFv.tex}}

\newcommand{\sailfndecodeSomeMTCzero}{ \lstinputlisting[language=sail]{sail_latex/sailfndecodeSomeMTCzero.tex}}

\newcommand{\sailsailfndecodeSomeMTCzerov}{ \lstinputlisting[language=sail]{sail_latex/sailsailfndecodeSomeMTCzerov.tex}}

\newcommand{\sailfndecodeSome}{ \lstinputlisting[language=sail]{sail_latex/sailfndecodeSome.tex}}

\newcommand{\sailsailfndecodeSomev}{ \lstinputlisting[language=sail]{sail_latex/sailsailfndecodeSomev.tex}}

\newcommand{\sailsailsailfndecodeSomevv}{ \lstinputlisting[language=sail]{sail_latex/sailsailsailfndecodeSomevv.tex}}

\newcommand{\sailsailsailsailfndecodeSomevvv}{ \lstinputlisting[language=sail]{sail_latex/sailsailsailsailfndecodeSomevvv.tex}}

\newcommand{\sailfndecodeSomeRDHWR}{ \lstinputlisting[language=sail]{sail_latex/sailfndecodeSomeRDHWR.tex}}

\newcommand{\sailfndecodeSomeERET}{ \lstinputlisting[language=sail]{sail_latex/sailfndecodeSomeERET.tex}}

\newcommand{\sailfndecodeSomeCGetPerm}{ \lstinputlisting[language=sail]{sail_latex/sailfndecodeSomeCGetPerm.tex}}

\newcommand{\sailfndecodeSomeCGetType}{ \lstinputlisting[language=sail]{sail_latex/sailfndecodeSomeCGetType.tex}}

\newcommand{\sailfndecodeSomeCGetBase}{ \lstinputlisting[language=sail]{sail_latex/sailfndecodeSomeCGetBase.tex}}

\newcommand{\sailfndecodeSomeCGetLen}{ \lstinputlisting[language=sail]{sail_latex/sailfndecodeSomeCGetLen.tex}}

\newcommand{\sailfndecodeSomeCGetTag}{ \lstinputlisting[language=sail]{sail_latex/sailfndecodeSomeCGetTag.tex}}

\newcommand{\sailfndecodeSomeCGetSealed}{ \lstinputlisting[language=sail]{sail_latex/sailfndecodeSomeCGetSealed.tex}}

\newcommand{\sailfndecodeSomeCGetCauserd}{ \lstinputlisting[language=sail]{sail_latex/sailfndecodeSomeCGetCauserd.tex}}

\newcommand{\sailfndecodeSomeCReturn}{ \lstinputlisting[language=sail]{sail_latex/sailfndecodeSomeCReturn.tex}}

\newcommand{\sailfndecodeSomeCGetOffset}{ \lstinputlisting[language=sail]{sail_latex/sailfndecodeSomeCGetOffset.tex}}

\newcommand{\sailfndecodeSomeCSetCausert}{ \lstinputlisting[language=sail]{sail_latex/sailfndecodeSomeCSetCausert.tex}}

\newcommand{\sailfndecodeSomeCAndPerm}{ \lstinputlisting[language=sail]{sail_latex/sailfndecodeSomeCAndPerm.tex}}

\newcommand{\sailfndecodeSomeCToPtr}{ \lstinputlisting[language=sail]{sail_latex/sailfndecodeSomeCToPtr.tex}}

\newcommand{\sailfndecodeSomeCPtrCmp}{ \lstinputlisting[language=sail]{sail_latex/sailfndecodeSomeCPtrCmp.tex}}

\newcommand{\sailsailfndecodeSomeCPtrCmpv}{ \lstinputlisting[language=sail]{sail_latex/sailsailfndecodeSomeCPtrCmpv.tex}}

\newcommand{\sailsailsailfndecodeSomeCPtrCmpvv}{ \lstinputlisting[language=sail]{sail_latex/sailsailsailfndecodeSomeCPtrCmpvv.tex}}

\newcommand{\sailsailsailsailfndecodeSomeCPtrCmpvvv}{ \lstinputlisting[language=sail]{sail_latex/sailsailsailsailfndecodeSomeCPtrCmpvvv.tex}}

\newcommand{\sailsailsailsailsailfndecodeSomeCPtrCmpvvvv}{ \lstinputlisting[language=sail]{sail_latex/sailsailsailsailsailfndecodeSomeCPtrCmpvvvv.tex}}

\newcommand{\sailsailsailsailsailsailfndecodeSomeCPtrCmpvvvvv}{ \lstinputlisting[language=sail]{sail_latex/sailsailsailsailsailsailfndecodeSomeCPtrCmpvvvvv.tex}}

\newcommand{\sailsailsailsailsailsailsailfndecodeSomeCPtrCmpvvvvvv}{ \lstinputlisting[language=sail]{sail_latex/sailsailsailsailsailsailsailfndecodeSomeCPtrCmpvvvvvv.tex}}

\newcommand{\sailsailsailsailsailsailsailsailfndecodeSomeCPtrCmpvvvvvvv}{ \lstinputlisting[language=sail]{sail_latex/sailsailsailsailsailsailsailsailfndecodeSomeCPtrCmpvvvvvvv.tex}}

\newcommand{\sailfndecodeSomeCIncOffset}{ \lstinputlisting[language=sail]{sail_latex/sailfndecodeSomeCIncOffset.tex}}

\newcommand{\sailfndecodeSomeCSetOffset}{ \lstinputlisting[language=sail]{sail_latex/sailfndecodeSomeCSetOffset.tex}}

\newcommand{\sailfndecodeSomeCSetBounds}{ \lstinputlisting[language=sail]{sail_latex/sailfndecodeSomeCSetBounds.tex}}

\newcommand{\sailfndecodeSomeCClearTag}{ \lstinputlisting[language=sail]{sail_latex/sailfndecodeSomeCClearTag.tex}}

\newcommand{\sailfndecodeSomeCFromPtr}{ \lstinputlisting[language=sail]{sail_latex/sailfndecodeSomeCFromPtr.tex}}

\newcommand{\sailfndecodeSomeCCheckPerm}{ \lstinputlisting[language=sail]{sail_latex/sailfndecodeSomeCCheckPerm.tex}}

\newcommand{\sailfndecodeSomeCCheckType}{ \lstinputlisting[language=sail]{sail_latex/sailfndecodeSomeCCheckType.tex}}

\newcommand{\sailfndecodeSomeCSeal}{ \lstinputlisting[language=sail]{sail_latex/sailfndecodeSomeCSeal.tex}}

\newcommand{\sailfndecodeSomeCUnseal}{ \lstinputlisting[language=sail]{sail_latex/sailfndecodeSomeCUnseal.tex}}

\newcommand{\sailfndecodeSomeCJALR}{ \lstinputlisting[language=sail]{sail_latex/sailfndecodeSomeCJALR.tex}}

\newcommand{\sailsailfndecodeSomeCJALRv}{ \lstinputlisting[language=sail]{sail_latex/sailsailfndecodeSomeCJALRv.tex}}

\newcommand{\sailsailfndecodeSomeCGetCauserdv}{ \lstinputlisting[language=sail]{sail_latex/sailsailfndecodeSomeCGetCauserdv.tex}}

\newcommand{\sailfndecodeSomeCSetCausers}{ \lstinputlisting[language=sail]{sail_latex/sailfndecodeSomeCSetCausers.tex}}

\newcommand{\sailfndecodeSomeCGetPCCcd}{ \lstinputlisting[language=sail]{sail_latex/sailfndecodeSomeCGetPCCcd.tex}}

\newcommand{\sailsailsailfndecodeSomeCJALRvv}{ \lstinputlisting[language=sail]{sail_latex/sailsailsailfndecodeSomeCJALRvv.tex}}

\newcommand{\sailsailfndecodeSomeCCheckPermv}{ \lstinputlisting[language=sail]{sail_latex/sailsailfndecodeSomeCCheckPermv.tex}}

\newcommand{\sailsailfndecodeSomeCCheckTypev}{ \lstinputlisting[language=sail]{sail_latex/sailsailfndecodeSomeCCheckTypev.tex}}

\newcommand{\sailsailfndecodeSomeCClearTagv}{ \lstinputlisting[language=sail]{sail_latex/sailsailfndecodeSomeCClearTagv.tex}}

\newcommand{\sailfndecodeSomeCMOVX}{ \lstinputlisting[language=sail]{sail_latex/sailfndecodeSomeCMOVX.tex}}

\newcommand{\sailsailsailsailfndecodeSomeCJALRvvv}{ \lstinputlisting[language=sail]{sail_latex/sailsailsailsailfndecodeSomeCJALRvvv.tex}}

\newcommand{\sailsailfndecodeSomeCGetPermv}{ \lstinputlisting[language=sail]{sail_latex/sailsailfndecodeSomeCGetPermv.tex}}

\newcommand{\sailsailfndecodeSomeCGetTypev}{ \lstinputlisting[language=sail]{sail_latex/sailsailfndecodeSomeCGetTypev.tex}}

\newcommand{\sailsailfndecodeSomeCGetBasev}{ \lstinputlisting[language=sail]{sail_latex/sailsailfndecodeSomeCGetBasev.tex}}

\newcommand{\sailsailfndecodeSomeCGetLenv}{ \lstinputlisting[language=sail]{sail_latex/sailsailfndecodeSomeCGetLenv.tex}}

\newcommand{\sailsailfndecodeSomeCGetTagv}{ \lstinputlisting[language=sail]{sail_latex/sailsailfndecodeSomeCGetTagv.tex}}

\newcommand{\sailsailfndecodeSomeCGetSealedv}{ \lstinputlisting[language=sail]{sail_latex/sailsailfndecodeSomeCGetSealedv.tex}}

\newcommand{\sailsailfndecodeSomeCGetOffsetv}{ \lstinputlisting[language=sail]{sail_latex/sailsailfndecodeSomeCGetOffsetv.tex}}

\newcommand{\sailfndecodeSomeCGetPCCSetOffset}{ \lstinputlisting[language=sail]{sail_latex/sailfndecodeSomeCGetPCCSetOffset.tex}}

\newcommand{\sailfndecodeSomeCReadHwr}{ \lstinputlisting[language=sail]{sail_latex/sailfndecodeSomeCReadHwr.tex}}

\newcommand{\sailfndecodeSomeCWriteHwr}{ \lstinputlisting[language=sail]{sail_latex/sailfndecodeSomeCWriteHwr.tex}}

\newcommand{\sailfndecodeSomeCGetAddr}{ \lstinputlisting[language=sail]{sail_latex/sailfndecodeSomeCGetAddr.tex}}

\newcommand{\sailsailfndecodeSomeCSealv}{ \lstinputlisting[language=sail]{sail_latex/sailsailfndecodeSomeCSealv.tex}}

\newcommand{\sailsailfndecodeSomeCUnsealv}{ \lstinputlisting[language=sail]{sail_latex/sailsailfndecodeSomeCUnsealv.tex}}

\newcommand{\sailsailfndecodeSomeCAndPermv}{ \lstinputlisting[language=sail]{sail_latex/sailsailfndecodeSomeCAndPermv.tex}}

\newcommand{\sailsailfndecodeSomeCSetOffsetv}{ \lstinputlisting[language=sail]{sail_latex/sailsailfndecodeSomeCSetOffsetv.tex}}

\newcommand{\sailsailfndecodeSomeCSetBoundsv}{ \lstinputlisting[language=sail]{sail_latex/sailsailfndecodeSomeCSetBoundsv.tex}}

\newcommand{\sailfndecodeSomeCSetBoundsExact}{ \lstinputlisting[language=sail]{sail_latex/sailfndecodeSomeCSetBoundsExact.tex}}

\newcommand{\sailsailfndecodeSomeCIncOffsetv}{ \lstinputlisting[language=sail]{sail_latex/sailsailfndecodeSomeCIncOffsetv.tex}}

\newcommand{\sailfndecodeSomeCBuildCap}{ \lstinputlisting[language=sail]{sail_latex/sailfndecodeSomeCBuildCap.tex}}

\newcommand{\sailfndecodeSomeCCopyType}{ \lstinputlisting[language=sail]{sail_latex/sailfndecodeSomeCCopyType.tex}}

\newcommand{\sailfndecodeSomeCCSeal}{ \lstinputlisting[language=sail]{sail_latex/sailfndecodeSomeCCSeal.tex}}

\newcommand{\sailsailfndecodeSomeCToPtrv}{ \lstinputlisting[language=sail]{sail_latex/sailsailfndecodeSomeCToPtrv.tex}}

\newcommand{\sailsailfndecodeSomeCFromPtrv}{ \lstinputlisting[language=sail]{sail_latex/sailsailfndecodeSomeCFromPtrv.tex}}

\newcommand{\sailfndecodeSomeCSub}{ \lstinputlisting[language=sail]{sail_latex/sailfndecodeSomeCSub.tex}}

\newcommand{\sailsailfndecodeSomeCMOVXv}{ \lstinputlisting[language=sail]{sail_latex/sailsailfndecodeSomeCMOVXv.tex}}

\newcommand{\sailsailsailfndecodeSomeCMOVXvv}{ \lstinputlisting[language=sail]{sail_latex/sailsailsailfndecodeSomeCMOVXvv.tex}}

\newcommand{\sailsailsailsailsailsailsailsailsailfndecodeSomeCPtrCmpvvvvvvvv}{ \lstinputlisting[language=sail]{sail_latex/sailsailsailsailsailsailsailsailsailfndecodeSomeCPtrCmpvvvvvvvv.tex}}

\newcommand{\sailsailsailsailsailsailsailsailsailsailfndecodeSomeCPtrCmpvvvvvvvvv}{ \lstinputlisting[language=sail]{sail_latex/sailsailsailsailsailsailsailsailsailsailfndecodeSomeCPtrCmpvvvvvvvvv.tex}}

\newcommand{\sailsailsailsailsailsailsailsailsailsailsailfndecodeSomeCPtrCmpvvvvvvvvvv}{ \lstinputlisting[language=sail]{sail_latex/sailsailsailsailsailsailsailsailsailsailsailfndecodeSomeCPtrCmpvvvvvvvvvv.tex}}

\newcommand{\sailsailsailsailsailsailsailsailsailsailsailsailfndecodeSomeCPtrCmpvvvvvvvvvvv}{ \lstinputlisting[language=sail]{sail_latex/sailsailsailsailsailsailsailsailsailsailsailsailfndecodeSomeCPtrCmpvvvvvvvvvvv.tex}}

\newcommand{\sailsailsailsailsailsailsailsailsailsailsailsailsailfndecodeSomeCPtrCmpvvvvvvvvvvvv}{ \lstinputlisting[language=sail]{sail_latex/sailsailsailsailsailsailsailsailsailsailsailsailsailfndecodeSomeCPtrCmpvvvvvvvvvvvv.tex}}

\newcommand{\sailsailsailsailsailsailsailsailsailsailsailsailsailsailfndecodeSomeCPtrCmpvvvvvvvvvvvvv}{ \lstinputlisting[language=sail]{sail_latex/sailsailsailsailsailsailsailsailsailsailsailsailsailsailfndecodeSomeCPtrCmpvvvvvvvvvvvvv.tex}}

\newcommand{\sailsailsailsailsailsailsailsailsailsailsailsailsailsailsailfndecodeSomeCPtrCmpvvvvvvvvvvvvvv}{ \lstinputlisting[language=sail]{sail_latex/sailsailsailsailsailsailsailsailsailsailsailsailsailsailsailfndecodeSomeCPtrCmpvvvvvvvvvvvvvv.tex}}

\newcommand{\sailsailsailsailsailsailsailsailsailsailsailsailsailsailsailsailfndecodeSomeCPtrCmpvvvvvvvvvvvvvvv}{ \lstinputlisting[language=sail]{sail_latex/sailsailsailsailsailsailsailsailsailsailsailsailsailsailsailsailfndecodeSomeCPtrCmpvvvvvvvvvvvvvvv.tex}}

\newcommand{\sailfndecodeSomeCTestSubset}{ \lstinputlisting[language=sail]{sail_latex/sailfndecodeSomeCTestSubset.tex}}

\newcommand{\sailfndecodeSomeCBX}{ \lstinputlisting[language=sail]{sail_latex/sailfndecodeSomeCBX.tex}}

\newcommand{\sailsailfndecodeSomeCBXv}{ \lstinputlisting[language=sail]{sail_latex/sailsailfndecodeSomeCBXv.tex}}

\newcommand{\sailfndecodeSomeCBZ}{ \lstinputlisting[language=sail]{sail_latex/sailfndecodeSomeCBZ.tex}}

\newcommand{\sailsailfndecodeSomeCBZv}{ \lstinputlisting[language=sail]{sail_latex/sailsailfndecodeSomeCBZv.tex}}

\newcommand{\sailsailfndecodeSomeCReturnv}{ \lstinputlisting[language=sail]{sail_latex/sailsailfndecodeSomeCReturnv.tex}}

\newcommand{\sailfndecodeSomeCCall}{ \lstinputlisting[language=sail]{sail_latex/sailfndecodeSomeCCall.tex}}

\newcommand{\sailfndecodeSomeClearRegs}{ \lstinputlisting[language=sail]{sail_latex/sailfndecodeSomeClearRegs.tex}}

\newcommand{\sailsailfndecodeSomeClearRegsv}{ \lstinputlisting[language=sail]{sail_latex/sailsailfndecodeSomeClearRegsv.tex}}

\newcommand{\sailsailsailfndecodeSomeClearRegsvv}{ \lstinputlisting[language=sail]{sail_latex/sailsailsailfndecodeSomeClearRegsvv.tex}}

\newcommand{\sailsailsailsailfndecodeSomeClearRegsvvv}{ \lstinputlisting[language=sail]{sail_latex/sailsailsailsailfndecodeSomeClearRegsvvv.tex}}

\newcommand{\sailfndecodeSomeCIncOffsetImmediate}{ \lstinputlisting[language=sail]{sail_latex/sailfndecodeSomeCIncOffsetImmediate.tex}}

\newcommand{\sailfndecodeSomeCSetBoundsImmediate}{ \lstinputlisting[language=sail]{sail_latex/sailfndecodeSomeCSetBoundsImmediate.tex}}

\newcommand{\sailfndecodeSomeCLoad}{ \lstinputlisting[language=sail]{sail_latex/sailfndecodeSomeCLoad.tex}}

\newcommand{\sailsailfndecodeSomeCLoadv}{ \lstinputlisting[language=sail]{sail_latex/sailsailfndecodeSomeCLoadv.tex}}

\newcommand{\sailsailsailfndecodeSomeCLoadvv}{ \lstinputlisting[language=sail]{sail_latex/sailsailsailfndecodeSomeCLoadvv.tex}}

\newcommand{\sailsailsailsailfndecodeSomeCLoadvvv}{ \lstinputlisting[language=sail]{sail_latex/sailsailsailsailfndecodeSomeCLoadvvv.tex}}

\newcommand{\sailsailsailsailsailfndecodeSomeCLoadvvvv}{ \lstinputlisting[language=sail]{sail_latex/sailsailsailsailsailfndecodeSomeCLoadvvvv.tex}}

\newcommand{\sailsailsailsailsailsailfndecodeSomeCLoadvvvvv}{ \lstinputlisting[language=sail]{sail_latex/sailsailsailsailsailsailfndecodeSomeCLoadvvvvv.tex}}

\newcommand{\sailsailsailsailsailsailsailfndecodeSomeCLoadvvvvvv}{ \lstinputlisting[language=sail]{sail_latex/sailsailsailsailsailsailsailfndecodeSomeCLoadvvvvvv.tex}}

\newcommand{\sailsailsailsailsailsailsailsailfndecodeSomeCLoadvvvvvvv}{ \lstinputlisting[language=sail]{sail_latex/sailsailsailsailsailsailsailsailfndecodeSomeCLoadvvvvvvv.tex}}

\newcommand{\sailsailsailsailsailsailsailsailsailfndecodeSomeCLoadvvvvvvvv}{ \lstinputlisting[language=sail]{sail_latex/sailsailsailsailsailsailsailsailsailfndecodeSomeCLoadvvvvvvvv.tex}}

\newcommand{\sailsailsailsailsailsailsailsailsailsailfndecodeSomeCLoadvvvvvvvvv}{ \lstinputlisting[language=sail]{sail_latex/sailsailsailsailsailsailsailsailsailsailfndecodeSomeCLoadvvvvvvvvv.tex}}

\newcommand{\sailsailsailsailsailsailsailsailsailsailsailfndecodeSomeCLoadvvvvvvvvvv}{ \lstinputlisting[language=sail]{sail_latex/sailsailsailsailsailsailsailsailsailsailsailfndecodeSomeCLoadvvvvvvvvvv.tex}}

\newcommand{\sailsailsailsailsailsailsailsailsailsailsailsailfndecodeSomeCLoadvvvvvvvvvvv}{ \lstinputlisting[language=sail]{sail_latex/sailsailsailsailsailsailsailsailsailsailsailsailfndecodeSomeCLoadvvvvvvvvvvv.tex}}

\newcommand{\sailsailsailsailsailsailsailsailsailsailsailsailsailfndecodeSomeCLoadvvvvvvvvvvvv}{ \lstinputlisting[language=sail]{sail_latex/sailsailsailsailsailsailsailsailsailsailsailsailsailfndecodeSomeCLoadvvvvvvvvvvvv.tex}}

\newcommand{\sailsailsailsailsailsailsailsailsailsailsailsailsailsailfndecodeSomeCLoadvvvvvvvvvvvvv}{ \lstinputlisting[language=sail]{sail_latex/sailsailsailsailsailsailsailsailsailsailsailsailsailsailfndecodeSomeCLoadvvvvvvvvvvvvv.tex}}

\newcommand{\sailfndecodeSomeCStore}{ \lstinputlisting[language=sail]{sail_latex/sailfndecodeSomeCStore.tex}}

\newcommand{\sailsailfndecodeSomeCStorev}{ \lstinputlisting[language=sail]{sail_latex/sailsailfndecodeSomeCStorev.tex}}

\newcommand{\sailsailsailfndecodeSomeCStorevv}{ \lstinputlisting[language=sail]{sail_latex/sailsailsailfndecodeSomeCStorevv.tex}}

\newcommand{\sailsailsailsailfndecodeSomeCStorevvv}{ \lstinputlisting[language=sail]{sail_latex/sailsailsailsailfndecodeSomeCStorevvv.tex}}

\newcommand{\sailsailsailsailsailfndecodeSomeCStorevvvv}{ \lstinputlisting[language=sail]{sail_latex/sailsailsailsailsailfndecodeSomeCStorevvvv.tex}}

\newcommand{\sailsailsailsailsailsailfndecodeSomeCStorevvvvv}{ \lstinputlisting[language=sail]{sail_latex/sailsailsailsailsailsailfndecodeSomeCStorevvvvv.tex}}

\newcommand{\sailsailsailsailsailsailsailfndecodeSomeCStorevvvvvv}{ \lstinputlisting[language=sail]{sail_latex/sailsailsailsailsailsailsailfndecodeSomeCStorevvvvvv.tex}}

\newcommand{\sailsailsailsailsailsailsailsailfndecodeSomeCStorevvvvvvv}{ \lstinputlisting[language=sail]{sail_latex/sailsailsailsailsailsailsailsailfndecodeSomeCStorevvvvvvv.tex}}

\newcommand{\sailfndecodeSomeCSC}{ \lstinputlisting[language=sail]{sail_latex/sailfndecodeSomeCSC.tex}}

\newcommand{\sailsailfndecodeSomeCSCv}{ \lstinputlisting[language=sail]{sail_latex/sailsailfndecodeSomeCSCv.tex}}

\newcommand{\sailfndecodeSomeCLC}{ \lstinputlisting[language=sail]{sail_latex/sailfndecodeSomeCLC.tex}}

\newcommand{\sailsailfndecodeSomeCLCv}{ \lstinputlisting[language=sail]{sail_latex/sailsailfndecodeSomeCLCv.tex}}

\newcommand{\sailsailsailfndecodeSomeCLCvv}{ \lstinputlisting[language=sail]{sail_latex/sailsailsailfndecodeSomeCLCvv.tex}}

\newcommand{\sailfndecodeSomeCtwoDumprt}{ \lstinputlisting[language=sail]{sail_latex/sailfndecodeSomeCtwoDumprt.tex}}

\newcommand{\sailfndecodeSomeRI}{ \lstinputlisting[language=sail]{sail_latex/sailfndecodeSomeRI.tex}}



\newcommand{\sailfnexecuteDADDIU}{ \lstinputlisting[language=sail]{sail_latex/sailfnexecuteDADDIU.tex}}

\newcommand{\sailfnexecuteDADDU}{ \lstinputlisting[language=sail]{sail_latex/sailfnexecuteDADDU.tex}}

\newcommand{\sailfnexecuteDADDI}{ \lstinputlisting[language=sail]{sail_latex/sailfnexecuteDADDI.tex}}

\newcommand{\sailfnexecuteDADD}{ \lstinputlisting[language=sail]{sail_latex/sailfnexecuteDADD.tex}}

\newcommand{\sailfnexecuteADD}{ \lstinputlisting[language=sail]{sail_latex/sailfnexecuteADD.tex}}

\newcommand{\sailfnexecuteADDI}{ \lstinputlisting[language=sail]{sail_latex/sailfnexecuteADDI.tex}}

\newcommand{\sailfnexecuteADDU}{ \lstinputlisting[language=sail]{sail_latex/sailfnexecuteADDU.tex}}

\newcommand{\sailfnexecuteADDIU}{ \lstinputlisting[language=sail]{sail_latex/sailfnexecuteADDIU.tex}}

\newcommand{\sailfnexecuteDSUBU}{ \lstinputlisting[language=sail]{sail_latex/sailfnexecuteDSUBU.tex}}

\newcommand{\sailfnexecuteDSUB}{ \lstinputlisting[language=sail]{sail_latex/sailfnexecuteDSUB.tex}}

\newcommand{\sailfnexecuteSUB}{ \lstinputlisting[language=sail]{sail_latex/sailfnexecuteSUB.tex}}

\newcommand{\sailfnexecuteSUBU}{ \lstinputlisting[language=sail]{sail_latex/sailfnexecuteSUBU.tex}}

\newcommand{\sailfnexecuteAND}{ \lstinputlisting[language=sail]{sail_latex/sailfnexecuteAND.tex}}

\newcommand{\sailfnexecuteANDI}{ \lstinputlisting[language=sail]{sail_latex/sailfnexecuteANDI.tex}}

\newcommand{\sailfnexecuteOR}{ \lstinputlisting[language=sail]{sail_latex/sailfnexecuteOR.tex}}

\newcommand{\sailfnexecuteORI}{ \lstinputlisting[language=sail]{sail_latex/sailfnexecuteORI.tex}}

\newcommand{\sailfnexecuteNOR}{ \lstinputlisting[language=sail]{sail_latex/sailfnexecuteNOR.tex}}

\newcommand{\sailfnexecuteXOR}{ \lstinputlisting[language=sail]{sail_latex/sailfnexecuteXOR.tex}}

\newcommand{\sailfnexecuteXORI}{ \lstinputlisting[language=sail]{sail_latex/sailfnexecuteXORI.tex}}

\newcommand{\sailfnexecuteLUI}{ \lstinputlisting[language=sail]{sail_latex/sailfnexecuteLUI.tex}}

\newcommand{\sailfnexecuteDSLL}{ \lstinputlisting[language=sail]{sail_latex/sailfnexecuteDSLL.tex}}

\newcommand{\sailfnexecuteDSLLthreetwo}{ \lstinputlisting[language=sail]{sail_latex/sailfnexecuteDSLLthreetwo.tex}}

\newcommand{\sailfnexecuteDSLLV}{ \lstinputlisting[language=sail]{sail_latex/sailfnexecuteDSLLV.tex}}

\newcommand{\sailfnexecuteDSRA}{ \lstinputlisting[language=sail]{sail_latex/sailfnexecuteDSRA.tex}}

\newcommand{\sailfnexecuteDSRAthreetwo}{ \lstinputlisting[language=sail]{sail_latex/sailfnexecuteDSRAthreetwo.tex}}

\newcommand{\sailfnexecuteDSRAV}{ \lstinputlisting[language=sail]{sail_latex/sailfnexecuteDSRAV.tex}}

\newcommand{\sailfnexecuteDSRL}{ \lstinputlisting[language=sail]{sail_latex/sailfnexecuteDSRL.tex}}

\newcommand{\sailfnexecuteDSRLthreetwo}{ \lstinputlisting[language=sail]{sail_latex/sailfnexecuteDSRLthreetwo.tex}}

\newcommand{\sailfnexecuteDSRLV}{ \lstinputlisting[language=sail]{sail_latex/sailfnexecuteDSRLV.tex}}

\newcommand{\sailfnexecuteSLL}{ \lstinputlisting[language=sail]{sail_latex/sailfnexecuteSLL.tex}}

\newcommand{\sailfnexecuteSLLV}{ \lstinputlisting[language=sail]{sail_latex/sailfnexecuteSLLV.tex}}

\newcommand{\sailfnexecuteSRA}{ \lstinputlisting[language=sail]{sail_latex/sailfnexecuteSRA.tex}}

\newcommand{\sailfnexecuteSRAV}{ \lstinputlisting[language=sail]{sail_latex/sailfnexecuteSRAV.tex}}

\newcommand{\sailfnexecuteSRL}{ \lstinputlisting[language=sail]{sail_latex/sailfnexecuteSRL.tex}}

\newcommand{\sailfnexecuteSRLV}{ \lstinputlisting[language=sail]{sail_latex/sailfnexecuteSRLV.tex}}

\newcommand{\sailfnexecuteSLT}{ \lstinputlisting[language=sail]{sail_latex/sailfnexecuteSLT.tex}}

\newcommand{\sailfnexecuteSLTI}{ \lstinputlisting[language=sail]{sail_latex/sailfnexecuteSLTI.tex}}

\newcommand{\sailfnexecuteSLTU}{ \lstinputlisting[language=sail]{sail_latex/sailfnexecuteSLTU.tex}}

\newcommand{\sailfnexecuteSLTIU}{ \lstinputlisting[language=sail]{sail_latex/sailfnexecuteSLTIU.tex}}

\newcommand{\sailfnexecuteMOVN}{ \lstinputlisting[language=sail]{sail_latex/sailfnexecuteMOVN.tex}}

\newcommand{\sailfnexecuteMOVZ}{ \lstinputlisting[language=sail]{sail_latex/sailfnexecuteMOVZ.tex}}

\newcommand{\sailfnexecuteMFHI}{ \lstinputlisting[language=sail]{sail_latex/sailfnexecuteMFHI.tex}}

\newcommand{\sailfnexecuteMFLO}{ \lstinputlisting[language=sail]{sail_latex/sailfnexecuteMFLO.tex}}

\newcommand{\sailfnexecuteMTHI}{ \lstinputlisting[language=sail]{sail_latex/sailfnexecuteMTHI.tex}}

\newcommand{\sailfnexecuteMTLO}{ \lstinputlisting[language=sail]{sail_latex/sailfnexecuteMTLO.tex}}

\newcommand{\sailfnexecuteMUL}{ \lstinputlisting[language=sail]{sail_latex/sailfnexecuteMUL.tex}}

\newcommand{\sailfnexecuteMULT}{ \lstinputlisting[language=sail]{sail_latex/sailfnexecuteMULT.tex}}

\newcommand{\sailfnexecuteMULTU}{ \lstinputlisting[language=sail]{sail_latex/sailfnexecuteMULTU.tex}}

\newcommand{\sailfnexecuteDMULT}{ \lstinputlisting[language=sail]{sail_latex/sailfnexecuteDMULT.tex}}

\newcommand{\sailfnexecuteDMULTU}{ \lstinputlisting[language=sail]{sail_latex/sailfnexecuteDMULTU.tex}}

\newcommand{\sailfnexecuteMADD}{ \lstinputlisting[language=sail]{sail_latex/sailfnexecuteMADD.tex}}

\newcommand{\sailfnexecuteMADDU}{ \lstinputlisting[language=sail]{sail_latex/sailfnexecuteMADDU.tex}}

\newcommand{\sailfnexecuteMSUB}{ \lstinputlisting[language=sail]{sail_latex/sailfnexecuteMSUB.tex}}

\newcommand{\sailfnexecuteMSUBU}{ \lstinputlisting[language=sail]{sail_latex/sailfnexecuteMSUBU.tex}}

\newcommand{\sailfnexecuteDIV}{ \lstinputlisting[language=sail]{sail_latex/sailfnexecuteDIV.tex}}

\newcommand{\sailfnexecuteDIVU}{ \lstinputlisting[language=sail]{sail_latex/sailfnexecuteDIVU.tex}}

\newcommand{\sailfnexecuteDDIV}{ \lstinputlisting[language=sail]{sail_latex/sailfnexecuteDDIV.tex}}

\newcommand{\sailfnexecuteDDIVU}{ \lstinputlisting[language=sail]{sail_latex/sailfnexecuteDDIVU.tex}}

\newcommand{\sailfnexecuteJ}{ \lstinputlisting[language=sail]{sail_latex/sailfnexecuteJ.tex}}

\newcommand{\sailfnexecuteJAL}{ \lstinputlisting[language=sail]{sail_latex/sailfnexecuteJAL.tex}}

\newcommand{\sailfnexecuteJR}{ \lstinputlisting[language=sail]{sail_latex/sailfnexecuteJR.tex}}

\newcommand{\sailfnexecuteJALR}{ \lstinputlisting[language=sail]{sail_latex/sailfnexecuteJALR.tex}}

\newcommand{\sailfnexecuteBEQ}{ \lstinputlisting[language=sail]{sail_latex/sailfnexecuteBEQ.tex}}

\newcommand{\sailfnexecuteBCMPZ}{ \lstinputlisting[language=sail]{sail_latex/sailfnexecuteBCMPZ.tex}}

\newcommand{\sailfnexecuteSYSCALL}{ \lstinputlisting[language=sail]{sail_latex/sailfnexecuteSYSCALL.tex}}

\newcommand{\sailfnexecuteBREAK}{ \lstinputlisting[language=sail]{sail_latex/sailfnexecuteBREAK.tex}}

\newcommand{\sailfnexecuteWAIT}{ \lstinputlisting[language=sail]{sail_latex/sailfnexecuteWAIT.tex}}

\newcommand{\sailfnexecuteTRAPREG}{ \lstinputlisting[language=sail]{sail_latex/sailfnexecuteTRAPREG.tex}}

\newcommand{\sailfnexecuteTRAPIMM}{ \lstinputlisting[language=sail]{sail_latex/sailfnexecuteTRAPIMM.tex}}

\newcommand{\sailfnexecuteLoad}{ \lstinputlisting[language=sail]{sail_latex/sailfnexecuteLoad.tex}}

\newcommand{\sailfnexecuteStore}{ \lstinputlisting[language=sail]{sail_latex/sailfnexecuteStore.tex}}

\newcommand{\sailfnexecuteLWL}{ \lstinputlisting[language=sail]{sail_latex/sailfnexecuteLWL.tex}}

\newcommand{\sailfnexecuteLWR}{ \lstinputlisting[language=sail]{sail_latex/sailfnexecuteLWR.tex}}

\newcommand{\sailfnexecuteSWL}{ \lstinputlisting[language=sail]{sail_latex/sailfnexecuteSWL.tex}}

\newcommand{\sailfnexecuteSWR}{ \lstinputlisting[language=sail]{sail_latex/sailfnexecuteSWR.tex}}

\newcommand{\sailfnexecuteLDL}{ \lstinputlisting[language=sail]{sail_latex/sailfnexecuteLDL.tex}}

\newcommand{\sailfnexecuteLDR}{ \lstinputlisting[language=sail]{sail_latex/sailfnexecuteLDR.tex}}

\newcommand{\sailfnexecuteSDL}{ \lstinputlisting[language=sail]{sail_latex/sailfnexecuteSDL.tex}}

\newcommand{\sailfnexecuteSDR}{ \lstinputlisting[language=sail]{sail_latex/sailfnexecuteSDR.tex}}

\newcommand{\sailfnexecuteCACHE}{ \lstinputlisting[language=sail]{sail_latex/sailfnexecuteCACHE.tex}}

\newcommand{\sailfnexecuteSYNC}{ \lstinputlisting[language=sail]{sail_latex/sailfnexecuteSYNC.tex}}

\newcommand{\sailfnexecuteMFCzero}{ \lstinputlisting[language=sail]{sail_latex/sailfnexecuteMFCzero.tex}}

\newcommand{\sailfnexecuteHCF}{ \lstinputlisting[language=sail]{sail_latex/sailfnexecuteHCF.tex}}

\newcommand{\sailfnexecuteMTCzero}{ \lstinputlisting[language=sail]{sail_latex/sailfnexecuteMTCzero.tex}}

\newcommand{\sailfnexecuteTLBWI}{ \lstinputlisting[language=sail]{sail_latex/sailfnexecuteTLBWI.tex}}

\newcommand{\sailfnexecuteTLBWR}{ \lstinputlisting[language=sail]{sail_latex/sailfnexecuteTLBWR.tex}}

\newcommand{\sailfnexecuteTLBR}{ \lstinputlisting[language=sail]{sail_latex/sailfnexecuteTLBR.tex}}

\newcommand{\sailfnexecuteTLBP}{ \lstinputlisting[language=sail]{sail_latex/sailfnexecuteTLBP.tex}}

\newcommand{\sailfnexecuteRDHWR}{ \lstinputlisting[language=sail]{sail_latex/sailfnexecuteRDHWR.tex}}

\newcommand{\sailfnexecuteERET}{ \lstinputlisting[language=sail]{sail_latex/sailfnexecuteERET.tex}}

\newcommand{\sailfnexecuteCGetPerm}{ \lstinputlisting[language=sail]{sail_latex/sailfnexecuteCGetPerm.tex}}

\newcommand{\sailfnexecuteCGetType}{ \lstinputlisting[language=sail]{sail_latex/sailfnexecuteCGetType.tex}}

\newcommand{\sailfnexecuteCGetBase}{ \lstinputlisting[language=sail]{sail_latex/sailfnexecuteCGetBase.tex}}

\newcommand{\sailfnexecuteCGetOffset}{ \lstinputlisting[language=sail]{sail_latex/sailfnexecuteCGetOffset.tex}}

\newcommand{\sailfnexecuteCGetLen}{ \lstinputlisting[language=sail]{sail_latex/sailfnexecuteCGetLen.tex}}

\newcommand{\sailfnexecuteCGetTag}{ \lstinputlisting[language=sail]{sail_latex/sailfnexecuteCGetTag.tex}}

\newcommand{\sailfnexecuteCGetSealed}{ \lstinputlisting[language=sail]{sail_latex/sailfnexecuteCGetSealed.tex}}

\newcommand{\sailfnexecuteCGetAddr}{ \lstinputlisting[language=sail]{sail_latex/sailfnexecuteCGetAddr.tex}}

\newcommand{\sailfnexecuteCGetPCC}{ \lstinputlisting[language=sail]{sail_latex/sailfnexecuteCGetPCC.tex}}

\newcommand{\sailfnexecuteCGetPCCSetOffset}{ \lstinputlisting[language=sail]{sail_latex/sailfnexecuteCGetPCCSetOffset.tex}}

\newcommand{\sailfnexecuteCGetCause}{ \lstinputlisting[language=sail]{sail_latex/sailfnexecuteCGetCause.tex}}

\newcommand{\sailfnexecuteCSetCause}{ \lstinputlisting[language=sail]{sail_latex/sailfnexecuteCSetCause.tex}}

\newcommand{\sailfnexecuteCReadHwr}{ \lstinputlisting[language=sail]{sail_latex/sailfnexecuteCReadHwr.tex}}

\newcommand{\sailfnexecuteCWriteHwr}{ \lstinputlisting[language=sail]{sail_latex/sailfnexecuteCWriteHwr.tex}}

\newcommand{\sailfnexecuteCAndPerm}{ \lstinputlisting[language=sail]{sail_latex/sailfnexecuteCAndPerm.tex}}

\newcommand{\sailfnexecuteCToPtr}{ \lstinputlisting[language=sail]{sail_latex/sailfnexecuteCToPtr.tex}}

\newcommand{\sailfnexecuteCSub}{ \lstinputlisting[language=sail]{sail_latex/sailfnexecuteCSub.tex}}

\newcommand{\sailfnexecuteCPtrCmp}{ \lstinputlisting[language=sail]{sail_latex/sailfnexecuteCPtrCmp.tex}}

\newcommand{\sailfnexecuteCIncOffset}{ \lstinputlisting[language=sail]{sail_latex/sailfnexecuteCIncOffset.tex}}

\newcommand{\sailfnexecuteCIncOffsetImmediate}{ \lstinputlisting[language=sail]{sail_latex/sailfnexecuteCIncOffsetImmediate.tex}}

\newcommand{\sailfnexecuteCSetOffset}{ \lstinputlisting[language=sail]{sail_latex/sailfnexecuteCSetOffset.tex}}

\newcommand{\sailfnexecuteCSetBounds}{ \lstinputlisting[language=sail]{sail_latex/sailfnexecuteCSetBounds.tex}}

\newcommand{\sailfnexecuteCSetBoundsImmediate}{ \lstinputlisting[language=sail]{sail_latex/sailfnexecuteCSetBoundsImmediate.tex}}

\newcommand{\sailfnexecuteCSetBoundsExact}{ \lstinputlisting[language=sail]{sail_latex/sailfnexecuteCSetBoundsExact.tex}}

\newcommand{\sailfnexecuteCClearTag}{ \lstinputlisting[language=sail]{sail_latex/sailfnexecuteCClearTag.tex}}

\newcommand{\sailfnexecuteCMOVX}{ \lstinputlisting[language=sail]{sail_latex/sailfnexecuteCMOVX.tex}}

\newcommand{\sailfnexecuteClearRegs}{ \lstinputlisting[language=sail]{sail_latex/sailfnexecuteClearRegs.tex}}

\newcommand{\sailfnexecuteCFromPtr}{ \lstinputlisting[language=sail]{sail_latex/sailfnexecuteCFromPtr.tex}}

\newcommand{\sailfnexecuteCBuildCap}{ \lstinputlisting[language=sail]{sail_latex/sailfnexecuteCBuildCap.tex}}

\newcommand{\sailfnexecuteCCopyType}{ \lstinputlisting[language=sail]{sail_latex/sailfnexecuteCCopyType.tex}}

\newcommand{\sailfnexecuteCCheckPerm}{ \lstinputlisting[language=sail]{sail_latex/sailfnexecuteCCheckPerm.tex}}

\newcommand{\sailfnexecuteCCheckType}{ \lstinputlisting[language=sail]{sail_latex/sailfnexecuteCCheckType.tex}}

\newcommand{\sailfnexecuteCTestSubset}{ \lstinputlisting[language=sail]{sail_latex/sailfnexecuteCTestSubset.tex}}

\newcommand{\sailfnexecuteCSeal}{ \lstinputlisting[language=sail]{sail_latex/sailfnexecuteCSeal.tex}}

\newcommand{\sailfnexecuteCCSeal}{ \lstinputlisting[language=sail]{sail_latex/sailfnexecuteCCSeal.tex}}

\newcommand{\sailfnexecuteCUnseal}{ \lstinputlisting[language=sail]{sail_latex/sailfnexecuteCUnseal.tex}}

\newcommand{\sailfnexecuteCCall}{ \lstinputlisting[language=sail]{sail_latex/sailfnexecuteCCall.tex}}

\newcommand{\sailsailfnexecuteCCallv}{ \lstinputlisting[language=sail]{sail_latex/sailsailfnexecuteCCallv.tex}}

\newcommand{\sailfnexecuteCReturn}{ \lstinputlisting[language=sail]{sail_latex/sailfnexecuteCReturn.tex}}

\newcommand{\sailfnexecuteCBX}{ \lstinputlisting[language=sail]{sail_latex/sailfnexecuteCBX.tex}}

\newcommand{\sailfnexecuteCBZ}{ \lstinputlisting[language=sail]{sail_latex/sailfnexecuteCBZ.tex}}

\newcommand{\sailfnexecuteCJALR}{ \lstinputlisting[language=sail]{sail_latex/sailfnexecuteCJALR.tex}}

\newcommand{\sailfnexecuteCLoad}{ \lstinputlisting[language=sail]{sail_latex/sailfnexecuteCLoad.tex}}

\newcommand{\sailfnexecuteCStore}{ \lstinputlisting[language=sail]{sail_latex/sailfnexecuteCStore.tex}}

\newcommand{\sailfnexecuteCSC}{ \lstinputlisting[language=sail]{sail_latex/sailfnexecuteCSC.tex}}

\newcommand{\sailfnexecuteCLC}{ \lstinputlisting[language=sail]{sail_latex/sailfnexecuteCLC.tex}}

\newcommand{\sailfnexecuteCtwoDump}{ \lstinputlisting[language=sail]{sail_latex/sailfnexecuteCtwoDump.tex}}

\newcommand{\sailfnexecuteRI}{ \lstinputlisting[language=sail]{sail_latex/sailfnexecuteRI.tex}}



\newcommand{\sailsupportedinstructions}{\label{zsupportedzyinstructions} \lstinputlisting[language=sail]{sail_latex/sailsupportedinstructions.tex}}

\newcommand{\sailfnsupportedinstructions}{\label{zsupportedzyinstructions} \lstinputlisting[language=sail]{sail_latex/sailfnsupportedinstructions.tex}}


  \prefixval{my_function}
  \prefixfn{my_function}
\end{lstlisting}

The generated definitions are created wrapped in customisable macros
that can be overridden to change the formatting of the Sail code. For
\verb+\sailfn+ there is a macro \verb+\saildocfn+ that must be defined,
and similarly for the other Sail toplevel types.

\paragraph{Cross-referencing} For each macro \verb+\sail+\emph{X}\verb+{id}+ there is a macro
\verb+\sailref+\emph{X}\verb+{id}{text}+ which creates a
hyper-reference to the original definition. This requires the
hyper-ref package.

\subsection{Other options}

Here we summarize most of the other options available for
Sail. Debugging options (usually for debugging Sail itself) are
indicated by starting with the letter \verb+d+.

\begin{itemize}
\item {\verb+-v+} Print the Sail version.

\item {\verb+-help+} Print a list of options.

\item {\verb+-no_warn+} Turn off warnings.

\item {\verb+-enum_casts+} Allow elements of enumerations to be
  automatically cast to numbers.

\item \verb+-memo_z3+ Memoize calls to the Z3 solver. This can greatly
  improve typechecking times if you are repeatedly typechecking the
  same specification while developing it.

\item \verb+-no_lexp_bounds_check+ Turn off bounds checking in the left
  hand side of assignments.

\item \verb+-no_effects+ Turn off effect checking. May break some
  backends that assume effects are properly checked.

\item \verb+-undefined_gen+ Generate functions that create undefined
  values of user-defined types. Every type \ll{T} will get a
  \ll{undefined_T} function created for it. This flag is set
  automatically by some backends that want to re-write \ll{undefined}.

\item \verb+-just_check+ Force Sail to terminate immediately after
  typechecking.

\item \verb+-dno_cast+ Force Sail to never perform type coercions
  under any circumstances.

\item \verb+-dtc_verbose <verbosity>+ Make the typechecker print a
  trace of typing judgements. If the verbosity level is 1, then this
  should only include fairly readable judgements about checking and
  inference rules. If verbosity is 2 then it will include a large
  amount of debugging information. This option can be useful to
  diagnose tricky type-errors, especially if the error message isn't
  very good.

\item \verb+-ddump_tc_ast+ Write the typechecked AST to stdout after
  typechecking

\item \verb+-ddump_rewrite_ast <prefix>+ Write the AST out after each
  re-writing pass. The output from each pass is placed in a file
  starting with \verb+prefix+.

\item \verb+-dsanity+ Perform extra sanity checks on the AST.

\item \verb+-dmagic_hash+ Allow the \# symbol in identifiers. It's
  currently used as a magic symbol to separate generated identifiers
  from those the user can write, so this option allows for the output
  of the various other debugging options to be fed back into Sail.
\end{itemize}
