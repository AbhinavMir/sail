\section{A tutorial \riscv\ example}
\label{sec:riscv}

We introduce the basic features of Sail via a small example from our
\riscv\ model that includes just two instructions: add immediate and
load double. We defining the default order (see \ref{sec:vec} for
details), and including the Sail prelude.

\begin{lstlisting}
default Order dec
$include <prelude.sail>
\end{lstlisting}

\noindent The Sail prelude is very minimal, and it is expected that
Sail specifications will usually build upon it. Some Sail
specifications are derived from pre-existing pseudocode, which already
use specific idioms---our Sail ARM specification uses ZeroExtend and
SignExtend mirroring the ASL code, whereas our MIPS and RISC-V
specification use EXTZ and EXTS for the same functions. Therefore for
this example we define zero-extension and sign-extension functions as:

\sailval{EXTZ}
\sailfn{EXTZ}

\sailval{EXTS}
\sailfn{EXTS}

We now define an integer type synonym xlen, which for this example
will be equal to 64. Sail supports definitions which are generic over
both regular types, and integers (think const generics in C++, but
more expressive). We also create a type \ll{xlenbits} for bitvectors
of length \ll{xlen}.

\sailtype{xlen}
\sailtype{xlen_bytes}
\sailtype{xlenbits}

For the purpose of this example, we also introduce a type synonym for
bitvectors of length 5, which represent registers.

\sailtype{regbits}

We now set up some basic architectural state. First creating a
register of type \ll{xlenbits} for both the program counter \ll{PC}, and
the next program counter, \ll{nextPC}. We define the general purpose
registers as a vector of 32 \ll{xlenbits} bitvectors. The \ll{dec}
keyword isn't important in this example, but Sail supports two
different numbering schemes for (bit)vectors \ll{inc}, for most
significant bit is zero, and \ll{dec} for least significant bit is
zero. We then define a getter and setter for the registers, which
ensure that the zero register is treated specially (in
\riscv\ register 0 is always hardcoded to be 0). Finally we overload
both the read (\ll{rX}) and write (\ll{wX}) functions as simply
\ll{X}. This allows us to write registers as \ll{X(r) = value} and
read registers as \ll{value = X(r)}. Sail supports flexible ad-hoc
overloading, and has an expressive l-value language in assignments,
with the aim of allowing pseudo-code like definitions.

\begin{lstlisting}
register PC : xlenbits
register nextPC : xlenbits

register Xs : vector(32, dec, xlenbits)
\end{lstlisting}

\sailval{rX}
\sailfn{rX}

\sailval{wX}
\sailfn{wX}

\sailoverloadAX

We also give a function \ll{MEMr} for reading memory, this function
just points at a builtin we have defined elsewhere.  The builtin is
very general, so we also derive a simpler \ll{read_mem} function.

\sailval{MEMr}

\sailvalreadMem
\sailfnreadMem

It's common when defining architecture specifications to break
instruction semantics down into separate functions that handle
decoding (possibly even in several stages) into custom intermediate
datatypes and executing the decoded instructions. However it's often
desirable to group the relevant parts of these functions and datatypes
together in one place, as they would usually be found in an
architecture reference manual. To support this Sail supports
\emph{scattered} definitions. First we give types for the execute and
decode functions, as well as the \ll{ast} union.

\sailtype{iop}

\begin{lstlisting}
scattered union ast

val decode : bits(32) -> option(ast)

val execute : ast -> unit
\end{lstlisting}

Now we provide the clauses for the add-immediate \ll{ast} type, as
well as its execute and decode clauses. We can define the decode
function by directly pattern matching on the bitvector representing
the instruction. Sail supports vector concatenation patterns (\ll{@}
is the vector concatenation operator), and uses the types provided
(e.g. \ll{bits(12)} and \ll{regbits}) to destructure the vector in the
correct way. We use the \ll{EXTS} function that sign-extends
its argument.

\begin{lstlisting}
union clause ast = ITYPE : (bits(12), regbits, regbits, iop)
\end{lstlisting}

\sailfclITYPEdecode
\sailfclITYPEexecute

\noindent Now we do the same thing for the load-double instruction:

\begin{lstlisting}
union clause ast = LOAD : (bits(12), regbits, regbits)
\end{lstlisting}

\sailfclLOADdecode
\sailfclLOADexecute

\noindent Finally we define the fallthrough case for the decode function. Note
that the clauses in a scattered function will be matched in the order
they appear in the file.
